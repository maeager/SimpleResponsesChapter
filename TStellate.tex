%%%%%%%%%%%%%%%%%%%%%%%%%%%%%%%%%%%%%%%%%%%%% 

\section{T~stellate cell units}


\subsection{Background}
\label{sec:backgrTS}


% \begin{itemize}
% \item histology staining
% \begin{itemize}
% \item multipolar (Nissl staining) and stellate (Golgi impregnation) \citep{Osen:1969,BrawerMorestEtAl:1974,Lorente:1981}
% \end{itemize}
% \item axonal projections
% \begin{itemize}
% \item \textbf{T} trapezoid body or ventral acoustic stria (T stellate), and \textbf{D} dorsal acoustic stria (D stellate) \citep{OertelWuEtAl:1990}
% \item T stellates project locally in VCN and deep layer of DCN $\rightarrow$ match Chopper \citep{RhodeOertelEtAl:1983,SmithRhode:1989}
% \item D stellate projections wide in VCN,DCN and cCN $\rightarrow$ match OnC \citep{SmithRhode:1989}
% \end{itemize}
% \item somatic innervation: type 1 (few somatic) type II (many somatic and dendritic) \citep{Cant:1981}
% \item anatomical and physiological: \citet{SmithRhode:1989}
% \begin{itemize}
% \item T stellate $\rightarrow$ type 1 $\rightarrow$ chopping PSTH
% \item D stellate $\rightarrow$ type 2 $\rightarrow$ OnC PSTH
% \end{itemize}
% \item Chopper subdivision:
% \begin{itemize}
% \item sustained and transient (no evidence for anatomical differences in T\~{}stellate cells)
% \end{itemize}
% \item dendritic differences \citep{DoucetRyugo:1997,DoucetRyugo:2006}
% \begin{itemize}
% \item planar (T) and radial (D)
% \end{itemize}
% \end{itemize}

% \subsection{Acoustic Response of T stellate cells}
% \label{sec-1_2}



% \begin{itemize}
% \item regular, tonic response to tones \citep{RhodeOertelEtAl:1983,SmithRhode:1989,BlackburnSachs:1989}
% \item ``Chopping'' precise regular timing that degrades throughout stimulus\citep{YoungRobertEtAl:1988,BlackburnSachs:1989}
% \begin{itemize}
% \item sustained (70\%) $\rightarrow$ constant rate, ISI histogram sharp, CV \textless 0.3, CV constant
% \item transient (30\%) $\rightarrow$ rate decreases, CV starts below 0.3 then varies
% \end{itemize}
% \item Inhibition
% \begin{itemize}
% \item Gly, GABA tuned on frequency to reduce peak excitation \citep{CasparyBackoffEtAl:1994}
% \item inhibitory side bands mainly D stellate \citep{FerragamoGoldingEtAl:1998a} but periolivary also contribute \citep{AdamsWarr:1976,Adams:1983,ShoreHelfertEtAl:1991,OstapoffBensonEtAl:1997}
% \end{itemize}
% \end{itemize}
%\citep{PalombiCaspary:1992,RhodeSmith:1986,NelkenYoung:1994,PaoliniClareyEtAl:2005,PaoliniClareyEtAl:2004}  

% \begin{itemize}
% \item sustained firing despite AN adaptation
% \begin{itemize}
% \item signals the sound intensity consistently, hence precise level information
% \item Phasic also do level, but tonic suits encoding of spectrum across population since encoding the peaks and valleys is relatively independent of time after onset of sound \citep{BlackburnSachs:1990,May:2003,MayPrellEtAl:1998,MaySachs:1998}
% \item suits encong of envelope of sounds, important for speech (envelops under 50 Hz \citep{ShannonZengEtAl:1995}
% \end{itemize}
% \item AM coding in choppers encoded over wide range of intensities\citep{RhodeGreenberg:1994,FrisinaSmithEtAl:1990}
% \begin{itemize}
% \item other work in Am coding by CN neurons \citep{Moller:1972,Moller:1974a,Moller:1974,MooreCashin:1974,Frisina:1984,PalmerWinterEtAl:1986,KimRhodeEtAl:1986,WinterPalmer:1990a,Palmer:1990,PalmerWinter:1992,FrisinaSmithEtAl:1990a,Frisina:1983,GorodetskaiaBibikov:1985,RhodeGreenberg:1994,ShofnerSheftEtAl:1996,FrisinaKarcichEtAl:1996,DAngeloSterbingEtAl:2003,Aggarwal:2003}
% \end{itemize}
% \item phasic firing in AN maintained by bushy
% \begin{itemize}
% \item phasic info important: enhances formant transitions, and provides accurate information about the location of sound sources even in reverberant environments, critical in hearing \citep{DelgutteKiang:1984,DelgutteKiang:1984a,DelgutteKiang:1984b,DelgutteKiang:1984c,DelgutteKiang:1984d,DavoreIhlefeldEtAl:2009}
% \end{itemize}
% \end{itemize}
% \subsection{Mechanisms making phasic input tonic}
% \label{sec-1_3}


%    steady depolarising current shows intracellular ability to be tonic \citep{Oertel:1983,OertelWuEtAl:1988} BUT - how does the input remain stable given AN adaptation?


% \begin{enumerate}
% \item selective processing of HSR and LSR input
% \begin{enumerate}
% \item feed-forward excitation in TS cells
% \begin{itemize}
% \item axon collaterals in local isofrequency (most cells in PVCN are TS cells)
% \end{itemize}
% \item co-activation of phasic inhibition
% \begin{itemize}
% \item DS inhibition ispi and contralaterally
% \begin{itemize}
% \item onset inhibition strongest, affecting TS cells after first spike
% \item broad tuning sharpens FSL
% \end{itemize}
% \item TV sharply tuned inhibition (Ferr98)
% \begin{itemize}
% \item TV response variable and non-monotonic
% \item \citep{Rhode:1999}  labelled TV cells phasic in aneasthetised cats
% \item unanesthetised cats and gerbils are phasic or tonic  \citep{DingVoigt:1997,ShofnerYoung:1985}
% \end{itemize}
% \item Others - Glycine from ipsi periolivary region, GABA from both
%           periolivary regions \citep{AdamsWarr:1976,ShoreHelfertEtAl:1991,OstapoffBensonEtAl:1997}
% \end{itemize}
% \item Absense of LT potassium in TS
% \begin{itemize}
% \item labelled \citep{ManisMarx:1991,BalOertel:2001,FerragamoOertel:2002,CaoShatadalEtAl:2007}
% \item unlabelled \citep{RothmanManis:2003,RothmanManis:2003a,RothmanManis:2003b,Rothman:1999}
% \end{itemize}
% \item Activation of NMDA
% \begin{itemize}
% \item \citep{CaoOertel:2010} shows TS cells activate large currents through NMDA receptors
% \item NMDA longer lasting, reducing phasic nature of input
% \end{itemize}
% \item Little synaptic depression
% \begin{itemize}
% \item SD less tahn bushy and octopus \citep{WuOertel:1987,ChandaXu-Friedman:2010,CaoOertel:2010}
% \item excitation of TS adapts less than other VCN neurons
% \end{itemize}
% \end{enumerate}
% \end{enumerate}
% \subsection{Mechanisms of tonic firing obscure temporal features}
% \label{sec-1_4}



% \begin{itemize}
% \item sFSL largest in TS of core VCN units by 1 msec $\rightarrow$ onset inhibition + longer integration time \citep{GisbergenGrashuisEtAl:1975,GisbergenGrashuisEtAl:1975a,GisbergenGrashuisEtAl:1975b,YoungRobertEtAl:1988,PaoliniClareyEtAl:2004}
% \item integration window longest for choppers \citep{McGinleyOertel:2006}
% \item inhibition from high CF units alters FSL to tones \citep{Wickesberg:1996}
% \item Onset: Volley of Excitation + feedforward excitation + DS inhibition
% \item After onset: Phasic excitation + feedforward excitation + NMDA
% activation + TV inhibition (+ small DS inhibition) + GABA inhibition = stable
% excitation but loss of temporal features
% \end{itemize}

% \subsection{Neuromodulatory effects in T stellate cells}
% \label{sec-1_5}



% \begin{itemize}
% \item sensitive to neuromodulatory currents \citep{FujinoOertel:2001}
% \begin{itemize}
% \item high input resistance $\rightarrow$ amplify small current inputs \citep{FujinoOertel:2001}
% \item no LKT in TS,  LKT makes bushy and optopus insensitive to steady currents \citep{OertelFujino:2001,McGinleyOertel:2006}
% \item Ih higher in TS \& activated more at lower potentials than in bushy and octopus, so that it is less active at rest
% \item high resistance $\rightarrow$ greater voltage changes in small modulating current $\rightarrow$ Ih can be modulated by G-protein coupledreceptors, hence making TS more excitable when Ih activated \citep{RodriguesOertel:2006}
% \end{itemize}
% \end{itemize}

% \begin{enumerate}
% \item Driving inputs
% \end{enumerate}
% Proximal dendrites and at the soma:

% \begin{itemize}
% \item ANF provide glutamatergic excitation for T stellates  \citep{Cant:1981,FerragamoGoldingEtAl:1998a,Alibardi:1998a}
% \begin{itemize}
% \item only 5 or 6 in mice \citep{FerragamoGoldingEtAl:1998a,CaoOertel:2010}
% \end{itemize}
% \item Recurrent excitation from other T stellate cells \citep{FerragamoGoldingEtAl:1998a}
% \item Glycine from DS cells \citep{FerragamoGoldingEtAl:1998a}
% \item Glycine from TV cells \citep{WickesbergOertel:1990,ZhangOertel:1993b}
% \item Neuromodulatory
% \end{itemize}
%      No signs of PSP or PSCs hence distal or G-protein coupled, effects on time-course minimal
     
% a. Golgi cells (GABA)

% \begin{itemize}
% \item no IPSPs or IPSCs but presence of GABAa receptors and response changes to bicuculine \citep{WuOertel:1986,OertelWickesberg:1993,FerragamoGoldingEtAl:1998a}
% \item dend filter obscures PSPs
% \item Golgi cells are GABAergic and lie within the granule cell domains around the VCN and terminate near the fine distal dendrites of T stellate cells
% \end{itemize}
% b. Periolivary cells (GABA + GAD - glutamic acid decarboxylase) 

% \begin{itemize}
% \item observed in PVCN close to TS \citep{AdamsMugnaini:1987}
% \item GAD effectively Glycine \citep{GoldingOertel:1997}
% \item can also arise from GABAergic neurons in ipsi LNTB and DM Periolivary
% \end{itemize}


The work by Tony Paolini, Janine Clarey, Karina Needham and others at the Royal
Victorian Eye and Ear Hospital were pivotal in collecting the T stellate cell
experimental data used in this section.  Accurate developing my network model of
the cochlear nucleus, in particular my interest in chopper units and T~stellate
cells.  Figure~\ref{fig:PaoliniAIV} shows the intracellular acoustic
classification of chopper units in the rat into three distinct types
\citep{PaoliniClareyEtAl:2005}: CS chopper sustained, and two transient choppers
CT1 and CT2.  This work expands on the extracellular classification by
\citep{BlackburnSachs:1989} into slowly or transiently adapting and sustained
choppers in cats. The intracellular traces in Figure~\ref{fig:PaoliniAIV} will
form the basis for the optimisation routine of T~stellate cell model


\begin{figure}[htb]
  \centering
\subfloat[Chopper Sustained]{\resizebox{0.3\textwidth}{!}{\includegraphics{TStellate/CS-01-864-004}}}\hfill%\quad%
\subfloat[Chopper Transient 1]{\resizebox{0.3\textwidth}{!}{\includegraphics{TStellate/CT1-01-857-007}}}\hfill%\\
\subfloat[Chopper Transient 2]{\resizebox{0.3\textwidth}{!}{\includegraphics{TStellate/CT2-01-305-014}}}%\quad%
%\subfloat[Onset Chopper]{\resizebox{0.35\textwidth}{!}{\includegraphics{TStellate/OC-99-812-013}}}
\caption[Average intracelluar response data in stellate cells in rats.]{Average
  intracelluar response to CF tone 30dB above depolarisation threshold in
  stellate
  cells~\citep[Reproduced~from~][]{PaoliniClareyEtAl:2005}. A. Sustained chopper
  unit 01-864-004. B. Transient chopper type 1 unit 01-857-007, CF
  8.9kHz. C. Transient chopper type 2 unit 01-305-014. D. Onset chopper unit
  99-812-013. Hyperpolarisation after tone indicated by
  asterix.  \label{fig:PaoliniAIV}}
\end{figure}

\yellownote{Permission needed for Paolini plots}


The categorisation via coefficient of variation is shown in
Figure~\ref{fig:PaoliniCVdata}. Sustained choppers maintain a stable CV below 0.2
throughout the entire stimulus. The transient chopper optimisation had two types
defined by \citep{PaoliniClareyEtAl:2005}.  The first CT type is categorised
with CV starting below 0.2 then rising, hence the name transient, but stays
below 0.3.  The second CT type is regular in the first 10 ms period, but rises
to 0.3 or above throughout the stimulus.

\begin{figure}[htb]
  \centering
  \resizebox{0.5\textwidth}{!}{\includegraphics{TStellate/PaoliniCV.eps}}
  \caption{Regularity in chopper units \citep[Data reproduced from
Fig.~2,~][]{PaoliniClareyEtAl:2005}}
  \label{fig:PaoliniCVdata}
\end{figure}



$\overline{g}_{\rm KHT},\bar{g}_{\rm KHT}$

\subsection{Implementation}

\yellownote{Para 1: Nordlie table~\ref{tab:TSModelSummary}A. Using R\&M
  type 1-t, Other previous models}


\yellownote{Synaptic inputs known and unknown, included and not included in
  model.  Previous work include and excludes}



Figure~\ref{fig:TSinputs} shows the expected response of a T~stellate cell
to individual connections from different cell types. The membrane
parameters for the single compartment T~stellate cell model are default
except for sodium conductance set to zero. In this example, excitation from
the afferent \ANF~inputs (\LSR~Figure~\ref{fig:TSinputs}A and
\HSR~Figure~\ref{fig:TSinputs}B) show a large depolarisation.  \HSR~inputs
show a rapid onset and a slowly adapting sustained depolarisation.


\yellownote{Actual parameters: diameter =19.5$\mu$m, $\overline{g}_{\rm
    Na}=0$, $\overline{g}_{\rm KHT}=0.0189416$ S cm$^{-2}$,
  $\overline{g}_{\rm leak}=0.000473539$ S cm$^{-2}$, $\overline{g}_{\rm
    h}=6.20392e-05$ S cm$^{-2}$, $\overline{g}_{\rm KA}=0.01539$ S
  cm$^{-2}$,E$_{\rm rev}=-65$ mV, E$_{\rm Na}=50$ mV, E$_{\rm K}=-70$ mV.}


\begin{figure}[htb]
  \centering
  \resizebox{0.6\textwidth}{!}{\includegraphics{TStellate/baseline_exc}}
  \resizebox{0.6\textwidth}{!}{\includegraphics{TStellate/baseline_inh}}
  \caption[Response of T~stellate cells to isolated synaptic
  inputs]{Intracellular membrane voltage response of a T~stellate cell model to
    isolated synaptic inputs. A pure tone stimulus of 8.2~kHz at 85 dB~SPL was
    presented to the CN network. The CF of the recorded T~stellate cell was
    8.267~kHz.  Single stimulus responses are shown as a thin line and average
    response over 25 repetitions is shown as the dark line. A. 30 LSR ANF
    synapses. B. 20 HSR ANF synapsexs. C. 20 D stellate cell glycinergic
    synapses. D. 15 Golgi cell GABA$_{\rm A}$ synapses. All weights were set to
    $0.0005\,\mu{\rm S}$ and the sodium conductance set to zero.  The parameters
    for synapases were: excitatory (tau = 0.36 msec), glycinergic (tau1=0.4 and
    tau2=2.5 msec), and GABAergic (tau1=0.26 and tau2=5.43
    msec).\label{fig:TSinputs}}
\end{figure}

\begin{figure}[htb]
  \centering
  \resizebox{0.6\textwidth}{!}{\includegraphics{TStellate/baseline_exc}}
  \resizebox{0.6\textwidth}{!}{\includegraphics{TStellate/baseline_jitter.eps}}
  % \caption[Response of T~stellate cells to isolated synaptic
  % inputs]{Intracellular membrane voltage response of a T~stellate cell
  %   model to isolated synaptic inputs. A pure tone stimulus of 8.2~kHz at
  %   85 dB~SPL was presented to the CN network. The CF of the recorded
  %   T~stellate cell was 8.267~kHz.  Single stimulus responses are shown as
  %   a thin line and average response over 25 repetitions is shown as the
  %   dark line. A. 30 LSR ANF synapses. B. 20 HSR ANF synapsexs. C. 20 D
  %   stellate cell glycinergic synapses. D. 15 Golgi cell GABA$_{\rm A}$
  %   synapses. All weights were set to $0.0005\,\mu{\rm S}$ and the sodium
  %   conductance set to zero.  The parameters for synapases were: excitatory
  %   (tau = 0.36 msec), glycinergic (tau1=0.4 and tau2=2.5 msec), and
  %   GABAergic (tau1=0.26 and tau2=5.43 msec).\label{fig:TSExcinputs}}
  \caption[]{Jitter of AN input T~stellate Optimisation results}
  \label{fig:CSjitter}
\end{figure}




\clearpage
\subsection{Optimisation Results}

Choosing optimisation data - average population data or individual exemplar
based on Paolini rat data.


What is not shown in Figure~\ref{fig:TSinputs} is firstly the non-linear
dynamics of the action potential in the neural model; and secondly the
hyperpolarisation activation of I$_{\rm h}$ and I$_{\rm KA}$ currents due
to inhibitory input.  The second factor is important in enhancing the first
spike response across the T~stellate cells \citep{PaoliniClareyEtAl:2004}to
an oncoming stimulus, since the \OnC~units will respond earlier and provide
fast inhibition

%       \begin{figure}[htb]
%         \centering\resizebox{0.95\textwidth}{!}{%
%           \includegraphics{RateLevel/psthsingle90.0.eps}%
% %          \includegraphics{RateLevel/TS_ratelevel.eps}}
%       \end{figure}

The final error for the optimisation routine was calculated as the square
root of the mean squared error between the first 4 CV periods and the three
voltage measures. The measures of membrane voltage were independant of RMP
included a) the onset ratio ($v_{\rm 5-10}/v_{\rm 20-30}$), b) adaptation
of sustained depolarisation ($v_{\rm 20-21} - v_{\rm 50-51}$), and c)
post-tone hyperpolarization ($v_{\rm RMP} - v_{\rm 60-61}$).


\subsubsection{Sustained chopper}

Figure~\ref{fig:CSresults} shows the optimsation results for the sustained
chopper (CS) unit 01-864-004 from \citep{PaoliniClareyEtAl:2005}, with CF
3.9~kHz. This unit had a depolarisation threshold at 10 dB and a firing rate
saturation at 75 dB. The stimulus for the optimisation routine was a pure tone
30~dB above depolarisation threshold at CF (3.9~kHz 40~dB SPL), 5 ms on/off
cosine squared ramp.  A 20 ms delay was included to allow the model to reach a
sustained RMP. The CS model unit had CF of 3.872~kHz due to the separation of
frequency channels in the model.

The best parameters found by the optimisation and shown in
Figure~\ref{fig:CSresults}, were very high excitatory input (\wLSRTS 0.00286796,
\wHSRTS 0.00201089) and moderate to high inhibition (\wDSTS 0.000291181, \wTVTS
0.000884174, \wGLGTS 0.000749812).  A low input stimulus level means that the
majority of synaptic input came from HSR fibres.

There was a slight increase in the leak conductance ($\gleak=6.57345e-4$) from
the initial value (4.907e-4), causing a decrease in the input resistance to
120~M$\Omega$ from 163~M$\Omega$. The final error, calculated with more weight
on the CV variables, was 0.839952.  Post-tone hyperpolarisation (reference
1.30035, best CS model -0.029843) and the first CV value (ref 0.1543, best CS
model 0.49706) had the greatest difference. The poor initial CV response is due
to the irregular first spike.  The large excitatory HSR input causes spikes
occuring after tone onset, just before the stimulus invoked spike arrival at the
TS cell 3 ms later, but still included in the CV calculation.


\yellownote{Note the CS optimisation had added weight to the CV values in the
  error calculation.  The PSTH and IV trace do not even show a decent chopping
  behaviour.}  

\yellownote{The peaks for the IV are 3.66, 6.77, and 10.99
  ms. The AN model input is affected by the rat audiogram, the low CF and low
  level.  This shifts the arrival time of spikes to the TS model, and a raw IV
  comparison would need to be shifted by approx 2 ms.  The onset ratio and the
  first CV value are effected by this.}

\yellownote{Synaptic weight in the model above 0.002 will initiate a spike at
  each input spike, similar to the calyx synapse on bushy cells. This means that
  the TS cell will respond with a primary-like or primary with notch
  PSTH. Dendritic filtering could improve this but the high synaptic weights do
  not seem like a realistic physiological option for small bouton endings in the
  CN.}

\begin{figure}[htb]
  \centering
 
\resizebox{0.95\textwidth}{!}{\includegraphics{TStellate_CS/multiplotIVCV.eps}}
  \caption[CS T~stellate Optimisation results]{Optimisation results for the CS
    type T~stellate cell. The reference data for CS unit 01-864-004 from
    \citep{PaoliniClareyEtAl:2005}, with CF=3.9~kHz, is shown in blue and the
    optimised model is shown in black (CF=3.872 kHz). The stimulus (pure tone
    3.9~kHz stimulus, 40dB SPL, 20 ms delay, 40 ms post-tone, 5 ms on/off ramp)
    was repeted 50 times for this figure but during the optimisation routine the
    repetition was 25. A. PSTH B. Coefficient of variation averaged over 10 ms
    periods C. Average membrane voltage (note shaded areas indicate areas used
    in the measures in D) D. Measures of membrane voltage including a) onset
    ratio ($v_{\rm 5-10}/v_{\rm 20-30}$), b) adaptation of sustained
    depolarisation ($v_{\rm 20-21} - v_{\rm 50-51}$ mV), and c) post-tone
    hyperpolarization ($v_{\rm RMP} - v_{\rm 60-61}$ mV). RMP is calculated from
    -10 to 0 ms.  The final error for the CS unit model was 0.839952, with
    double weighting for CV values.  The RMP for the reference unit and best
    parameter model was -45.1 and -64.22, respectively.  }
  \label{fig:CSresults}
\end{figure}

\yellownote{The RMP of the experimental data was relatively high, and if this
  was produced in a model the spontaneous rate would be close to 100 sp/s. The
  IV measures allow for comparison of the experimental and model data that is
  independent of the RMP.}

\yellownote{Hand tuning creates many types of CS models, but the best effects
  are when you smooth out the input with dendritic filtering.  This can be used in a
  single compartment using a higher time constant on the synapse or better,
  using a double-exponential synapse with longer onset and decay time
  constants. Very high excitatory inputs saturates the cell and it will spike
  repetedly in a regular fashion but the onset is too precise. See CT2 IV trace
  for an example of precise onset. I was able to generated a good IV trace when
  the full trace was the error function at higher SPL and higher CF, but this
  was unable to generate the same trace for 3.9 KHz at 40 dB or a sustained
  CV. }

\clearpage
\subsubsection{Transient Chopper 1}

The CT1 unit chosen for the optimisation was unit 01-857-007
\citep{PaoliniClareyEtAl:2005} (CF=8.2~kHz).  The depolarisation threshold for
this unit was 55~dB, close to the average 47.2~dB for CT1 units. The stimulus
for the optimisation routine was an 85~dB 8.2 kHz pure tone and the exemplar TS
cell model had CF 8.27~kHz.


The optimisation results of CT1 ares shown in Figure~\ref{fig:CT1results} with a
final error of 0.839141 without weighting.  Moderate excitatory synaptic weights
(\wLSRTS 0.000822193, \wHSRTS 0.000620294) is balanced with moderate to low
glycinergic synaptic input (\wDSTS 0.000148518, \wTVTS 0.000195416). The
GABAergic input weight was very low (\wGLGTS 1.70742e-05), but the loud input
stimulus (85 dB) would cause reasonable inhibition to the TS cell throughout the
stimulus.  The leak conductance had a slight decrease (\gleak 0.000409819).


\begin{figure}[htb]
  \centering
%  \resizebox{0.95\textwidth}{!}{\includegraphics{TStellate_CT1/optplot.eps}}
%\resizebox{0.95\textwidth}{!}{\includegraphics{TStellate/TStellate_CT1/multiplotIVCV}}\\
%  \resizebox{0.95\textwidth}{!}{\includegraphics{TStellate_CT1/multiplotIVCV}}
 \resizebox{0.95\textwidth}{!}{\includegraphics{TStellate_CT1/multiplotIVCV3}}
  \caption[CT1 T~stellate Optimisation results]{CT1 T~stellate cell model
    optimisation results. The RMP for the reference unit and best parameter
    model was -55.98 and -64.97 mV, respectively. The final error was 0.839141,
    without weighting. See Figure~\ref{fig:CSresults} for figure configuration.
    A. PSTH, B. CV, C. Average membrane voltage, and D. Voltage measures.}
  \label{fig:CT1results}
\end{figure}

\clearpage
\subsubsection{Transient Chopper 2}


The second transiently adapting chopper unit's difference to CT1 is reflected in
its average membrane potential and CV. CT2 has a higher reduction in onset
depolarization (90.2\%) compared to CT1, and post-tone inhibition is greater
(8.82 mV).  The exemplar CT2 unit (01-305-014, CF 12.4~kHz \citep*{PaoliniClareyEtAl:2005}),
has a low depolarising threshold (5 dB) so the stimulus is set to 35 dB SPL. The model T stellate cell had a CF of 12.427 kHz.

%    chopper transient type 2
%    // unit 01-305-014 \citep{PaoliniClareyEtAl:2005}
%    //CV (10ms segs) first 10m below 0.2, then rises above 0.3
%    // rate saturation at 5 dB
%    spl      = 35   //30 dB above depolarisation threshold
%    tonefreq = 12400 //Hz

% #RMP	-60.6332	-64.957
% #IVOnset	0.902367	0.962511
% #IVAdaptation	8.82784	7.8878
% #IVOffset	3.24212	2.18856

Figure~\ref{fig:CT2results} shows the results of the CT2 optimsiation.  The
resting membrane potential for the exemplar and model CT2 unit was -60.6332
-64.957 mV, respectively.


\begin{figure}[htb]
  \centering
 %\resizebox{0.95\textwidth}{!}{\includegraphics{/media/data/Work/cnstellate/TStellate_CT2/multiplotIVCV}}\\
% 
\resizebox{0.95\textwidth}{!}{\includegraphics{TStellate_CT2/multiplotIVCV3}}
  \caption[CT2 T~stellate Optimisation results]{CT2 T~stellate Optimisation
    results.  The RMP for the reference unit and best CT1 model was -60.63 and
   -64.95, respectively.  See Figure~\ref{fig:CSresults} for figure
    configuration.  A. PSTH, B. CV, C. Average membrane voltage, and D. Voltage
    measures.}
  \label{fig:CT2results}
\end{figure}



{\small% % D~----------------------------------
\noindent%
\begin{tabularx}{\linewidth}{|X|c|c|c|c|c|}
\hdr{6}{F}{Optimisation} \\ \hline
       \textbf{Parameters}         &    \textbf{Name}     & \textbf{Range} &\multicolumn{3}{|c|}{\textbf{Best Values}} \\
                                   &                      &                & CS&    CT1     & CT2 \\\hline
Weight of HSR syn on TS  ($\mu$S)  &       \wHSRTS        &  [1e-5,0.03]   &   &0.000708301 & 0.00168911\\
Weight of LSR syn on TS  ($\mu$S)  &       \wLSRTS        &  [1e-5,0.03]   &   & (0.0025)   & 0.00140628\\
 %  Number of synapses, LSR to TS  &       \nLSRTV        &     [0,64]     &   &            & \\
 %  Number of synapses, HSR to TS  &       \nHSRTV        &     [0,64]     &   &            & \\
 Weight of DS syn on TS  ($\mu$S)  &        \wDSTS        &  [1e-5,0.03]   &   &    --      & --\\
%Number of DS syn on TS  ($\mu$S)  &        \nDSTS        &     [0,64]     &   &            & \\
 Weight of TV syn on TS  ($\mu$S)  &        \wTVTS        &  [1e-5,0.03]   &   &    --      & --\\
%Number of TV syn on TS  ($\mu$S)  &        \nTVTS        &     [0,64]     &   &            & \\
Weight of GLG syn on TS  ($\mu$S)  &       \wGLGTS        &  [1e-5,0.03]   &   &    --      & --\\
%Number of GLG syn on TS  ($\mu$S) &       \nGLGTS        &     [0,64]     &   &            & \\ \hline
TS~leak conductance (S cm$^{-2}$)  & $\overline{g}_{\rm leak}$ &  [1e-6,0.003] &   & 0.000860145 & 8.9326e-05\\ 
 TS~KHT conductance (S cm$^{-2}$)  & $\overline{g}_{\rm KHT}$  &  [1e-6,0.003]&   &  (0.0189)   & 0.0226787\\ 
    TS~reversal potential (mV)     &    E$_{\rm rev}$     &   [-45,-70]    &   &  -54.26    & -63.502\\ \hline
                       \multicolumn{3}{|l|}{Error}                         &   &  11.886    & 31.481\\ \hline 
\end{tabularx}
}




\subsection{Verification}

% \subsubsection{Effect of DS: time frequency enhancement}

% Go on Paolini's proposal

% \subsubsection{Effect of TV: comb filter and echo suppression}


%       \subsection{Tone Response}

%       \begin{figure}[htb]
%         \centering\resizebox{0.95\textwidth}{!}{%
%           \includegraphics{RateLevel/psthsingle90.0.eps}%
% %          \includegraphics{RateLevel/TS_ratelevel.eps}}
%       \end{figure}
%       \begin{figure}[htb]
%         \centering\resizebox{0.95\textwidth}{!}{%
%  %         \includegraphics{RateLevel/response_area.0.eps}%
%   %        \includegraphics{RateLevel/response_area_log2.0.eps}}
%       \end{figure}
%       \begin{figure}[htb]
%         \centering\resizebox{0.95\textwidth}{!}{%
%           % \includegraphics{RateLevel/response_area.0.eps}
%           \includegraphics{RateLevel/psthall90.0.eps}%
%           \includegraphics{RateLevel/psthVlevel.0.eps}}
%       \end{figure}


%       \clearpage
%       \subsection{Noise Response}
%       \begin{figure}[htb]
%         \centering\resizebox{0.95\textwidth}{!}{%
%           \includegraphics{NoiseRateLevel/psthsingle120.0.eps}%
%           \includegraphics{NoiseRateLevel/TS_ratelevel.eps}}
%       \end{figure}
%       \begin{figure}[htb]
%         \centering\resizebox{0.95\textwidth}{!}{%
%           \includegraphics{NoiseRateLevel/response_area.0.eps}%
%           \includegraphics{NoiseRateLevel/response_area_log2.0.eps}}
%       \end{figure}
%       \begin{figure}[htb]
%         \centering\resizebox{0.95\textwidth}{!}{%
%           % \includegraphics{RateLevel/response_area.0.eps}
%           \includegraphics{NoiseRateLevel/psthall90.0.eps}%
%           \includegraphics{NoiseRateLevel/psthVlevel.0.eps}}
%       \end{figure}


%       \clearpage
%       \subsection{Masked Noise and Tone}
%       \begin{figure}[h!]
%        \centering\resizebox{0.95\textwidth}{!}{\includegraphics{MaskedRateLevel/psthsingle90.0.eps}\includegraphics{MaskedRateLevel/TS_ratelevel.eps}}
%       \end{figure}
%       \begin{figure}[h!]
%         \centering\resizebox{0.95\textwidth}{!}{%
%           \includegraphics{MaskedRateLevel/response_area.0.eps}%
%           \includegraphics{MaskedRateLevel/response_area_log2.0.eps}}
%       \end{figure}

%       \begin{figure}[h!]
%         \centering\resizebox{0.95\textwidth}{!}{%
%           % \includegraphics{RateLevel/response_area.0.eps}
%           \includegraphics{MaskedRateLevel/psthall90.0.eps}%
%           \includegraphics{MaskedRateLevel/psthVlevel.0.eps}}
%       \end{figure}
%       \clearpage
%       \subsection{Masked Response Area}
%       \begin{figure}[h!]
%         \centering\resizebox{0.95\textwidth}{!}{%
%           \includegraphics{MaskedResponseCurve/psthsingle5810.0.eps}%
%           \includegraphics{MaskedResponseCurve/TS_masked.eps}}
%       \end{figure}
%       \begin{figure}[h!]
%         \centering\resizebox{0.95\textwidth}{!}{%
%           \includegraphics{MaskedResponseCurve/response_area.0.eps}%
%           \includegraphics{MaskedResponseCurve/response_area_log2log2.0.eps}}
%       \end{figure}

%       \begin{figure}[h!]
%         \centering\resizebox{0.95\textwidth}{!}{%
%           % \includegraphics{RateLevel/response_area.0.eps}
%           \includegraphics{MaskedResponseCurve/psthall5810.0.eps}%
%           \includegraphics{MaskedResponseCurve/psthVmod.0.eps}}
%       \end{figure}
%       \clearpage


%%% Local Variables: 
%%% mode: latex
%%% mode: tex-fold
%%% TeX-master: "SimpleResponses"
%%% TeX-PDF-mode: nil
%%% End: 
