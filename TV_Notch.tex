
\graphicspath{{/media/data/Work/cnstellate/TV_notch/}{/media/data/Work/Responses/}{/media/data/Work/cnstellate/}{/media/data/Work/thesis/ans2010/gfx/}}
\newpage
\section[TV Cell Model]{Tuberculo-ventral cell model: Asymmetric broadband inhibition }
\label{sec:tv-cell-model}
% - A ------------------------------------------------------------------------------

\subsection{Background}

Tuberculoventral (TV) cells are glycinergic, inhibitory cells found in
the deep layers of the DCN that send axon collaterals to the
VCN\@. They are characterized as having a non-monotonic response to
tones with increasing sound level and respond poorly to broadband
noise \citep{SpirouDavisEtAl:1999,NelkenYoung:1997,ReissYoung:2005}.
Anterograde labeling in the DCN suggests TV cells project
tonotopically to the AVCN not just on-CF, but also to the low and high
frequency side bands in the AVCN
\citep{MunirathinamOstapoffEtAl:2004,OstapoffMorestEtAl:1999}.
Ultra-structural labeling of synapses in the rat DCN suggest TV cells
are inhibited by glycinergic DS cells and from sources in the DCN but
excitatory inputs were not found from TS cells in the rat
\citep{Rubio:2005}. Evidence in the mouse suggests otherwise since
intracellular responses from labeled TV cells in the mouse show clear
excitatory input from TS cells and diffuse inhibitory input from DS
cells \citep{ZhangOertel:1993b,WickesbergOertel:1993}.

\medskip{} 

TV or vertical cells receive monosynaptic excitatory input from
auditory nerve fibers
\citep{OertelWu:1989}\citep{ZhangOertel:1993b}. Taken together, these
two results suggest that low SR auditory nerve fibers may form the
major excitatory input to type II cells. If true, this hypothesis also
could explain the finding that type II units have consistently higher
thresholds than DCN principal cells \citep{YoungBrownell:1976} because
low SR auditory nerve fibers also have elevated thresholds relative to
the lowest threshold auditory nerve fibers
\citep{Liberman:1978}. However, patterns of auditory nerve innervation
of the DCN are most consistent with high SR fiber innervation of
vertical cell somata and low SR fiber innervation of dendrites
\citep{Liberman:1993}. In that case, the low spontaneous rates and
high sound thresholds of type II units might be caused by a high
intrinsic electrical threshold \citep{HancockDavisEtAl:1997}; this is
consistent with the responses of vertical cells to intracellular
current injection \citep{DingVoigt:1997,ZhangOertel:1993b}.


Type II units also supply an inhibitory input to the VCN
\citep{WickesbergOertel:1990}, but the role of type II terminals in
the VCN is less clear. Three different hypotheses have been
raised. The first is that this projection modulates the response
thresholds of VCN neurons \citep{PaoliniClark:1998}.  The role of type
II units in spectral processing is that of a narrowband
inhibitor. Responses of DCN principal cells are strongly inhibited by
this narrowband source. As a result, DCN principal cells are inhibited
by sharp spectral peaks close to their BF
\citep{SpirouDavisEtAl:1999}.



\subsubsection{Key design factors}


\textbf{Morphological}
\begin{itemize}
\item vertical/multipolar cell in deep layer of DCN \citep{Rhode:1999}
\item receive small number of ANF syn to dend
\item receive large number of Gly and GABA syn to soma and dendrite
\end{itemize}

\textbf{Intracellular}
\begin{itemize}
\item type I current clamp response
\item presence of glycine \citep{OertelWickesberg:1993}
\end{itemize}


\textbf{Physiological}
\begin{itemize}
\item Type II, wide chopper PSTH
  \citep{Rhode:1999,SpirouDavisEtAl:1999}
\item Narrow response area, non-monotonic RL
\item poor response to noise and clicks
\item asymmetric response to notch noise \citep{ReissYoung:2005}
\end{itemize}


\begin{itemize}
\item Rat model (no TS-TV) but has been shown in other mammals
\item Unable to include other DCN inputs
\item Model must show \DSTV inhibition and offset of distribution


\item Notch noise stimulus $\rightarrow$ need more TV cells across
  frequency
\item Input SPL and weight of excitation affect spiking output
\item Larger network $\rightarrow$ Computational problems
\item Solution: Paralellise model
\end{itemize}


\subsection{Implementation}

{
\small\linespread{0.5}
\begin{table}[htb]
    \caption{Tuberculoventral cell model summary}
    \label{tab:TVNotchModelSummary}
\end{table}
\noindent%
\begin{tabularx}{\textwidth}{|l|X|}\hline %
\hdr{2}{A}{Model Summary}\\\hline
         \textbf{Populations}          & HSR and LSR ANFs, Golgi, DS, and TV cells \\\hline
          \textbf{Topology}            & Tono-topicity of the rat AN and CN \\\hline
        \textbf{Connectivity}          & ANF$\to$\{GLG, DS, TV\}, GLG$\to$DS, DS$\to$TV  \\\hline
         \textbf{Input model}          & ANF~model: Instantaneous-rate Poisson neural model \citep{ZilanyBruce:2007} \\ \hline
\multirow{3}{*}{\textbf{Neuron model}} & GLG: Instantaneous-rate Poisson neural model\\
                                       & DS: HH-like single-compartment model (Type I-II \RM model)\\ 
                                       & TV: HH-like single-compartment model (Type I-classic \RM model) \\\hline
       \textbf{Channel models}         & $I_{\textrm{Na}}$, $I_{\textrm{KHT}}$, $I_{\textrm{KLT}}$, $I_{\textrm{KA}}$ and $I_{\textrm{h}}$ \citep{RothmanManis:2003b}\\\hline
\multirow{2}{*}{\textbf{Synapse model}} & Excitatory: AMPA glutamatergic receptor (single-exponential)\\
&  Inhibitory: \GABAa GABAergic receptor (double-exponential), Glycinergic receptor (double-exponential) \\\hline
%            \textbf{Input}             & Notch-noise stimulus \\\hline
%\textbf{Optimisation}    & Parameters for \GLGDS are optimised based on experimental click recovery date from \citet{BackoffPalombiEtAl:1997}. The praxis method is used for optimisation.  \\\hline
%\textbf{Measurements}    &  Spikes of TV units recorded and PSTH genereated. First spike latency, mean rate and variance of TV units calculated. Fitting data was compared against experimental data of a Type II~\DCN~unit~\citep[Figure~9]{ReissYoung:2005}.\\\hline
\end{tabularx}
\vspace{1ex}

% - B -----------------------------------------------------------------------------
\noindent%
\begin{tabularx}{\textwidth}{|l|X|X|}\hline
\hdr{3}{B}{Populations}\\\hline
\textbf{Name} &    \textbf{Elements}    & \textbf{Size} \\\hline
     HSR      &    Poisson generator    & $N_{\text{HSR}} = 50$ per freq.\ channel \\\hline
     LSR      &    Poisson generator    & $N_{\text{LSR}}= 30$  per freq.\ channel \\\hline
     GLG      &    Poisson generator    & $N_{\text{GLG}}= 1$  per freq.\ channel  \\\hline
     DS       &   Type I-II \RM model    & $N_{\text{DS}}= 1$ per freq.\ channel \\\hline
     TV       & Type I-classic \RM model & $N_{\text{TV}}= 1$ per freq.\ channel\\\hline
\end{tabularx}
\vspace{1ex}

% - C ------------------------------------------------------------------------------
\noindent%
\begin{tabularx}{\textwidth}{|l|l|l|X|}\hline
\hdr{4}{C}{Connectivity}\\\hline
\textbf{Name}  & \textbf{Source} & \textbf{Target} & \textbf{Pattern} \\\hline
%   ANF$\to$DS &       ANF       &   D~Stellate    & Skewed Gaussian, centered at CF, spread below CF \sANFDSl, spread above CF \sANFDSh \\\hline
    \ANFTV     &    LSR, HSR     &       TV        & 
Narrowband connection on CF, zero spread, weight \wLSRTV and \wHSRTV, number \nLSRTV and \nHSRTV, delay \dANFTV \\\hline
%   GLG$\to$DS &      Golgi      &   D~Stellate    & Gaussian, centered at CF with spread \sGLGDS \\\hline
    \DSTV      &       DS        &       TV        & 
Gaussian convergence, centered on CF, spread \protect{$\sigma^2 = \sGLGDS$}, weight \wGLGDS, number \nGLGDS, delay $\dGLGDS=0.5$ ms \\\hline
\multicolumn{4}{|>{\centering}X|}{\ANFGLG, \ANFDS, and \GLGDS from Table~\ref{tab:TVModelSummary} }\\\hline
\end{tabularx}
% , uniform weight \wANFDS for all synapses, number \nLSRDS \& \nHSRDS, delay \dANFDS
\vspace{1ex}

% - D ------------------------------------------------------------------------------
\noindent%
\begin{tabularx}{\textwidth}{|l|X|}\hline
\hdr{2}{D}{Neuron and Synapse Model}\\\hline
        \textbf{Name}          & TV cell model \\\hline
        \textbf{Type}          & Type I-classic \RM model \citep{RothmanManis:2003b}, conductance synapse input \\\hline
\textbf{Subthreshold dynamics} & Na, KHT, Ih, and leak currents \\\hline
       \textbf{Spiking}        & Emit spike when $v(t) \geq \theta$  \\\hline
\end{tabularx}
\vspace{1ex}
% \noindent\begin{tabularx}{\textwidth}{|p{0.150.95\textwidth}|X|}\hline
% \hdr{2}{D}{Neuron and Synapse Model}\\\hline
% \textbf{Name} &  \\\hline
% \textbf{Type} & \\\hline
% \raisebox{-4.5ex}{\parbox{0.95\textwidth}{\textbf{Subthreshold dynamics}}} &
% \rule{1em}{0em}\vspace*{-3.5ex}
%     \begin{equation*}
%       \begin{array}{r@{\;=\;}lll}
%       \tau \dot{V}(t) & -V(t) + R I(t) &\text{if} & t > t^*+\tau_{\text{rp}} \\
%       V(t) & V_{\text{r}} & \text{else} \\[2ex]
%       I(t) & \multicolumn{3}{l}{\frac{\tau}{R} \sum_{\tilde{t}} w
%         \delta(t-(\tilde{t}+\Delta))}
%       \end{array}
%     \end{equation*}
% \vspace*{-2.5ex}\rule{1em}{0em}
%  \\\hline
% \multirow{3}{*}{\textbf{Spiking}} &
%    If $V(t-)<\theta \wedge V(t+)\geq \theta$
% \vspace*{-1ex}
% \begin{enumerate}\setlength{\itemsep}{-0.5ex}
% \item set $t^* = t$
% \item emit spike with time-stamp $t^*$
% \end{enumerate}
% \vspace*{-4ex}\rule{1em}{0em}
% \\\hline
% \end{tabularx}
%\vspace{2ex}

\noindent%
\begin{tabularx}{\textwidth}{|l|X|}\hline %
\hdr{2}{E}{Optimisation}\\\hline
\textbf{Input Stimulus} & Notch-noise stimulus based on \citet{ReissYoung:2005}. Stop-band filtered white noise (60 dB SPL, 50 ms duration, 2 ms cosine squared on\slash off ramp, 20 ms delay), 30 dB half-octave stop-band width, centred on the middle of the network (5.8 kHz)\\\hline
%\multicolumn{2}{|c|}{\begin{minipage}[c]{0.8\textwidth} \includegraphics[width=0.8\textwidth,keepaspectratio]{./gfx/Notch-Wl-12.5kHz-0.5.eps} \end{minipage}}\\\hline
\textbf{Parameters} &     
\wHSRTV,
\wLSRTV,
\wDSTV, \nDSTV

\\\hline

    \textbf{Input}      & Stimulus induced Poisson spike trains from \GLG units, \HSR and \LSR\ \ANFs, and natural synaptic input from \DS units\\\hline
\textbf{Fitness Function} & Spiking output of all 100 TV units across the network recorded over 25 repetitions.\\
% %\multicolumn{2}{|c|}{\begin{minipage}[c]{0.8\textwidth} \includegraphics[width=0.8\textwidth,keepaspectratio]{./gfx/AN_rateplace_12.5_0.5.eps}\end{minipage}}\\\hline
%\textbf{Measurements}    & PSTH sampled at each click for 2 ms to measure click recovery\\\hline
% %\textbf{Optimisation}    & Parameters for \GLGDS are optimised based on experimental click recovery date from \citet{BackoffPalombiEtAl:1997}. The praxis method is used for optimisation.  \\\hline
    &  PSTH of TV cells, calculated for first spike latency and mean rate. Fitting data was compared against experimental data of a Type II \DCN unit \citep[Figure~9]{ReissYoung:2005}. \\\hline
\end{tabularx}
\vspace{1ex}

%  \textbf{Assumptions}    & The spread ANF to DS cells (\sANFDSh,\sANFDSl) is arbitrary at this point and will be explored in the next experiment.\\ \hline
%   \textbf{Function}     & Weighted mean squared error see listing below  \\ \hline


% % D~----------------------------------
% \begin{tabularx}{\linewidth}{|X|c|c|c|}
% \hdr{4}{F}{Optimisation} \\ \hline
%               \textbf{Parameters}                & \textbf{Name} & \textbf{Range} & \textbf{Best Values} \\\hline 
%         Weight of DS syn on TV  ($\mu$S)         &    \wDSTV     &  [1e-5,0.005]  & 0.0029 \\
%        Weight of ANF syn on TV  ($\mu$S)         &    \wANFTV    &  [1e-5,0.005]  & 0.00017 \\
%          Number of synapses, LSR to TV           &    \nLSRTV    &     [0,64]     & 8           \\
%          Number of synapses, HSR to TV           &    \nHSRTV    &     [0,64]     & 14          \\
% Spread of DS connections onto TV (channel units) &    \sDSTV     &     [0,10]     & 2.1         \\
% Offset of DS connections onto TV (channel units) &    \oDSTV     &     [0,10]     & 0.24        \\ \hline
% \end{tabularx}
}


%%% Local Variables: 
%%% mode: latex
%%% TeX-master: "SimpleResponses"
%%% TeX-PDF-mode: nil
%%% End: 


\clearpage
\subsection{Results} 

\begin{figure}[htb]
  \centering
\includegraphics[keepaspectratio,width=0.8\textwidth]{./gfx/TV_Reiss}
\caption{Experimental Data of a single Type-II DCN unit \citep{ReissYoung:2005}, Fig.~9.}
  \label{fig:TVReissFig9}
\end{figure}


\begin{figure}[tbh]
  \centering
%\resizebox{5in}{!}{
\turnbox{90}{\small{Rate (sp/s)}}%
%\includegraphics[keepaspectratio=true,width=0.45\textwidth]{AN_rateplace_10_0.5.eps}\includegraphics[keepaspectratio=true,width=0.45\textwidth]{AN_rateplace_12.5_0.5.eps}\\
%\includegraphics[keepaspectratio=true,width=0.45\textwidth]{CN_rateplace_10_0.5.eps}\includegraphics[keepaspectratio=true,width=0.45\textwidth]{CN_rateplace_12.5_0.5.eps}
%\small{Freq\. Channel}
%}
\caption{AN (top) and CN rate-place profiles from the CN stellate
  model in response to half and 1 octave notch noise inputs. }
\label{fig:TVResults}
\end{figure}


\textbf{Error} 0.0167  (MSE Normalised rate between 5-40kHz channels)



\subsection{Optimisation Results}


% \clearpage
% \newpage
\section{Verification}

% \subsection{Tone Response}
% \begin{figure}[h!]
% \centering\resizebox{0.95\textwidth}{!}{%
% \includegraphics{RateLevel/psthsingle90.1.eps}%
% \includegraphics{RateLevel/TV_ratelevel.eps}}
% \end{figure}
% \begin{figure}[h!]
% \centering\resizebox{0.95\textwidth}{!}{%
% \includegraphics{RateLevel/response_area.1.eps}%
% \includegraphics{RateLevel/response_area_log2.1.eps}}
% \end{figure}
% \begin{figure}[h!]
% \centering\resizebox{0.95\textwidth}{!}{%
% %\includegraphics{RateLevel/response_area.1.eps}
% \includegraphics{RateLevel/psthall90.1.eps}%
% \includegraphics{RateLevel/psthVlevel.1.eps}}
% \end{figure}


% \clearpage
% \subsection{Noise Response}
% \begin{figure}[h!]
% \centering\resizebox{0.95\textwidth}{!}{%
% \includegraphics{NoiseRateLevel/psthsingle120.1.eps}%
% \includegraphics{NoiseRateLevel/TV_ratelevel.eps}}
% \end{figure}
% \begin{figure}[h!]
% \centering\resizebox{0.95\textwidth}{!}{%
% \includegraphics{NoiseRateLevel/response_area.1.eps}%
% \includegraphics{NoiseRateLevel/response_area_log2.1.eps}}
% \end{figure}
% \begin{figure}[h!]
% \centering\resizebox{0.95\textwidth}{!}{%
% %\includegraphics{RateLevel/response_area.1.eps}
% \includegraphics{NoiseRateLevel/psthall90.1.eps}%
% \includegraphics{NoiseRateLevel/psthVlevel.1.eps}}
% \end{figure}


% \clearpage
% \subsection{Masked Noise and Tone}
% \begin{figure}[h!]
% \centering\resizebox{0.95\textwidth}{!}{\includegraphics{MaskedRateLevel/psthsingle90.1.eps}\includegraphics{MaskedRateLevel/TV_ratelevel.eps}}
% \end{figure}
% \begin{figure}[h!]
% \centering\resizebox{0.95\textwidth}{!}{%
% \includegraphics{MaskedRateLevel/response_area.1.eps}%
% \includegraphics{MaskedRateLevel/response_area_log2.1.eps}}
% \end{figure}
% \begin{figure}[h!]
% \centering\resizebox{0.95\textwidth}{!}{%
% %\includegraphics{RateLevel/response_area.1.eps}
% \includegraphics{MaskedRateLevel/psthall90.1.eps}%
% \includegraphics{MaskedRateLevel/psthVlevel.1.eps}}
% \end{figure}
% \clearpage
% \subsection{Masked Response Area}
% \begin{figure}[h!]
% \centering\resizebox{0.95\textwidth}{!}{%
% \includegraphics{MaskedResponseCurve/psthsingle5810.1.eps}%
% \includegraphics{MaskedResponseCurve/TV_masked.eps}}
% \end{figure}
% \begin{figure}[h!]
% \centering\resizebox{0.95\textwidth}{!}{%
% \includegraphics{MaskedResponseCurve/response_area.1.eps}%
% \includegraphics{MaskedResponseCurve/response_area_log2log2.1.eps}}
% \end{figure}
% \begin{figure}[h!]
% \centering\resizebox{0.95\textwidth}{!}{%
% %\includegraphics{RateLevel/response_area.1.eps}
% \includegraphics{MaskedResponseCurve/psthall5810.1.eps}%
% \includegraphics{MaskedResponseCurve/psthVmod.1.eps}}
% \end{figure}
% \clearpage


% % - F -----------------------------------------------------------------------------

% \noindent\begin{tabularx}{0.95\textwidth}{|X|}\hline
% \hdr{1}{F}{Measurements}\\\hline
% %\\\hline
% \end{tabularx}



 

%%% Local Variables: 
%%% mode: latex
%%% TeX-master: "SimpleResponses"
%%% TeX-PDF-mode: nil
%%% End: 
