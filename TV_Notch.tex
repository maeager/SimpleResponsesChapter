
% LaTeX file for generating the Model Description Table in Fig. 5 of
%
%  Nordlie E, Gewaltig M-O, Plesser HE (2009)
%  Towards Reproducible Descriptions of Neuronal Network Models.
%  PLoS Comput Biol 5(8): e1000456.
%
%  Paper URL : http://dx.doi.org/10.1371/journal.pcbi.1000456
%  Figure URL: http://dx.doi.org/10.1371/journal.pcbi.1000456.g005
%
% This file is released under a
%
%   Creative Commons Attribution, non-commercial, share-alike licence
%   http://creativecommons.org/licenses/by-nc-sa/3.0/de/deed.en
%
% with the following specifications:
%
%  1. When publishing tables generated from this LaTeX file and modified
%     versions of it, you must cite the paper by Nordlie et al given above.
%
%  2. The non-commercial clause applies only to the distribution of THIS FILE
%     and LaTeX source code files derived from it. You may commercially publish
%     documents generated using this file and derivatived versions of this file.
%
% Contact: Hans Ekkehard Plesser, UMB (hans.ekkehard.plesser at umb.no)

\documentclass{article}

%%
%% Custom Hyphenations
%%
%%
\hyphenation{cross-talk au-di-tory adap-t-a-tion phe-nom-en-o-lo-g-i-cal
  syn-a-pse co-inc-id-ence Tub-er-culo-vent-ral glyc-in-ergic psycho-phys-ical
  asym-met-ric ex-plor-at-ory pot-as-sium op-ti-mi-sa-tion au-di-t-ory system
  Neuro-in-for-mat-ics mar-g-in-al par-a-m-eters Rh-ode Neu-ro-fit-ter
  elec-tro-phys-io-log-i-cal in-fer-ring the con-nec-tiv-i-ty with-in
  non-lin-ear ap-pr-oa-ch-es stel-late mi-cro-cir-cuit show-ing synap-tic
  in-ter-ac-tion iso-lam-inar pop-u-la-tion evo-l-ved com-part-ment
  con-duc-tance mod-els Hod-g-kin Hux-l-ey gluta-m-at-er-gic gen-o-mes
  Theu-nis-sen re-sp-on-se co-inc-id-ence det-e-ctor exp-eri-men-tal
  ac-cu-mu-lated neu-ro-science cre-ate de-tailed ex-ten-sively stud-ied
  In-tra-cel-lu-lar neur-rons in-tra-cel-lu-lar in-ves-ti-ga-tion se-quen-tial
  de-ter-min-ed cat-e-gor-is-ed im-me-di-ate sen-si-tiv-ity bicu-cu-line
  mul-ti-ple in-hib-it-ory com-mis-sural path-way re-cip-ro-cal fre-quency
  po-si-tion func-tion de-lay prep-a-ra-tions iso-lated oct-o-pus in-sen-si-tive
  Chop-per im-preg-na-tion phys-i-o-log-i-cal ef-fect GABA-er-gic in-puts on-set
  chop-pers pri-mar-ily  gen-er-ate ar-ray gaus-sian ran-dom num-bers
  in-ves-ti-gat-ing im-p-or-tant phys-i-o-log-i-cal mech-a-nisms}
%%
%% Sample custom-configuration
%%
%%   You are encouraged to modify the following section with any of your
%%   own custom commands, packages, etc.
%%

%error 'You should modify this section and remove this error.'

% for URLs
\usepackage{url}

% AMS packages
%\usepackage{amsfonts}
\usepackage{amssymb}
\usepackage[fleqn]{amsmath}   % displayed equations flush left
%\setlength{\mathindent}{0em}
%\usepackage{amsthm}
\usepackage[mathscr]{eucal}

\newcommand{\vect}[1]{\mathbf{#1}}


% Allow equations to break over pages...
\interdisplaylinepenalty=2500
% Command to stop equation breaks
% Note: enclose this in braces when used...
\newcommand{\donotsplitoverpages}{\interdisplaylinepenalty=10000}

%% Graphics
% \ifx\pdftexversion\undefined
%  \usepackage[dvips]{graphicx}
% \else
%  \usepackage[pdftex]{graphicx}
% \fi

%% My Graphics and Hyperlinks stuff
\usepackage{ifpdf}
 \ifpdf
   \pdfoutput=1
   \usepackage[pdftex]{graphicx}  % uncomment if using graphicx
\usepackage[final,          % override "draft" which means "do nothing"
            colorlinks,     % rather than outlining them in boxes
            linkcolor=black, % override truly awful colour choices
            citecolor=black, %   (ditto)
            urlcolor=black,  %   (ditto)
            ]{hyperref}

 \ifx\pdfoutput\undefined \usepackage[ps2pdf,
 bookmarks=true,
 bookmarksnumbered=true,
 breaklinks=true,
            final,          % override "draft" which means "do nothing"
            colorlinks,     % rather than outlining them in boxes
            linkcolor=black, % override truly awful colour choices
            citecolor=black, %   (ditto)
            urlcolor=black,  %   (ditto)
            ]{hyperref}

% \usepackage[pdftex]{hyperref}  % uncomment if using hyperref
%  \usepackage[ps2pdf]{thumbpdf}
 \DeclareGraphicsExtensions{.eps,.bmp}
  \else
 \DeclareGraphicsExtensions{.png,.pdf,.jpg,.JPEG}
  \usepackage{epstopdf}
 %\usepackage[pdftex,bookmarks=true,bookmarksnumbered=true,breaklinks=true]{hyperref}
  \pdfadjustspacing=1
  \usepackage[pdftex]{thumbpdf}
  \fi
 \else
   \usepackage[dvips]{graphicx}  % uncomment if using graphicx
    % comment if not using hyperref
 \usepackage[final,          % override "draft" which means "do nothing"
            colorlinks,     % rather than outlining them in boxes
            linkcolor=black, % override truly awful colour choices
            citecolor=black, %   (ditto)
            urlcolor=black,  %   (ditto)
             ]{hyperref}
 \DeclareGraphicsExtensions{.eps,.bmp}
\fi
% Enable IEEE macros
%\usepackage{IEEEtrantools}

% Use a plain bibliography style
%\bibliographystyle{plain}
% Use the IEEE bibliography style (sorted)
%\bibliographystyle{IEEEtrans}
% Use the IEEE bibliography style (unsorted; order of reference)
%\bibliographystyle{IEEEtran}

% For isolated bibliographies
\usepackage{bibunits}

\usepackage{color}
%\usepackage[noadjust]{cite}
\usepackage{caption}
\usepackage{breakurl} % necessary to break URLs when using LaTeX-> dvips -> Ps2PDF, must be after hyperref

% For cool tables
\usepackage{array}
\usepackage{tabularx}  % automatically adjusts column width in tables
\usepackage{multirow}  % allows entries spanning several rows
\usepackage{colortbl}  % allows coloring tables


% For algorithms
%\usepackage{algorithm}
%\usepackage{algorithmic}

% For cases
\usepackage{sublabel}

% For theroem numbers having the chapter included
%\usepackage{style/chngcntr}

% For cool theorem styles
%\usepackage[amsthm]{ntheorem}
%%\theorembodyfont{\normalfont}
%
%% Theorem definition
%\newtheorem{theorem}{Theorem}
%\counterwithin{theorem}{chapter}
%
%% Corollary definition
%\newtheorem{corollary}{Corollary}
%\counterwithin{corollary}{chapter}
%
%% Result definition
%\newtheorem{result}{Result}
%\counterwithin{result}{chapter}
%
%% Lemma definition
%\newtheorem{lemma}{Lemma}
%\counterwithin{lemma}{chapter}
%
%% Proposition definition
%\newtheorem{proposition}{Proposition}
%\counterwithin{proposition}{chapter}
%
%% Definition definition!
%\newtheorem{definition}{Definition}
%\counterwithin{definition}{chapter}
%
%% Remark definition (no counter?)
%\newenvironment{remark}{\emph{Remark:~}}{}
%
%% Fact definition (no counter?)
%\newenvironment{fact}{\emph{Fact:~}}{}

% (Re)Set the figure path
\newcommand{\setfigurepath}[1]{%
\ifx\figurepath\undefined
	\newcommand{\figurepath}{#1}
\else
	\renewcommand{\figurepath}{#1}
\fi%
}

% Used in the continued list environment below
\newcounter{continuedlist}

% Continued list environment
\newenvironment{continuedlist}{%
	\begin{enumerate}%
		% Space out each item
		\setlength{\itemsep}{1.25em}%
		% Start the enumeration from the previous value
		\setcounter{enumi}{\value{continuedlist}} %
}{ %
  % Save the counter to continue it later
  \setcounter{continuedlist}{\value{enumi}}%
  \end{enumerate}%
  % \vspace{1.25em}% 
  \vspace{1em}%
}

% Spaced out list environment
\newenvironment{spacedoutlist}{%
	\begin{itemize}%
		% Space out each item
		\setlength{\itemsep}{1.25em}%
}{\end{itemize}}


%% My added packages


\usepackage[usenames,dvipsnames]{xcolor}
\definecolor{halfgray}{gray}{0.55}

\usepackage{listings}
\lstset{language=C++,%[LaTeX]Tex,%
    keywordstyle=\color{RoyalBlue},%\bfseries,
    basicstyle=\small\sffamily,
    identifierstyle=\color{NavyBlue},
    commentstyle=\color{Green}\rmfamily,
    stringstyle=\sffamily,
    numbers=left,%none,%
    numberstyle=\scriptsize,%\tiny
    stepnumber=5,
    numbersep=8pt,
    showstringspaces=false,
    breaklines=true,
    %frameround=ftff,
    %frame=single
    %frame=L
    lineskip=-5pt
}


% \lstset{language=Octave,                % choose the language of the code
% basicstyle=\footnotesize,       % the size of the fonts that are used for the code
% numbers=left,                   % where to put the line-numbers
% numberstyle=\footnotesize,      % the size of the fonts that are used for the line-numbers
% stepnumber=2,                   % the step between two line-numbers. If it's 1 each line will be numbered
% numbersep=5pt,                  % how far the line-numbers are from the code
% backgroundcolor=\color{white},  % choose the background color. You must add \usepackage{color}
% showspaces=false,               % show spaces adding particular underscores
% showstringspaces=false,         % underline spaces within strings
% showtabs=false,                 % show tabs within strings adding particular underscores
% frame=single,			% adds a frame around the code
% tabsize=2,			% sets default tabsize to 2 spaces
% captionpos=b,			% sets the caption-position to bottom
% breaklines=true,		% sets automatic line breaking
% breakatwhitespace=false,	% sets if automatic breaks should only happen at whitespace
% lineskip=-5pt  %
% }

\ifx\setcitestyle\undefined
\usepackage[sort,round,authoryear]{natbib}
\setcitestyle{aysep={}} % J Neurophys formatting
\fi

\usepackage{xspace}
\usepackage{rotating}
\usepackage{tikz}
\usepackage{calc}

\newcommand{\hdr}[3]{%
\multicolumn{#1}{|l|}{\color{white}\cellcolor[gray]{0.0}%
\textbf{\makebox[0.05\linewidth][l]{#2}\hspace{0.45\linewidth}\makebox[0pt][c]{#3}}%
%\textbf{\makebox[0pt]{#2}\hspace{0.5\linewidth}\makebox[0pt][c]{#3}}%
}}
 % Nordelie table environment
\newenvironment{ntab}[4]{
\noindent\begin{tabularx}{\linewidth}{#1}\hline 
\multicolumn{#2}{|l|}{\color{white}\cellcolor[gray]{0.0}\textbf{#3\hfill{}{#4}\hfill{}}}
}{
\end{tabularx}
\vspace{1ex}
}



\usepackage[colorinlistoftodos,backgroundcolor=yellow!35,textsize=footnotesize]{todonotes}
\newcommand{\yellownote}[1]{\todo[inline]{#1}}


\usepackage{scrtime}
%\usepackage{mparhack}

%\setlength{\parskip}{0ex} or {1.2ex}
%\setlength{\parindent}{0em} or {0ex}

\usepackage[australian]{babel}

\usepackage{booktabs,ltxtable,ctable,dcolumn}
\newcommand{\otoprule}{\midrule[\heavyrulewidth]}

\usepackage{doipubmed}

% For subfigures
\usepackage{keyval}
\usepackage[config,labelfont={sf,bf}]{subfig}
%\captionsetup[table]{position=top}
%\captionsetup[subtable]{position=top}
%\usepackage[heightadjust=all,valign=t]{floatrow}
%\usepackage{fr-subfig}
%\floatsetup{style=Plaintop}
%\usepackage{subfigure}

\usepackage{lscape}

\newcommand{\code}[1]{\mbox{\normalfont\texttt{#1}}}
\newcommand{\progname}[1]{\mbox{\normalfont\textsf{#1}}}
\newcommand{\figfont}[1]{\large{\textbf{\textsf{#1}}}}

%% Glossary

% ANF to Golgi
\newcommand{\ANFGLG}{\protect\ensuremath{\mbox{ANF} \to \mbox{GLG}\xspace}}
\newcommand{\HSRGLG}{\protect\ensuremath{\mbox{HSR} \to \mbox{GLG}\xspace}}
\newcommand{\LSRGLG}{\protect\ensuremath{\mbox{LSR} \to \mbox{GLG}\xspace}}
\newcommand{\wANFGLG}{\protect\ensuremath{w_{\ANFGLG}\xspace}}
\newcommand{\wLSRGLG}{\protect\ensuremath{w_{\LSRGLG}\xspace}}
\newcommand{\wHSRGLG}{\protect\ensuremath{w_{\HSRGLG}\xspace}}
\newcommand{\nLSRGLG}{\protect\ensuremath{n_{\LSRGLG}\xspace}}
\newcommand{\nHSRGLG}{\protect\ensuremath{n_{\HSRGLG}\xspace}}
\newcommand{\sANFGLG}{\protect\ensuremath{s_{\ANFGLG}\xspace}}
\newcommand{\sLSRGLG}{\protect\ensuremath{s_{\LSRGLG}\xspace}}
\newcommand{\sHSRGLG}{\protect\ensuremath{s_{\HSRGLG}\xspace}}
\newcommand{\dANFGLG}{\protect\ensuremath{d_{\ANFGLG}\xspace}}

%ANF to D-stellate
\newcommand{\ANFDS}{\protect\ensuremath{\mbox{ANF} \to \mbox{DS}\xspace}}
\newcommand{\HSRDS}{\protect\ensuremath{\mbox{HSR} \to \mbox{DS}\xspace}}
\newcommand{\LSRDS}{\protect\ensuremath{\mbox{LSR} \to \mbox{DS}\xspace}}
\newcommand{\wANFDS}{\ensuremath{w_{\ANFDS}\xspace}}
\newcommand{\wLSRDS}{\protect\ensuremath{w_{\LSRDS}\xspace}}
\newcommand{\wHSRDS}{\protect\ensuremath{w_{\HSRDS}\xspace}}
\newcommand{\nLSRDS}{\protect\ensuremath{n_{\LSRDS}\xspace}}
\newcommand{\nHSRDS}{\protect\ensuremath{n_{\HSRDS}\xspace}}
\newcommand{\dANFDS}{\protect\ensuremath{d_{\ANFDS}\xspace}}
\newcommand{\sANFDSh}{\protect\ensuremath{s^+_{\ANFDS}\xspace}}
\newcommand{\sANFDSl}{\protect\ensuremath{s^-_{\ANFDS}\xspace}}

%ANF to T-stellate
\newcommand{\ANFTS}{\protect\ensuremath{\mbox{ANF} \to \mbox{TS}\xspace}}
\newcommand{\HSRTS}{\protect\ensuremath{\mbox{HSR} \to \mbox{TS}\xspace}}
\newcommand{\LSRTS}{\protect\ensuremath{\mbox{LSR} \to \mbox{TS}\xspace}}
\newcommand{\wANFTS}{\protect\ensuremath{w_{\ANFTS}\xspace}}
\newcommand{\nLSRTS}{\protect\ensuremath{n_{\LSRTS}\xspace}}
\newcommand{\nHSRTS}{\protect\ensuremath{n_{\HSRTS}\xspace}}
\newcommand{\sANFTS}{\protect\ensuremath{s_{\ANFTS}\xspace}}
\newcommand{\dANFTS}{\protect\ensuremath{d_{\ANFTS}\xspace}}

%ANF to Tuberculoventral
\newcommand{\ANFTV}{\ensuremath{\mbox{ANF} \to \mbox{TV}\xspace}}
\newcommand{\HSRTV}{\ensuremath{\mbox{HSR} \to \mbox{TV}\xspace}}
\newcommand{\LSRTV}{\ensuremath{\mbox{LSR} \to \mbox{TV}\xspace}}
\newcommand{\wANFTV}{\ensuremath{w_{\ANFTV}\xspace}}
\newcommand{\nLSRTV}{\ensuremath{n_{\LSRTV}\xspace}}
\newcommand{\nHSRTV}{\ensuremath{n_{\HSRTV}\xspace}}
\newcommand{\sANFTV}{\ensuremath{s_{\ANFTV}\xspace}}
\newcommand{\dANFTV}{\ensuremath{d_{\ANFTV}\xspace}}

%GLG to T-stellate
\newcommand{\GLGTS}{\protect\ensuremath{\mbox{GLG} \to \mbox{TS}\xspace}}
\newcommand{\wGLGTS}{\protect\ensuremath{w_{\GLGTS}\xspace}}
\newcommand{\nGLGTS}{\protect\ensuremath{n_{\GLGTS}\xspace}}
\newcommand{\sGLGTS}{\protect\ensuremath{s_{\GLGTS}\xspace}}
\newcommand{\dGLGTS}{\protect\ensuremath{d_{\GLGTS}\xspace}}
%GLG to D-stellate
\newcommand{\GLGDS}{\protect\ensuremath{\mbox{GLG} \to \mbox{DS}\xspace}}
\newcommand{\wGLGDS}{\protect\ensuremath{w_{\GLGDS}\xspace}}
\newcommand{\nGLGDS}{\protect\ensuremath{n_{\GLGDS}\xspace}}
\newcommand{\sGLGDS}{\protect\ensuremath{s_{\GLGDS}\xspace}}
\newcommand{\dGLGDS}{\protect\ensuremath{d_{\GLGDS}\xspace}}

% % TS to Golgi
% \newcommand{\GLGTS}{\protect\ensuremath{\mbox{GLG} \to \mbox{TS}\xspace}}
% \newcommand{\wTSGLG}{\protect\ensuremath{w_{}}\xspace}}
% \newcommand{\nTSGLG}{\protect\ensuremath{n_{\mbox{TS} \to \mbox{GLG}}\xspace}}
% \newcommand{\sTSGLG}{\protect\ensuremath{s_{\mbox{TS} \to \mbox{GLG}}\xspace}}
% \newcommand{\dTSGLG}{\protect\ensuremath{d_{\mbox{TS} \to \mbox{GLG}}\xspace}}

%TS to D-stellate
\newcommand{\TSDS}{\protect\ensuremath{\mbox{TS} \to \mbox{DS}\xspace}}
\newcommand{\wTSDS}{\protect\ensuremath{w_{\TSDS}\xspace}}
\newcommand{\nTSDS}{\protect\ensuremath{n_{\TSDS}\xspace}}
\newcommand{\sTSDS}{\protect\ensuremath{s_{\TSDS}\xspace}}
\newcommand{\dTSDS}{\protect\ensuremath{d_{\TSDS}\xspace}}
%TS to T-stellate
\newcommand{\TSTS}{\protect\ensuremath{\mbox{TS} \to \mbox{TS}\xspace}}
\newcommand{\wTSTS}{\protect\ensuremath{w_{\TSTS}\xspace}}
\newcommand{\nTSTS}{\protect\ensuremath{n_{\TSTS}\xspace}}
\newcommand{\sTSTS}{\protect\ensuremath{s_{\TSTS}\xspace}}
\newcommand{\dTSTS}{\protect\ensuremath{d_{\TSTS}\xspace}}

%TS to Tuberculoventral
\newcommand{\TSTV}{\protect\ensuremath{\mbox{TS} \to \mbox{TV}\xspace}}
\newcommand{\wTSTV}{\protect\ensuremath{w_{\TSTV}\xspace}}
\newcommand{\nTSTV}{\protect\ensuremath{n_{\TSTV}\xspace}}
\newcommand{\sTSTV}{\protect\ensuremath{s_{\TSTV}\xspace}}
\newcommand{\dTSTV}{\protect\ensuremath{d_{\TSTV}\xspace}}

% DS to Golgi
\newcommand{\DSGLG}{\protect\ensuremath{\mbox{DS} \to \mbox{GLG}\xspace}}
\newcommand{\wDSGLG}{\protect\ensuremath{w_{\DSGLG}\xspace}}
\newcommand{\nDSGLG}{\protect\ensuremath{n_{\DSGLG}\xspace}}
\newcommand{\sDSGLG}{\protect\ensuremath{s_{\DSGLG}\xspace}}
\newcommand{\dDSGLG}{\protect\ensuremath{d_{\DSGLG}\xspace}}

% %DS to D-stellate
% \newcommand{\DSDS}{\protect\ensuremath{\mbox{DS} \to \mbox{DS}\xspace}}
% \newcommand{\wDSDS}{\protect\ensuremath{w_{\DSDS}\xspace}}
% \newcommand{\nDSDS}{\protect\ensuremath{n_{\DSDS}\xspace}}
% \newcommand{\sDSDS}{\protect\ensuremath{s_{\DSDS}\xspace}}
% \newcommand{\dDSDS}{\protect\ensuremath{d_{\DSDS}\xspace}}

%DS to T-stellate
\newcommand{\DSTS}{\protect\ensuremath{\mbox{DS} \to \mbox{TS}\xspace}}
\newcommand{\wDSTS}{\protect\ensuremath{w_{\DSTS}\xspace}}
\newcommand{\nDSTS}{\protect\ensuremath{n_{\DSTS}\xspace}}
\newcommand{\sDSTS}{\protect\ensuremath{s_{\DSTS}\xspace}}
\newcommand{\dDSTS}{\protect\ensuremath{d_{\DSTS}\xspace}}

%DS to Tuberculoventral
\newcommand{\DSTV}{\protect\ensuremath{\mbox{DS} \to \mbox{TS}\xspace}}
\newcommand{\wDSTV}{\protect\ensuremath{w_{\DSTV}\xspace}}
\newcommand{\nDSTV}{\protect\ensuremath{n_{\DSTV}\xspace}}
\newcommand{\sDSTV}{\protect\ensuremath{s_{\DSTV}\xspace}}
\newcommand{\dDSTV}{\protect\ensuremath{d_{\DSTV}\xspace}}
\newcommand{\oDSTV}{\protect\ensuremath{o_{\DSTV}\xspace}}

% TV to Golgi
\newcommand{\TVGLG}{\protect\ensuremath{\mbox{TV} \to \mbox{GLG}\xspace}}
\newcommand{\wTVGLG}{\protect\ensuremath{w_{\TVGLG}\xspace}}
\newcommand{\nTVGLG}{\protect\ensuremath{n_{\TVGLG}\xspace}}
\newcommand{\sTVGLG}{\protect\ensuremath{s_{\TVGLG}\xspace}}
\newcommand{\dTVGLG}{\protect\ensuremath{d_{\TVGLG}\xspace}}

%TV to D-stellate
\newcommand{\TVDS}{\protect\ensuremath{\mbox{TV} \to \mbox{DS}\xspace}}
\newcommand{\wTVDS}{\protect\ensuremath{w_{\TVDS}\xspace}}
\newcommand{\nTVDS}{\protect\ensuremath{n_{\TVDS}\xspace}}
\newcommand{\sTVDS}{\protect\ensuremath{s_{\TVDS}\xspace}}
\newcommand{\dTVDS}{\protect\ensuremath{d_{\TVDS}\xspace}}

%TV to T-stellate
\newcommand{\TVTS}{\protect\ensuremath{\mbox{TV} \to \mbox{TS}\xspace}}
\newcommand{\wTVTS}{\protect\ensuremath{w_{\TVTS}\xspace}}
\newcommand{\nTVTS}{\protect\ensuremath{n_{\TVTS}\xspace}}
\newcommand{\sTVTS}{\protect\ensuremath{s_{\TVTS}\xspace}}
\newcommand{\dTVTS}{\protect\ensuremath{d_{\TVTS}\xspace}}

%TV to Tuberculoventral
\newcommand{\TVTV}{\protect\ensuremath{\mbox{TV} \to \mbox{TV}\xspace}}
\newcommand{\wTVTV}{\protect\ensuremath{w_{\TVTV}\xspace}}
\newcommand{\nTVTV}{\protect\ensuremath{n_{\TVTV}\xspace}}
\newcommand{\sTVTV}{\protect\ensuremath{s_{\TVTV}\xspace}}
\newcommand{\dTVTV}{\protect\ensuremath{d_{\TVTV}\xspace}}


%Other common symbols
\newcommand{\GABAa}{\ensuremath{\mbox{GABA}_\mbox{A}}}


\usepackage[margin=0.4in]{geometry} % get enough space on page

\usepackage{tabularx}  % automatically adjusts column width in tables
\usepackage{multirow}  % allows entries spanning several rows
\usepackage{colortbl}  % allows coloring tables

%\usepackage[fleqn]{amsmath}   % displayed equations flush left
%\setlength{\mathindent}{0em}

% use Helvetica for text, Pazo math fonts
\usepackage{mathpazo}
\usepackage[scaled=.95]{helvet}
\renewcommand\familydefault{\sfdefault}

\renewcommand\arraystretch{1.2}  % slightly more space in tables

\pagestyle{empty}  % no header of footer


\usepackage[dvips]{graphicx}
\graphicspath{{/media/data/Work/cnstellate/TV_notch/}{/media/data/Work/Responses/}{/media/data/Work/cnstellate/}{/media/data/Work/thesis/ans2010/gfx/}}


\begin{document}

\section{Assymetric broadband inhibition of Tuberculo-ventral cells}
% - A ------------------------------------------------------------------------------



\noindent
\begin{tabularx}{0.95\textwidth}{|l|X|}\hline %
\hdr{2}{A}{Model Summary}\\\hline
\textbf{Populations}     & Five: HSR \& LSR ANFs, Golgi, DS, and TV cells \\\hline
\textbf{Topology}        & Tono-topicity of the rat AN and CN \\\hline
\textbf{Connectivity}    & ANF$\to${Golgi,DS,TV}, Golgi$\to$DS, DS$\to$TV  \\\hline
\textbf{Neuron model}    &\begin{minipage}{0.5\textwidth}
Golgi cell \begin{itemize}
\item instantaneous-rate Poisson spike trains
\item weighted sum of LSR instantaneous-rate vectors
\item smoothing due to alpha function kernel
\end{itemize}
D-stellate cell\begin{itemize}
\item biophysically-based HH-like single-compartment model
\item type I-II current-clamp model
\end{itemize}
Tuberculo-ventral cell \begin{itemize}
\item biophysically-based HH-like single-compartment model
\item type I-c current-clamp model \citep{RothmanManis:2003b}
\end{itemize}
\end{minipage}\\\hline
\textbf{Channel models}  &  $I_{\textrm{Na}}$, $I_{\textrm{KHT}}$, $I_{\textrm{KLT}}$, $I_{\textrm{KA}}$ and $I_{\textrm{h}}$ \citep{RothmanManis:2003b}\\\hline
\textbf{Synapse model}   & AMPA (\textit{ExpSyn}), GABA$_{\rm A}$ (\textit{Exp2Syn}), Glycine (\textit{Exp2Syn}) \\\hline
\textbf{Input}           &  Notch-noise (Stop-band filtered white noise) \\\hline
\textbf{Measurements}    &  First spikes and PSTH of TV cells, calculated for first spike latency, mean rate and variance \\\hline
\end{tabularx}

\vspace{2ex}

% - B -----------------------------------------------------------------------------

\noindent\begin{tabularx}{0.95\textwidth}{|l|l|X|}\hline
\hdr{3}{B}{Populations}\\\hline
\textbf{Name} &            \textbf{Elements}            & \textbf{Size} \\\hline
     HSR      &            Poisson generator            & $N_{\text{HSR}} = 50$ per freq.\ channel \\\hline
     LSR      &            Poisson generator            & $N_{\text{LSR}}= 20$  per freq.\ channel \\\hline
     GLG      &            Poisson generator            & $N_{\text{GLG}}= 1$  per freq.\ channel  \\\hline
     DS       & Single-Compartment H-H model (type I-II)& $N_{\text{DS}}= 1$ per freq\. channel \\\hline
     TV       & Single-Compartment H-H model (type I-c) & $N_{\text{TV}}= 1$ at CF=XXX kHz (channel 50)\\\hline
\end{tabularx}

\vspace{2ex}

% - C ------------------------------------------------------------------------------

\noindent\begin{tabularx}{0.95\textwidth}{|l|l|l|X|}\hline
\hdr{4}{C}{Connectivity}\\\hline
\textbf{Name} &  \textbf{Source}  & \textbf{Target}  & \textbf{Pattern} \\\hline
   \ANFDS     & ANF (HSR and LSR) &    D-Stellate    & skewed Gaussian, centered at CF, spread below CF \sANFDSl, spread above CF \sANFDSh, uniform weight \wANFDS for all synapses, number \nLSRDS \& \nHSRDS, delay \dANFDS \\\hline
   \ANFTV     & ANF (HSR and LSR) & Tuberculoventral & Narrowband, centered at CF,spread negligent , uniform weight \wANFDS for all synapses, number \nLSRDS \& \nHSRDS, delay \dANFDS \\\hline
   \GLGDS     &       Golgi       &    D-Stellate    & Gaussian, centered at CF with spread \sGLGDS, uniform weight \wGLGDS, number \nGLGDS, delay \dGLGDS \\\hline
    \DSTV     &    D-Stellate     & Tuberculoventral & Gaussian, centered at CF with spread \sGLGDS, uniform weight \wGLGDS, number \nGLGDS, delay \dGLGDS \\\hline
\end{tabularx}

\vspace{2ex}

% - D ------------------------------------------------------------------------------

% \noindent\begin{tabularx}{0.95\textwidth}{|p{0.150.95\textwidth}|X|}\hline
% \hdr{2}{D}{Neuron and Synapse Model}\\\hline
% \textbf{Name} &  \\\hline
% \textbf{Type} & \\\hline
% \raisebox{-4.5ex}{\parbox{0.95\textwidth}{\textbf{Subthreshold dynamics}}} &
% \rule{1em}{0em}\vspace*{-3.5ex}
%     \begin{equation*}
%       \begin{array}{r@{\;=\;}lll}
%       \tau \dot{V}(t) & -V(t) + R I(t) &\text{if} & t > t^*+\tau_{\text{rp}} \\
%       V(t) & V_{\text{r}} & \text{else} \\[2ex]
%       I(t) & \multicolumn{3}{l}{\frac{\tau}{R} \sum_{\tilde{t}} w
%         \delta(t-(\tilde{t}+\Delta))}
%       \end{array}
%     \end{equation*}
% \vspace*{-2.5ex}\rule{1em}{0em}
%  \\\hline
% \multirow{3}{*}{\textbf{Spiking}} &
%    If $V(t-)<\theta \wedge V(t+)\geq \theta$
% \vspace*{-1ex}
% \begin{enumerate}\setlength{\itemsep}{-0.5ex}
% \item set $t^* = t$
% \item emit spike with time-stamp $t^*$
% \end{enumerate}
% \vspace*{-4ex}\rule{1em}{0em}
% \\\hline
% \end{tabularx}

\vspace{2ex}

\noindent
\begin{tabularx}{0.95\textwidth}{|l|X|}\hline %
\hdr{2}{B}{Input/Ouput}\\\hline
\textbf{Input Stimulus}  & Notch-noise (Stop-band filtered white noise)  \\\hline
%\multicolumn{2}{|c|}{\begin{minipage}[c]{0.8\textwidth}
%\includegraphics[width=0.8\textwidth,keepaspectratio]{./gfx/Notch-Wl-12.5kHz-0.5.eps}
%\end{minipage}}\\\hline
\textbf{Output} & Output of 100 TV cells, across the network, with 25 repetitions\\\hline
%\multicolumn{2}{|c|}{\begin{minipage}[c]{0.8\textwidth}%
%\includegraphics[width=0.8\textwidth,keepaspectratio]{./gfx/AN_rateplace_12.5_0.5.eps}
%\end{minipage}}\\\hline
%\textbf{Measurements}    & PSTH sampled at each click for 2 ms to measure click recovery\\\hline
%\textbf{Optimisation}    & Parameters for \GLGDS are optimised based on experimental click recovery date from \citet{BackoffPalombiEtAl:1997}. The praxis method is used for optimisation.  \\\hline
\textbf{Measurements}    &  First spikes and PSTH of TV cells, calculated for first spike latency, mean rate and variance. Fitting data was compared against experimental data of a Type-II DCN unit \citep{ReissYoung:2005}, Fig.~9. \\\hline
\end{tabularx}
\vspace{2ex}

\noindent\begin{tabularx}{0.95\textwidth}{|l|X|}\hline %{0.95\textwidth}
\hdr{2}{C}{Optimisation} \\ \hline
\textbf{Type}       & Hand-tuning and Principle-axis method \\\hline
\textbf{Parameters}   & \\\hline
Syn.~weight \DSTV & $\wDSTV \quad\to\quad [0.00001,0.05]\quad\mu{\rm S}$ \\\hline
Syn.~weight ANF to DS       & $\wANFTV \quad\to\quad [0.00001,0.05] \quad \mu{\rm S}$\\\hline
No.~LSR to DS       & $\nLSRTV \quad\to\quad [0.00001,0.05] \quad \mu{\rm S}$\\\hline
No.~HSR to DS       & $\nHSRTV \quad\to\quad [0.00001,0.05] \quad \mu{\rm S}$\\\hline
Spread of \DSTV       & $\sDSTV \quad\to\quad [0,5] \quad {\rm Channels}$\\\hline
Offset \DSTV       & $\oDSTV \quad\to\quad [0,5] \quad {\rm Channels}$\\\hline
%  \textbf{Assumptions}    & The spread ANF to DS cells (\sANFDSh,\sANFDSl) is arbitrary at this point and will be explored in the next experiment.\\ \hline
%   \textbf{Function}     & Weighted mean squared error see listing below  \\ \hline
\end{tabularx}
% D-----------------------------------

\vspace{2ex}

\noindent\begin{tabularx}{0.95\textwidth}{|l|X|}\hline %{0.95\textwidth}
\hdr{2}{D}{Results} \\\hline %\multicolumn{2}{|X|}{\includegraphics[keepaspectratio,width=0.8\textwidth]{./gfx/TV_Reiss}}\\%\multicolumn{2}{|X|}{Experimental Data of a single Type-II DCN unit \citep{ReissYoung:2005}, Fig.~9.}\\\hline
\multicolumn{2}{|c|}{\turnbox{90}{\small{Rate (sp/s)}}%
\begin{minipage}[c]{0.95\textwidth}
\includegraphics[keepaspectratio=true,width=0.45\textwidth]{AN_rateplace_10_0.5.eps}%
\includegraphics[keepaspectratio=true,width=0.45\textwidth]{AN_rateplace_12.5_0.5.eps}\\
\includegraphics[keepaspectratio=true,width=0.45\textwidth]{CN_rateplace_10_0.5.eps}%
\includegraphics[keepaspectratio=true,width=0.45\textwidth]{CN_rateplace_12.5_0.5.eps}\\
\centering\small{Freq. Channel}\end{minipage}}\\
\multicolumn{2}{|X|}{Fig. 7: AN (top) and CN rate-place profiles from the CN stellate model in response to half and 1 octave notch noise inputs. }\\\hline
\textbf{Best Parameters} & {\begin{minipage}[c]{0.6\textwidth}\vspace{1ex}
$\wDSTV=0.0029 \quad\mu{\rm S}$ \\
$\wANFTV=0.00017 \quad\mu{\rm S}$ \\
$\nHSRTV=8 \quad \mu{\rm S}$\\
$\nLSRTV=14 \quad \mu{\rm S}$\\
$\sDSTV=2.1$ \\
$\oDSTV=0.24$ \\
\end{minipage}}\\\hline
\textbf{Error} & 0.0167  (MSE Normalised rate between 5-40kHz channels)\\\hline
\end{tabularx}



\clearpage
\newpage
\section{Verification}

\subsection{Tone Response}
\begin{figure}[h!]
\centering\resizebox{0.95\textwidth}{!}{%
\includegraphics{RateLevel/psthsingle90.1.eps}%
\includegraphics{RateLevel/TV_ratelevel.eps}}
\end{figure}
\begin{figure}[h!]
\centering\resizebox{0.95\textwidth}{!}{%
\includegraphics{RateLevel/response_area.1.eps}%
\includegraphics{RateLevel/response_area_log2.1.eps}}
\end{figure}
\begin{figure}[h!]
\centering\resizebox{0.95\textwidth}{!}{%
%\includegraphics{RateLevel/response_area.1.eps}
\includegraphics{RateLevel/psthall90.1.eps}%
\includegraphics{RateLevel/psthVlevel.1.eps}}
\end{figure}


\clearpage
\subsection{Noise Response}
\begin{figure}[h!]
\centering\resizebox{0.95\textwidth}{!}{%
\includegraphics{NoiseRateLevel/psthsingle120.1.eps}%
\includegraphics{NoiseRateLevel/TV_ratelevel.eps}}
\end{figure}
\begin{figure}[h!]
\centering\resizebox{0.95\textwidth}{!}{%
\includegraphics{NoiseRateLevel/response_area.1.eps}%
\includegraphics{NoiseRateLevel/response_area_log2.1.eps}}
\end{figure}
\begin{figure}[h!]
\centering\resizebox{0.95\textwidth}{!}{%
%\includegraphics{RateLevel/response_area.1.eps}
\includegraphics{NoiseRateLevel/psthall90.1.eps}%
\includegraphics{NoiseRateLevel/psthVlevel.1.eps}}
\end{figure}


\clearpage
\subsection{Masked Noise and Tone}
\begin{figure}[h!]
\centering\resizebox{0.95\textwidth}{!}{\includegraphics{MaskedRateLevel/psthsingle90.1.eps}\includegraphics{MaskedRateLevel/TV_ratelevel.eps}}
\end{figure}
\begin{figure}[h!]
\centering\resizebox{0.95\textwidth}{!}{%
\includegraphics{MaskedRateLevel/response_area.1.eps}%
\includegraphics{MaskedRateLevel/response_area_log2.1.eps}}
\end{figure}
\begin{figure}[h!]
\centering\resizebox{0.95\textwidth}{!}{%
%\includegraphics{RateLevel/response_area.1.eps}
\includegraphics{MaskedRateLevel/psthall90.1.eps}%
\includegraphics{MaskedRateLevel/psthVlevel.1.eps}}
\end{figure}
\clearpage
\subsection{Masked Response Area}
\begin{figure}[h!]
\centering\resizebox{0.95\textwidth}{!}{%
\includegraphics{MaskedResponseCurve/psthsingle5810.1.eps}%
\includegraphics{MaskedResponseCurve/TV_masked.eps}}
\end{figure}
\begin{figure}[h!]
\centering\resizebox{0.95\textwidth}{!}{%
\includegraphics{MaskedResponseCurve/response_area.1.eps}%
\includegraphics{MaskedResponseCurve/response_area_log2log2.1.eps}}
\end{figure}
\begin{figure}[h!]
\centering\resizebox{0.95\textwidth}{!}{%
%\includegraphics{RateLevel/response_area.1.eps}
\includegraphics{MaskedResponseCurve/psthall5810.1.eps}%
\includegraphics{MaskedResponseCurve/psthVmod.1.eps}}
\end{figure}
\clearpage


% - F -----------------------------------------------------------------------------

% \noindent\begin{tabularx}{0.95\textwidth}{|X|}\hline
% \hdr{1}{F}{Measurements}\\\hline
% %\\\hline
% \end{tabularx}

% ---------------------------------------------------------------------------------
%%%%%%%%%%%%%%%%%%%%%%%%%%%%%%%%%%%%%%%%%%%%%%%%%%%%%%
 \bibliographystyle{plainnat}%bmc_article} % Style BST file
 \bibliography{../manuscript/bib/MyBib}



\end{document}

