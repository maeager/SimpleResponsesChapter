%===================================
\subsection{Optimisation Results}

Figure~\ref{fig:GolgiTestResult} shows the output of the test optimisation trials for the Golgi cell model.
The testing trial used only five sound levels (0, 15, 55, 75 and 85 dB \SPL) and detected the mean rate from the instantaneous profile in its fitting routine.
The best response obtained a minimum root mean squared error of 11.63 spikes/sec against the five points in the target experimental data of unit S03-07 (CF=21~kHz) from \citep{GhoshalKim:1996}.
A rate-level curve (green circles, Figure~\ref{fig:GolgiTestResult}) was generated from the spiking output only to show a big discrepancy in the spike-based rate-level and the monotonic rate based rate-level.
The lack of low level response and a higher threshold indicated the need for some \HSR~input into the Golgi cell model.

%\smallskip{}


\begin{figure}[htb]
  \centering
\resizebox{0.6\textwidth}{!}{\includegraphics{GolgiRateLevel_result2.eps}}\\
  \caption[Initial results of Golgi cell model]{Initial trial results of the Golgi cell model optimisation.
Responses of the Golgi cell model (blue triangles) compared five five sound level (0,15, 55, 75 and 85 dB SPL) against 5 point in the target response (red squares).
The eventual best optimisation response obtained a minimum error of 11.63 spikes/s (root mean squared).
A spike response (green circles) was generated from the spiking output of the Golgi cell model using the final parameters. \label{fig:GolgiTestResult}}
\end{figure}

The final optimisation routine with 22 levels and a Golgi cell model with \HSR~and \LSR~\ANF~inputs was used to generate a closer fit to the \citeauthor{GhoshalKim:1996}  data.
Figure~\ref{fig:GolgiResult} shows the rate-level output of the best model response and its best combination of parameters are shown in Table~\ref{tab:GolgiCellModelSummary}E.
The root mean squared error of the best response was 4.48~spikes per second.

The parameters in Table~\ref{tab:GolgiCellResults} were within the range of expected values.
\LSR~inputs to the Golgi cell model out-weighted \HSR~inputs by more than a factor of 10.
The monotonic response of \LSR~fibres at high sound levels were necessary to create the large dynamic range in the Golgi cell model, the \HSR~fibres were just as necessary to provide some low level activity.
The spontaneous rate parameter matches the base response of unit S03-07 in Figure~\ref{fig:GolgiResult}.
The smoothing filter time constant of 5 ms is a typical value in membrane time constants for neural models and fits with the input resistance in intracellular recordings of Golgi cells \citep{FerragamoGoldingEtAl:1998}.

The input spread parameter is not well constrained by the optimisation fitness routine with a pure tone input and a single neuron, but the result is satisfactory given the uncertainty in \LSR~fibre's axonal organisation in the \GCD\@. 
The dendritic widths in Golgi cells are around 100 microns and the frequency separation laminae in the \VCN~core is approximately 70 microns, giving an expected result of 1.5 connectivity spread hence the result of 2.48 channels gives added frequency spread from \LSR~fibres.

%\smallskip{}


Table~\ref{tab:GolgiCellModelSummary}E result table.\\
{\small%% Result table
\noindent%
\begin{table}[htb]
  \centering
\begin{tabularx}{\textwidth}{|X|c|c|c|}\hline %{\textwidth}
\hdr{4}{}{GLG model parameters} \\ \hline
                \textbf{Parameters}                 & \textbf{Name} & \textbf{Range} & \textbf{Best Values} \\\hline
     Spatial spread LSR$\to$GLG (channel unit)      &  $\sANFGLG$   &     [0,10]     & 2.48  \\\hline
        Smoothing filter time constant (ms)         &    $\Gtau$    &     [0,20]     & 5.01  \\\hline
          Weighted sum of HSR~(unit-less)           &  $\wHSRGLG$   &     [0,5]      & 0.517 \\\hline
          Weighted sum of LSR~(unit-less)           &  $\wLSRGLG$   &     [0,5]      & 0.0487\\\hline
Spontaneous rate in Golgi cell model (spikes / sec) &   $\Gspon$    &     [0,50]     & 3.73  \\\hline
\end{tabularx}
  \caption{Golgi cell model optimisation parameters}
  \label{tab:GolgiCellResults}
\end{table}
}

\begin{figure}[htb]
  \centering
  % \resizebox{3.5in}{!}{\includegraphics{NoFigure}} \\
\includegraphics[keepaspectratio=true,width=0.6\textwidth]{GolgiRateLevel_result.eps}\\
  % \hspace{1cm}\figfont{A}\hfill\\
  %\resizebox{\textwidth}{!}{\includegraphics{GolgiRateLevel_result2.eps}} \\
  % \hspace{1cm}\figfont{B}\hfill \\
  \caption[Golgi cell model optimisation results]{Golgi cell model optimisation result trials against unit S03-07 (CF 21~kHz) from \citet{GhoshalKim:1996}.
A more detailed optimisation with 22 levels and included HSR inputs in the Golgi cell model generated a closer fit to the Ghoshal and Kim data.
The final root mean squared error was 4.48 spikes/s.
 \label{fig:GolgiResult}}
\end{figure}



%   % \includegraphics[width=0.6\textwidth,angle=-90]{GolgiRateLevelActualFit}\\
%   % \caption{Optimisation Results for Golgi Model using Rate Level data from
%   %     \label{Ch3:fig:GolgiFit}}
%   %   \includegraphics[width=0.8\textwidth]{GolgiRateLevel}\\
%   %   \caption{Optimisation Results for Golgi Model using Rate Level data from
%   %     \label{Ch3:fig:GolgiRL}}

%   %   \includegraphics[width=0.8\textwidth]{golgi_RateLevel_opt}\\
%   %   \caption{Optimisation Results for Golgi Model using Rate Level data from
%   %     \label{Ch3:fig:GolgiRL}}
%   % \includegraphics[width=0.8\textwidth,angle=-90]{GolgiRateLevel2}\\
%     %   \caption{Optimisation Results for Golgi Model using Rate Level data
%     %   from     \label{Ch3:fig:GolgiRL}}
%   \begin{figure}[htb]
%     \centering
% \includegraphics[width=0.6\textwidth,angle=-90]{GolgiRateLevelActualFit}\\
%     \caption{Optimisation Results for Golgi Model using Rate Level data from
%       \label{Ch3:fig:GolgiFit}}
%   \end{figure}
%   \begin{figure}[htb]
%     \centering
%     \includegraphics[width=0.8\textwidth]{GolgiRateLevel}\\
%     \caption{Optimisation Results for Golgi Model using Rate Level data from
%       \label{Ch3:fig:GolgiRL}}
%   \end{figure}
%   \begin{figure}[htb]
%     \centering
%     \includegraphics[width=0.8\textwidth]{golgi_RateLevel_opt}\\
%     \caption{Optimisation Results for Golgi Model using Rate Level data from
%       \label{Ch3:fig:GolgiRL}}
%   \end{figure}
%   \begin{figure}[htb]
%     \centering
% \includegraphics[width=0.8\textwidth,angle=-90]{GolgiRateLevel2}\\
%     \caption{Optimisation Results for Golgi Model using Rate Level data from
%       \label{Ch3:fig:GolgiRL}}
%   \end{figure}

%   \clearpage \newpage

%===================================
\subsection{Verification Results of Golgi Cell Model \label{sec:Golgi:verif-golgi-cell}}
%   \subsubsection{Tone Responses}


After settling with the above optimised parameters, the Golgi cell model was run with typical inputs to determine it's behaviour outside of the optimisation routine.
The Golgi cell model was tested across the entire network using tones, noise and tones plus noise stimuli. Figure~\ref{fig:Golgi_verification}A, B and D show the response of a Golgi cell model at the centre of the network (CF=5.8 kHz) and had monotonic responses to tones and noise similar to other Ghoshal and Kim units (Figure~\ref{fig:GolgiKimFig2}).  Figure~\ref{fig:Golgi_verification}C shows the response of all \GLG units in the network to a 5.8~kHz tone, increased from 0 to 90 dB~{SPL}.
 %\smallskip{}

\begin{figure}[htb]
%\centering
{\figfont{A}\hspace{0.5\textwidth}\figfont{B}\hfill}\\
%\resizebox{0.95\textwidth}{!}{
\includegraphics[keepaspectratio=true,width=0.48\textwidth]{ResponsesNoComp/G_ratelevel_combined.eps}%
\includegraphics[keepaspectratio=true,width=0.48\textwidth]{ResponsesNoComp/RateLevel/psthsingle90-3.eps}\\
%}\\
{\figfont{C}\hspace{0.5\textwidth}\figfont{D}\hfill}\\
%\resizebox{0.95\textwidth}{!}{
\includegraphics[keepaspectratio=true,width=0.48\textwidth]{ResponsesNoComp/RateLevel/response_area-3.eps}%
\includegraphics[keepaspectratio=true,width=0.48\textwidth]{ResponsesNoComp/MaskedResponseCurve3/15/G_masked.eps}\\
%}\\
% }}
%\resizebox{0.45\textwidth}{!}{\includegraphics{ResponsesNoComp/RateLevel/psthsingle90-3.eps}}\\
%\resizebox{0.45\textwidth}{!}{\includegraphics{ResponsesNoComp/RateLevel/psthsingle50-3.eps}}\\
\caption[Optimised Golgi cell model responses]{Response of optimised Golgi cell model at the centre of the network (CF=5.8~kHz). 
A. Rate level responses to tone, noise and tone plus noise. 
B. PSTH at 90 dB~SPL\.  
C. Response area equivalent using all GLG units in the network. 
D. Masked noise-tone response of the central unit to 15 dB masking noise and frequencies one octave above and below its CF.} \label{fig:Golgi_verification}
\end{figure}




%   \begin{figure}[h]
%     \centering\resizebox{0.95\textwidth}{!}{%
%     \includegraphics{RateLevel/response_area-3.eps}%
%     \includegraphics{RateLevel/response_area_log2.3.eps}}
%   \end{figure}
%   \begin{figure}[h]
%     \centering\resizebox{0.95\textwidth}{!}{%
%     %     \includegraphics{RateLevel/response_area-3.eps}
%     \includegraphics{RateLevel/psthall90.3.eps}%
%     \includegraphics{RateLevel/psthVlevel.3.eps}}
%   \end{figure}



%   \clearpage
%   \subsubsection{Noise Responses}
%   \begin{figure}[h]
%     \centering\resizebox{0.95\textwidth}{!}{%
%     \includegraphics{NoiseRateLevel/psthsingle120-3.eps}%
%     \includegraphics{NoiseRateLevel/G_ratelevel.eps}}
%   \end{figure}
%   \begin{figure}[h]
%     \centering\resizebox{0.95\textwidth}{!}{%
%     \includegraphics{NoiseRateLevel/response_area-3.eps}%
%     \includegraphics{NoiseRateLevel/response_area_log2.3.eps}}
%   \end{figure}
%   \begin{figure}[h]
%     \centering\resizebox{0.95\textwidth}{!}{%
%     %     \includegraphics{RateLevel/response_area-3.eps}
%     \includegraphics{NoiseRateLevel/psthall90.3.eps}%
%     \includegraphics{NoiseRateLevel/psthVlevel.3.eps}}
%   \end{figure}


%   \clearpage
%   \subsubsection{Masking Responses}
%   \begin{figure}[h!]
% \centering\resizebox{0.95\textwidth}{!}{\includegraphics{MaskedRateLevel/psthsingle90-3.eps}\includegraphics{MaskedRateLevel/G_ratelevel.eps}}
%   \end{figure}
%   \begin{figure}[h!]
%     \centering\resizebox{0.95\textwidth}{!}{%
%     \includegraphics{MaskedRateLevel/response_area-3.eps}%
%     \includegraphics{MaskedRateLevel/response_area_log2.3.eps}}
%   \end{figure}

%   \begin{figure}[h!]
%     \centering\resizebox{0.95\textwidth}{!}{%
%     %     \includegraphics{RateLevel/response_area-3.eps}
%     \includegraphics{MaskedRateLevel/psthall90.3.eps}%
%     \includegraphics{MaskedRateLevel/psthVlevel.3.eps}}
%   \end{figure}
%   \clearpage

%   \begin{figure}[h!]
%     \centering\resizebox{0.95\textwidth}{!}{%
%     \includegraphics{MaskedResponseCurve/psthsingle5810.3.eps}%
%     \includegraphics{MaskedResponseCurve/G_masked.eps}}
%   \end{figure}
%   \begin{figure}[h!]
%     \centering\resizebox{0.95\textwidth}{!}{%
%     \includegraphics{MaskedResponseCurve/response_area-3.eps}%
% \includegraphics{MaskedResponseCurve/response_area_log2log2.3.eps}}
%   \end{figure}

%   \begin{figure}[h!]
%     \centering\resizebox{0.95\textwidth}{!}{%
%     %     \includegraphics{RateLevel/response_area-3.eps}
%     \includegraphics{MaskedResponseCurve/psthall5810.3.eps}%
%     \includegraphics{MaskedResponseCurve/psthVmod.3.eps}}
%   \end{figure}
%   \clearpage


%%% Local Variables:
%%% mode: latex
%%% mode: visual-line
%%% TeX-master: "SimpleResponses"
%%% TeX-PDF-mode: nil
%%% End:
