\documentclass{article}
% User defined commands

%%
%% Custom Hyphenations
%%
%%

%\hyphenation{cross-talk au-di-tory adap-ta-tion phe-nomeno-log-i-cal syn-a-pse}

%%
%% Sample custom-configuration
%%
%%   You are encouraged to modify the following section with any of your
%%   own custom commands, packages, etc.
%%

%error 'You should modify this section and remove this error.'

% for URLs
\usepackage{url}

% AMS packages
%\usepackage{amsfonts}
\usepackage{amssymb}
\usepackage[fleqn]{amsmath}   % displayed equations flush left
\setlength{\mathindent}{0em}
\usepackage{amsthm}
\usepackage[mathscr]{eucal}

% Allow equations to break over pages...
\interdisplaylinepenalty=2500
% Command to stop equation breaks
% Note: enclose this in braces when used...
\newcommand{\donotsplitoverpages}{\interdisplaylinepenalty=10000}

%% Graphics
%\ifx\pdftexversion\undefined
%  \usepackage[dvips]{graphicx}
%\else
%  \usepackage[pdftex]{graphicx}
%\fi
\usepackage{ifpdf}
 \ifpdf
   \pdfoutput=1
   \usepackage[pdftex]{graphicx}  % uncomment if using graphicx
\usepackage[final,          % override "draft" which means "do nothing"
            colorlinks,     % rather than outlining them in boxes
            linkcolor=blue, % override truly awful colour choices
            citecolor=blue, %   (ditto)
            urlcolor=blue,  %   (ditto)
            ]{hyperref}

 \ifx\pdfoutput\undefined \usepackage[ps2pdf,bookmarks=true,bookmarksnumbered=true,breaklinks=true,
            final,          % override "draft" which means "do nothing"
            colorlinks,     % rather than outlining them in boxes
            linkcolor=blue, % override truly awful colour choices
            citecolor=blue, %   (ditto)
            urlcolor=blue,  %   (ditto)
            ]{hyperref}

  % \usepackage[pdftex]{hyperref}  % uncomment if using hyperref
%  \usepackage[ps2pdf]{thumbpdf}
 \DeclareGraphicsExtensions{.eps,.bmp}
  \else
 \DeclareGraphicsExtensions{.png,.pdf,.jpg,.JPEG}
  \usepackage{epstopdf}
 %\usepackage[pdftex,bookmarks=true,bookmarksnumbered=true,breaklinks=true]{hyperref}
  \pdfadjustspacing=1
  \usepackage[pdftex]{thumbpdf}
  \fi
 \else
   \usepackage[dvips]{graphicx}  % uncomment if using graphicx
    % comment if not using hyperref
 \usepackage[final,          % override "draft" which means "do nothing"
            colorlinks,     % rather than outlining them in boxes
            linkcolor=blue, % override truly awful colour choices
            citecolor=blue, %   (ditto)
            urlcolor=blue,  %   (ditto)
            ]{hyperref}
 \DeclareGraphicsExtensions{.eps,.bmp}

\fi


% Enable IEEE macros
%\usepackage{IEEEtrantools}
% Use a plain bibliography style
%\bibliographystyle{plain}
% Use the IEEE bibliography style (sorted)
%\bibliographystyle{IEEEtrans}
% Use the IEEE bibliography style (unsorted; order of reference)
%\bibliographystyle{IEEEtran}

% For isolated bibliographies
\usepackage{bibunits}

\usepackage{color}
%\usepackage[noadjust]{cite}
\usepackage{caption}

% For cool tables
\usepackage{array}
\usepackage{tabularx}  % automatically adjusts column width in tables
\usepackage{multirow}  % allows entries spanning several rows
\usepackage{colortbl}  % allows coloring tables

% For subfigures
\usepackage{subfig}
%\usepackage{subfigure}

% For algorithms
%\usepackage{algorithm}
%\usepackage{algorithmic}

% For cases
\usepackage{sublabel}

% For theroem numbers having the chapter included
%\usepackage{style/chngcntr}

% For cool theorem styles
%\usepackage[amsthm]{ntheorem}
%%\theorembodyfont{\normalfont}
%
%% Theorem definition
%\newtheorem{theorem}{Theorem}
%\counterwithin{theorem}{chapter}
%
%% Corollary definition
%\newtheorem{corollary}{Corollary}
%\counterwithin{corollary}{chapter}
%
%% Result definition
%\newtheorem{result}{Result}
%\counterwithin{result}{chapter}
%
%% Lemma definition
%\newtheorem{lemma}{Lemma}
%\counterwithin{lemma}{chapter}
%
%% Proposition definition
%\newtheorem{proposition}{Proposition}
%\counterwithin{proposition}{chapter}
%
%% Definition definition!
%\newtheorem{definition}{Definition}
%\counterwithin{definition}{chapter}
%
%% Remark definition (no counter?)
%\newenvironment{remark}{\emph{Remark:~}}{}
%
%% Fact definition (no counter?)
%\newenvironment{fact}{\emph{Fact:~}}{}

% (Re)Set the figure path
\newcommand{\setfigurepath}[1]{%
\ifx\figurepath\undefined
	\newcommand{\figurepath}{#1}
\else
	\renewcommand{\figurepath}{#1}
\fi%
}

% Used in the continued list environment below
\newcounter{continuedlist}

% Continued list environment
% \newenvironment{continuedlist}{ %
% 	\begin{enumerate}%
% 		% Space out each item
% 		\setlength{\itemsep}{1.25em}%
% 		% Start the enumeration from the previous value
% 		\setcounter{enumi}{\value{continuedlist}} %
% }{ %
%   % Save the counter to continue it later
%   \setcounter{continuedlist}{\value{enumi}}%
%   \end{enumerate}%
%   % \vspace{1.25em}% 
%   \vspace{1em}%
% }

% % Spaced out list environment
% \newenvironment{spacedoutlist}{%
% 	\begin{itemize}%
% 		% Space out each item
% 		\setlength{\itemsep}{1.25em}%
% }{\end{itemize}}


\usepackage[usenames,dvipsnames]{xcolor}
\usepackage{listings}
\lstset{language=C++,%[LaTeX]Tex,%
    keywordstyle=\color{RoyalBlue},%\bfseries,
    basicstyle=\small\sffamily,
%    identifierstyle=\color{NavyBlue},
    commentstyle=\color{Green}\rmfamily,
    stringstyle=\sffamily,
    numbers=left,%none,%
    numberstyle=\scriptsize,%\tiny
    stepnumber=5,
    numbersep=8pt,
    showstringspaces=false,
    breaklines=true,
    %frameround=ftff,
    %frame=single
    %frame=L
    lineskip=-5pt
}


% \lstset{language=Octave,                % choose the language of the code
% basicstyle=\footnotesize,       % the size of the fonts that are used for the code
% numbers=left,                   % where to put the line-numbers
% numberstyle=\footnotesize,      % the size of the fonts that are used for the line-numbers
% stepnumber=2,                   % the step between two line-numbers. If it's 1 each line will be numbered
% numbersep=5pt,                  % how far the line-numbers are from the code
% backgroundcolor=\color{white},  % choose the background color. You must add \usepackage{color}
% showspaces=false,               % show spaces adding particular underscores
% showstringspaces=false,         % underline spaces within strings
% showtabs=false,                 % show tabs within strings adding particular underscores
% frame=single,			% adds a frame around the code
% tabsize=2,			% sets default tabsize to 2 spaces
% captionpos=b,			% sets the caption-position to bottom
% breaklines=true,		% sets automatic line breaking
% breakatwhitespace=false,	% sets if automatic breaks should only happen at whitespace
% lineskip=-5pt  %
% }


\usepackage[sort,round,authoryear,nonamebreak]{natbib}
\setcitestyle{aysep={}} % J Neurophys formatting



\usepackage{xspace}
\usepackage{rotating}

\newcommand{\hdr}[3]{%
\multicolumn{#1}{|l|}{\color{white}\cellcolor[gray]{0.0}%
\textbf{\makebox[0pt]{#2}\hspace{0.5\linewidth}\makebox[0pt][c]{#3}}%
}}

\usepackage[colorinlistoftodos,backgroundcolor=yellow!35,textsize=footnotesize]{todonotes}
\newcommand{\yellownote}[1]{\todo{#1}}



% \setlength{\marginparwidth}{1.1in}
% \let\oldmarginpar\marginpar
% \renewcommand\marginpar[1]{\-\oldmarginpar[\raggedleft\footnotesize #1]%
% {\raggedright\footnotesize #1}}


\usepackage{tikz}
\usepackage{calc}
\usepackage{mparhack}

\setlength{\parskip}{0ex}
\setlength{\parindent}{0ex}

\newlength{\yellownotewidth}
\setlength{\yellownotewidth}{3.5cm}
\newlength{\yellownoteheight}
\setlength{\yellownoteheight}{3.5cm}
%   -   -   -   -   -   -   -   -   -   -   -   -
% Yellow note...
%   -   -   -   -   -   -   -   -   -   -   -   -
% \newcommand{\yellownote}[1]{
% \marginpar{
%     \vspace{-0.5\yellownoteheight}
%         \begin{center}
%         \begin{tikzpicture}
% %            \draw[white,fill=gray!25,opacity=0.75,shift={(-0.125,-0.125)}] 
% %                (0,0) rectangle (\yellownotewidth,\yellownoteheight);
%             \draw[fill=yellow!35] (0,0) rectangle (\yellownotewidth,\yellownoteheight);
% %            \draw[opacity=0.45,fill=gray!50] (0.7\yellownotewidth,0) -- 
% %                (0.9\yellownotewidth,0.45) -- (\yellownotewidth,0.4) -- cycle;
%             \node[blue,below] at (0.5\yellownotewidth,\yellownoteheight) {
%                 \begin{minipage}{\yellownotewidth-1em}
%                     \scriptsize\sf#1
%                 \end{minipage}
%             };
%         \end{tikzpicture}
%         \end{center}
%         \vspace{0.5\yellownoteheight}
%     }
% }

%   -   -   -   -   -   -   -   -   -   -   -   -
% Resizeable - Yellow note...
%   -   -   -   -   -   -   -   -   -   -   -   -
\newcommand{\resizeableyellownote}[3]{
\setlength{\yellownotewidth}{#1cm}
\setlength{\yellownoteheight}{#2cm}
\marginpar{
    \vspace{-0.5\yellownoteheight}
        \begin{center}
        \begin{tikzpicture}
%            \draw[white,fill=gray!25,opacity=0.75,shift={(-0.125,-0.125)}] 
%                (0,0) rectangle (\yellownotewidth,\yellownoteheight);
            \draw[fill=yellow!35] (0,0) rectangle (\yellownotewidth,\yellownoteheight);
%            \draw[opacity=0.45,fill=gray!50] (0.7\yellownotewidth,0) -- 
%               (0.9\yellownotewidth,0.45) -- (\yellownotewidth,0.4) -- cycle;
            \node[blue,below] at (0.5\yellownotewidth,\yellownoteheight) {
                \begin{minipage}{\yellownotewidth-1em}
                    \scriptsize\sf#3
                \end{minipage}
            };
        \end{tikzpicture}
        \end{center}
        \vspace{0.5\yellownoteheight}
    }
}


%% Glossary

% ANF to Golgi
\newcommand{\ANFGLG}{\protect{\textrm{{ANF}}\ensuremath{\to}\textrm{{GLG}}\xspace}}
\newcommand{\HSRGLG}{\protect\ensuremath{\textrm{{HSR}}\to\textrm{{GLG}}\xspace}}
\newcommand{\LSRGLG}{\protect\ensuremath{\textrm{{LSR}}\to\textrm{{GLG}}\xspace}}
\newcommand{\wANFGLG}{\protect\ensuremath{w_{\ANFGLG}\xspace}}
\newcommand{\wLSRGLG}{\protect\ensuremath{w_{\LSRGLG}\xspace}}
\newcommand{\wHSRGLG}{\protect\ensuremath{w_{\HSRGLG}\xspace}}
\newcommand{\nLSRGLG}{\protect\ensuremath{n_{\LSRGLG}\xspace}}
\newcommand{\nHSRGLG}{\protect\ensuremath{n_{\HSRGLG}\xspace}}
\newcommand{\sANFGLG}{\protect\ensuremath{s_{\ANFGLG}\xspace}}
\newcommand{\sLSRGLG}{\protect\ensuremath{s_{\LSRGLG}\xspace}}
\newcommand{\sHSRGLG}{\protect\ensuremath{s_{\HSRGLG}\xspace}}
\newcommand{\dANFGLG}{\protect\ensuremath{d_{\ANFGLG}\xspace}}

%ANF to D-stellate
\newcommand{\ANFDS}{\protect\ensuremath{\textrm{{ANF}}\to\textrm{{DS}}\xspace}}
\newcommand{\HSRDS}{\protect\ensuremath{\textrm{{HSR}}\to\textrm{{DS}}\xspace}}
\newcommand{\LSRDS}{\protect\ensuremath{\textrm{{LSR}}\to\textrm{{DS}}\xspace}}
\newcommand{\wANFDS}{\ensuremath{w_{\ANFDS}\xspace}}
\newcommand{\wLSRDS}{\protect\ensuremath{w_{\LSRDS}\xspace}}
\newcommand{\wHSRDS}{\protect\ensuremath{w_{\HSRDS}\xspace}}
\newcommand{\nLSRDS}{\protect\ensuremath{n_{\LSRDS}\xspace}}
\newcommand{\nHSRDS}{\protect\ensuremath{n_{\HSRDS}\xspace}}
\newcommand{\dANFDS}{\protect\ensuremath{d_{\ANFDS}\xspace}}
\newcommand{\sANFDSh}{\protect\ensuremath{s^+_{\ANFDS}\xspace}}
\newcommand{\sANFDSl}{\protect\ensuremath{s^-_{\ANFDS}\xspace}}

%ANF to T-stellate
\newcommand{\ANFTS}{\protect\ensuremath{\textrm{{ANF}}\to\textrm{{TS}}\xspace}}
\newcommand{\HSRTS}{\protect\ensuremath{\textrm{{HSR}}\to\textrm{{TS}}\xspace}}
\newcommand{\LSRTS}{\protect\ensuremath{\textrm{{LSR}}\to\textrm{{TS}}\xspace}}
\newcommand{\wANFTS}{\protect\ensuremath{w_{\ANFTS}\xspace}}
\newcommand{\nLSRTS}{\protect\ensuremath{n_{\LSRTS}\xspace}}
\newcommand{\nHSRTS}{\protect\ensuremath{n_{\HSRTS}\xspace}}
\newcommand{\sANFTS}{\protect\ensuremath{s_{\ANFTS}\xspace}}
\newcommand{\dANFTS}{\protect\ensuremath{d_{\ANFTS}\xspace}}

%ANF to Tuberculoventral
\newcommand{\ANFTV}{\ensuremath{\textrm{{ANF}}\to\textrm{{TV}}\xspace}}
\newcommand{\HSRTV}{\ensuremath{\textrm{{HSR}}\to\textrm{{TV}}\xspace}}
\newcommand{\LSRTV}{\ensuremath{\textrm{{LSR}}\to\textrm{{TV}}\xspace}}
\newcommand{\wANFTV}{\ensuremath{w_{\ANFTV}\xspace}}
\newcommand{\nLSRTV}{\ensuremath{n_{\LSRTV}\xspace}}
\newcommand{\nHSRTV}{\ensuremath{n_{\HSRTV}\xspace}}
\newcommand{\sANFTV}{\ensuremath{s_{\ANFTV}\xspace}}
\newcommand{\dANFTV}{\ensuremath{d_{\ANFTV}\xspace}}

%GLG to T-stellate
\newcommand{\GLGTS}{\protect\ensuremath{\textrm{GLG}\to\textrm{TS}\xspace}}
\newcommand{\wGLGTS}{\protect\ensuremath{w_{\GLGTS}\xspace}}
\newcommand{\nGLGTS}{\protect\ensuremath{n_{\GLGTS}\xspace}}
\newcommand{\sGLGTS}{\protect\ensuremath{s_{\GLGTS}\xspace}}
\newcommand{\dGLGTS}{\protect\ensuremath{d_{\GLGTS}\xspace}}
%GLG to D-stellate
\newcommand{\GLGDS}{\protect\ensuremath{\textrm{{GLG}}\to\textrm{{DS}}\xspace}}
\newcommand{\wGLGDS}{\protect\ensuremath{w_{\GLGDS}\xspace}}
\newcommand{\nGLGDS}{\protect\ensuremath{n_{\GLGDS}\xspace}}
\newcommand{\sGLGDS}{\protect\ensuremath{s_{\GLGDS}\xspace}}
\newcommand{\dGLGDS}{\protect\ensuremath{d_{\GLGDS}\xspace}}

% % TS to Golgi
% \newcommand{\GLGTS}{\protect\ensuremath{\textrm{{GLG}}\to\textrm{{TS}}\xspace}}
% \newcommand{\wTSGLG}{\protect\ensuremath{w_{}}\xspace}}
% \newcommand{\nTSGLG}{\protect\ensuremath{n_{\textrm{TS}\to\textrm{GLG}}\xspace}}
% \newcommand{\sTSGLG}{\protect\ensuremath{s_{\textrm{TS}\to\textrm{GLG}}\xspace}}
% \newcommand{\dTSGLG}{\protect\ensuremath{d_{\textrm{TS}\to\textrm{GLG}}\xspace}}

%TS to D-stellate
\newcommand{\TSDS}{\protect\ensuremath{\textrm{{TS}}\to\textrm{{DS}}\xspace}}
\newcommand{\wTSDS}{\protect\ensuremath{w_{\TSDS}\xspace}}
\newcommand{\nTSDS}{\protect\ensuremath{n_{\TSDS}\xspace}}
\newcommand{\sTSDS}{\protect\ensuremath{s_{\TSDS}\xspace}}
\newcommand{\dTSDS}{\protect\ensuremath{d_{\TSDS}\xspace}}
%TS to T-stellate
\newcommand{\TSTS}{\protect\ensuremath{\textrm{TS}\to\textrm{TS}\xspace}}
\newcommand{\wTSTS}{\protect\ensuremath{w_{\TSTS}\xspace}}
\newcommand{\nTSTS}{\protect\ensuremath{n_{\TSTS}\xspace}}
\newcommand{\sTSTS}{\protect\ensuremath{s_{\TSTS}\xspace}}
\newcommand{\dTSTS}{\protect\ensuremath{d_{\TSTS}\xspace}}

%TS to Tuberculoventral
\newcommand{\TSTV}{\protect\ensuremath{\textrm{{TS}}\to\textrm{{TV}}\xspace}}
\newcommand{\wTSTV}{\protect\ensuremath{w_{\TSTV}\xspace}}
\newcommand{\nTSTV}{\protect\ensuremath{n_{\TSTV}\xspace}}
\newcommand{\sTSTV}{\protect\ensuremath{s_{\TSTV}\xspace}}
\newcommand{\dTSTV}{\protect\ensuremath{d_{\TSTV}\xspace}}

% DS to Golgi
\newcommand{\DSGLG}{\protect\ensuremath{\textrm{{DS}}\to\textrm{{GLG}}\xspace}}
\newcommand{\wDSGLG}{\protect\ensuremath{w_{\DSGLG}\xspace}}
\newcommand{\nDSGLG}{\protect\ensuremath{n_{\DSGLG}\xspace}}
\newcommand{\sDSGLG}{\protect\ensuremath{s_{\DSGLG}\xspace}}
\newcommand{\dDSGLG}{\protect\ensuremath{d_{\DSGLG}\xspace}}

% %DS to D-stellate
% \newcommand{\DSDS}{\protect\ensuremath{\textrm{{DS}}\to\textrm{{DS}}\xspace}}
% \newcommand{\wDSDS}{\protect\ensuremath{w_{\DSDS}\xspace}}
% \newcommand{\nDSDS}{\protect\ensuremath{n_{\DSDS}\xspace}}
% \newcommand{\sDSDS}{\protect\ensuremath{s_{\DSDS}\xspace}}
% \newcommand{\dDSDS}{\protect\ensuremath{d_{\DSDS}\xspace}}

%DS to T-stellate
\newcommand{\DSTS}{\protect\ensuremath{\textrm{DS}\to\textrm{TS}\xspace}}
\newcommand{\wDSTS}{\protect\ensuremath{w_{\DSTS}\xspace}}
\newcommand{\nDSTS}{\protect\ensuremath{n_{\DSTS}\xspace}}
\newcommand{\sDSTS}{\protect\ensuremath{s_{\DSTS}\xspace}}
\newcommand{\dDSTS}{\protect\ensuremath{d_{\DSTS}\xspace}}

%DS to Tuberculoventral
\newcommand{\DSTV}{\protect\ensuremath{\textrm{{DS}}\to\textrm{{TS}}\xspace}}
\newcommand{\wDSTV}{\protect\ensuremath{w_{\DSTV}\xspace}}
\newcommand{\nDSTV}{\protect\ensuremath{n_{\DSTV}\xspace}}
\newcommand{\sDSTV}{\protect\ensuremath{s_{\DSTV}\xspace}}
\newcommand{\dDSTV}{\protect\ensuremath{d_{\DSTV}\xspace}}
\newcommand{\oDSTV}{\protect\ensuremath{o_{\DSTV}\xspace}}

% TV to Golgi
\newcommand{\TVGLG}{\protect\ensuremath{\textrm{{TV}}\to\textrm{{GLG}}\xspace}}
\newcommand{\wTVGLG}{\protect\ensuremath{w_{\TVGLG}\xspace}}
\newcommand{\nTVGLG}{\protect\ensuremath{n_{\TVGLG}\xspace}}
\newcommand{\sTVGLG}{\protect\ensuremath{s_{\TVGLG}\xspace}}
\newcommand{\dTVGLG}{\protect\ensuremath{d_{\TVGLG}\xspace}}

%TV to D-stellate
\newcommand{\TVDS}{\protect\ensuremath{\textrm{{TV}}\to\textrm{{DS}}\xspace}}
\newcommand{\wTVDS}{\protect\ensuremath{w_{\TVDS}\xspace}}
\newcommand{\nTVDS}{\protect\ensuremath{n_{\TVDS}\xspace}}
\newcommand{\sTVDS}{\protect\ensuremath{s_{\TVDS}\xspace}}
\newcommand{\dTVDS}{\protect\ensuremath{d_{\TVDS}\xspace}}

%TV to T-stellate
\newcommand{\TVTS}{\protect\ensuremath{\textrm{{TV}}\to\textrm{{TS}}\xspace}}
\newcommand{\wTVTS}{\protect\ensuremath{w_{\TVTS}\xspace}}
\newcommand{\nTVTS}{\protect\ensuremath{n_{\TVTS}\xspace}}
\newcommand{\sTVTS}{\protect\ensuremath{s_{\TVTS}\xspace}}
\newcommand{\dTVTS}{\protect\ensuremath{d_{\TVTS}\xspace}}

%TV to Tuberculoventral
\newcommand{\TVTV}{\protect\ensuremath{\textrm{{TV}}\to\textrm{{TV}}\xspace}}
\newcommand{\wTVTV}{\protect\ensuremath{w_{\TVTV}\xspace}}
\newcommand{\nTVTV}{\protect\ensuremath{n_{\TVTV}\xspace}}
\newcommand{\sTVTV}{\protect\ensuremath{s_{\TVTV}\xspace}}
\newcommand{\dTVTV}{\protect\ensuremath{d_{\TVTV}\xspace}}


%Other common symbols
\newcommand{\GABAa}{\textrm{GABA}$_{\textrm{A}}\xspace$}

%\usepackage[margin=0.2in]{geometry} % get enough space on page

\usepackage{tabularx} % automatically adjusts column width in tables
\usepackage{multirow} % allows entries spanning several rows
\usepackage{colortbl} % allows coloring tables
% \usepackage{natbib}
% \usepackage[fleqn]{amsmath} % displayed equations flush left
% \setlength{\mathindent}{0em}

% use Helvetica for text, Pazo math fonts
\usepackage{mathpazo} \usepackage[scaled=.95]{helvet}
\renewcommand\familydefault{\sfdefault}

\renewcommand\arraystretch{1.2} % slightly more space in tables

\pagestyle{empty} % no header of footer

% \hdr{ncols}{label}{title}
%
% Typeset header bar across table with ncols columns with label at
% left margin and centered title
%


\graphicspath{{/media/data/Work/cnstellate/golgi_RateLevel/}{/media/data/Work/Responses/}{/media/data/Work/cnstellate/Responses/}}

\begin{document}

%%%%%%%%%%%%%%%%%%%%%%%%%%%%%%%%%%%%%%%%%%%%%
%%%%%%%%%%%%%%%%%%%%%%%%%%%%%%%%%%%%%%%%%%%%%
%%%%%%%%%%%%%%%%%%%%%%%%%%%%%%%%%%%%%%%%%%%%%

\section{Golgi Cell Optimisation: Modelling GABAegic VCN units from
  monotonic marginal shell units}

% Modelling in the auditory periphery has benefited extensively from
% the work of Liberman, Greewood, Patterson, Young, Sachs and others,
% in acoustic \it{in vivo} experiments.


The source of GABAergic inputs to cells in the mammalian CN is somewhat
contentious. Despite studies showing that GABAergic inputs to the CN generally
arise in the peri-olivary regions of the medulla in cats
\citep{OstapoffBensonEtAl:1997} and birds
\citep{LachicaRubsamenEtAl:1995,YangMonsivaisEtAl:1999}, slice preparations of
the isolated murine VCN have shown sensitivity to bicuculine in TS and DS cells
\citep{FerragamoGoldingEtAl:1998a}.  The only known source of GABA intrinsic to
the VCN are the golgi cells of the granule cell domain (GCD) overlying the VCN
\citep[Fig.~\ref{fig:CNdiagram}]{Mugnaini:1985,FerragamoGoldingEtAl:1998}. Inputs
to golgi cells are more complicated than VCN cells in the core regions. Golgi
cells are sparse in the GCD, surrounded by the many, smaller excitatory granule
cells. Extracellular recordings from the GCD (most likely from golgi cells
since granule cell somata are $<10 \mu m$) indicate a monotonic increase in
firing rate with increasing sound intensity \citep{GhoshalKim:1997}. The
presence of GABAergic inputs to TS, DS and TV cells has been verified by
labeled terminals adjacent to the soma and dendrites
\citep{SmithRhode:1989,AwatramaniTurecekEtAl:2005,BabalianRyugoEtAl:2003} and
release from inhibition in their response areas with ionotopopheretic
application of the GABA antagonist, bicuculine
\citep{EvansZhao:1998,CasparyBackoffEtAl:1994,BackoffShadduckEtAl:1999,FerragamoGoldingEtAl:1998a}.
Post-onset GABAergic inhibition in DS cells is a major influence on the PSTH of
On-C neurons \citep{FerragamoGoldingEtAl:1998a,EvansZhao:1998}. Latency of
excitation to auditory nerve shocks suggests golgi cells are activated by type
II ANFs and low spontaneous rate type I ANFs
\citep{BensonBerglundEtAl:1996,FerragamoGoldingEtAl:1998}. Therefore, type II
and LS type I ANFs could be involved in gain control through GABAergic
modulation of activity in the VCN.

The lack of extensive experimental data meant that a phenomenological model
would be preferred over the Hogkin-Huxley type neural model. A number of steps
were taken to investigate the golgi cell model.

The known GABAergic input to VCN units comes primarily from the superior olive
(ref), but the presence of active GABA synapses in isolated VCN slices by
Ferragammo \citet{FerragamoGoldingEtAl:1998a} led to further investigation of
golgi cells in the granule cell domain by \citet{FerragamoGoldingEtAl:1998a}.
Ferragamo et~al.~showed that golgi cells have a classic type-I current
response, which suggest they integrate inputs.  Ghosal and Kim
\citeyear{GhoshalKim:1997} suggested that golgi cells provide some sort of
automatic gain control to the principle VCN units, through their monotonic
responses to tones and noise.





% (Reference) showed that in adult animals high-spontaneous rate ANFs do not
% project to the GCD; they do show that low-spontaneous rate ANFs do project into
% the GCD, albeit more profusely than in the core on the VCN.
   

In the creation of the golgi cell model, we can reduce the explicit
responses of Golgi cells down to:
\begin{itemize}
\item type-I~current clamp response,
\item monotonic response to tones and noise, and
\item delay from ANFs of \~0.7~ms relative to the core VCN units.
\end{itemize}


In Chapter \ref{Chapter5:GA} and previous publications \citep{EagerGraydenEtAl:2006a}


\subsection{Golgi Model}

Monotonic rate-level data from GCD in VCN \citep{GhoshalKim:1996} unit S03-07
(CF 21~kHz) was used to optimise parameters ${\rm golgi\_spon}$, \wLSRGLG,
\wHSRGLG, and  \sANFGLG.  The optimisation method was a default  using the principle-axis method.


To summarise the neural model used in this optimisation, I have
reduced the information to tabular format as suggested by
\citet{NordlieGewaltigEtAl:2009}.


\todo{* Model Summary 
* Populations 
* Connectivity 
* Neuron and Synapse Model 
* Input 
* Measurement
}





\begin{figure}[hp!]
  \centering
  \includegraphics[width=0.9\textwidth,trim=0 110mm 1 55mm]{./gfx/GolgiFilter}
  \caption{ Poisson instantaneous-rate profile was generated using ANF
    profiles. $s$ is the spread of frequency channels based, $w$ is
    the weighted sum of HSR and LSR instantaneous-rate vectors,
    $\alpha$ is a smoothing function (alpha function kernel).}
\end{figure}





\vspace{2ex}
% - A
% ------------------------------------------------------------------------------
\noindent
\begin{tabularx}{\textwidth}{|l|X|}\hline %
\hdr{2}{A}{Model Summary}\\\hline 
 \textbf{Populations}   & ANF(HSR,LSR) and Golgi \\\hline 
   \textbf{Topology}    & Tonotopic - 100 frequency channels based on
   Rat basilar membrane \citep{Greenwood:1990} and audiogram \citep{HeffnerKoayEtAl:2001}\\\hline
 \textbf{Connectivity}  & Place-based Gaussian spread of connections \\\hline
 \textbf{Neuron model}  & Poisson instantaneous-rate model with refractory effects \\\hline
\textbf{Channel models} & --- \\\hline 
\textbf{Synapse model}  & alpha function smoothing kernel \\\hline
    \textbf{Input}      & Independent fixed-rate Poisson spike trains to all neurons \\\hline
 \textbf{Measurements}  & Mean rate, spike activity \\\hline
 \textbf{Neuron model}  &
\begin{minipage}[c]{0.5\textwidth} 
ANFs: \citeauthor{ZilanyBruceEtAl:2009},  instantaneous-rate Poisson model  \\
Golgi: instantaneous-rate Poisson model developed from ANF inputs\\
\end{minipage} \\\hline
\end{tabularx}

% - B -----------------------------------------------------------------------------

\noindent\begin{tabularx}{\linewidth}{|l|X|X|}\hline %{\linewidth}
\hdr{3}{B}{Populations}\\\hline
  \textbf{Name}   & \textbf{Elements} & \textbf{Number} \\\hline
    HSR     & Zilany \& Bruce model        & $N_{\text{HSR}} = 50$ per freq. channel \\\hline
    LSR     & Zilany \& Bruce model        & $N_{\text{LSR}}= 20$  per freq. channel \\\hline
    GLG     & Filtered instantaneous rate model, Poisson spike-generator with refractory effects & $N_{\text{GLG}}= 1$  per freq. channel  \\\hline
\end{tabularx}
\vspace{2ex}

% - C ------------------------------------------------------------------------------

\noindent\begin{tabularx}{\linewidth}{|l|l|l|X|}\hline
\hdr{4}{C}{Connectivity}\\\hline
\textbf{Name} & \textbf{Source} & \textbf{Target} & \textbf{Pattern} \\\hline
  $\textrm{ANF} \to \textrm{GLG}$ & ANF (HSR and LSR) & Golgi & \begin{minipage}[c]{0.5\textwidth}
    Gaussian, centered at CF, spread of LSR \sLSRGLG was optimised, spread of HSR \sHSRGLG is fixed due to its replication of granule cells in the model, weight for LSR \wLSRGLG and HSR \wHSRGLG are determined  for all synapses, number \nLSRDS and \nHSRDS, delay \dANFGLG added to smoothing function to ensure conductance and dendritic filtering are included.
\end{minipage} \\\hline
 \end{tabularx}

\vspace{2ex}
% - D ------------------------------------------------------------------------------
\noindent\begin{tabularx}{\linewidth}{|p{0.15\linewidth}|X|}\hline
\hdr{2}{D}{Neuron and Synapse Model}\\\hline
\textbf{Name} & Golgi Phenomenological Model \\\hline
\textbf{Type} & Poisson instantaneous-rate model, ANF inst. rate input\\\hline
\textbf{Golgi Phenomenological Model} & \begin{minipage}[c]{0.6\textwidth}
$\mathbf{w}_{LSR} = N(\textrm{CF channel},\sLSRGLG)$,  $\mathbf{w}_{HSR} = N(\textrm{CF channel},\sHSRGLG)$  \\ 
\texttt{for \textit{i}=0, nchannels} \\
	$\mathbf{x}_i = \mathbf{w}_{LSR}(i)\mathbf{LSR}_i+\mathbf{w}_{HSR}(i)\cdot\mathbf{HSR}_i$ \\
\texttt{end}
	$\mathbf{x} = \mathbf{x}_i\circledast\mathbf{a}$  //Convolve profile with Alpha kernel\\
\end{minipage} \\\hline
% \multirow{3}{*}{\textbf{Spiking}} &
%    If $V(t-)<\theta \wedge V(t+)\geq \theta$
% \vspace*{-1ex}
% \begin{enumerate}\setlength{\itemsep}{-0.5ex}
% \item set $t^* = t$
% \item emit spike with time-stamp $t^*$
% \end{enumerate}
% \vspace*{-4ex}\rule{1em}{0em}
% \\\hline
\end{tabularx}

\vspace{2ex}



% - B
% -----------------------------------------------------------------------------
\vspace{2ex}
\noindent
\begin{tabularx}{\textwidth}{|l|X|}\hline %
\hdr{2}{B}{Input/Ouput}\\\hline \textbf{Input Stimulus} & Rate Level
function, 21~kHz tone at SPL -15 to 85 dB (20 ms delay, 2ms cosine squared
      on/off ramp)\\\hline \textbf{Measurements}        & Mean rate of instantaneous rate
profile or PSTH sampled from poisson spike-generator (25 repetitions). \\\hline
\end{tabularx}
\vspace{2ex}

\noindent\begin{tabularx}{\textwidth}{|X|X|X|X|}\hline
\hdr{4}{C}{Connectivity}\\\hline 
         \textbf{Name}          &  \textbf{Source}  & 
        \textbf{Target}         & \textbf{Pattern} \\\hline 
$\textrm{ANF} \to \textrm{GLG}$ & ANF (HSR and LSR) & Golgi & 
\begin{minipage}[c]{0.3\textwidth} Gaussian, centered at CF, spread
of LSR \sANFGLG was optimised, spread of HSR was fixed due to the
lack of evidence replication of granule cells in the model, weight
for LSR \wLSRGLG and HSR \wHSRGLG are determined for all synapses,
number \nLSRDS and \nHSRDS, delay \dANFGLG added to smoothing
function to ensure conductance and dendritic filtering are
included.
\end{minipage} \\\hline
\end{tabularx}
\vspace{2ex}


% % - D
% % ------------------------------------------------------------------------------
% % \begin{etabular}{|l|l|X|}%{3}{A}{Parameters}
%   \noindent\begin{tabularx}{\textwidth}{|l|r|c|X|}\hline
%     \hdr{4}{A}{Parameters}\\\hline \textbf{Name} &
%     \textbf{Initial}&\textbf{Range}&\textbf{Comments} \\ \hline
%     \wANFGLG& 1 	&	& Spontaneous rate  \\
%     \nLSRGLG& 0.5 	&	&weight of LSR ANFs to Golgi cells          \\
%     \nHSRGLG& 0.1 	&	&weight of HSR ANFs to Golgi cells        \\
%     \sANFGLG& 3 	&	&spatial spread of LSR ANFs to DS cells             \\
%     \dANFGLG& - & &delay of ANFs to Golgi \\ \hline \hline
%   \end{tabularx}
%   \vspace{2ex}
% % \end{etabular}

\noindent
\begin{tabularx}{\textwidth}{|l|X|}\hline %{\textwidth}
\hdr{2}{C}{Optimisation} \\ \hline 
        \textbf{Type}         & Simplex method
\\\hline 
     \textbf{Parameters}      & \\\hline 
   Spatial spread \ANFGLG     & $\sANFGLG \quad\to [0,10] $ \\\hline 
Dendritic Filter time constant& $\tau_{\ANFGLG} \quad\to [0,20]$ ms\\\hline 
     Weighted sum of HSR      & $\wHSRGLG \quad\to [0,5] $\\\hline 
     Weighted sum of HSR      & $\wLSRGLG\quad\to [0,5] $\\\hline 
      Spontaneous Rate        & ${\rm golgi\_spon} \to [0,50]$ sp/ms \\\hline
\end{tabularx}

% \includegraphics[width=0.6\textwidth,angle=-90]{GolgiRateLevelActualFit}\\
% \caption{Optimisation Results for Golgi Model using Rate Level data
%   from }\label{Ch3:fig:GolgiFit}
% \includegraphics[width=0.8\textwidth]{GolgiRateLevel}\\
% \caption{Optimisation Results for Golgi Model using Rate Level data
%   from }\label{Ch3:fig:GolgiRL}

% \includegraphics[width=0.8\textwidth]{golgi_RateLevel_opt}\\
% \caption{Optimisation Results for Golgi Model using Rate Level data
%   from }\label{Ch3:fig:GolgiRL}
% \includegraphics[width=0.8\textwidth,angle\todo=-90]{GolgiRateLevel2}\\
% \caption{Optimisation Results for Golgi Model using Rate Level data
%   from }\label{Ch3:fig:GolgiRL}

\subsection{Results}
Fig. 4: Golgi model (Green) and spike based output (Pink) was used to
fit the experimental data of unit S03-07 (CF 21~kHz) from
\citep{GhoshalKim:1996} (Red).  LSR mean rate (Blue) of 21~kHz unit is
monotonic with a high threshold.

\begin{figure}[htb]
  \centering
  \resizebox{0.8\textwidth}{!}{\includegraphics{./gfx/GhoshalKim}}
\end{figure}


\begin{figure}[htb]
  \centering \turnbox{90}{\small{Rate (sp/s)}}
  \resizebox{0.8\textwidth}{!}{\includegraphics[angle=-90]{./gfx/GolgiRateLevel2}}\\
  \small{Level (dB SPL)}
\end{figure}

\subsubsection{Best Parameters}

\begin{minipage}[c]{0.5\textwidth}%
  $\sANFGLG = 2.48 $ \\
  $\tau_{\ANFGLG} = 5.0$ ms\\
  $\wHSRGLG = 0.517 $ \\
  $\wLSRGLG = 0.0487 $ \\
  \textit{golgi\_spon} $=  3.727 $ sp/ms \\
\end{minipage}

\textbf{Error} 0.021 (MSE re max rate)



\clearpage \newpage
\section{Verification}
\subsection{Tone Response}

\begin{figure}[h]
  \centering\resizebox{0.95\textwidth}{!}{%
    \includegraphics{RateLevel/psthsingle90.3.eps}%
    \includegraphics{RateLevel/G_ratelevel.eps}}
\end{figure}
\begin{figure}[h]
  \centering\resizebox{0.95\textwidth}{!}{%
    \includegraphics{RateLevel/response_area.3.eps}%
    \includegraphics{RateLevel/response_area_log2.3.eps}}
\end{figure}
\begin{figure}[h]
  \centering\resizebox{0.95\textwidth}{!}{%
    % \includegraphics{RateLevel/response_area.3.eps}
    \includegraphics{RateLevel/psthall90.3.eps}%
    \includegraphics{RateLevel/psthVlevel.3.eps}}
\end{figure}



\clearpage
\subsection{Noise Response}
\begin{figure}[h]
  \centering\resizebox{0.95\textwidth}{!}{%
    \includegraphics{NoiseRateLevel/psthsingle120.3.eps}%
    \includegraphics{NoiseRateLevel/G_ratelevel.eps}}
\end{figure}
\begin{figure}[h]
  \centering\resizebox{0.95\textwidth}{!}{%
    \includegraphics{NoiseRateLevel/response_area.3.eps}%
    \includegraphics{NoiseRateLevel/response_area_log2.3.eps}}
\end{figure}
\begin{figure}[h]
  \centering\resizebox{0.95\textwidth}{!}{%
    % \includegraphics{RateLevel/response_area.3.eps}
    \includegraphics{NoiseRateLevel/psthall90.3.eps}%
    \includegraphics{NoiseRateLevel/psthVlevel.3.eps}}
\end{figure}


\clearpage
\subsection{Masked Noise and Tone}
\begin{figure}[h!]
  \centering\resizebox{0.95\textwidth}{!}{\includegraphics{MaskedRateLevel/psthsingle90.3.eps}\includegraphics{MaskedRateLevel/G_ratelevel.eps}}
\end{figure}
\begin{figure}[h!]
  \centering\resizebox{0.95\textwidth}{!}{%
    \includegraphics{MaskedRateLevel/response_area.3.eps}%
    \includegraphics{MaskedRateLevel/response_area_log2.3.eps}}
\end{figure}

\begin{figure}[h!]
  \centering\resizebox{0.95\textwidth}{!}{%
    % \includegraphics{RateLevel/response_area.3.eps}
    \includegraphics{MaskedRateLevel/psthall90.3.eps}%
    \includegraphics{MaskedRateLevel/psthVlevel.3.eps}}
\end{figure}
\clearpage
\subsection{Masked Response Area}
\begin{figure}[h!]
  \centering\resizebox{0.95\textwidth}{!}{%
    \includegraphics{MaskedResponseCurve/psthsingle5810.3.eps}%
    \includegraphics{MaskedResponseCurve/G_masked.eps}}
\end{figure}
\begin{figure}[h!]
  \centering\resizebox{0.95\textwidth}{!}{%
    \includegraphics{MaskedResponseCurve/response_area.3.eps}%
    \includegraphics{MaskedResponseCurve/response_area_log2log2.3.eps}}
\end{figure}

\begin{figure}[h!]
  \centering\resizebox{0.95\textwidth}{!}{%
    % \includegraphics{RateLevel/response_area.3.eps}
    \includegraphics{MaskedResponseCurve/psthall5810.3.eps}%
    \includegraphics{MaskedResponseCurve/psthVmod.3.eps}}
\end{figure}
\clearpage





% \todo{add stuff here}



% %%%%%%%%%%%%%%%%%%%%%%%%%%%%%%%%%%%%%%%%%%%%%%%%%%%%%%
\bibliographystyle{plainnat}%bmc_article} % Style BST file
\bibliography{../manuscript/bib/MyBib}
 
\end{document}
