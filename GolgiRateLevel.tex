
%%%%%%%%%%%%%%%%%%%%%%%%%%%%%%%%%%%%%%%%%%%%%
\graphicspath{{../figures/}{./gfx/}{/media/data/Work/cnstellate/golgi/}{/media/data/Work/cnstellate/}}
%%%%%%%%%%%%%%%%%%%%%%%%%%%%%%%%%%%%%%%%%%%%%

\section[Golgi Cell Model]{Golgi Cell Model: Optimisation using monotonic
  rate-level responses in marginal shell units}
\label{sec:GolgiCellModel}

\subsection{Background}

The presence of GABAergic inputs to TS, DS and TV cells has been verified by
labeled terminals adjacent to the soma and dendrites
\citep{SmithRhode:1989,AwatramaniTurecekEtAl:2005,BabalianRyugoEtAl:2003} and
release from inhibition in their response areas with ionotopopheretic
application of the \GABAa antagonist, bicuculine
\citep{EvansZhao:1998,CasparyBackoffEtAl:1994,BackoffShadduckEtAl:1999,FerragamoGoldingEtAl:1998a}. The
source of GABAergic inputs to cells in the mammalian CN is somewhat
contentious. Studies show that GABAergic inputs to the CN generally arise in the
peri-olivary regions of the medulla in cats \citep{OstapoffBensonEtAl:1997} and
birds \citep{LachicaRubsamenEtAl:1995,YangMonsivaisEtAl:1999}. Slice
preparations of the isolated murine VCN show strong and immediate sensitivity to
bicuculine in TS and DS cells from a source within the CN complex
\citep{FerragamoGoldingEtAl:1998a}.  The only known source of GABA intrinsic to
the VCN are the Golgi cells of the granule cell domain (GCD) overlying the VCN
\citep[Fig.~\ref{fig:CNdiagram}]{Mugnaini:1985,FerragamoGoldingEtAl:1998}.

%\medskip{}

% \yellownote{TODO: Clean up paragraph} Other studies in the rat cochlear
% nucleus relating to the Golgi cell or GABA:
% \begin{itemize}
% \item \citep{MugnainiOsenEtAl:1980} Fine structure of granule cells and
%   related inter-neurons (termed {Golgi} cells) in the cochlear nuclear complex
%   of cat, rat and mouse
% \item \GABAa expression in the rat brainstem \citep{CamposCaboEtAl:2001}
% \item \citep{Alibardi:2003a} Ultrastructural distribution of glycinergic and
%   {{GABAergic}} neurons and axon terminals in the rat dorsal cochlear nucleus,
%   with emphasis on granule cell areas
% \item \citep{AwatramaniTurecekEtAl:2005} Staggered {Development} of
%   {GABAergic} and {Glycinergic} {Transmission} in the {MNTB}
% \end{itemize}

% %\medskip{}

% \yellownote{TODO: Expand role of GABA, or combine with previous para} Role of
% GABA in the VCN
% \begin{itemize}
% \item Effects of microiontophoretically applied glycine and {GABA} on neuronal
%   response patterns in the cochlear nuclei \citep{CasparyHaveyEtAl:1979}
% \end{itemize}
% \citep{Alibardi:2003a} rat CN complex -> Golgi-stellate cells (fusiform layer:
% 2) in DCN contact granule and unipolar brush cells


Inputs to Golgi cells are more complicated than VCN cells in the core
regions. Golgi cells are sparse in the GCD, surrounded by the many, smaller
excitatory granule cells, that form small en passant endings. Type II ANFs
create diffuse glutamatergic release sites in the GCD
\citep{HurdHutsonEtAl:1999,BensonBrown:2004} that may stimulate NMDA glutamate
receptors in Golgi cells \citep{FerragamoGoldingEtAl:1998a}.

%\medskip{}

Extracellular recordings from labelled Golgi cells is not available in the
literature.  The GCD (or marginal shell in cats) has been studied by one group
\citet{GhoshalKim:1997}, and any extracellular spikes are most likely from Golgi
cells since granule cell somata are less than $10{}\mu{m}$ and very narrow
axons. The majority of recorded units showed a monotonic increase in firing rate
with increasing sound intensity
\citep[Figure~\ref{fig:GolgiKimFig2}][]{GhoshalKim:1997}.  The general
assumption of the functional role of Golgi cells is to regulate granule cells
but they may also provide automatic gain control to the principle VCN units,
through their monotonic responses to tones and noise.

%\medskip{}

The known GABAergic input to VCN units comes primarily from the superior olive
(ref), but the presence of active GABA synapses in isolated VCN slices by
\citet{FerragamoGoldingEtAl:1998} led to further investigation of Golgi cells in
the granule cell domain. Intracellular recordings of Golgi cells have a classic
type-I current response, which suggest they are simple integrators, and their
response to AN shocks were delayed by approximately 0.7~ms relative to the core
VCN units \citep{FerragamoGoldingEtAl:1998}.

%\medskip{}

% (Reference) showed that in adult animals high-spontaneous rate ANFs do not
% project to the GCD; they do show that low-spontaneous rate ANFs do project
% into the GCD, albeit more profusely than in the core on the VCN.
   
\begin{figure}[ht!]
  \centering
  % \resizebox{3.5in}{!}{\includegraphics{NoFigure}}
  \resizebox{0.5\textwidth}{!}{\includegraphics{GhoshalKim}}
  \caption{Rate level response of 6 units \citep[from Fig.~2]{GhoshalKim:1997}. Unit S03-07 (CF 22.7~kHz) will be the unit chosen to optimise the Golgi cell model.}\label{fig:GolgiKimFig2}
\end{figure}
  \yellownote{Inclusion of Ghoshal figure needs permission, fill in caption}

\subsection{Implementation}

\todo{see david's clean up}In Chapter~\ref{sec:GAChapter}, the Golgi cell model was implemented as a
single-compartment conductance neuron. Due to the unavailability of sufficient
data regarding \emph{in vivo} Golgi cell responses, the decision to simulate the
Golgi cell model as a Poisson neuron rather than simulating the neural membrane
with Hodgkin-Huxley models.  The instantaneous response profile of Golgi cells
use inputs from the auditory model's instantaneous rate outputs.  A number of
steps were taken to investigate the Golgi cell model. A further detailed
explanation of the implementation is in the chapter
Appendix~\ref{sec:chp3appendix}.  %\medskip{}

In the creation of the Golgi cell model, we can reduce the explicit responses of
Golgi cells down to three major details: a) Golgi cells are integrators due to
their type-I~current clamp response \citep{FerragamoGoldingEtAl:1998}, b) Golgi
cells are (most likely) monotonic to tone and noise rate increases
\citep{GhoshalKim:1997}, and c) they have a significant delay of first spike
latency relative to the core VCN units \citep{GhoshalKim:1997}. The lack of
extensive experimental data regarding type-II ANF units and granule cell
response to acoustic input meant that a Poisson rate neural model would be
preferred over the Hodgkin-Huxley type neural model.  %\medskip{}
% by $\mu(f=CF)$ and $\sigma$ variables, which control

The Golgi cell model is implemented as an instantaneous-rate Poisson rate model,
shown in Figure~\ref{fig:GolgiDiagram}. Connections across frequency channels of
HSR and LSR ANF responses to Golgi cells were determined the Gaussian
distribution in units of channel separation in the network. The weighted sum of
HSR and LSR instantaneous-rate vectors are smoothed out by an alpha function,
Eq.~\ref{eq:alpha_Golgi}, mimicking a synaptic and dendritic smoothing filter.
\begin{equation}\label{eq:alpha_Golgi}
  \alpha(t)=t*exp(-t/\tau_{G})
\end{equation}
The alpha filter length was 10 times the time constant, $\tau_{G}$, and was
normalised so that the area under the function was 1. The final step in the Golgi cell model instantaneous Poisson profile was to correct for spontaneous activity and negative values.

Although HSR ANF terminals do not generally project into the GCD, they are
included in this model to provide some low level sound-induced activity.
%\medskip{}

\begin{figure}[h!]
  \centering
  % \resizebox{3.5in}{!}{\includegraphics{NoFigure}}\\
  \resizebox{0.9\textwidth}{!}{\includegraphics{gfx/GolgiDiagram.eps}}\\
  \caption{The Golgi instantaneous-rate profile was generated using a weighted
    sum ANF profiles and a alpha function smoothing filter to mimic dendritic
    and synaptic filtering. The Gaussian spread of connections is independent
    for HSR and LSR auditory filters, with the mean equal to CF channel of
    unit. The alpha function includes a delay of 2.5~ms, which is 0.7~ms greater
    than the core VCN units as shown by \citet{GhoshalKim:1997}. The final stage
    sets the spontaneous rate by addition at t=0 and changes any negative values
    to zero in the rate profile. }
%     across frequency channels is Gaussian, and $\mathbf{w}$ is
%    the weighted sum of HSR and LSR instantaneous-rate vectors,
%     $\alpha$ is the synaptic and dendritic smoothing function.
  \label{fig:GolgiDiagram}
\end{figure}

%\medskip{}

Monotonic rate-level data from GCD in VCN \citep{GhoshalKim:1996} unit S03-07
(CF 21~kHz) was used to optimise parameters \Gspon, \wLSRGLG, \wHSRGLG, and
\sANFGLG\@.  The table~\ref{tab:GolgiCellModelSummary} shows the model details
to be used by the optimisation process.

{\small%\linespread{0.5}
  \begin{table}[htb]
    \caption{Golgi cell model summary (Nordlie format)}
    \label{tab:GolgiCellModelSummary}
  \end{table}
\noindent%
\begin{tabularx}{\textwidth}{|l|X|}\hline %
\hdr{2}{A}{Model Summary}\\\hline 
%\begin{ntab}{|l|X|}{2}{\ref{tab:GolgiCellModelSummary} A}{Model Summary}\\\hline
 \textbf{Populations}  & ANF~(HSR, LSR) and Golgi cells \\\hline 
  \textbf{Topology}    & Tonotopic - $N_{\text{channel}}=100$ frequency channels (0.2--40 kHz) separated evenly based on even place positions on the basilar membrane \citep{Greenwood:1990}\\\hline
\textbf{Connectivity}  & Place-based Gaussian spread of connections from ANF to GLG \\\hline
\textbf{Input model}  & ANF~model: instantaneous-rate Poisson model \citep{ZilanyBruce:2007} \\\hline
\textbf{Neuron model}  & Golgi cell model: instantaneous-rate Poisson model developed from ANF inputs \\\hline
\textbf{Synapse model} & Synapto-dendritic smoothing filter (alpha function) \\\hline
    \textbf{Input}     & Pure tones (22.7 kHz, 50 ms, 5 ms on/off ramp, 20 ms delay), intensity range 0--100 dB~SPL   \\\hline
\textbf{Measurements}  & Mean firing rate of Golgi cell instantaneous rate profile or PSTH sampled from Poisson spike-generator (25 repetitions) \\\hline
\end{tabularx}
\vspace{1ex}
%\end{ntab}

% - B ------------------------------------------------------------------------
\noindent%
\begin{tabularx}{\textwidth}{|l|X|X|}\hline%
\hdr{3}{B}{Populations}\\\hline
\textbf{Name} &                         \textbf{Elements}                          & \textbf{Number} \\\hline
     HSR      & Auditory nerve fibre \citep{ZilanyBruce:2007,ZilanyBruceEtAl:2009} & N/A \\\hline
     LSR      & Auditory nerve fibre \citep{ZilanyBruce:2007,ZilanyBruceEtAl:2009} & N/A \\\hline
     GLG      &                 Instantaneous-rate Poisson neuron                  & 1 unit (CF 22.7 kHz, channel 76)  \\\hline
\end{tabularx}
\vspace{2ex}

% - C ------------------------------------------------------------------------------
\noindent
\begin{tabularx}{\textwidth}{|l|l|l|X|}\hline%
\hdr{4}{C}{Connectivity}    \\\hline
     \textbf{Name}       & \textbf{Source} & \textbf{Target} & \textbf{Pattern} \\\hline
\multirow{2}{*}{ANF$\to$GLG} &      LSR       &      GLG       & Gaussian spatial spread centered on CF. Fixed variance $\sHSRGLG=2$. \\
                         &      HSR       &      GLG       & Gaussian spatial spread centered on CF. LSR variance, \sLSRGLG, to be optimised.\\\hline
\end{tabularx}
\vspace{1ex}

% - D ------------------------------------------------------------------------------
\noindent%
\begin{tabularx}{\linewidth}{|p{0.1\linewidth}|X|}\hline
\hdr{2}{D}{Neuron and Synapse Model}\\\hline
 \textbf{Name} & Golgi cell \\\hline
 \textbf{Type} & Instantaneous-rate Poisson generator with refractory effects, derived from AN model inputs \\\hline 
%\raisebox{-4.5ex}{\parbox{\linewidth}{\textbf{Model Dynamics}}} & 
 \textbf{Model Dynamics} & \rule{1em}{0em}\vspace*{-3.5ex}\begin{equation*}
      \begin{array}{r@{\;=\;}ll}
  \mathbf{w}_{L,H}   &                   w_{LSR,HSR \to GLG} \cdot \mathcal{N}(i,\sigma),                    & \sigma^2 = \sLSRGLG, i=\text{channel position} \\ 
     \alpha(t)       &                     \left( t \cdot \exp(\frac{-t}{\Gtau}) \right)                     & \text{synapto-dendritic filter}\dag \\
        g(t)         & r\left(\mathbf{w}_{L}\bullet\mathbf{L}+\mathbf{w}_{H}\bullet\mathbf{H}\right) & \text{matrix algebra and rectifying function}\\ %\mathbf{H},\mathbf{L} \to f(\text{channel},t) & \mathbf{w}_{H,L} \to f(\text{channel})\\
%      r(\mathbf{x}) &                   \max\{\mathbf{x}(t-\dANFGLG) - x(0) + \Gspon,0\}                    & \\
        G(t)         &                                 \alpha(t)\,\ast\,g(t)                                 &                      \text{convolution of $\alpha(t)$ and $g(t)$}\\%                       & \text{if } G(t) < 0 & G(t)=0 \\
\end{array}
  \end{equation*}
\vspace*{-2.5ex}\rule{1em}{0em} 
 \\\hline
 \textbf{Spiking} & Renewal process with refractory effects  \citep{ZilanyBruce:2007,Jackson:2003} \\\hline
\end{tabularx}
\vspace{2ex} 
$\dag$\footnotesize{Synaptic filter is normalised, by setting the
  area under the alpha function to one. For a large enough filter length, the
  alpha function integral ($\int \alpha(t) dt = (-\Gtau^2 - t \cdot \Gtau)\cdot
  \exp(-\frac{t}{\Gtau})$) approximately equals $\Gtau^2$. In this case $10
  \times \Gtau$ is used for the filter length.}  
}

% - E -----------------------------------------------------------------------------
% \noindent\begin{tabularx}{\linewidth}{|l|X|}\hline %
% \hdr{2}{\ref{tab:GolgiCellModelSummary} E}{Input\slash Output}\\\hline 
% \textbf{Input Stimulus} & Rate Level function, 21~kHz tone at SPL -15 to 85 dB (20 ms delay, 2ms cosine squared on\slash off ramp)\\\hline 
%  \textbf{Measurements}  & Mean rate of instantaneous rate profile or PSTH sampled from Poisson spike-generator (25 repetitions). \\\hline
% \end{tabularx}
% \vspace{1ex}


% \begin{ntab}{4}{|X|c|c|c|}{E}{Optimisation NTAB}
% \textbf{Parameters}             &    \textbf{Name}     & \textbf{Range} & \textbf{Best Values} \\\hline 
%  Spatial spread $\ANFGLG$ (channel unit)   &      $\sANFGLG$      &     [0,10]     & 2.48  \\\hline 
%  Synaptodendritic filter time constant (ms)     &   $\tau_{\ANFGLG}$     &     [0,20]       & 5.01  \\\hline 
%       Weighted sum of HSR (unitless)       &      $\wHSRGLG$      &     [0,5]      & 0.517 \\\hline 
%       Weighted sum of LSR (unitless)       &      $\wLSRGLG$      &     [0,5]      & 0.0487\\\hline 
% Golgi spontaneous rate (spikes per second) & \texttt{golgi\_spon} &     [0,50]     & 3.73  \\\hline
% \end{ntab}

%%% Local Variables: 
%%% mode: latex
%%% TeX-master: "SimpleResponses"
%%% TeX-PDF-mode: nil
%%% End: 
 
\yellownote{complete explanation of tables }
\clearpage
% Due to its replication of granule cells in the model, weight for LSR
% (\wLSRGLG) and HSR (\wHSRGLG) are determined for all synapses, number \nLSRDS
% and \nHSRDS, delay \dANFGLG added to smoothing function to ensure conductance
% and dendritic filtering are included.

% \subsubsection{Key design factors}

% \yellownote{TODO: expand para, include fig ref} Choosing neural model: HH-type
% or Poisson - Problem of monotonic excitation at low levels - Spread of ANF to
% GCD ARE broader than core VCN - are we spoiling the broth too early?


% \includegraphics[width=0.6\textwidth,angle=-90]{GolgiRateLevelActualFit}\\
% \caption{Optimisation Results for Golgi Model using Rate Level data from
% }\label{Ch3:fig:GolgiFit}
%   \includegraphics[width=0.8\textwidth]{GolgiRateLevel}\\
%   \caption{Optimisation Results for Golgi Model using Rate Level data from
% }\label{Ch3:fig:GolgiRL}

%   \includegraphics[width=0.8\textwidth]{golgi_RateLevel_opt}\\
%   \caption{Optimisation Results for Golgi Model using Rate Level data from
% }\label{Ch3:fig:GolgiRL}
%   \includegraphics[width=0.8\textwidth,angle\todo=-90]{GolgiRateLevel2}\\
%   \caption{Optimisation Results for Golgi Model using Rate Level data from
% }\label{Ch3:fig:GolgiRL}


 \subsection{Results}

\todo{ Para on initial tests}

\todo{ Para on main results }

\begin{figure}[htb]
  \centering
  % \resizebox{3.5in}{!}{\includegraphics{NoFigure}} \\
  \resizebox{0.8\textwidth}{!}{\includegraphics{./gfx/GolgiRateLevel_result2.eps}}\\
  \caption{Initial trial results of the Golgi cell model optimisation.
    Responses of the Golgi cell model (blue triangles) compared five five sound level (0,
    15, 55, 75 and 85 dB SPL) 
    against 5 point in the target response (red squares). The eventual
    best optimisation response obtained a minimum error of 11.63
    spikes/s (root mean squared). A spike response
    (green circles) was generated from the spiking output of the Golgi
    cell model using the final parameters. 
  }\label{fig:GolgiTestResult}
\end{figure}


{ %% Result table
\noindent\begin{tabularx}{\linewidth}{|X|c|c|c|}\hline %{\textwidth}
\hdr{4}{\ref{tab:GolgiCellModelSummary} E}{Optimisation} \\ \hline 
           \textbf{Parameters}             &    \textbf{Name}     & \textbf{Range} & \textbf{Best Values} \\\hline 
 Spatial spread $\ANFGLG$ (channel unit)   &      $\sANFGLG$      &     [0,10]     & 2.48  \\\hline 
 Synaptodendritic filter time constant (ms)&   $\tau_{\ANFGLG}$     &     [0,20]       & 5.01  \\\hline 
      Weighted sum of HSR (unitless)       &      $\wHSRGLG$      &     [0,5]      & 0.517 \\\hline 
      Weighted sum of LSR (unitless)       &      $\wLSRGLG$      &     [0,5]      & 0.0487\\\hline 
Spontaneous rate in Golgi cell model (spikes per second) & \Gspon &     [0,50]     & 3.73  \\\hline
\end{tabularx}
}

\begin{figure}[htb]
  \centering
  % \resizebox{3.5in}{!}{\includegraphics{NoFigure}} \\
  \resizebox{\textwidth}{!}{\includegraphics{./gfx/GolgiRateLevel_result.eps}} \\
  % \hspace{1cm}\figfont{A}\hfill\\
  % \resizebox{\textwidth}{!}{\includegraphics{./gfx/GolgiRateLevel_result2.eps}} \\
  % \hspace{1cm}\figfont{B}\hfill \\
  % \resizebox{\textwidth}{!}{\includegraphics{./gfx/GolgiRateLevel_result.eps}} \\
  \caption{Golgi cell model optimisation result trials against unit
    S03-07 (CF 21~kHz) from \citet{GhoshalKim:1996}.  A more detailed
    optimisation with 22 levels and included HSR inputs in the Golgi
    cell model generated a closer fit to the Ghoshal and Kim data.
    The final root mean squared error was 4.48 spikes/s.
  }\label{fig:GolgiResult}
\end{figure}

Figure~\ref{fig:GolgiTestResult} shows the output of the test optimisation trials
for the Golgi cell model.  The testing trial used only five sound levels (0, 15,
55, 75 and 85 dB SPL) and detected the mean rate from the instantaneous profile
in its fitting routine. The best response obtained a minimum error of 11.63
spikes/s (root mean squared error) against the five points in the target
experimental data of unit S03-07 (CF 21~kHz) from \citep{GhoshalKim:1996}. A
rate-level curve (green circles, Fig.~\ref{fig:GolgiResult}A) was generated from the spiking output only to
show a big discrepancy in the spike-based rate-level and the monotonic
rate-based rate-level. The lack of low level response and a higher threshold
indicated the need for some HSR input into the Golgi cell model.  
%\medskip{}

Figure~\ref{fig:GolgiResult} shows the more detailed (and final) optimisation routine with
22 levels and an enhanced Golgi cell model was used to generate a closer fit to
the Ghoshal and Kim data ().  The final error was 4.48 spikes/s, which is a 60\%
reduction from the initial optimisation.
\yellownote{We are satisfied with the Golgi cell response, now show a plot of
  simple responses of a generic cell in the centre of the network.}

%   % \includegraphics[width=0.6\textwidth,angle=-90]{GolgiRateLevelActualFit}\\
%   % \caption{Optimisation Results for Golgi Model using Rate Level data from
%   % }\label{Ch3:fig:GolgiFit}
%   %   \includegraphics[width=0.8\textwidth]{GolgiRateLevel}\\
%   %   \caption{Optimisation Results for Golgi Model using Rate Level data from
%   % }\label{Ch3:fig:GolgiRL}

%   %   \includegraphics[width=0.8\textwidth]{golgi_RateLevel_opt}\\
%   %   \caption{Optimisation Results for Golgi Model using Rate Level data from
%   % }\label{Ch3:fig:GolgiRL}
%   %   \includegraphics[width=0.8\textwidth,angle=-90]{GolgiRateLevel2}\\
%     %   \caption{Optimisation Results for Golgi Model using Rate Level data
%     %   from }\label{Ch3:fig:GolgiRL}





%   \begin{figure}[htb]
%     \centering
%     \includegraphics[width=0.6\textwidth,angle=-90]{GolgiRateLevelActualFit}\\
%     \caption{Optimisation Results for Golgi Model using Rate Level data from
%   }\label{Ch3:fig:GolgiFit}
%   \end{figure}

%   \begin{figure}[htb]
%     \centering
%     \includegraphics[width=0.8\textwidth]{GolgiRateLevel}\\
%     \caption{Optimisation Results for Golgi Model using Rate Level data from
%   }\label{Ch3:fig:GolgiRL}
%   \end{figure}

%   \begin{figure}[htb]
%     \centering
%     \includegraphics[width=0.8\textwidth]{golgi_RateLevel_opt}\\
%     \caption{Optimisation Results for Golgi Model using Rate Level data from
%   }\label{Ch3:fig:GolgiRL}
%   \end{figure}

%   \begin{figure}[htb]
%     \centering
%     \includegraphics[width=0.8\textwidth,angle=-90]{GolgiRateLevel2}\\
%     \caption{Optimisation Results for Golgi Model using Rate Level data from
%   }\label{Ch3:fig:GolgiRL}
%   \end{figure}





%   \clearpage \newpage
   \subsection{Verification of Golgi cell model}
%   \subsubsection{Tone Responses}




\begin{figure}[h]
     \centering
     \resizebox{0.95\textwidth}{!}{\includegraphics{\includegraphics{./ResponsesNoCompression/G_ratelevel.eps}\includegraphics{./ResponsesNoCompression/RateLevel/response_area.3.eps}\includegraphics{./ResponsesNoCompression/RateLevel/psthsingle90.3.eps}}
\resizebox{0.95\textwidth}{!}{\includegraphics{./ResponsesNoCompression/MaskedResponseCurve/G_ratelevel.eps} \includegraphics{./ResponsesNoCompression/MaskedResponseCurve/psthsingle5810.3.eps}}
\label{fig:Golgi_verification}\caption{Response of optimised Golgi cell model at the centre of the network (CF=5.8~kHz). }
 \end{figure}

%\resizebox{0.95\textwidth}{!}{\includegraphics{./ResponsesNoCompression/MaskedResponseCurve3/G_masked.eps}\includegraphics{./ResponsesNoCompression/NoiseRateLevel/G_ratelevel.eps}}


%   \begin{figure}[h]
%     \centering\resizebox{0.95\textwidth}{!}{%
%     \includegraphics{RateLevel/response_area.3.eps}%
%     \includegraphics{RateLevel/response_area_log2.3.eps}}
%   \end{figure}
%   \begin{figure}[h]
%     \centering\resizebox{0.95\textwidth}{!}{%
%     %     \includegraphics{RateLevel/response_area.3.eps}
%     \includegraphics{RateLevel/psthall90.3.eps}%
%     \includegraphics{RateLevel/psthVlevel.3.eps}}
%   \end{figure}



%   \clearpage
%   \subsubsection{Noise Responses}
%   \begin{figure}[h]
%     \centering\resizebox{0.95\textwidth}{!}{%
%     \includegraphics{NoiseRateLevel/psthsingle120.3.eps}%
%     \includegraphics{NoiseRateLevel/G_ratelevel.eps}}
%   \end{figure}
%   \begin{figure}[h]
%     \centering\resizebox{0.95\textwidth}{!}{%
%     \includegraphics{NoiseRateLevel/response_area.3.eps}%
%     \includegraphics{NoiseRateLevel/response_area_log2.3.eps}}
%   \end{figure}
%   \begin{figure}[h]
%     \centering\resizebox{0.95\textwidth}{!}{%
%     %     \includegraphics{RateLevel/response_area.3.eps}
%     \includegraphics{NoiseRateLevel/psthall90.3.eps}%
%     \includegraphics{NoiseRateLevel/psthVlevel.3.eps}}
%   \end{figure}


%   \clearpage
%   \subsubsection{Masking Responses}
%   \begin{figure}[h!]
%     \centering\resizebox{0.95\textwidth}{!}{\includegraphics{MaskedRateLevel/psthsingle90.3.eps}\includegraphics{MaskedRateLevel/G_ratelevel.eps}}
%   \end{figure}
%   \begin{figure}[h!]
%     \centering\resizebox{0.95\textwidth}{!}{%
%     \includegraphics{MaskedRateLevel/response_area.3.eps}%
%     \includegraphics{MaskedRateLevel/response_area_log2.3.eps}}
%   \end{figure}

%   \begin{figure}[h!]
%     \centering\resizebox{0.95\textwidth}{!}{%
%     %     \includegraphics{RateLevel/response_area.3.eps}
%     \includegraphics{MaskedRateLevel/psthall90.3.eps}%
%     \includegraphics{MaskedRateLevel/psthVlevel.3.eps}}
%   \end{figure}
%   \clearpage

%   \begin{figure}[h!]
%     \centering\resizebox{0.95\textwidth}{!}{%
%     \includegraphics{MaskedResponseCurve/psthsingle5810.3.eps}%
%     \includegraphics{MaskedResponseCurve/G_masked.eps}}
%   \end{figure}
%   \begin{figure}[h!]
%     \centering\resizebox{0.95\textwidth}{!}{%
%     \includegraphics{MaskedResponseCurve/response_area.3.eps}%
%     \includegraphics{MaskedResponseCurve/response_area_log2log2.3.eps}}
%   \end{figure}

%   \begin{figure}[h!]
%     \centering\resizebox{0.95\textwidth}{!}{%
%     %     \includegraphics{RateLevel/response_area.3.eps}
%     \includegraphics{MaskedResponseCurve/psthall5810.3.eps}%
%     \includegraphics{MaskedResponseCurve/psthVmod.3.eps}}
%   \end{figure}
%   \clearpage


%%% Local Variables: 
%%% mode: latex
%%% TeX-master: "SimpleResponses"
%%% TeX-PDF-mode: nil
%%% End: 
