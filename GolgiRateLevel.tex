
%%%%%%%%%%%%%%%%%%%%%%%%%%%%%%%%%%%%%%%%%%%%%
\graphicspath{{../figures/}{./gfx/}{/media/data/Work/cnstellate/golgi/}{/media/data/Work/Resposnes/}{/media/data/Work/cnstellate/Responses/}}
%%%%%%%%%%%%%%%%%%%%%%%%%%%%%%%%%%%%%%%%%%%%%

\section[Golgi Cell Model]{Golgi Cell Model: Optimisation using
  monotonic rate-level responses in marginal shell units}
\label{sec:GolgiCellModel}

\subsection{Background}

The presence of GABAergic inputs to TS, DS and TV cells has been verified by
labeled terminals adjacent to the soma and dendrites
\citep{SmithRhode:1989,AwatramaniTurecekEtAl:2005,BabalianRyugoEtAl:2003} and
release from inhibition in their response areas with ionotopopheretic
application of the \GABAa antagonist, bicuculine
\citep{EvansZhao:1998,CasparyBackoffEtAl:1994,BackoffShadduckEtAl:1999,FerragamoGoldingEtAl:1998a}. The
source of GABAergic inputs to cells in the mammalian CN is somewhat
contentious. Studies show that GABAergic inputs to the CN generally arise in the
peri-olivary regions of the medulla in cats \citep{OstapoffBensonEtAl:1997} and
birds \citep{LachicaRubsamenEtAl:1995,YangMonsivaisEtAl:1999}. Slice
preparations of the isolated murine VCN show strong and immediate sensitivity to
bicuculine in TS and DS cells from a source within the CN complex
\citep{FerragamoGoldingEtAl:1998a}.  The only known source of GABA intrinsic to
the VCN are the golgi cells of the granule cell domain (GCD) overlying the VCN
\citep[Fig.~\ref{fig:CNdiagram}]{Mugnaini:1985,FerragamoGoldingEtAl:1998}.

\medskip{}

% \yellownote{TODO: Clean up paragraph}
% Other studies in the rat cochlear nucleus relating to the Golgi cell or GABA:
% \begin{itemize}
% \item \citep{MugnainiOsenEtAl:1980} Fine structure of granule cells
%   and related inter-neurons (termed {Golgi} cells) in the cochlear
%   nuclear complex of cat, rat and mouse
% \item \GABAa expression in the rat brainstem  \citep{CamposCaboEtAl:2001}
% \item \citep{Alibardi:2003a} Ultrastructural distribution of
%   glycinergic and {{GABAergic}} neurons and axon terminals in the rat
%   dorsal cochlear nucleus, with emphasis on granule cell areas
% \item \citep{AwatramaniTurecekEtAl:2005} Staggered {Development} of
%   {GABAergic} and {Glycinergic} {Transmission} in the {MNTB}
% \end{itemize}

% \medskip{}

% \yellownote{TODO: Expand role of GABA, or combine with previous para}
% Role of GABA in the VCN
% \begin{itemize}
% \item Effects of microiontophoretically applied glycine and {GABA} on
%   neuronal response patterns in the cochlear nuclei
%   \citep{CasparyHaveyEtAl:1979}
% \end{itemize}
% \citep{Alibardi:2003a} rat CN complex -> Golgi-stellate cells
% (fusiform layer: 2) in DCN contact granule and unipolar brush cells

\medskip{}

Inputs to golgi cells are more complicated than VCN cells in the core
regions. Golgi cells are sparse in the GCD, surrounded by the many,
smaller excitatory granule cells, that form small en passant
endings. Type II ANFs create diffuse glutamatergic release sites in
the GCD \citep{HurdHutsonEtAl:1999,BensonBrown:2004} that may
stimulate NMDA glutamate receptors in golgi cells
\citep{FerragamoGoldingEtAl:1998a}.

\medskip{}

Extracellular recordings from labelled golgi cells is not available in the
literature.  The GCD (or marginal shell in cats) has been studied by one group
\citet{GhoshalKim:1997}, and any extracellular spikes are most likely from golgi
cells since granule cell somata are less than $10{}\mu{m}$ and very narrow
axons. The majority of recorded units showed a monotonic increase in firing rate
with increasing sound intensity \citep[Figure~\ref{fig:GolgiKimFig2}][]{GhoshalKim:1997}.  The general assumption
of the functional role of golgi cells is to regulate granule cells but they may
also provide automatic gain control to the principle VCN units, through their
monotonic responses to tones and noise.

\medskip{}

The known GABAergic input to VCN units comes primarily from the superior olive
(ref), but the presence of active GABA synapses in isolated VCN slices by
\citet{FerragamoGoldingEtAl:1998} led to further investigation of golgi cells in
the granule cell domain. Intracellular recordings of golgi cells have a classic
type-I current response, which suggest they are simple integrators, and their
response to AN shocks were delayed by approximately 0.7~ms relative to the core
VCN units \citep{FerragamoGoldingEtAl:1998}.

\medskip{}

% (Reference) showed that in adult animals high-spontaneous rate ANFs do not
% project to the GCD; they do show that low-spontaneous rate ANFs do project into
% the GCD, albeit more profusely than in the core on the VCN.
   
\yellownote{Inclusion of Ghoshal figure needs permission, fill in caption}
\begin{figure}[ht!]
  \centering
%\resizebox{3.5in}{!}{\includegraphics{NoFigure}}
 \resizebox{\textwidth}{!}{\includegraphics{GhoshalKim}}
\caption{Rate level response of 6 units \citep[from~Fig.~2]{GhoshalKim:1997}. %S03-07 (CF 22.7~kHz) 
}\label{fig:GolgiKimFig2}
\end{figure}

\subsection{Implementation}

In Chapter~\ref{sec:GAChapter} and previous publications
\citep{EagerGraydenEtAl:2006a}, the golgi cell model was implemented
as a single-compartment conductance neuron. Due to the unavailability
of sufficient data regarding \emph{in vivo} golgi cell responses, we
have decided to simulate the response of golgi cells using inputs from
the auditory model's instantaneous rate outputs rather than simulating
the neural membrane with Hodgkin-Huxley models.  A number of steps
were taken to investigate the golgi cell model. A further detailed
explanation of the implementation is in the chapter
Appendix~\ref{sec:chp3appendix}.
\medskip{}

In the creation of the golgi cell model, we can reduce the explicit responses of
Golgi cells down to three major details: a) golgi cells are integrators due to
their type-I~current clamp response \citep{FerragamoGoldingEtAl:1998}, b) golgi
cells are (most likely) monotonic to tone and noise rate increases
\citep{GhoshalKim:1997}, and c) they have a significant delay of first spike
latency relative to the core VCN units \citep{GhoshalKim:1997}. The lack of
extensive experimental data regarding type-II ANF units and granule cell
response to acoustic input meant that a Poisson rate neural model would be
preferred over the Hodgkin-Huxley type neural model.

\medskip{}

The golgi cell model is implemented as an instantaneous-rate Poisson rate model,
shown in Figure~\ref{fig:GolgiDiagram}. Connections across frequency channels of
HSR and LSR ANF responses to golgi cells were determined by $\mu(f)$ and
$\sigma$ variables, which control the Gaussian distribution in units of channel
separation in the network. The weighted sum of HSR and LSR instantaneous-rate
vectors are smoothed out by $\alpha$ is the synaptic and dendritic smoothing
function, then corrected for spontaneous activity.  Although HSR ANF terminals do not generally project into the GCD, they are necessary to provide some level of low level sound activity.

 \begin{figure}[h!]
   \centering
%   \resizebox{3.5in}{!}{\includegraphics{NoFigure}}\\
  \resizebox{0.9\textwidth}{!}{\includegraphics{gfx/GolgiDiagram.eps}}\\
  \caption{The golgi instantaneous-rate profile was generated using a weighted sum ANF
     profiles and a alpha function smoothing filter to mimic dendritic and synaptic filtering. The Gaussian spread of connections is independent for HSR and LSR auditory filters, with the mean equal to CF channel of unit. The alpha function includes a delay of 2.5 ms, 0.7 ms greater than the core VCN units as shown by \citet{GhoshalKim:1997}. The final stage sets the spontaneous rate by addition at t=0 and changes any negative values to zero in the rate profile.  
%     across frequency channels is Gaussian, and $\mathbf{w}$ is
%    the weighted sum of HSR and LSR instantaneous-rate vectors,
%     $\alpha$ is the synaptic and dendritic smoothing function.
}\label{fig:GolgiDiagram}
 \end{figure}

\medskip{}

Monotonic rate-level data from GCD in VCN \citep{GhoshalKim:1996} unit
S03-07 (CF 21~kHz) was used to optimise parameters ${\rm
golgi\_spon}$, \wLSRGLG, \wHSRGLG, and \sANFGLG\@.  The table~\ref{tab:GolgiCellModelSummary}
shows the model details to be used by the optimisation process. 

{\small%\linespread{0.5}
  \begin{table}[htb]
    \caption{Golgi cell model summary (Nordlie format)}
    \label{tab:GolgiCellModelSummary}
  \end{table}
\noindent%
\begin{tabularx}{\textwidth}{|l|X|}\hline %
\hdr{2}{A}{Model Summary}\\\hline 
%\begin{ntab}{|l|X|}{2}{\ref{tab:GolgiCellModelSummary} A}{Model Summary}\\\hline
 \textbf{Populations}  & ANF~(HSR, LSR) and Golgi cells \\\hline 
  \textbf{Topology}    & Tonotopic - $N_{\text{channel}}=100$ frequency channels (0.2--40 kHz) separated evenly based on even place positions on the basilar membrane \citep{Greenwood:1990}\\\hline
\textbf{Connectivity}  & Place-based Gaussian spread of connections from ANF to GLG \\\hline
\textbf{Input model}  & ANF~model: instantaneous-rate Poisson model \citep{ZilanyBruce:2007} \\\hline
\textbf{Neuron model}  & Golgi cell model: instantaneous-rate Poisson model developed from ANF inputs \\\hline
\textbf{Synapse model} & Synapto-dendritic smoothing filter (alpha function) \\\hline
    \textbf{Input}     & Pure tones (22.7 kHz, 50 ms, 5 ms on/off ramp, 20 ms delay), intensity range 0--100 dB~SPL   \\\hline
\textbf{Measurements}  & Mean firing rate of Golgi cell instantaneous rate profile or PSTH sampled from Poisson spike-generator (25 repetitions) \\\hline
\end{tabularx}
\vspace{1ex}
%\end{ntab}

% - B ------------------------------------------------------------------------
\noindent%
\begin{tabularx}{\textwidth}{|l|X|X|}\hline%
\hdr{3}{B}{Populations}\\\hline
\textbf{Name} &                         \textbf{Elements}                          & \textbf{Number} \\\hline
     HSR      & Auditory nerve fibre \citep{ZilanyBruce:2007,ZilanyBruceEtAl:2009} & N/A \\\hline
     LSR      & Auditory nerve fibre \citep{ZilanyBruce:2007,ZilanyBruceEtAl:2009} & N/A \\\hline
     GLG      &                 Instantaneous-rate Poisson neuron                  & 1 unit (CF 22.7 kHz, channel 76)  \\\hline
\end{tabularx}
\vspace{2ex}

% - C ------------------------------------------------------------------------------
\noindent
\begin{tabularx}{\textwidth}{|l|l|l|X|}\hline%
\hdr{4}{C}{Connectivity}    \\\hline
     \textbf{Name}       & \textbf{Source} & \textbf{Target} & \textbf{Pattern} \\\hline
\multirow{2}{*}{ANF$\to$GLG} &      LSR       &      GLG       & Gaussian spatial spread centered on CF. Fixed variance $\sHSRGLG=2$. \\
                         &      HSR       &      GLG       & Gaussian spatial spread centered on CF. LSR variance, \sLSRGLG, to be optimised.\\\hline
\end{tabularx}
\vspace{1ex}

% - D ------------------------------------------------------------------------------
\noindent%
\begin{tabularx}{\linewidth}{|p{0.1\linewidth}|X|}\hline
\hdr{2}{D}{Neuron and Synapse Model}\\\hline
 \textbf{Name} & Golgi cell \\\hline
 \textbf{Type} & Instantaneous-rate Poisson generator with refractory effects, derived from AN model inputs \\\hline 
%\raisebox{-4.5ex}{\parbox{\linewidth}{\textbf{Model Dynamics}}} & 
 \textbf{Model Dynamics} & \rule{1em}{0em}\vspace*{-3.5ex}\begin{equation*}
      \begin{array}{r@{\;=\;}ll}
  \mathbf{w}_{L,H}   &                   w_{LSR,HSR \to GLG} \cdot \mathcal{N}(i,\sigma),                    & \sigma^2 = \sLSRGLG, i=\text{channel position} \\ 
     \alpha(t)       &                     \left( t \cdot \exp(\frac{-t}{\Gtau}) \right)                     & \text{synapto-dendritic filter}\dag \\
        g(t)         & r\left(\mathbf{w}_{L}\bullet\mathbf{L}+\mathbf{w}_{H}\bullet\mathbf{H}\right) & \text{matrix algebra and rectifying function}\\ %\mathbf{H},\mathbf{L} \to f(\text{channel},t) & \mathbf{w}_{H,L} \to f(\text{channel})\\
%      r(\mathbf{x}) &                   \max\{\mathbf{x}(t-\dANFGLG) - x(0) + \Gspon,0\}                    & \\
        G(t)         &                                 \alpha(t)\,\ast\,g(t)                                 &                      \text{convolution of $\alpha(t)$ and $g(t)$}\\%                       & \text{if } G(t) < 0 & G(t)=0 \\
\end{array}
  \end{equation*}
\vspace*{-2.5ex}\rule{1em}{0em} 
 \\\hline
 \textbf{Spiking} & Renewal process with refractory effects  \citep{ZilanyBruce:2007,Jackson:2003} \\\hline
\end{tabularx}
\vspace{2ex} 
$\dag$\footnotesize{Synaptic filter is normalised, by setting the
  area under the alpha function to one. For a large enough filter length, the
  alpha function integral ($\int \alpha(t) dt = (-\Gtau^2 - t \cdot \Gtau)\cdot
  \exp(-\frac{t}{\Gtau})$) approximately equals $\Gtau^2$. In this case $10
  \times \Gtau$ is used for the filter length.}  
}

% - E -----------------------------------------------------------------------------
% \noindent\begin{tabularx}{\linewidth}{|l|X|}\hline %
% \hdr{2}{\ref{tab:GolgiCellModelSummary} E}{Input\slash Output}\\\hline 
% \textbf{Input Stimulus} & Rate Level function, 21~kHz tone at SPL -15 to 85 dB (20 ms delay, 2ms cosine squared on\slash off ramp)\\\hline 
%  \textbf{Measurements}  & Mean rate of instantaneous rate profile or PSTH sampled from Poisson spike-generator (25 repetitions). \\\hline
% \end{tabularx}
% \vspace{1ex}


% \begin{ntab}{4}{|X|c|c|c|}{E}{Optimisation NTAB}
% \textbf{Parameters}             &    \textbf{Name}     & \textbf{Range} & \textbf{Best Values} \\\hline 
%  Spatial spread $\ANFGLG$ (channel unit)   &      $\sANFGLG$      &     [0,10]     & 2.48  \\\hline 
%  Synaptodendritic filter time constant (ms)     &   $\tau_{\ANFGLG}$     &     [0,20]       & 5.01  \\\hline 
%       Weighted sum of HSR (unitless)       &      $\wHSRGLG$      &     [0,5]      & 0.517 \\\hline 
%       Weighted sum of LSR (unitless)       &      $\wLSRGLG$      &     [0,5]      & 0.0487\\\hline 
% Golgi spontaneous rate (spikes per second) & \texttt{golgi\_spon} &     [0,50]     & 3.73  \\\hline
% \end{ntab}

%%% Local Variables: 
%%% mode: latex
%%% TeX-master: "SimpleResponses"
%%% TeX-PDF-mode: nil
%%% End: 


% Due to its replication of granule cells in the model, weight for LSR
% (\wLSRGLG) and HSR (\wHSRGLG) are determined for all synapses, number
% \nLSRDS and \nHSRDS, delay \dANFGLG added to smoothing function to
% ensure conductance and dendritic filtering are included.

% \subsubsection{Key design factors}

% \yellownote{TODO: expand para, include fig ref}
% Choosing neural model: HH-type or Poisson
%  - Problem of monotonic excitation at low levels
%  - Spread of ANF to GCD ARE broader than core VCN
%   - are we spoiling the broth too early? 


% \includegraphics[width=0.6\textwidth,angle=-90]{GolgiRateLevelActualFit}\\
% \caption{Optimisation Results for Golgi Model using Rate Level data
%   from }\label{Ch3:fig:GolgiFit}
% \includegraphics[width=0.8\textwidth]{GolgiRateLevel}\\
% \caption{Optimisation Results for Golgi Model using Rate Level data
%   from }\label{Ch3:fig:GolgiRL}

% \includegraphics[width=0.8\textwidth]{golgi_RateLevel_opt}\\
% \caption{Optimisation Results for Golgi Model using Rate Level data
%   from }\label{Ch3:fig:GolgiRL}
% \includegraphics[width=0.8\textwidth,angle\todo=-90]{GolgiRateLevel2}\\
% \caption{Optimisation Results for Golgi Model using Rate Level data
%   from }\label{Ch3:fig:GolgiRL}


\subsection{Results}



\begin{figure}[htb]
  \centering 
%  \resizebox{3.5in}{!}{\includegraphics{NoFigure}} \\
\hspace{2pt}\figfont{A}\hspace{0.5\textwidth}\figfont{B}\hfill\\
\resizebox{\textwidth}{!}{\includegraphics{./gfx/GolgiRateLevel_result2.eps}\hspace{1cm}\includegraphics{./gfx/GolgiRateLevel_result.eps}} \\
% \hspace{1cm}\figfont{A}\hfill\\
% \resizebox{\textwidth}{!}{\includegraphics{./gfx/GolgiRateLevel_result2.eps}} \\
% \hspace{1cm}\figfont{B}\hfill \\
% \resizebox{\textwidth}{!}{\includegraphics{./gfx/GolgiRateLevel_result.eps}} \\
\caption{Golgi cell model optimisation result trials against unit S03-07 (CF 21~kHz) from \citep{GhoshalKim:1996}. A. The initial optimisation used only five sound levels (0, 15, 55, 75 and 85 dB SPL) and rate-model in its routine. The eventual best optimisation response (red squares) obtained a minimum error of 11.63 spikes/s (root mean squared) against the target response (blue). A rate-level curve (green circles) was generated from the spiking output only to show a big discrepancy in the spike-based rate-level and the monotonic rate-based rate-level. The lack of low level response indicated the need for some HSR input into the golgi cell model. B. A more detailed optimisation with 22 levels and included HSR inputs in the golgi cell model generated a closer fit to the Ghoshal and Kim data.  The final root mean squared error was 4.48 spikes/s.  }\label{fig:GolgiResult}
\end{figure}



Figure~\ref{fig:GolgiResult} shows the output of the optimisation trials for the
golgi cell model.  The final result produced a root mean squared error of 0.021
spikes per second (normalised to maximum rate of the fitting data). Golgi model
and spike based output (Pink) was used to fit the experimental data of unit
S03-07 (CF 21~kHz) from \citep{GhoshalKim:1996} (Red).  LSR mean rate (Blue) of
21~kHz unit is monotonic with a high threshold.  The best parameters, in
table~\ref{tab:GolgiCellModelSummary}E, show the effect of HSR inputs is minimal
with the dominant input to golgi cells coming from LSR fibres.


Golgi cell model optimisation result trials against unit S03-07 (CF 21~kHz) from
\citep{GhoshalKim:1996}. A. The initial optimisation used only five sound levels
(0, 15, 55, 75 and 85 dB SPL) and rate-model in its routine. The eventual best
optimisation response (red squares) obtained a minimum error of 11.63 spikes/s
(root mean squared) against the target response (blue). A rate-level curve
(green circles) was generated from the spiking output only to show a big
discrepancy in the spike-based rate-level and the monotonic rate-based
rate-level. The lack of low level response indicated the need for some HSR input
into the golgi cell model. B. A more detailed optimisation with 22 levels and
included HSR inputs in the golgi cell model generated a closer fit to the
Ghoshal and Kim data.  The final root mean squared error was 4.48 spikes/s.


%\yellownote{TODO: expand results output}



% \begin{figure}[htb]
%   \centering %\turnbox{90}{\small{Rate (sp/s)}}
% %  \resizebox{3.5in}{!}{\includegraphics{NoFigure}} \\
%   \resizebox{\textwidth}{!}{\includegraphics[angle=-90]{./gfx/GolgiRateLevel_result.eps}} \\
%   \caption{Golgi cell model optimisation results with the best response obtaining a minimum error 0.021 (spikes/s, average mean squared error of experimental and data points). }\label{fig:GolgiResult}
% \end{figure}


%   % \includegraphics[width=0.6\textwidth,angle=-90]{GolgiRateLevelActualFit}\\
%   % \caption{Optimisation Results for Golgi Model using  Rate Level data from }\label{Ch3:fig:GolgiFit}
%   % \includegraphics[width=0.8\textwidth]{GolgiRateLevel}\\
%   % \caption{Optimisation Results for Golgi Model using  Rate Level data from }\label{Ch3:fig:GolgiRL}

%   % \includegraphics[width=0.8\textwidth]{golgi_RateLevel_opt}\\
%   % \caption{Optimisation Results for Golgi Model using  Rate Level data from }\label{Ch3:fig:GolgiRL}
%   % \includegraphics[width=0.8\textwidth,angle=-90]{GolgiRateLevel2}\\
%     % \caption{Optimisation Results for Golgi Model using  Rate Level data from }\label{Ch3:fig:GolgiRL}





% \begin{figure}[htb]
% \centering
%   \includegraphics[width=0.6\textwidth,angle=-90]{GolgiRateLevelActualFit}\\
%   \caption{Optimisation Results for Golgi Model using  Rate Level data from }\label{Ch3:fig:GolgiFit}
% \end{figure}

% \begin{figure}[htb]
% \centering
%   \includegraphics[width=0.8\textwidth]{GolgiRateLevel}\\
%   \caption{Optimisation Results for Golgi Model using  Rate Level data from }\label{Ch3:fig:GolgiRL}
% \end{figure}

% \begin{figure}[htb]
% \centering
%   \includegraphics[width=0.8\textwidth]{golgi_RateLevel_opt}\\
%   \caption{Optimisation Results for Golgi Model using  Rate Level data from }\label{Ch3:fig:GolgiRL}
% \end{figure}

% \begin{figure}[htb]
% \centering
%   \includegraphics[width=0.8\textwidth,angle=-90]{GolgiRateLevel2}\\
%   \caption{Optimisation Results for Golgi Model using  Rate Level data from }\label{Ch3:fig:GolgiRL}
% \end{figure}





% \clearpage \newpage
%\subsection{Verification of Golgi cell model}
% \subsubsection{Tone Responses}

% \begin{figure}[h]
%   \centering\resizebox{0.95\textwidth}{!}{%
%     \includegraphics{RateLevel/psthsingle90.3.eps}%
%     \includegraphics{RateLevel/G_ratelevel.eps}}
% \end{figure}
% \begin{figure}[h]
%   \centering\resizebox{0.95\textwidth}{!}{%
%     \includegraphics{RateLevel/response_area.3.eps}%
%     \includegraphics{RateLevel/response_area_log2.3.eps}}
% \end{figure}
% \begin{figure}[h]
%   \centering\resizebox{0.95\textwidth}{!}{%
%     % \includegraphics{RateLevel/response_area.3.eps}
%     \includegraphics{RateLevel/psthall90.3.eps}%
%     \includegraphics{RateLevel/psthVlevel.3.eps}}
% \end{figure}



% \clearpage
% \subsubsection{Noise Responses}
% \begin{figure}[h]
%   \centering\resizebox{0.95\textwidth}{!}{%
%     \includegraphics{NoiseRateLevel/psthsingle120.3.eps}%
%     \includegraphics{NoiseRateLevel/G_ratelevel.eps}}
% \end{figure}
% \begin{figure}[h]
%   \centering\resizebox{0.95\textwidth}{!}{%
%     \includegraphics{NoiseRateLevel/response_area.3.eps}%
%     \includegraphics{NoiseRateLevel/response_area_log2.3.eps}}
% \end{figure}
% \begin{figure}[h]
%   \centering\resizebox{0.95\textwidth}{!}{%
%     % \includegraphics{RateLevel/response_area.3.eps}
%     \includegraphics{NoiseRateLevel/psthall90.3.eps}%
%     \includegraphics{NoiseRateLevel/psthVlevel.3.eps}}
% \end{figure}


% \clearpage
% \subsubsection{Masking Responses}
% \begin{figure}[h!]
%   \centering\resizebox{0.95\textwidth}{!}{\includegraphics{MaskedRateLevel/psthsingle90.3.eps}\includegraphics{MaskedRateLevel/G_ratelevel.eps}}
% \end{figure}
% \begin{figure}[h!]
%   \centering\resizebox{0.95\textwidth}{!}{%
%     \includegraphics{MaskedRateLevel/response_area.3.eps}%
%     \includegraphics{MaskedRateLevel/response_area_log2.3.eps}}
% \end{figure}

% \begin{figure}[h!]
%   \centering\resizebox{0.95\textwidth}{!}{%
%     % \includegraphics{RateLevel/response_area.3.eps}
%     \includegraphics{MaskedRateLevel/psthall90.3.eps}%
%     \includegraphics{MaskedRateLevel/psthVlevel.3.eps}}
% \end{figure}
% \clearpage

% \begin{figure}[h!]
%   \centering\resizebox{0.95\textwidth}{!}{%
%     \includegraphics{MaskedResponseCurve/psthsingle5810.3.eps}%
%     \includegraphics{MaskedResponseCurve/G_masked.eps}}
% \end{figure}
% \begin{figure}[h!]
%   \centering\resizebox{0.95\textwidth}{!}{%
%     \includegraphics{MaskedResponseCurve/response_area.3.eps}%
%     \includegraphics{MaskedResponseCurve/response_area_log2log2.3.eps}}
% \end{figure}

% \begin{figure}[h!]
%   \centering\resizebox{0.95\textwidth}{!}{%
%     % \includegraphics{RateLevel/response_area.3.eps}
%     \includegraphics{MaskedResponseCurve/psthall5810.3.eps}%
%     \includegraphics{MaskedResponseCurve/psthVmod.3.eps}}
% \end{figure}
% \clearpage


%%% Local Variables: 
%%% mode: latex
%%% TeX-master: "SimpleResponses"
%%% TeX-PDF-mode: nil
%%% End: 
