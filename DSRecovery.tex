
%\graphicspath{{/media/data/Work/cnstellate/DS_ClickRecovery/}{/media/data/Work/Responses/}{../figures/}{./gfx/}}
\section[DS Cell Model]{D-Stellate Cell Model: optimisation using click recovery
  responses}\label{sec:d-stellate-cell-model}

\subsection{Introduction}\label{sec:DS:introduction}

This section shows the GABAergic input and intrinsic cell properties influence
the behaviour Onset chopper units.  Onset-chopper units in the mammalian \VCN
have a wide-ranging influence on the primary cells of the \VCN, especially
stellate and bushy neurons \citep{RhodeSmithEtAl:1983}, the ipsilateral \DCN
(type II and type IV EIRA units) and the contra\-lateral CN
\citep{NeedhamPaolini:2007}.

\smallskip{}

% Large multipolar or stellate cells in the \VCN have been shown to have 3-4
% long dendrites stretching 200 microns (or one third of the \VCN) and their
% axonal collaterals cover the same region in the \VCN, almost one half of the
% \DCN, and are one source of the commissural projection to the contralateral
% cochlear nucleus \citep{NeedhamPaolini:2007}.
%%%%%%%%%%%%%%%%%%% Copied from original jneurometh article
  
D-stellate (DS) cells are large multipolar neurons in the \VCN and have an
onset-chopping (On-C) PSTH to tones and noise \citep{SmithRhode:1989,
  NeedhamPaolini:2006}.  They typically have 3-4 long dendrites stretching
200 microns (or one third of the \VCN) and their axonal collaterals cover
the same region in the \VCN, almost one half of the \DCN, and are one
source of the commissural projection to the contra-lateral cochlear nucleus
\citep{Cant:1992,Cant:1981,SchofieldCant:1996,CantBenson:2003,
  NeedhamPaolini:2007, PaoliniClark:1999}. Intracellular responses to
sounds indicate the bandwidth of inputs to DS neurons typically ranges from
two octaves below CF to one octave above CF \citep{PalmerJiangEtAl:1996,
  JiangPalmerEtAl:1996, PaoliniClark:1999}. DS cell axon terminals contain
the inhibitory neurotransmitter glycine and synapse widely in the \VCN and
\DCN\@.  They also send a commissural projection to the contralateral
cochlear nucleus that mediates fast inhibition between the nuclei
\citep{NeedhamPaolini:2003, NeedhamPaolini:2006, Oertel:1997}.

\smallskip{}

Post-onset GABAergic inhibition in DS cells is a major influence on the PSTH of
On-C neurons \citep{FerragamoGoldingEtAl:1998a,EvansZhao:1998}. Latency of
excitation to auditory nerve shocks suggests Golgi cells are activated by type
II ANFs and low spontaneous rate type I ANFs \citep{BensonBerglundEtAl:1996,
  FerragamoGoldingEtAl:1998}. Therefore, type II and LS type I ANFs could be
involved in gain control through GABAergic modulation of activity in the \VCN.

\smallskip{}

% \yellownote{Discuss AM coding by DS cells} GABA blockers in the \VCN has the
% effects of changing the behaviour in the response to AM in the IC
% \citep{CasparyPalombiEtAl:2002}.  AM coding effects of GABA in the Chinchilla
% CN \citep{BackoffShadduckEtAl:1999}.  \citep{CasparyBackoffEtAl:1994} Caspary
% and colleagues worked on the effects of GABA in in the \VCN\@.

% Zhang and Winter looked at the response area of \VCN onset units to determine
% GABA on/off freq.

% Smith and Rhode, Smith and others looked at OnC response area and two-tone


\subsection{Implementation}\label{sec:DS:implementation}
 

Key elements in designing D-stellate cell model are
shown in the Nordlie table~\ref{tab:DScellModelSummary}A. A type I-II single
compartment neuron by \citet{RothmanManis:2003b} has the characteristics of a
onset chopper unit and has previously been used to simulate a DS cell model. The
choice of having a large multipolar neuron without dendrites was based on
computational efficiency and ensuring that the model fit within the criteria for
DS cells. The electrotonic dendrites of DS cells mean that the filtering in DS
cells primarily controls the height of EPSCs reaching the soma
\citep{WhiteYoungEtAl:1994}, hence a single compartment with graded weights
should suffice. The inhibitory effect of Golgi cells onto DS units are delayed and occurs
through a the slow peak of \GABAa inhibition, so optimisation of number and
weight parameters is straight forward. ANF spread onto DS cells is well
documented (-2 octaves below, 1 octave above), hence a decision made to fix this
parameter due to large computational task of calculating bandwidth. Initial
optimisation procedures were not successful at constraining the short delay
recovery responses (2,3,4 ms), hence the DS leak and KLT conductance parameters
were included to allow cell's input resistance behaviour to fit fast acting
behaviour in the cell.

%{\small%\linespread{0.5}
  \begin{table}[ht]
    \caption{D~stellate cell  model summary}
    \label{tab:DScellModelSummary}
  \end{table}
\noindent%
\begin{tabularx}{\textwidth}{|l|X|}\hline %
\hdr{2}{A}{Model Summary}\\\hline
         \textbf{Populations}          & ANF (HSR,LSR), Golgi, and  D~stellate cells\\\hline
          \textbf{Topology}            & Tonotopic, auditory system of the rat  \\\hline
        \textbf{Connectivity}          & Gaussian spread dependent on morphology and afferent connections  \\\hline
         \textbf{Input model}          & Auditory nerve model \citep{ZilanyBruce:2007}\\\hline
\multirow{2}{*}{\textbf{Neuron model}} & Golgi: instantaneous-rate Poisson spike trains\\
                                       & D~stellate: Type 1-2 R\&M single compartment neuron\\ \hline
       \textbf{Channel models}         & $I_{\textrm{Na}}$, $I_{\textrm{KHT}}$, $I_{\textrm{KLT}}$, $I_{\textrm{KA}}$ and $I_{\textrm{h}}$ \citep{RothmanManis:2003b} \\\hline
        \textbf{Synapse model}         & Conductance synapses: excitatory (single-exponential), GABAergic (double-exponential) \\\hline
       \textbf{Input Stimulus}         & Mask/Recovery click trains with delay 2, 3, 4 and 8
ms, separated by 50 ms\\\hline
        \textbf{Measurements}          & PSTH sampled at each recovery click for 2 ms to measure click recovery\\\hline
% PSTHs were generated from 25
%   stimulus repetitions. Each response to a click is measured for 2 ms
%   after the minimum first spike latency for the unit.  The unit used
%   in the optimisation has a CF = 5.8~kHz (channel no. 50).\\ \hline
\end{tabularx}
\vspace{2ex}

% - B -----------------------------------------------------------------------------

\noindent%
\begin{tabularx}{\textwidth}{|l|X|X|}\hline %{\textwidth}
\hdr{3}{B}{Populations}\\\hline
\textbf{Name} &               \textbf{Elements}                & \textbf{Number} \\\hline
     HSR      & Auditory nerve fibre \citep{ZilanyBruce:2007}  & $N_{\text{HSR}} = 50\times{}N_\mathsf{channel}$ \\\hline
     LSR      & Auditory nerve fibre \citep{ZilanyBruce:2007}  & $N_{\text{LSR}} = 20\times{}N_\mathsf{channel}$ \\\hline
     GLG      & Instantaneous-rate Poisson neuron        & $N_{\text{GLG}} = 1\times{}N_\mathsf{channel}$ \\\hline
     DS       & Type I-II \citeauthor{RothmanManis:2003b} model & 1 unit at channel 50, $CF = 5.6$ kHz \\\hline
\end{tabularx}
\vspace{2ex}

% - C ------------------------------------------------------------------------------

\noindent%
\begin{tabularx}{\textwidth}{|l|l|l|X|}\hline
\hdr{4}{C}{Connectivity}\\\hline
        \textbf{Name}          & \textbf{Source} & \textbf{Target} & \textbf{Pattern} \\\hline
$\textrm{HSR} \to \textrm{DS}$ &       ANF       &   D~Stellate    & skewed Gaussian, centered at CF, spread below CF \sANFDSl, spread above CF \sANFDSh, uniform weight \wANFDS\ for all synapses, number \nLSRDS\ and \nHSRDS, delay \dANFDS \\\hline
$\textrm{LSR} \to \textrm{DS}$ &       ANF       &   D~Stellate    & skewed Gaussian, centered at CF, spread below CF \sANFDSl, spread above CF \sANFDSh, uniform weight \wANFDS\ for all synapses, number \nLSRDS\ and \nHSRDS, delay \dANFDS \\\hline

$\textrm{GLG} \to \textrm{DS}$ &      Golgi      &   D~Stellate    & Gaussian, centered at CF with spread \sGLGDS, uniform weight \wGLGDS, number \nGLGDS, delay \dGLGDS \\\hline
\end{tabularx}

\vspace{2ex}




% - D ------------------------------------------------------------------------------

\noindent%
\begin{tabularx}{\textwidth}{|l|X|}\hline
\hdr{2}{D}{Neuron and Synapse Model}\\\hline
 \textbf{Name} & DS cell model \\\hline
 \textbf{Type} & Type 1-2 \citep{RothmanManis:2003b}, conductance synapse input \\\hline
 \textbf{Spiking} & Emit spike when $V(t)\geq \theta$  \\\hline
 \end{tabularx}

\vspace{2ex}
}


%%% Local Variables: 
%%% mode: latex
%%% TeX-master: "SimpleResponses"
%%% TeX-PDF-mode: nil
%%% End: 
 

\smallskip{}

Optimisation parameters for \GLGDS are optimised based on experimental click
recovery data from \citep{BackoffPalombiEtAl:1997}.  Fixed parameters included
the number of Golgi cells to DS cells ($\nGLGDS = 25$), the spread of ANFs to DS
cells $\ANFDS$, and the extra delay from the auditory nerve.  The first spike
latency in DS cells 2.8 ± 0.09 ms \citep{RhodeSmith:1986}.The addition of 0.5 ms
to \ANFDS connections is a combination of conductance and synaptic delay. The
effect of Golgi cells on DS is delayed by the extra 0.7~ms delay from ANF to
Golgi, plus the slow peak of \GABAa inhibition.  The spread ANF to DS cells
(\sANFDSh,\sANFDSl) is arbitrary at this point and fairly broad so the estimate
is set so that 2 octaves below and 1 octave above CF are within 2 standard
deviations \citep{PaoliniClark:1999}.


% \begin{figure}[htb]
%   \centering
%   \includegraphics[width=0.5\textwidth]{DS_ClickRecovery_DSpsth}\label{fig:DSClickRecoveryPSTH}\\
%   \includegraphics[width=0.5\textwidth]{DS_ClickRecovery_Gpsth}\label{fig:GClickRecoveryPSTH}
%   \caption{PSTH response of a D-stellate cell from the click recovery stimulus
%   used in the optimisation.}
% \end{figure}



\subsection{Results}\label{sec:DS:results}

The DS cell optimisation function, derived from mask and recovery responses to clicks by \citet{BackoffPalombiEtAl:1997}, was tested 


the is the square root of a mean squared error between vectors from the
experimental and simulated data. The vector elements were calculated from the
number of spikes in a 2 ms window after a click.  The input stimulus was a
series of masker-probe clicks, with intervals of 2, 3, 4, 8, and 16 ms,
separated by 50 ms, and repeated 25 stimulus times. Each response to a click is
measured for 2 ms after the minimum first spike latency for the unit.  The unit
used in the optimisation has a CF of 5.8~kHz (equivalent to channel no. 50).


\smallskip{}

% \noindent\begin{tabularx}{\textwidth}{|l|X|}\hline %{\textwidth}
% \hdr{2}{D}{Results} \\\hline
% \end{minipage}}\\\hline
% \textbf{Error} & 0.006671    unweighted (MSE of recovery spike rate / mask rate)\\\hline
% & 0.01447    final result (MSE of recovery spike rate / mask rate)\\\hline
% \end{tabularx}

{% - E ------------------------------------------------------------------------------
\noindent
\begin{tabularx}{\textwidth}{|X|c|c|c|}\hline %{\textwidth}
\hdr{4}{E}{Optimisation} \\ \hline
        \textbf{Parameters}          &   \textbf{Name}  & \textbf{Range} & \textbf{Best Values} \\\hline 
      Weight of GLG on DS (nS)       &     \wGLGDS      &   [0.01,50]    & 0.532 \\	\hline	
    Weight of HSR syn on DS (nS)     &     \wHSRDS      &   [0.01,50]    & 0.16 \\	   \hline
   Weight of LSR syn on DS  (nS)     &     \wLSRDS      &   [0.01,50]    & 13.1 \\	    \hline
 \GABAa synapse rise constant  (ms)  &  $\tau_{GABA1}$  &  [0.01,10.0]   & 5.432\\	     \hline
 \GABAa synapse decay constant (ms)  &  $\tau_{GABA2}$  &   [0.1,50.0]   & 0.262\\	    \hline
DS cell leak conductance (mS/cm$^2$) & $\bar{g}_{leak}$ &  [1e-5,0.05]   & 0.0163 \\ \hline
\end{tabularx}
\vspace{2ex}
}

The optimisation parameters show a clear favouritism toward the LSR input rather
than the HSR input to DS units. While this may not seem ideal for fast
co-incidence detection, the large number of HSR synapses makes up for the small
weight that was obtained in the optimisation.

\begin{figure}[htb!]
  \centering
  % \resizebox{0.9\textwidth}{!}{\includegraphics{NoFigure}}
  \resizebox{0.8\textwidth}{!}{\includegraphics[height=\textwidth,keepaspectratio,angle=-90]{DS_ClickRecovery_result}}
  \caption{Optimisation results of click recovery behaviour in DS cell
    (CF=5.8~kHz). The optimal response (blue circle) is obtained from Fig.~3 in
    \citet{BackoffPalombiEtAl:1997}, representing the click recovery response of
    an OnC unit (CF=5.8~kHz).  Best result (green triangles)
  } \label{fig:DS_ClickRecovery_result}
\end{figure}



% \begin{figure}
%   \includegraphics[width=0.5\textwidth]{DS_ClickRecovery_OptVars.eps}\\
% %   \includegraphics[width=0.5\textwidth]{DS_ClickRecovery_Output.eps}\label{Ch3:fig:DSClickRecoveryOutput}
%   \caption{Final Output Data of the D-stellate Click Recovery optimisation }
% \end{figure}
% \begin{figure}
%   \includegraphics[keepaspectratio=true,width=0.8\textwidth]{DS_ClickRecovery_Example1.eps}\\
%   \includegraphics[keepaspectratio=true,width=0.8\textwidth]{DS_ClickRecovery_Example10.eps}\\
%   \includegraphics[keepaspectratio=true,width=0.8\textwidth]{DS_ClickRecovery_Example13.eps}\\
%   \includegraphics[keepaspectratio=true,width=0.8\textwidth]{DS_ClickRecovery_Example19.eps}\\
%   \caption{Click Recovery optimisation functions}
% \end{figure}


% \begin{figure}
%   \includegraphics[keepaspectratio=true,angle=-90,width=0.8\textwidth]{DS_ClickRecovery_result1.eps}\\
% \end{figure}


% \begin{figure}
%   \includegraphics[keepaspectratio=true,angle=-90,width=0.8\textwidth]{DS_ClickRecovery_result2.eps}\\
%   \caption{Click Recovery optimisation }
% \end{figure}


% \begin{figure}
%   \begin{center}
%     \includegraphics[keepaspectratio=true]{DS_ClickRecovery_handtuned.eps}\\
%     \includegraphics[keepaspectratio=true,angle=-90,width=0.8\textwidth]{DS_ClickRecovery_result_handtuned.eps}
%     \caption{Handtuned}
%     \label{hantuned}
%   \end{center}
% \end{figure}

% \begin{figure}
%   \begin{center}
% %     \includegraphics[keepaspectratio=true]{DS_ClickRecovery_handtuned.eps}\\
%     \includegraphics[keepaspectratio=true,angle=-90,width=0.8\textwidth]{gfx/DS_ClickRecovery_result_unweighted_8.eps}\\
%     \includegraphics[keepaspectratio=true,angle=-90,width=0.8\textwidth]{gfx/DS_ClickRecovery_result_weighted_0.eps}
%     \caption{Handtuned}
%     \label{hantuned}
%   \end{center}
% \end{figure}

\clearpage
% \newpage
\subsection{Verification}\label{sec:DS:verification}

\yellownote{Small presentation of PSTH, RL, NRL, MRC and RA Leave AM responses
  till next chapter}
% \subsection{Tone Responses}
% \begin{figure}[h!]
%   \centering\resizebox{\textwidth}{!}{%
%   \includegraphics{RateLevel/psthsingle90.2.eps}%
%   \includegraphics{RateLevel/DS_ratelevel.eps}}
% \end{figure}
% \begin{figure}[h!]
%   \centering\resizebox{\textwidth}{!}{%
%   \includegraphics{RateLevel/response_area.2.eps}%
%   \includegraphics{RateLevel/response_area_log2.2.eps}}
% \end{figure}
% \begin{figure}[h!]
%   \centering\resizebox{\textwidth}{!}{%
% %   \includegraphics{RateLevel/response_area.2.eps}
%   \includegraphics{RateLevel/psthall90.2.eps}%
%   \includegraphics{RateLevel/psthVlevel.2.eps}}
% \end{figure}


% \clearpage
% \subsection{Noise Responses}
% \begin{figure}[h!]
%   \centering\resizebox{\textwidth}{!}{%
%   \includegraphics{NoiseRateLevel/psthsingle120.2.eps}%
%   \includegraphics{NoiseRateLevel/DS_ratelevel.eps}}
% \end{figure}
% \begin{figure}[h!]
%   \centering\resizebox{\textwidth}{!}{%
%   \includegraphics{NoiseRateLevel/response_area.2.eps}%
%   \includegraphics{NoiseRateLevel/response_area_log2.2.eps}}
% \end{figure}
% \begin{figure}[h!]
%   \centering\resizebox{\textwidth}{!}{%
% %   \includegraphics{RateLevel/response_area.2.eps}
%   \includegraphics{NoiseRateLevel/psthall90.2.eps}%
%   \includegraphics{NoiseRateLevel/psthVlevel.2.eps}}
% \end{figure}


% \clearpage
% \subsection{Masked Noise and Tone Responses}
% \begin{figure}[h!]
%   \centering\resizebox{\textwidth}{!}{\includegraphics{MaskedRateLevel/psthsingle90.2.eps}\includegraphics{MaskedRateLevel/DS_ratelevel.eps}}
% \end{figure}
% \begin{figure}[h!]
%   \centering\resizebox{\textwidth}{!}{%
%   \includegraphics{MaskedRateLevel/response_area.2.eps}%
%   \includegraphics{MaskedRateLevel/response_area_log2.2.eps}}
% \end{figure}

% \begin{figure}[h!]
%   \centering\resizebox{\textwidth}{!}{%
% %   \includegraphics{RateLevel/response_area.2.eps}
%   \includegraphics{MaskedRateLevel/psthall90.2.eps}%
%   \includegraphics{MaskedRateLevel/psthVlevel.2.eps}}
% \end{figure}
% \clearpage
% \subsection{Masked Response Area}
% \begin{figure}[h!]
%   \centering\resizebox{\textwidth}{!}{%
%   \includegraphics{MaskedResponseCurve/psthsingle5810.2.eps}%
%   \includegraphics{MaskedResponseCurve/DS_masked.eps}}
% \end{figure}
% \begin{figure}[h!]
%   \centering\resizebox{\textwidth}{!}{%
%   \includegraphics{MaskedResponseCurve/response_area.2.eps}%
%   \includegraphics{MaskedResponseCurve/response_area_log2log2.2.eps}}
% \end{figure}

% \begin{figure}[h!]
%   \centering\resizebox{\textwidth}{!}{%
% %   \includegraphics{RateLevel/response_area.2.eps}
%   \includegraphics{MaskedResponseCurve/psthall5810.2.eps}%
%   \includegraphics{MaskedResponseCurve/psthVmod.2.eps}}
% \end{figure}
% \clearpage


\subsection{(Optional) Effects of $g_{leak}$ and $g_{KLT}$ on DS resting membrane
  potential}\label{sec:DS:effects-g_leak-g_lkt}

\yellownote{This section is optional}


The resting membrane potential of these large multipolar cells has been shown to
be in the range of 3-5 MOhms \yellownote{citation needed here}. A quick
observation of the parameter space around the optimisation results for
$g_{leak}$ and $g_{KLT}$ is shown in Figure~\ref{fig:leakVltk}.

\begin{figure}[h!]
  \centering 
\resizebox{\textwidth}{!}{\includegraphics{NoFigure}}
%\resizebox{\textwidth}{!}{\includegraphics{leakvltk}}
  \caption{Resting Membrane potential calculated for leak conductance and KLT
    conductance changes around the previously obtained best values for these
    parameters.}\label{fig:leakVltk}
\end{figure}


%%% Local Variables: 
%%% mode: latex
%%% mode: tex-fold
%%% TeX-master: "SimpleResponses"
%%% TeX-PDF-mode: nil
%%% End: 

