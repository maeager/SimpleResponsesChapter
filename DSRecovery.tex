
\graphicspath{{/media/data/Work/cnstellate/DS_ClickRecovery/}{/media/data/Work/Responses/}{../figures/}{./gfx/}}
\section[DS Cell Model]{D-Stellate Cell Model: optimisation using click recovery responses}
\label{sec:d-stellate-cell-model}


This section shows the GABAergic input and intrinsic cell properties
influence the behaviour Onset chopper units.  Onset-chopper units in
the mammalian VCN have a wide-ranging influence on the primary cells
of the VCN (stellate and bushy neurons \citep{RhodeSmithEtAl:1983}),
the ipsilateral DCN (type II and type IV EIRA units) and the
contra\-lateral CN \citep{NeedhamPaolini:2007}.

\medskip{}


% Large multipolar or stellate cells in the VCN have been shown to have
% 3-4 long dendrites stretching 200 microns (or one third of the VCN)
% and their axonal collaterals cover the same region in the VCN, almost
% one half of the DCN, and are one source of the commisural projection
% to the contralateral cochlear nucleus \citep{NeedhamPaolini:2007}.
%%%%%%%%%%%%%%%%%%%Copied from original jneurometh article



\subsection{Background}

  
D-stelate (DS) cells have an onset-chopping (On-C) PSTH to tones and
noise \citep{SmithRhode:1989,NeedhamPaolini:2006}. Intracellular
responses to sounds indicate the bandwidth of inputs to DS neurons
typically ranges from two octaves below CF to one octave above CF
\citep{PalmerJiangEtAl:1996,PaoliniClark:1999}. DS cell axon terminals
contain the inhibitory neurotransmitter glycine and synapse widely in
the VCN and DCN\@.  They also send a commissural projection to the
contralateral cochlear nucleus that mediates fast inhibition between
the nuclei \citep{NeedhamPaolini:2003,NeedhamPaolini:2006}.
\citep{Oertel:1997}


\medskip{}

Post-onset GABAergic inhibition in DS cells is a major influence on
the PSTH of On-C neurons
\citep{FerragamoGoldingEtAl:1998a,EvansZhao:1998}. Latency of
excitation to auditory nerve shocks suggests golgi cells are activated
by type II ANFs and low spontaneous rate type I ANFs
\citep{BensonBerglundEtAl:1996,FerragamoGoldingEtAl:1998}. Therefore,
type II and LS type I ANFs could be involved in gain control through
GABAergic modulation of activity in the VCN.


\medskip{}

\yellownote{Discuss AM coding by DS cells}  GABA
blockers in the VCN has the effects of changing the behaviour in the
response to AM in the IC \citep{CasparyPalombiEtAl:2002}.  AM coding
effects of GABA in the Chinchilla CN  \citep{BackoffShadduckEtAl:1999}.  \citep{CasparyBackoffEtAl:1994}
Caspary and colleagues worked on the effects of GABA in in the VCN\@.

Zhang and Winter looked at the response area of VCN onset units to
determine GABA on/off freq.

Smith and Rhode, Smith and others looked at OnC respnse area and two-tone



\subsection{Implementation}
 
Key factors in designing D-stellate cell model

\medskip{}

Choosing neural model: type I-II Rothman and Manis model
  - with/without dendrites
  - variable KLT, leak conductance

\begin{itemize}
\item  ANF spread to DS cells well documented (decision made to
    fix params due to large computational task of calc response area) 
\item  short delay recovery responses (2,3,4 ms) were not successful upon
    first model, included DS leak and KLT conductances to allow cell
    behaviour to be fit
\item The effect of Golgi cells on DS is delayed by the extra 0.7 ms
    delay from ANF to Golgi, plus the slow peak of \GABAa inhibition.
\end{itemize}

\medskip{}

Optimisation parameters for \GLGDS are optimised based on experimental
click recovery data from \citep{BackoffPalombiEtAl:1997}.  Fixed
parameters included the number of golgi cells to DS cells ($\nGLGDS =
25$), the spread of ANFs to DS cells $\ANFDS$, and the extra delay
from the auditory nerve.  The first spike latency in DS cells 2.8 ±
0.09 ms \citep{RhodeSmith:1986}.The addition of 0.5 ms to \ANFDS
connections is a combination of conductance and synaptic delay. The
effect of Golgi cells on DS is delayed by the extra 0.7~ms delay from
ANF to Golgi, plus the slow peak of \GABAa inhibition.  The spread ANF
to DS cells (\sANFDSh,\sANFDSl) is arbitrary at this point and fairly
broad so the estimate is set so that 2 octaves below and 1 octave
above CF are within 2 standard deviations \citep{PaoliniClark:1999}.
\medskip{}

The optimisation function was a weighted mean squared error between
the experimental and simulated data, from an array where the elements
were the number of spikes in a 2 ms window after a click.  The input
stimulus was a series of masker/response clicks, with the click
intervals of 2, 3, 4, 8, 16 ms, and a separation of 50 ms.


 PSTHs were generated from 25
   stimulus repetitions. Each response to a click is measured for 2 ms
   after the minimum first spike latency for the unit.  The unit  used
   in the optimisation has a CF of 5.8~kHz (equivalent to channel no. 50).


% \begin{figure}[htb]
%   \centering
% \includegraphics[width=0.5\textwidth]{DS_ClickRecovery_DSpsth}\label{fig:DSClickRecoveryPSTH}\\
% \includegraphics[width=0.5\textwidth]{DS_ClickRecovery_Gpsth}\label{fig:GClickRecoveryPSTH}
%    \caption{PSTH response of a D-stellate cell from the click recovery stimulus used in the optimisation.}
%  \end{figure}



 % ---------------------------------------------------------------------------------
 \newpage
% \begin{lstlisting}
% func fun() {local f
%       //Modify Variables
%       param.w.x[glg][ds] = $2
%       param.w.x[hsr][ds] = $3
%       param.w.x[lsr][ds] = $3
%       //Modify the network
%       {create_cells() connect_cells(fileroot) SetRates()}
%       // Simulate the network for N reps
%       for j=0, reps-1{
%          print j
%          GenSpikes()
%          run()
%          DSvec.append(dstellate[50][0].spiketimes)
%          //print startsw()-x, "secs"
%       }
%       DSvec = DSvec.histogram(0,tstop,0.1)
%
%       objref errorvec
%       errorvec = new Vector()
%       //Find the mean number of spikes in the first click
%       maxrate = (DSvec.sum(240,260) + DSvec.sum(740,760)+ DSvec.sum(1340,1360))/3
%       //Calc ratio of number of spikes in second click relative to mean first click
%       errorvec.append( DSvec.sum(260,280) / maxrate )
%       errorvec.append( DSvec.sum(780,800) / maxrate )
%       errorvec.append( DSvec.sum(1420,1440) / maxrate )
%       errorvec.plot(g
%     return errorvec.meansqerr(targetclick)
% }
% \end{lstlisting}


{%
\small\linespread{0.5}
\begin{table}[!pt]
    \caption{D~stellate cell  model summary}
    \label{tab:DScellModelSummary}
%\noindent%
\begin{tabularx}{\textwidth}{|l|X|}\hline %
\hdr{2}{A}{Model Summary}\\\hline
         \textbf{Populations}           & ANF (HSR, LSR), GLG, and  DS cell models\\\hline
           \textbf{Topology}            & Tonotopic, auditory system of the rat, 100 frequency channels  \\\hline
         \textbf{Connectivity}          & Gaussian spread dependent on morphology and afferent connections  \\\hline
         \textbf{Input model}           & ANF~model: Instantaneous-rate Poisson neural model  \citep{ZilanyBruceEtAl:2009} \\\hline
\multirow{2}{*}{\textbf{Neuron model}}  & \GLG cell: \GLG neural model (see Section \ref{sec:Ch3:GolgiModel}).\\
                                        & DS cell model: Type I-II \RM single compartment neural model\\ \hline
        \textbf{Channel models}         & $I_{\textrm{Na}}$, $I_{\textrm{KHT}}$, $I_{\textrm{KLT}}$, $I_{\textrm{KA}}$ and $I_{\textrm{h}}$ \citep{RothmanManis:2003b} \\\hline
\multirow{2}{*}{\textbf{Synapse model}} & Excitatory: \AMPA glutamatergic receptor (single-exponential)\\
                                        & Inhibitory: \GABAA GABAergic receptor (double-exponential), \GlyR glycinergic receptor (double-exponential) \\\hline
       % \textbf{Synapse model}         & Conductance synapses: excitatory (single-exponential), GABAergic (double-exponential) \\\hline
%        \textbf{Input Stimulus}         & Five mask and recovery click pairs separated by 50 ms\\\hline
%         \textbf{Measurements}          & PSTH sampled at each click for 2 ms to measure recovery from masking clicks\\\hline
\end{tabularx}
\vspace{1ex}
% - B -----------------------------------------------------------------------------
\noindent%
\begin{tabularx}{\textwidth}{|l|X|X|}\hline %{\textwidth}
\hdr{3}{B}{Populations}\\\hline
\textbf{Name} &               \textbf{Elements}                & \textbf{Number} \\\hline
     HSR      & Auditory nerve fibre \citep{ZilanyBruceEtAl:2009}  & $N_{\text{HSR}} = 50$ per channel \\\hline
     LSR      & Auditory nerve fibre \citep{ZilanyBruceEtAl:2009}                       & $N_{\text{LSR}} = 20$ per channel \\\hline
     GLG      & Instantaneous-rate Poisson neuron        & $N_{\text{GLG}} = 1$ per channel \\\hline
    \multirow{2}{*}{DS}       & \multirow{2}{*}{Type I-II \RM model} &  Click Recovery: 1 unit at channel 50, CF$ = 5.6$ kHz \\
&& Rate Level: 1 unit at channel 76, CF = 11.1 KHz \\\hline
\end{tabularx}
\vspace{1ex}
% - C ------------------------------------------------------------------------------
\noindent%
\begin{tabularx}{\textwidth}{|l|l|l|X|}\hline
\hdr{4}{C}{Connectivity}\\\hline
     \textbf{Name}      & \textbf{Source} & \textbf{Target} & \textbf{Pattern} \\\hline
\ANFDS &
%{\begin{minipage}\begin{center}
 \HSR,\,\LSR
%\end{center} \end{minipage}}
&       \DS       &
Skewed Gaussian convergence, centred on CF, spread below  CF $\sigma^2 = \sANFDSl$, spread above CF $\sigma^2 = \sANFDSh$, delay  \dANFDS.  Weight and number differ for HSR and  LSR connections ( \wHSRDS,  \nHSRDS, \wLSRDS, \nLSRDS) \\\hline
       $\GLGDS$         &       \GLG       &       \DS        &
Gaussian convergence, centred on CF, spread $\sigma^2 = \sGLGDS$, uniform weight \wGLGDS, number \nGLGDS, delay \dGLGDS \\\hline
\multicolumn{4}{|X|}{\ANFGLG from previous section, see Table~\ref{tab:GolgiCellModelSummary}C.}\\\hline
\end{tabularx}
\vspace{1ex}
% - D ------------------------------------------------------------------------------
\end{table}
\begin{table}[!pt]
    {Table~\ref{tab:DScellModelSummary}: D~stellate cell  model summary - continued}\\
\noindent%
\begin{tabularx}{\textwidth}{|l|X|}\hline
\hdr{2}{D}{Neuron and Synapse Model}\\\hline
 \textbf{Name} & DS cell model \\\hline
 \textbf{Type} & Type I-II \citep{RothmanManis:2003b}, conductance synapse input \\\hline
\textbf{Subthreshold dynamics} & Na, KLT, KHT, Ih, and leak currents \\\hline
 \textbf{Spiking} & Emit spike when $V(t)\geq \theta$  \\\hline
 \end{tabularx}
%\vspace{1ex}
\noindent%
\begin{tabularx}{\textwidth}{|l|X|}\hline %
\hdr{2}{E}{Optimisation - Click Recovery}\\\hline
\textbf{Input Stimulus}  & Mask and recovery click pairs, with delay 16, 2, 8, 4, and 3 ms (in this order), separated by 50 ms   \\\hline
     \textbf{Parameters}      &
      \wGLGDS,
      \wHSRDS,
      \wLSRDS,
$\tau_{\rm GABA-1}$,
$\tau_{\rm GABA-2}$,
      \DS \gleak    \\\hline
% \textbf{Measurements}   &  Spiking output of DS unit, in channel 50, from 25 repetitions and collected in a PSTH.  The PSTH was sampled at each click for 2 ms to measure click recovery. Idle times were recorded for spontaneous activity and level of ANF excitation.\\\hline
% PSTHs were generated from 25 stimulus repetitions. Each response to a click is measured for a period of 2 ms.  The sample period was delayed by 4 ms, an estimate of the auditory delay and minimum first spike latency for the DS unit.  The unit used   in the optimisation has a CF = 5.8~kHz (channel no. 50).\\ \hline
% %\textbf{Optimisation} & Parameters for \GLGDS are optimised based on experimental click recovery date from \citet{BackoffPalombiEtAl:1997}. The praxis method is used for optimisation.  \\\hline
 %\textbf{Measurements}  & First spikes and PSTH of TV cells, calculated for first spike latency, mean rate and variance. Fitting data was compared against experimental data of a Type-II~\DCN~unit~\citep{ReissYoung:2005}, Fig.~9. \\\hline

\textbf{Fitness Function} & Weighted mean squared error between masker-probe rate ratios of \DS cell  model and experimental \DS cell \citep{BackoffPalombiEtAl:1997} to pairs of clicks. Idle rates were recorded for spontaneous activity and levels of ANF excitation were used as additional penalties. \\\hline
\end{tabularx}
\vspace{1ex}
\noindent%
\begin{tabularx}{\textwidth}{|l|X|}\hline %
\hdr{2}{F}{Optimisation - Rate Level}\\\hline
\textbf{Input Stimulus}  & 1. Tone rate Level function, 11.1~kHz tone at 30 to 100 dB SPL in 5 dB intervals (50 ms duration, 2 ms cosine squared on\slash off ramp, 20 ms delay). 2. Noise rate level, broad-band noise at 40 to 95 db SPL in 5 dB intervals   (50 ms duration, 2 ms cosine squared on\slash off ramp, 20 ms delay). \\\hline
\textbf{Parameters}      &
      \wGLGDS, \nGLGDS,
      \wHSRDS, \nHSRDS,
      \wLSRDS, \nLSRDS
          \\\hline
\textbf{Fitness Function} & RMS error between \DS cell model (CF 11.1 kHz) and experimental \OnC unit \citep[CF~10.9~kHz,][]{ArnottWallaceEtAl:2004} at each input stimulus in tone and noise rate levels. \\\hline
\end{tabularx}
\end{table}
}

%%% Local Variables:
%%% mode: latex
%%% TeX-master: "SimpleResponses"
%%% TeX-PDF-mode: nil
%%% End:



\clearpage

\begin{figure}[htb]
\centering
\resizebox{5in}{!}{\includegraphics{NoFigure}}
%\includegraphics[keepaspectratio,angle=-90,width=0.8\textwidth]{DSClickRecoveryExpData}
\caption{Experimental Data of GABAergic influence on D-stellate cells from \citep{BackoffPalombiEtAl:1997}, Fig.~3.}\label{Ch3:fig:DSClickRecoveryExpData}
\end{figure}

%\parsep

\clearpage
\subsection{Results}

% \noindent\begin{tabularx}{\textwidth}{|l|X|}\hline %{\textwidth}
% \hdr{2}{D}{Results} \\\hline
% \end{minipage}}\\\hline
% \textbf{Error} & 0.006671    unweighted (MSE of recovery spike rate / mask rate)\\\hline
% & 0.01447    final result (MSE of recovery spike rate / mask rate)\\\hline
% \end{tabularx}

\begin{figure}[htb!]
  \centering
\resizebox{0.9\textwidth}{!}{\includegraphics{NoFigure}}
%\resizebox{3.5in}{!}{\includegraphics[angle=-90]{./gfx/DS_ClickRecovery_result.eps}}
\caption{Experimental Data ({\color{green} Green}) of GABAergic influence on D-stellate cells from \citep{BackoffPalombiEtAl:1997}, Fig.~3.  Best result ({\color{blue} Blue}) shown in figure below.} \label{fig:DS_ClickRecovery_result}  
\end{figure}


% \begin{figure}
%   \includegraphics[width=0.5\textwidth]{DS_ClickRecovery_OptVars.eps}\\
% %  \includegraphics[width=0.5\textwidth]{DS_ClickRecovery_Output.eps}\label{Ch3:fig:DSClickRecoveryOutput}
%   \caption{Final Output Data of the D-stellate Click Recovery optimisation }
% \end{figure}

% \begin{figure}
%   \includegraphics[keepaspectratio=true,width=0.8\textwidth]{DS_ClickRecovery_Example1.eps}\\
%   \includegraphics[keepaspectratio=true,width=0.8\textwidth]{DS_ClickRecovery_Example10.eps}\\
%   \includegraphics[keepaspectratio=true,width=0.8\textwidth]{DS_ClickRecovery_Example13.eps}\\
%   \includegraphics[keepaspectratio=true,width=0.8\textwidth]{DS_ClickRecovery_Example19.eps}\\
%   \caption{Click Recovery optimisation functions}
% \end{figure}




% \begin{figure}
%   \includegraphics[keepaspectratio=true,angle=-90,width=0.8\textwidth]{DS_ClickRecovery_result1.eps}\\
% \end{figure}


% \begin{figure}
%   \includegraphics[keepaspectratio=true,angle=-90,width=0.8\textwidth]{DS_ClickRecovery_result2.eps}\\
%   \caption{Click Recovery optimisation }
% \end{figure}




% \begin{figure}
% \begin{center}
% \includegraphics[keepaspectratio=true]{DS_ClickRecovery_handtuned.eps}\\
% \includegraphics[keepaspectratio=true,angle=-90,width=0.8\textwidth]{DS_ClickRecovery_result_handtuned.eps}
% \caption{Handtuned}
% \label{hantuned}
% \end{center}
% \end{figure}

% \begin{figure}
% \begin{center}
% %\includegraphics[keepaspectratio=true]{DS_ClickRecovery_handtuned.eps}\\
% \includegraphics[keepaspectratio=true,angle=-90,width=0.8\textwidth]{gfx/DS_ClickRecovery_result_unweighted_8.eps}\\
% \includegraphics[keepaspectratio=true,angle=-90,width=0.8\textwidth]{gfx/DS_ClickRecovery_result_weighted_0.eps}
% \caption{Handtuned}
% \label{hantuned}
% \end{center}
% \end{figure}

 \clearpage
%\newpage
\subsection{Verification}

 \subsection{Tone Responses}
% \begin{figure}[h!]
% \centering\resizebox{\textwidth}{!}{%
% \includegraphics{RateLevel/psthsingle90.2.eps}%
% \includegraphics{RateLevel/DS_ratelevel.eps}}
% \end{figure}
% \begin{figure}[h!]
% \centering\resizebox{\textwidth}{!}{%
% \includegraphics{RateLevel/response_area.2.eps}%
% \includegraphics{RateLevel/response_area_log2.2.eps}}
% \end{figure}
% \begin{figure}[h!]
% \centering\resizebox{\textwidth}{!}{%
% %\includegraphics{RateLevel/response_area.2.eps}
% \includegraphics{RateLevel/psthall90.2.eps}%
% \includegraphics{RateLevel/psthVlevel.2.eps}}
% \end{figure}


% \clearpage
 \subsection{Noise Responses}
% \begin{figure}[h!]
% \centering\resizebox{\textwidth}{!}{%
% \includegraphics{NoiseRateLevel/psthsingle120.2.eps}%
% \includegraphics{NoiseRateLevel/DS_ratelevel.eps}}
% \end{figure}
% \begin{figure}[h!]
% \centering\resizebox{\textwidth}{!}{%
% \includegraphics{NoiseRateLevel/response_area.2.eps}%
% \includegraphics{NoiseRateLevel/response_area_log2.2.eps}}
% \end{figure}
% \begin{figure}[h!]
% \centering\resizebox{\textwidth}{!}{%
% %\includegraphics{RateLevel/response_area.2.eps}
% \includegraphics{NoiseRateLevel/psthall90.2.eps}%
% \includegraphics{NoiseRateLevel/psthVlevel.2.eps}}
% \end{figure}


% \clearpage
 \subsection{Masked Noise and Tone Responses}
% \begin{figure}[h!]
% \centering\resizebox{\textwidth}{!}{\includegraphics{MaskedRateLevel/psthsingle90.2.eps}\includegraphics{MaskedRateLevel/DS_ratelevel.eps}}
% \end{figure}
% \begin{figure}[h!]
% \centering\resizebox{\textwidth}{!}{%
% \includegraphics{MaskedRateLevel/response_area.2.eps}%
% \includegraphics{MaskedRateLevel/response_area_log2.2.eps}}
% \end{figure}

% \begin{figure}[h!]
% \centering\resizebox{\textwidth}{!}{%
% %\includegraphics{RateLevel/response_area.2.eps}
% \includegraphics{MaskedRateLevel/psthall90.2.eps}%
% \includegraphics{MaskedRateLevel/psthVlevel.2.eps}}
% \end{figure}
% \clearpage
% \subsection{Masked Response Area}
% \begin{figure}[h!]
% \centering\resizebox{\textwidth}{!}{%
% \includegraphics{MaskedResponseCurve/psthsingle5810.2.eps}%
% \includegraphics{MaskedResponseCurve/DS_masked.eps}}
% \end{figure}
% \begin{figure}[h!]
% \centering\resizebox{\textwidth}{!}{%
% \includegraphics{MaskedResponseCurve/response_area.2.eps}%
% \includegraphics{MaskedResponseCurve/response_area_log2log2.2.eps}}
% \end{figure}

% \begin{figure}[h!] 
% \centering\resizebox{\textwidth}{!}{%
% %\includegraphics{RateLevel/response_area.2.eps}
% \includegraphics{MaskedResponseCurve/psthall5810.2.eps}%
% \includegraphics{MaskedResponseCurve/psthVmod.2.eps}}
% \end{figure}
% \clearpage

 

%%% Local Variables: 
%%% mode: latex
%%% mode: tex-fold
%%% TeX-master: "SimpleResponses"
%%% TeX-PDF-mode: nil
%%% End: 
