
\graphicspath{{/media/data/Work/cnstellate/DS_ClickRecovery/}{/media/data/Work/Responses/}}
\section[DS Cell Model]{D-Stellate Cell Model: optimisation using click recovery responses}
\label{sec:d-stellate-cell-model}
% - A ------------------------------------------------------------------------------

Onset-chopper units in the mammalian VCN have a wide-ranging influence
on the primary cells of the VCN (stellate and bushy cells), the
ipsilateral DCN and the contralateral CN \citep{NeedhamPaolini:2007}.  

This experiment shows the GABAergic input and intrinsic
cell properties  influence the behaviour Onset chopper units


** D Stellate Cell Model:  Experimental evidence
Morphological 
  - large Multipolar cell
  - electrotonic dendrites stretching 200 microns (one third of VCN), one half DCN, cCN 
  - receive large number of ANF syn to dend and soma
  - receive large number of Gly and GABA syn to soma dend
Intracellular
  + type I-II current clamp response
  + presence of Ih and small amounts of KLT currents
  + delay from shock to ANFs less than 1 ms 
Physiological 
  - Onset chopper PSTH, fast narrow FSL 
  - Wide response area (+1 oct and -2 oct)
  - high sync index, Low-pass MTF
  - monotonic response to tones and noise 

** D Stellate Cell Model:  Key factors in designing D-stellate cell model

Choosing neural model: type I-II Rothman and Manis model
  - with/without dendrites
  - variable KLT, leak conductance

Problems
  - ANF spread to DS cells well documented (decision made to
    fix params due to large computational task of calc response area) 
  - short delay recovery responses (2,3,4 ms) were not successful upon
    first model, included DS leak and KLT conductances to allow cell
    behaviour to be fit
  - The effect of Golgi cells on DS is delayed by the extra 0.7 ms delay from ANF to Golgi, plus the slow peak of \GABAa inhibition.


%%%%%%%%%%%%%%%%%%%Copied from original jneurometh article

\subsection{Background}

  
D-stelate (DS) cells have an onset-chopping (On-C) PSTH to tones and noise
\citep{SmithRhode:1989,NeedhamPaolini:2006}. Intracellular responses to sounds
indicate the bandwidth of inputs to DS neurons typically ranges from two
octaves below CF to one octave above CF
\citep{PalmerJiangEtAl:1996,PaoliniClark:1999}. DS cell axon terminals contain
the inhibitory neurotransmitter glycine and synapse widely in the VCN and DCN.
They also send a commissural projection to the contralateral cochlear nucleus
that mediates fast inhibition between the nuclei
\citep{NeedhamPaolini:2003,NeedhamPaolini:2006}. 
%%%%%%%%%%%%%%%%%%%%%%%%%%%%%%%%%%%%%%%%%%%%%%%%

Post-onset GABAergic inhibition in DS cells is a major influence on the PSTH
of On-C neurons \citep{FerragamoGoldingEtAl:1998a,EvansZhao:1998}. Latency of
excitation to auditory nerve shocks suggests golgi cells are activated by type
II ANFs and low spontaneous rate type I ANFs
\citep{BensonBerglundEtAl:1996,FerragamoGoldingEtAl:1998}. Therefore, type II
and LS type I ANFs could be involved in gain control through GABAergic
modulation of activity in the VCN.




\subsection{Implementation}

\textbf{Optimisation}    
 Parameters for \GLGDS are optimised based on experimental click recovery data from \citep{BackoffPalombiEtAl:1997}.

\noindent
\begin{tabularx}{\textwidth}{|l|X|}\hline %
%
\hdr{2}{A}{Model Summary}\\\hline
\textbf{Populations}     & ANF (HSR,LSR), Golgi, D-stellate \\\hline
\textbf{Topology}        & Tonotopic,  Auditory system of the rat  \\\hline
\textbf{Connectivity}    & Gaussian spread dependent on morphology and afferent connections  \\\hline
\textbf{Auditory model}  & \citep{ZilanyBruce:2008} ANF phenomenological instantaneous-rate Poisson spike trains\\\hline
\multirow{2}{*}{\textbf{Neuron model}}    & Golgi: instantaneous-rate Poisson spike trains
&D-stellate: HH-like single-compartment model (Type I-II RM model)
\textbf{Channel models}  & $I_{\textrm{Na}}$, $I_{\textrm{KHT}}$, $I_{\textrm{KLT}}$, $I_{\textrm{KA}}$ and $I_{\textrm{h}}$ \citep{RothmanManis:2003b} \\\hline
\textbf{Synapse model}   & Conductance synapses: excitatory (single-exponential), GABAergic (double-exponential) \\\hline
\textbf{Input Stimulus}  & Mask/Recovery click trains with delay 2, 3, 4 and 8
ms, separated by 50 ms\\\hline
\textbf{Measurements}    & PSTH sampled at each recovery click for 2 ms to measure click recovery\\\hline
\end{tabularx}
\vspace{2ex}

% - B -----------------------------------------------------------------------------

\noindent
\begin{tabularx}{\textwidth}{|l|X|X|}\hline %{\textwidth}
\hdr{3}{B}{Populations}\\\hline
\textbf{Name} &            \textbf{Elements}            & \textbf{Number} \\\hline
     HSR      &            Poisson generator with refractory effects           & $N_{\text{HSR}} = 50$ per freq.\ channel \\\hline
     LSR      &            ''            & $N_{\text{LSR}}= 20$  per freq.\ channel \\\hline
     GLG      &            ''            & $N_{\text{GLG}}= 1$  per freq.\ channel  \\\hline
     DS       &  Type I-II RM model & $N_{\text{DS}}= 1$ at CF=5.6~kHz \\\hline
\end{tabularx}
\vspace{2ex}

% - C ------------------------------------------------------------------------------

\noindent
\begin{tabularx}{\textwidth}{|l|l|l|X|}\hline
\hdr{4}{C}{Connectivity}\\\hline
        \textbf{Name}          &  \textbf{Source}  & \textbf{Target} & \textbf{Pattern} \\\hline
$\textrm{ANF} \to \textrm{DS}$ & ANF (HSR and LSR) &   D-Stellate    & skewed Gaussian, centered at CF, spread below CF \sANFDSl, spread above CF \sANFDSh, uniform weight \wANFDS for all synapses, number \nLSRDS and \nHSRDS, delay \dANFDS \\\hline
$\textrm{GLG} \to \textrm{DS}$ &       Golgi       &   D-Stellate    & Gaussian, centered at CF with spread \sGLGDS, uniform weight \wGLGDS, number \nGLGDS, delay \dGLGDS \\\hline
\end{tabularx}

\vspace{2ex}

% - D ------------------------------------------------------------------------------



%\noindent\begin{tabularx}{\textwidth}{|p{0.150.95\textwidth}|X|}\hline
%\hdr{2}{D}{Neuron and Synapse Model}\\\hline
% \textbf{Name} & Poisson spike generator \\\hline
% \textbf{Type} & Leaky integrate-and-fire, $\delta$-current input\\\hline
% \raisebox{-4.5ex}{\parbox{\textwidth}{\textbf{Subthreshold dynamics}}} &
% \rule{1em}{0em}\vspace*{-3.5ex}
%     \begin{equation*}
%       \begin{array}{r@{\;=\;}lll}
% \tau \dot{V}(t) & -V(t) + R I(t) & \text{if} & t > t^*+\tau_{\text{rp}} \\
%      V(t)       &  V_{\text{r}}  & \text{else} \\[2ex]
%      I(t)       & \multicolumn{3}{l}{\frac{\tau}{R} \sum_{\tilde{t}} w \delta(t-(\tilde{t}+\Delta))}
% \end{array}
%     \end{equation*}
% \vspace*{-2.5ex}\rule{1em}{0em}
%  \\\hline
 % \multirow{3}{*}{\textbf{Spiking}} &   If $V(t-)<\theta \wedge V(t+)\geq \theta$ \vspace*{-1ex}
 % \begin{enumerate}\setlength{\itemsep}{-0.5ex}
 % \item set $t^* = t$
 % \item emit spike with time-stamp $t^*$
 % \end{enumerate}
 % \vspace*{-4ex}\rule{1em}{0em} \\\hline
 % \end{tabularx}

\vspace{2ex}

% - E ------------------------------------------------------------------------------

\noindent
\begin{tabularx}{\textwidth}{|l|X|}\hline %{\textwidth}
\hdr{2}{E}{Optimisation} \\ \hline
      \textbf{Type}       & Principle-axis method \\\hline
   \textbf{Parameters}   & \\
 & $\wGLGDS \quad\to\quad [0.00001,0.05]\quad\mu{\rm S}$ \\
 & $\wANFDS \mathrm{ HSR }\quad [0.00001,0.05] \quad \mu{\rm S}$\\\hline
 & $\wANFDS \mathrm{ LSR }\quad [0.00001,0.05] \quad \mu{\rm S}$\\\hline
 & DS \GABAa synapse $ \tau_{GABA2} \quad [0.1,50.0]\quad {\rm ms}$\\\hline
 & DS leak conductance $ \bar{g}_{leak} \quad [0.00001,0.05] \quad \mathrm{Scm}^{-2}$\\\hline

\textbf{Fixed Parameters} & \\ \hline
$\nGLGDS = 25$ & \\ \hline
      $\wANFGLG$                    & \\ \hline
      $\nLSRGLG$                    & \\ \hline
      $\nHSRGLG$                    & \\ \hline
      $\sANFGLG $                   & \\ \hline
      $\dGLGDS = 0.5$ ms                    & Combination of conductance and synaptic delay. The effect of Golgi cells on DS is delayed by the extra 0.7~ms delay from ANF to Golgi, plus the slow peak of \GABAa inhibition.  \\\hline
  \textbf{Assumptions}    & The spread ANF to DS cells (\sANFDSh,\sANFDSl) is arbitrary at this point and will be explored in the next experiment.\\ \hline
   \textbf{Function}     & Weighted mean squared error see listing below  \\ \hline
\end{tabularx}
\vspace{2ex}




% - F -----------------------------------------------------------------------------

\noindent\begin{tabularx}{\textwidth}{|X|}\hline
\hdr{1}{F}{Measurements}\\\hline
PSTHs were generated from 25 stimulus repetitions. Each response to a click is measured for 2 ms after the minimum first spike latency for the unit.  The unit used in the optimisation has a CF = 5.8~kHz (channel no. 50).\\ \hline
\begin{minipage}[c]{0.6\textwidth}
\vspace{1cm}
DS Ouput \hspace{2in} Golgi Output
\includegraphics[width=0.5\textwidth]{DS_ClickRecovery_DSpsth}\label{Ch3:fig:DSClickRecoveryPSTH}\includegraphics[width=0.5\textwidth]{DS_ClickRecovery_Gpsth}\label{Ch3:fig:DSClickRecoveryPSTH}\\
  \captionsize{PSTH response of a D-stellate cell from the click recovery stimulus used in the optimisation.}
  \end{minipage}\\ \hline
\end{tabularx}

 % ---------------------------------------------------------------------------------
% \newpage
% \begin{lstlisting}
% func fun() {local f
%       //Modify Variables
%       param.w.x[glg][ds] = $2
%       param.w.x[hsr][ds] = $3
%       param.w.x[lsr][ds] = $3
%       //Modify the network
%       {create_cells() connect_cells(fileroot) SetRates()}
%       // Simulate the network for N reps
%       for j=0, reps-1{
%          print j
%          GenSpikes()
%          run()
%          DSvec.append(dstellate[50][0].spiketimes)
%          //print startsw()-x, "secs"
%       }
%       DSvec = DSvec.histogram(0,tstop,0.1)
%
%       objref errorvec
%       errorvec = new Vector()
%       //Find the mean number of spikes in the first click
%       maxrate = (DSvec.sum(240,260) + DSvec.sum(740,760)+ DSvec.sum(1340,1360))/3
%       //Calc ratio of number of spikes in second click relative to mean first click
%       errorvec.append( DSvec.sum(260,280) / maxrate )
%       errorvec.append( DSvec.sum(780,800) / maxrate )
%       errorvec.append( DSvec.sum(1420,1440) / maxrate )
%       errorvec.plot(g
%     return errorvec.meansqerr(targetclick)
% }
% \end{lstlisting}

\begin{tabularx}{\textwidth}{XXXX} 
\textbf{Parameters} &        \textbf{Range}         & \\ \hline
     \textbf{Parameters}      &  \textbf{Name}   & \textbf{Range} & \textbf{Best Values} \\\hline 
Weight of HSR syn on DS       &        \wHSRDS       &         [0.01,50] nS          &$0.532 \quad{\rm nS}$ \\          
Weight of LSR syn on DS       &        \wLSRDS       &         [0.01,50] nS          &$0.16 \quad{\rm nS}$\\            
\GABAa synapse rise constant  &    $\tau_{GABA1}$    &        [0.01,10.0] ms         &$13.1 \quad{\rm nS}$\\            
\GABAa synapse decay constant &    $\tau_{GABA2}$    &         [0.1,50.0] ms         &$5.432 \quad{\rm ms}$\\           
DS cell leak conductance      &   $\bar{g}_{leak}$   & [0.01,50] $\mu{\rm Scm}^{-2}$ &$0.262 \quad{\rm ms}$\\           
\end{tabularx}									      $0.0163 \quad\mathrm{Scm}^{-2}$\\ 

\begin{tabularx}{\textwidth}{lcX} 
\textbf{Fixed Parameters} & \textbf{Value}& \\ \hline
         \nGLGDS          &            25             & Number of GABA syn on DS cells                                                                                                                                                  \\
         \dGLGDS          &          0.5 ms           & Combination of conductance and synaptic delay.  \\
\end{tabularx}



\begin{figure}
\includegraphics[angle=-90,width=0.8\textwidth]{DSClickRecoveryExpData}\label{Ch3:fig:DSClickRecoveryExpData}
\caption{Experimental Data of GABAergic influence on D-stellate cells from \citep{BackoffPalombiEtAl:1997}, Fig.~3.}
\end{figure}

%\parsep

From the command line type:
\begin{verbatim}
$ ./i686/special DS_ClickRecovery.hoc
\end{verbatim}
in the \texttt{cnstellate} directory to simulate the optimisation for D-stellate click recovery.  The first run may take some time if the AN filters have not been previously saved, since the Zilany \& Bruce model requires 500~kHz resolution in the stimulus that is downsampled to 50~kHz.

\clearpage
\subsection{Results}


\noindent\begin{tabularx}{\textwidth}{|l|X|}\hline %{\textwidth}
\hdr{2}{D}{Results} \\\hline
\textbf{Best Parameters} &
{\begin{minipage}[c]{0.6\textwidth}
$\wGLGDS = 0.532 \quad{\rm nS}$ \\
 $\wHSRDS = 0.16 \quad{\rm nS}$\\
 $\wLSRDS = 13.1 \quad{\rm nS}$\\
 $\tau_{GABA2} = 5.432 \quad{\rm ms}$\\
$\tau_{GABA1} = 0.262 \quad{\rm ms}$\\
 $\bar{g}_{leak} = 0.0163 \quad\mathrm{Scm}^{-2}$\\
\end{minipage}}\\\hline
\textbf{Error} & 0.006671    unweighted (MSE of recovery spike rate / mask rate)\\\hline
& 0.01447    final result (MSE of recovery spike rate / mask rate)\\\hline
\end{tabularx}

\begin{figure}[hp!]
  \centering
\includegraphics[keepaspectratio=true,angle=-90,width=0.9\textwidth]{./gfx/DS_ClickRecovery_result.eps}
%\includegraphics[keepaspectratio,angle=-90,width=0.8\textwidth]{./gfx/DSClickRecoveryExpData}\\
\caption{Experimental Data ({\color{green} Green}) of GABAergic influence on D-stellate cells from \citep{BackoffPalombiEtAl:1997}, Fig.~3.  Best result ({\color{blue} Blue}) shown in figure below. }
\label{fig:DS_ClickRecovery_result}  
\end{figure}


% \begin{figure}
%   \includegraphics[width=0.5\textwidth]{DS_ClickRecovery_OptVars.eps}\\
% %  \includegraphics[width=0.5\textwidth]{DS_ClickRecovery_Output.eps}\label{Ch3:fig:DSClickRecoveryOutput}
%   \caption{Final Output Data of the D-stellate Click Recovery optimisation }
% \end{figure}

% \begin{figure}
%   \includegraphics[keepaspectratio=true,width=0.8\textwidth]{DS_ClickRecovery_Example1.eps}\\
%   \includegraphics[keepaspectratio=true,width=0.8\textwidth]{DS_ClickRecovery_Example10.eps}\\
%   \includegraphics[keepaspectratio=true,width=0.8\textwidth]{DS_ClickRecovery_Example13.eps}\\
%   \includegraphics[keepaspectratio=true,width=0.8\textwidth]{DS_ClickRecovery_Example19.eps}\\
%   \caption{Click Recovery optimisation functions}
% \end{figure}


% \begin{figure}
%   \includegraphics[keepaspectratio=true,angle=-90,width=0.8\textwidth]{DS_ClickRecovery_result.eps}\\
% \end{figure}


% \begin{figure}
%   \includegraphics[keepaspectratio=true,angle=-90,width=0.8\textwidth]{DS_ClickRecovery_result1.eps}\\
% \end{figure}


% \begin{figure}
%   \includegraphics[keepaspectratio=true,angle=-90,width=0.8\textwidth]{DS_ClickRecovery_result2.eps}\\
%   \caption{Click Recovery optimisation }
% \end{figure}




% \begin{figure}
% \begin{center}
% \includegraphics[keepaspectratio=true]{DS_ClickRecovery_handtuned.eps}\\
% \includegraphics[keepaspectratio=true,angle=-90,width=0.8\textwidth]{DS_ClickRecovery_result_handtuned.eps}
% \caption{Handtuned}
% \label{hantuned}
% \end{center}
% \end{figure}

% \begin{figure}
% \begin{center}
% %\includegraphics[keepaspectratio=true]{DS_ClickRecovery_handtuned.eps}\\
% \includegraphics[keepaspectratio=true,angle=-90,width=0.8\textwidth]{gfx/DS_ClickRecovery_result_unweighted_8.eps}\\
% \includegraphics[keepaspectratio=true,angle=-90,width=0.8\textwidth]{gfx/DS_ClickRecovery_result_weighted_0.eps}
% \caption{Handtuned}
% \label{hantuned}
% \end{center}
% \end{figure}


%% Example optimisation points used by praxis 
% \begin{figure}
% \begin{center}
% %\includegraphics[keepaspectratio=true]{DS_ClickRecovery_handtuned.eps}\\
% \includegraphics[keepaspectratio=true,width=0.5\textwidth]{Praxis_123.eps}
% \includegraphics[keepaspectratio=true,width=0.5\textwidth]{Praxis_456.eps}
% \caption{Handtuned}
% \label{hantuned}
% \end{center}
% \end{figure}


% \clearpage
\newpage
\subsection{Verification}

% \subsection{Tone Response}
% \begin{figure}[h!]
% \centering\resizebox{\textwidth}{!}{%
% \includegraphics{RateLevel/psthsingle90.2.eps}%
% \includegraphics{RateLevel/DS_ratelevel.eps}}
% \end{figure}
% \begin{figure}[h!]
% \centering\resizebox{\textwidth}{!}{%
% \includegraphics{RateLevel/response_area.2.eps}%
% \includegraphics{RateLevel/response_area_log2.2.eps}}
% \end{figure}
% \begin{figure}[h!]
% \centering\resizebox{\textwidth}{!}{%
% %\includegraphics{RateLevel/response_area.2.eps}
% \includegraphics{RateLevel/psthall90.2.eps}%
% \includegraphics{RateLevel/psthVlevel.2.eps}}
% \end{figure}


% \clearpage
% \subsection{Noise Response}
% \begin{figure}[h!]
% \centering\resizebox{\textwidth}{!}{%
% \includegraphics{NoiseRateLevel/psthsingle120.2.eps}%
% \includegraphics{NoiseRateLevel/DS_ratelevel.eps}}
% \end{figure}
% \begin{figure}[h!]
% \centering\resizebox{\textwidth}{!}{%
% \includegraphics{NoiseRateLevel/response_area.2.eps}%
% \includegraphics{NoiseRateLevel/response_area_log2.2.eps}}
% \end{figure}
% \begin{figure}[h!]
% \centering\resizebox{\textwidth}{!}{%
% %\includegraphics{RateLevel/response_area.2.eps}
% \includegraphics{NoiseRateLevel/psthall90.2.eps}%
% \includegraphics{NoiseRateLevel/psthVlevel.2.eps}}
% \end{figure}


% \clearpage
% \subsection{Masked Noise and Tone}
% \begin{figure}[h!]
% \centering\resizebox{\textwidth}{!}{\includegraphics{MaskedRateLevel/psthsingle90.2.eps}\includegraphics{MaskedRateLevel/DS_ratelevel.eps}}
% \end{figure}
% \begin{figure}[h!]
% \centering\resizebox{\textwidth}{!}{%
% \includegraphics{MaskedRateLevel/response_area.2.eps}%
% \includegraphics{MaskedRateLevel/response_area_log2.2.eps}}
% \end{figure}

% \begin{figure}[h!]
% \centering\resizebox{\textwidth}{!}{%
% %\includegraphics{RateLevel/response_area.2.eps}
% \includegraphics{MaskedRateLevel/psthall90.2.eps}%
% \includegraphics{MaskedRateLevel/psthVlevel.2.eps}}
% \end{figure}
% \clearpage
% \subsection{Masked Response Area}
% \begin{figure}[h!]
% \centering\resizebox{\textwidth}{!}{%
% \includegraphics{MaskedResponseCurve/psthsingle5810.2.eps}%
% \includegraphics{MaskedResponseCurve/DS_masked.eps}}
% \end{figure}
% \begin{figure}[h!]
% \centering\resizebox{\textwidth}{!}{%
% \includegraphics{MaskedResponseCurve/response_area.2.eps}%
% \includegraphics{MaskedResponseCurve/response_area_log2log2.2.eps}}
% \end{figure}

% \begin{figure}[h!] 
% \centering\resizebox{\textwidth}{!}{%
% %\includegraphics{RateLevel/response_area.2.eps}
% \includegraphics{MaskedResponseCurve/psthall5810.2.eps}%
% \includegraphics{MaskedResponseCurve/psthVmod.2.eps}}
% \end{figure}
% \clearpage

 

%%% Local Variables: 
%%% mode: latex
%%% TeX-master: "SimpleResponses"
%%% TeX-PDF-mode: nil
%%% End: 
