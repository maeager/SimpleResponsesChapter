

% % LaTeX file for generating the Model Description Table in Fig. 5 of
% %
% %  Nordlie E, Gewaltig M-O, Plesser HE (2009)
% %  Towards Reproducible Descriptions of Neuronal Network Models.
% %  PLoS Comput Biol 5(8): e1000456.
% %
% %  Paper URL : http://dx.doi.org/10.1371/journal.pcbi.1000456
% %  Figure URL: http://dx.doi.org/10.1371/journal.pcbi.1000456.g005
% %
% % This file is released under a
% %
% %   Creative Commons Attribution, non-commercial, share-alike licence
% %   http://creativecommons.org/licenses/by-nc-sa/3.0/de/deed.en
% %
% % with the following specifications:
% %
% %  1. When publishing tables generated from this LaTeX file and modified
% %     versions of it, you must cite the paper by Nordlie et al given above.
% %
% %  2. The non-commercial clause applies only to the distribution of THIS FILE
% %     and LaTeX source code files derived from it. You may commercially publish
% %     documents generated using this file and derivatived versions of this file.
% %
% % Contact: Hans Ekkehard Plesser, UMB (hans.ekkehard.plesser at umb.no)

% \documentclass{article}
% % User defined commands
% 
%%
%% Custom Hyphenations
%%
%%
\hyphenation{cross-talk au-di-tory adap-t-a-tion phe-nom-en-o-lo-g-i-cal
  syn-a-pse co-inc-id-ence Tub-er-culo-vent-ral glyc-in-ergic psycho-phys-ical
  asym-met-ric ex-plor-at-ory pot-as-sium op-ti-mi-sa-tion au-di-t-ory system
  Neuro-in-for-mat-ics mar-g-in-al par-a-m-eters Rh-ode Neu-ro-fit-ter
  elec-tro-phys-io-log-i-cal in-fer-ring the con-nec-tiv-i-ty with-in
  non-lin-ear ap-pr-oa-ch-es stel-late mi-cro-cir-cuit show-ing synap-tic
  in-ter-ac-tion iso-lam-inar pop-u-la-tion evo-l-ved com-part-ment
  con-duc-tance mod-els Hod-g-kin Hux-l-ey gluta-m-at-er-gic gen-o-mes
  Theu-nis-sen re-sp-on-se co-inc-id-ence det-e-ctor exp-eri-men-tal
  ac-cu-mu-lated neu-ro-science cre-ate de-tailed ex-ten-sively stud-ied
  In-tra-cel-lu-lar neur-rons in-tra-cel-lu-lar in-ves-ti-ga-tion se-quen-tial
  de-ter-min-ed cat-e-gor-is-ed im-me-di-ate sen-si-tiv-ity bicu-cu-line
  mul-ti-ple in-hib-it-ory com-mis-sural path-way re-cip-ro-cal fre-quency
  po-si-tion func-tion de-lay prep-a-ra-tions iso-lated oct-o-pus in-sen-si-tive
  Chop-per im-preg-na-tion phys-i-o-log-i-cal ef-fect GABA-er-gic in-puts on-set
  chop-pers pri-mar-ily  gen-er-ate ar-ray gaus-sian ran-dom num-bers
  in-ves-ti-gat-ing im-p-or-tant phys-i-o-log-i-cal mech-a-nisms}
%%
%% Sample custom-configuration
%%
%%   You are encouraged to modify the following section with any of your
%%   own custom commands, packages, etc.
%%

%error 'You should modify this section and remove this error.'

% for URLs
\usepackage{url}

% AMS packages
%\usepackage{amsfonts}
\usepackage{amssymb}
\usepackage[fleqn]{amsmath}   % displayed equations flush left
%\setlength{\mathindent}{0em}
%\usepackage{amsthm}
\usepackage[mathscr]{eucal}

\newcommand{\vect}[1]{\mathbf{#1}}


% Allow equations to break over pages...
\interdisplaylinepenalty=2500
% Command to stop equation breaks
% Note: enclose this in braces when used...
\newcommand{\donotsplitoverpages}{\interdisplaylinepenalty=10000}

%% Graphics
% \ifx\pdftexversion\undefined
%  \usepackage[dvips]{graphicx}
% \else
%  \usepackage[pdftex]{graphicx}
% \fi

%% My Graphics and Hyperlinks stuff
\usepackage{ifpdf}
 \ifpdf
   \pdfoutput=1
   \usepackage[pdftex]{graphicx}  % uncomment if using graphicx
\usepackage[final,          % override "draft" which means "do nothing"
            colorlinks,     % rather than outlining them in boxes
            linkcolor=black, % override truly awful colour choices
            citecolor=black, %   (ditto)
            urlcolor=black,  %   (ditto)
            ]{hyperref}

 \ifx\pdfoutput\undefined \usepackage[ps2pdf,
 bookmarks=true,
 bookmarksnumbered=true,
 breaklinks=true,
            final,          % override "draft" which means "do nothing"
            colorlinks,     % rather than outlining them in boxes
            linkcolor=black, % override truly awful colour choices
            citecolor=black, %   (ditto)
            urlcolor=black,  %   (ditto)
            ]{hyperref}

% \usepackage[pdftex]{hyperref}  % uncomment if using hyperref
%  \usepackage[ps2pdf]{thumbpdf}
 \DeclareGraphicsExtensions{.eps,.bmp}
  \else
 \DeclareGraphicsExtensions{.png,.pdf,.jpg,.JPEG}
  \usepackage{epstopdf}
 %\usepackage[pdftex,bookmarks=true,bookmarksnumbered=true,breaklinks=true]{hyperref}
  \pdfadjustspacing=1
  \usepackage[pdftex]{thumbpdf}
  \fi
 \else
   \usepackage[dvips]{graphicx}  % uncomment if using graphicx
    % comment if not using hyperref
 \usepackage[final,          % override "draft" which means "do nothing"
            colorlinks,     % rather than outlining them in boxes
            linkcolor=black, % override truly awful colour choices
            citecolor=black, %   (ditto)
            urlcolor=black,  %   (ditto)
             ]{hyperref}
 \DeclareGraphicsExtensions{.eps,.bmp}
\fi
% Enable IEEE macros
%\usepackage{IEEEtrantools}

% Use a plain bibliography style
%\bibliographystyle{plain}
% Use the IEEE bibliography style (sorted)
%\bibliographystyle{IEEEtrans}
% Use the IEEE bibliography style (unsorted; order of reference)
%\bibliographystyle{IEEEtran}

% For isolated bibliographies
\usepackage{bibunits}

\usepackage{color}
%\usepackage[noadjust]{cite}
\usepackage{caption}
\usepackage{breakurl} % necessary to break URLs when using LaTeX-> dvips -> Ps2PDF, must be after hyperref

% For cool tables
\usepackage{array}
\usepackage{tabularx}  % automatically adjusts column width in tables
\usepackage{multirow}  % allows entries spanning several rows
\usepackage{colortbl}  % allows coloring tables


% For algorithms
%\usepackage{algorithm}
%\usepackage{algorithmic}

% For cases
\usepackage{sublabel}

% For theroem numbers having the chapter included
%\usepackage{style/chngcntr}

% For cool theorem styles
%\usepackage[amsthm]{ntheorem}
%%\theorembodyfont{\normalfont}
%
%% Theorem definition
%\newtheorem{theorem}{Theorem}
%\counterwithin{theorem}{chapter}
%
%% Corollary definition
%\newtheorem{corollary}{Corollary}
%\counterwithin{corollary}{chapter}
%
%% Result definition
%\newtheorem{result}{Result}
%\counterwithin{result}{chapter}
%
%% Lemma definition
%\newtheorem{lemma}{Lemma}
%\counterwithin{lemma}{chapter}
%
%% Proposition definition
%\newtheorem{proposition}{Proposition}
%\counterwithin{proposition}{chapter}
%
%% Definition definition!
%\newtheorem{definition}{Definition}
%\counterwithin{definition}{chapter}
%
%% Remark definition (no counter?)
%\newenvironment{remark}{\emph{Remark:~}}{}
%
%% Fact definition (no counter?)
%\newenvironment{fact}{\emph{Fact:~}}{}

% (Re)Set the figure path
\newcommand{\setfigurepath}[1]{%
\ifx\figurepath\undefined
	\newcommand{\figurepath}{#1}
\else
	\renewcommand{\figurepath}{#1}
\fi%
}

% Used in the continued list environment below
\newcounter{continuedlist}

% Continued list environment
\newenvironment{continuedlist}{%
	\begin{enumerate}%
		% Space out each item
		\setlength{\itemsep}{1.25em}%
		% Start the enumeration from the previous value
		\setcounter{enumi}{\value{continuedlist}} %
}{ %
  % Save the counter to continue it later
  \setcounter{continuedlist}{\value{enumi}}%
  \end{enumerate}%
  % \vspace{1.25em}% 
  \vspace{1em}%
}

% Spaced out list environment
\newenvironment{spacedoutlist}{%
	\begin{itemize}%
		% Space out each item
		\setlength{\itemsep}{1.25em}%
}{\end{itemize}}


%% My added packages


\usepackage[usenames,dvipsnames]{xcolor}
\definecolor{halfgray}{gray}{0.55}

\usepackage{listings}
\lstset{language=C++,%[LaTeX]Tex,%
    keywordstyle=\color{RoyalBlue},%\bfseries,
    basicstyle=\small\sffamily,
    identifierstyle=\color{NavyBlue},
    commentstyle=\color{Green}\rmfamily,
    stringstyle=\sffamily,
    numbers=left,%none,%
    numberstyle=\scriptsize,%\tiny
    stepnumber=5,
    numbersep=8pt,
    showstringspaces=false,
    breaklines=true,
    %frameround=ftff,
    %frame=single
    %frame=L
    lineskip=-5pt
}


% \lstset{language=Octave,                % choose the language of the code
% basicstyle=\footnotesize,       % the size of the fonts that are used for the code
% numbers=left,                   % where to put the line-numbers
% numberstyle=\footnotesize,      % the size of the fonts that are used for the line-numbers
% stepnumber=2,                   % the step between two line-numbers. If it's 1 each line will be numbered
% numbersep=5pt,                  % how far the line-numbers are from the code
% backgroundcolor=\color{white},  % choose the background color. You must add \usepackage{color}
% showspaces=false,               % show spaces adding particular underscores
% showstringspaces=false,         % underline spaces within strings
% showtabs=false,                 % show tabs within strings adding particular underscores
% frame=single,			% adds a frame around the code
% tabsize=2,			% sets default tabsize to 2 spaces
% captionpos=b,			% sets the caption-position to bottom
% breaklines=true,		% sets automatic line breaking
% breakatwhitespace=false,	% sets if automatic breaks should only happen at whitespace
% lineskip=-5pt  %
% }

\ifx\setcitestyle\undefined
\usepackage[sort,round,authoryear]{natbib}
\setcitestyle{aysep={}} % J Neurophys formatting
\fi

\usepackage{xspace}
\usepackage{rotating}
\usepackage{tikz}
\usepackage{calc}

\newcommand{\hdr}[3]{%
\multicolumn{#1}{|l|}{\color{white}\cellcolor[gray]{0.0}%
\textbf{\makebox[0.05\linewidth][l]{#2}\hspace{0.45\linewidth}\makebox[0pt][c]{#3}}%
%\textbf{\makebox[0pt]{#2}\hspace{0.5\linewidth}\makebox[0pt][c]{#3}}%
}}
 % Nordelie table environment
\newenvironment{ntab}[4]{
\noindent\begin{tabularx}{\linewidth}{#1}\hline 
\multicolumn{#2}{|l|}{\color{white}\cellcolor[gray]{0.0}\textbf{#3\hfill{}{#4}\hfill{}}}
}{
\end{tabularx}
\vspace{1ex}
}



\usepackage[colorinlistoftodos,backgroundcolor=yellow!35,textsize=footnotesize]{todonotes}
\newcommand{\yellownote}[1]{\todo[inline]{#1}}


\usepackage{scrtime}
%\usepackage{mparhack}

%\setlength{\parskip}{0ex} or {1.2ex}
%\setlength{\parindent}{0em} or {0ex}

\usepackage[australian]{babel}

\usepackage{booktabs,ltxtable,ctable,dcolumn}
\newcommand{\otoprule}{\midrule[\heavyrulewidth]}

\usepackage{doipubmed}

% For subfigures
\usepackage{keyval}
\usepackage[config,labelfont={sf,bf}]{subfig}
%\captionsetup[table]{position=top}
%\captionsetup[subtable]{position=top}
%\usepackage[heightadjust=all,valign=t]{floatrow}
%\usepackage{fr-subfig}
%\floatsetup{style=Plaintop}
%\usepackage{subfigure}

\usepackage{lscape}

\newcommand{\code}[1]{\mbox{\normalfont\texttt{#1}}}
\newcommand{\progname}[1]{\mbox{\normalfont\textsf{#1}}}
\newcommand{\figfont}[1]{\large{\textbf{\textsf{#1}}}}


% %% Glossary

% ANF to Golgi
\newcommand{\ANFGLG}{\protect\ensuremath{\mbox{ANF} \to \mbox{GLG}\xspace}}
\newcommand{\HSRGLG}{\protect\ensuremath{\mbox{HSR} \to \mbox{GLG}\xspace}}
\newcommand{\LSRGLG}{\protect\ensuremath{\mbox{LSR} \to \mbox{GLG}\xspace}}
\newcommand{\wANFGLG}{\protect\ensuremath{w_{\ANFGLG}\xspace}}
\newcommand{\wLSRGLG}{\protect\ensuremath{w_{\LSRGLG}\xspace}}
\newcommand{\wHSRGLG}{\protect\ensuremath{w_{\HSRGLG}\xspace}}
\newcommand{\nLSRGLG}{\protect\ensuremath{n_{\LSRGLG}\xspace}}
\newcommand{\nHSRGLG}{\protect\ensuremath{n_{\HSRGLG}\xspace}}
\newcommand{\sANFGLG}{\protect\ensuremath{s_{\ANFGLG}\xspace}}
\newcommand{\sLSRGLG}{\protect\ensuremath{s_{\LSRGLG}\xspace}}
\newcommand{\sHSRGLG}{\protect\ensuremath{s_{\HSRGLG}\xspace}}
\newcommand{\dANFGLG}{\protect\ensuremath{d_{\ANFGLG}\xspace}}

%ANF to D-stellate
\newcommand{\ANFDS}{\protect\ensuremath{\mbox{ANF} \to \mbox{DS}\xspace}}
\newcommand{\HSRDS}{\protect\ensuremath{\mbox{HSR} \to \mbox{DS}\xspace}}
\newcommand{\LSRDS}{\protect\ensuremath{\mbox{LSR} \to \mbox{DS}\xspace}}
\newcommand{\wANFDS}{\ensuremath{w_{\ANFDS}\xspace}}
\newcommand{\wLSRDS}{\protect\ensuremath{w_{\LSRDS}\xspace}}
\newcommand{\wHSRDS}{\protect\ensuremath{w_{\HSRDS}\xspace}}
\newcommand{\nLSRDS}{\protect\ensuremath{n_{\LSRDS}\xspace}}
\newcommand{\nHSRDS}{\protect\ensuremath{n_{\HSRDS}\xspace}}
\newcommand{\dANFDS}{\protect\ensuremath{d_{\ANFDS}\xspace}}
\newcommand{\sANFDSh}{\protect\ensuremath{s^+_{\ANFDS}\xspace}}
\newcommand{\sANFDSl}{\protect\ensuremath{s^-_{\ANFDS}\xspace}}

%ANF to T-stellate
\newcommand{\ANFTS}{\protect\ensuremath{\mbox{ANF} \to \mbox{TS}\xspace}}
\newcommand{\HSRTS}{\protect\ensuremath{\mbox{HSR} \to \mbox{TS}\xspace}}
\newcommand{\LSRTS}{\protect\ensuremath{\mbox{LSR} \to \mbox{TS}\xspace}}
\newcommand{\wANFTS}{\protect\ensuremath{w_{\ANFTS}\xspace}}
\newcommand{\nLSRTS}{\protect\ensuremath{n_{\LSRTS}\xspace}}
\newcommand{\nHSRTS}{\protect\ensuremath{n_{\HSRTS}\xspace}}
\newcommand{\sANFTS}{\protect\ensuremath{s_{\ANFTS}\xspace}}
\newcommand{\dANFTS}{\protect\ensuremath{d_{\ANFTS}\xspace}}

%ANF to Tuberculoventral
\newcommand{\ANFTV}{\ensuremath{\mbox{ANF} \to \mbox{TV}\xspace}}
\newcommand{\HSRTV}{\ensuremath{\mbox{HSR} \to \mbox{TV}\xspace}}
\newcommand{\LSRTV}{\ensuremath{\mbox{LSR} \to \mbox{TV}\xspace}}
\newcommand{\wANFTV}{\ensuremath{w_{\ANFTV}\xspace}}
\newcommand{\nLSRTV}{\ensuremath{n_{\LSRTV}\xspace}}
\newcommand{\nHSRTV}{\ensuremath{n_{\HSRTV}\xspace}}
\newcommand{\sANFTV}{\ensuremath{s_{\ANFTV}\xspace}}
\newcommand{\dANFTV}{\ensuremath{d_{\ANFTV}\xspace}}

%GLG to T-stellate
\newcommand{\GLGTS}{\protect\ensuremath{\mbox{GLG} \to \mbox{TS}\xspace}}
\newcommand{\wGLGTS}{\protect\ensuremath{w_{\GLGTS}\xspace}}
\newcommand{\nGLGTS}{\protect\ensuremath{n_{\GLGTS}\xspace}}
\newcommand{\sGLGTS}{\protect\ensuremath{s_{\GLGTS}\xspace}}
\newcommand{\dGLGTS}{\protect\ensuremath{d_{\GLGTS}\xspace}}
%GLG to D-stellate
\newcommand{\GLGDS}{\protect\ensuremath{\mbox{GLG} \to \mbox{DS}\xspace}}
\newcommand{\wGLGDS}{\protect\ensuremath{w_{\GLGDS}\xspace}}
\newcommand{\nGLGDS}{\protect\ensuremath{n_{\GLGDS}\xspace}}
\newcommand{\sGLGDS}{\protect\ensuremath{s_{\GLGDS}\xspace}}
\newcommand{\dGLGDS}{\protect\ensuremath{d_{\GLGDS}\xspace}}

% % TS to Golgi
% \newcommand{\GLGTS}{\protect\ensuremath{\mbox{GLG} \to \mbox{TS}\xspace}}
% \newcommand{\wTSGLG}{\protect\ensuremath{w_{}}\xspace}}
% \newcommand{\nTSGLG}{\protect\ensuremath{n_{\mbox{TS} \to \mbox{GLG}}\xspace}}
% \newcommand{\sTSGLG}{\protect\ensuremath{s_{\mbox{TS} \to \mbox{GLG}}\xspace}}
% \newcommand{\dTSGLG}{\protect\ensuremath{d_{\mbox{TS} \to \mbox{GLG}}\xspace}}

%TS to D-stellate
\newcommand{\TSDS}{\protect\ensuremath{\mbox{TS} \to \mbox{DS}\xspace}}
\newcommand{\wTSDS}{\protect\ensuremath{w_{\TSDS}\xspace}}
\newcommand{\nTSDS}{\protect\ensuremath{n_{\TSDS}\xspace}}
\newcommand{\sTSDS}{\protect\ensuremath{s_{\TSDS}\xspace}}
\newcommand{\dTSDS}{\protect\ensuremath{d_{\TSDS}\xspace}}
%TS to T-stellate
\newcommand{\TSTS}{\protect\ensuremath{\mbox{TS} \to \mbox{TS}\xspace}}
\newcommand{\wTSTS}{\protect\ensuremath{w_{\TSTS}\xspace}}
\newcommand{\nTSTS}{\protect\ensuremath{n_{\TSTS}\xspace}}
\newcommand{\sTSTS}{\protect\ensuremath{s_{\TSTS}\xspace}}
\newcommand{\dTSTS}{\protect\ensuremath{d_{\TSTS}\xspace}}

%TS to Tuberculoventral
\newcommand{\TSTV}{\protect\ensuremath{\mbox{TS} \to \mbox{TV}\xspace}}
\newcommand{\wTSTV}{\protect\ensuremath{w_{\TSTV}\xspace}}
\newcommand{\nTSTV}{\protect\ensuremath{n_{\TSTV}\xspace}}
\newcommand{\sTSTV}{\protect\ensuremath{s_{\TSTV}\xspace}}
\newcommand{\dTSTV}{\protect\ensuremath{d_{\TSTV}\xspace}}

% DS to Golgi
\newcommand{\DSGLG}{\protect\ensuremath{\mbox{DS} \to \mbox{GLG}\xspace}}
\newcommand{\wDSGLG}{\protect\ensuremath{w_{\DSGLG}\xspace}}
\newcommand{\nDSGLG}{\protect\ensuremath{n_{\DSGLG}\xspace}}
\newcommand{\sDSGLG}{\protect\ensuremath{s_{\DSGLG}\xspace}}
\newcommand{\dDSGLG}{\protect\ensuremath{d_{\DSGLG}\xspace}}

% %DS to D-stellate
% \newcommand{\DSDS}{\protect\ensuremath{\mbox{DS} \to \mbox{DS}\xspace}}
% \newcommand{\wDSDS}{\protect\ensuremath{w_{\DSDS}\xspace}}
% \newcommand{\nDSDS}{\protect\ensuremath{n_{\DSDS}\xspace}}
% \newcommand{\sDSDS}{\protect\ensuremath{s_{\DSDS}\xspace}}
% \newcommand{\dDSDS}{\protect\ensuremath{d_{\DSDS}\xspace}}

%DS to T-stellate
\newcommand{\DSTS}{\protect\ensuremath{\mbox{DS} \to \mbox{TS}\xspace}}
\newcommand{\wDSTS}{\protect\ensuremath{w_{\DSTS}\xspace}}
\newcommand{\nDSTS}{\protect\ensuremath{n_{\DSTS}\xspace}}
\newcommand{\sDSTS}{\protect\ensuremath{s_{\DSTS}\xspace}}
\newcommand{\dDSTS}{\protect\ensuremath{d_{\DSTS}\xspace}}

%DS to Tuberculoventral
\newcommand{\DSTV}{\protect\ensuremath{\mbox{DS} \to \mbox{TS}\xspace}}
\newcommand{\wDSTV}{\protect\ensuremath{w_{\DSTV}\xspace}}
\newcommand{\nDSTV}{\protect\ensuremath{n_{\DSTV}\xspace}}
\newcommand{\sDSTV}{\protect\ensuremath{s_{\DSTV}\xspace}}
\newcommand{\dDSTV}{\protect\ensuremath{d_{\DSTV}\xspace}}
\newcommand{\oDSTV}{\protect\ensuremath{o_{\DSTV}\xspace}}

% TV to Golgi
\newcommand{\TVGLG}{\protect\ensuremath{\mbox{TV} \to \mbox{GLG}\xspace}}
\newcommand{\wTVGLG}{\protect\ensuremath{w_{\TVGLG}\xspace}}
\newcommand{\nTVGLG}{\protect\ensuremath{n_{\TVGLG}\xspace}}
\newcommand{\sTVGLG}{\protect\ensuremath{s_{\TVGLG}\xspace}}
\newcommand{\dTVGLG}{\protect\ensuremath{d_{\TVGLG}\xspace}}

%TV to D-stellate
\newcommand{\TVDS}{\protect\ensuremath{\mbox{TV} \to \mbox{DS}\xspace}}
\newcommand{\wTVDS}{\protect\ensuremath{w_{\TVDS}\xspace}}
\newcommand{\nTVDS}{\protect\ensuremath{n_{\TVDS}\xspace}}
\newcommand{\sTVDS}{\protect\ensuremath{s_{\TVDS}\xspace}}
\newcommand{\dTVDS}{\protect\ensuremath{d_{\TVDS}\xspace}}

%TV to T-stellate
\newcommand{\TVTS}{\protect\ensuremath{\mbox{TV} \to \mbox{TS}\xspace}}
\newcommand{\wTVTS}{\protect\ensuremath{w_{\TVTS}\xspace}}
\newcommand{\nTVTS}{\protect\ensuremath{n_{\TVTS}\xspace}}
\newcommand{\sTVTS}{\protect\ensuremath{s_{\TVTS}\xspace}}
\newcommand{\dTVTS}{\protect\ensuremath{d_{\TVTS}\xspace}}

%TV to Tuberculoventral
\newcommand{\TVTV}{\protect\ensuremath{\mbox{TV} \to \mbox{TV}\xspace}}
\newcommand{\wTVTV}{\protect\ensuremath{w_{\TVTV}\xspace}}
\newcommand{\nTVTV}{\protect\ensuremath{n_{\TVTV}\xspace}}
\newcommand{\sTVTV}{\protect\ensuremath{s_{\TVTV}\xspace}}
\newcommand{\dTVTV}{\protect\ensuremath{d_{\TVTV}\xspace}}


%Other common symbols
\newcommand{\GABAa}{\ensuremath{\mbox{GABA}_\mbox{A}}}

% \usepackage[margin=0.2in]{geometry} % get enough space on page

% \usepackage{tabularx}  % automatically adjusts column width in tables
% \usepackage{multirow}  % allows entries spanning several rows
% \usepackage{colortbl}  % allows coloring tables
% %\usepackage{natbib}
% %\usepackage[fleqn]{amsmath}   % displayed equations flush left
% %\setlength{\mathindent}{0em}

% % use Helvetica for text, Pazo math fonts
% \usepackage{mathpazo}
% \usepackage[scaled=.95]{helvet}
% \renewcommand\familydefault{\sfdefault}

% \renewcommand\arraystretch{1.2}  % slightly more space in tables

% \pagestyle{empty}  % no header of footer

% % \hdr{ncols}{label}{title}
% %
% % Typeset header bar across table with ncols columns
% % with label at left margin and centered title
% %


% \graphicspath{{/media/data/Work/cnstellate/DS_ClickRecovery/}{/media/data/Work/Responses/}}

% \begin{document}
\graphicspath{{/media/data/Work/cnstellate/DS_ClickRecovery/}{/media/data/Work/Responses/}}
\section{D-stellate cell click recovery: }
% - A ------------------------------------------------------------------------------

Onset-chopper units in the mammalian VCN have a wide-ranging influence
on the primary cells of the VCN (stellate and bushy cells), the
ipsilateral DCN and the contralateral CN \citep{NeedhamPaolini:2007}.  

This experiment shows the GABAergic input and intrinsic
cell properties  influence the behaviour Onset chopper units


** D Stellate Cell Model:  Experimental evidence
Morphological 
  - large Multipolar cell
  - electrotonic dendrites stretching 200 microns (one third of VCN), one half DCN, cCN 
  - receive large number of ANF syn to dend and soma
  - receive large number of Gly and GABA syn to soma dend
Intracellular
  + type I-II current clamp response
  + presence of Ih and small amounts of KLT currents
  + delay from shock to ANFs less than 1 ms 
Physiological 
  - Onset chopper PSTH, fast narrow FSL 
  - Wide response area (+1 oct and -2 oct)
  - high sync index, Low-pass MTF
  - monotonic response to tones and noise 

** D Stellate Cell Model:  Key factors in designing D-stellate cell model

Choosing neural model: type I-II Rothman and Manis model
  - with/without dendrites
  - variable KLT, leak conductance

Problems
  - ANF spread to DS cells well documented (decision made to
    fix params due to large computational task of calc response area) 
  - short delay recovery responses (2,3,4 ms) were not successful upon
    first model, included DS leak and KLT conductances to allow cell
    behaviour to be fit
  - The effect of Golgi cells on DS is delayed by the extra 0.7 ms delay from ANF to Golgi, plus the slow peak of \GABAa inhibition.


%%%%%%%%%%%%%%%%%%%Copied from original jneurometh article

DS cells have an onset-chopping (On-C) PSTH to tones and noise
\citep{SmithRhode:1989,NeedhamPaolini:2006}. Intracellular responses to sounds
indicate the bandwidth of inputs to DS neurons typically ranges from two
octaves below CF to one octave above CF
\citep{PalmerJiangEtAl:1996,PaoliniClark:1999}. DS cell axon terminals contain
the inhibitory neurotransmitter glycine and synapse widely in the VCN and DCN.
They also send a commissural projection to the contralateral cochlear nucleus
that mediates fast inhibition between the nuclei
\citep{NeedhamPaolini:2003,NeedhamPaolini:2006}. Tuberculoventral  (TV) cells
are glycinergic, inhibitory cells found in the deep layers of the DCN that send
axon collaterals to the VCN. They are characterized as having a non-monotonic
response to tones with increasing sound level and respond poorly to broadband
noise \citep{SpirouDavisEtAl:1999,NelkenYoung:1997,ReissYoung:2005}.
Anterograde labeling in the DCN suggests TV cells project tonotopically to the
AVCN not just on-CF, but also to the low and high frequency side bands in the
AVCN \citep{MunirathinamOstapoffEtAl:2004,OstapoffMorestEtAl:1999}.
Ultra-structural labeling of synapses in the rat DCN suggest TV cells are
inhibited by glycinergic DS cells and from sources in the DCN but excitatory
inputs were not found from TS cells in the rat \citep{Rubio:2005}. Evidence in
the mouse suggests otherwise since intracellular responses from labeled TV
cells in the mouse show clear excitatory input from TS cells and diffuse
inhibitory input from DS cells
\citep{ZhangOertel:1993b,WickesbergOertel:1993}.
%%%%%%%%%%%%%%%%%%%%%%%%%%%%%%%%%%%%%%%%%%%%%%%%



\subsection{Implementation}
\textbf{Optimisation}    
 Parameters for \GLGDS are optimised based on experimental click recovery data from \citep{BackoffPalombiEtAl:1997}.

\noindent
\begin{tabularx}{0.95\textwidth}{|l|X|}\hline %
%
\hdr{2}{A}{Model Summary}\\\hline
\textbf{Populations}     & ANF (HSR,LSR), Golgi, D-stellate \\\hline
\textbf{Topology}        & Tonotopic,  Auditory system of the rat  \\\hline
\textbf{Connectivity}    & Gaussian spread dependent on morphology and afferent connections  \\\hline
\textbf{Auditory model}  & \citep{ZilanyBruce:2008} ANF phenomenological instantaneous-rate Poisson spike trains\\\hline
\textbf{Neuron model}    &\begin{minipage}{0.5\textwidth}
Golgi \begin{itemize}
\item instantaneous-rate Poisson spike trains
\item weighted sum of LSR instantaneous-rate vectors
\item smoothing due to alpha function kernel
\end{itemize}
D-stellate\begin{itemize}
\item biophysically-based HH-like single-compartment model
\item type I-II current clamp model
\end{itemize}
\end{minipage}\\\hline
\textbf{Channel models}  & $I_{\textrm{Na}}$, $I_{\textrm{KHT}}$, $I_{\textrm{KLT}}$, $I_{\textrm{KA}}$ and $I_{\textrm{h}}$ \citep{RothmanManis:2003b} \\\hline
\textbf{Synapse model}   & Conductance synapses: excitatory (single-exponential), GABAergic (double-exponential) \\\hline
\textbf{Input Stimulus}  & Clicks with delay 2, 3, 4 and 8 ms\\\hline
\textbf{Measurements}    & PSTH sampled at each click for 2 ms to measure click recovery\\\hline
\end{tabularx}

\vspace{2ex}

% - B -----------------------------------------------------------------------------

\noindent
\begin{tabularx}{0.95\textwidth}{|l|X|X|}\hline %{0.95\textwidth}
\hdr{3}{B}{Populations}\\\hline
\textbf{Name} &            \textbf{Elements}            & \textbf{Number} \\\hline
     HSR      &            Poisson generator            & $N_{\text{HSR}} = 50$ per freq.\ channel \\\hline
     LSR      &            Poisson generator            & $N_{\text{LSR}}= 20$  per freq.\ channel \\\hline
     GLG      &            Poisson generator            & $N_{\text{GLG}}= 1$  per freq.\ channel  \\\hline
     DS       & Single-Compartment H-H model (type I-II)& $N_{\text{DS}}= 1$ at CF=5.6~kHz \\\hline
\end{tabularx}
\vspace{2ex}

% - C ------------------------------------------------------------------------------

\noindent
\begin{tabularx}{0.95\textwidth}{|l|l|l|X|}\hline
\hdr{4}{C}{Connectivity}\\\hline
        \textbf{Name}          &  \textbf{Source}  & \textbf{Target} & \textbf{Pattern} \\\hline
$\textrm{ANF} \to \textrm{DS}$ & ANF (HSR and LSR) &   D-Stellate    & skewed Gaussian, centered at CF, spread below CF \sANFDSl, spread above CF \sANFDSh, uniform weight \wANFDS for all synapses, number \nLSRDS and \nHSRDS, delay \dANFDS \\\hline
$\textrm{GLG} \to \textrm{DS}$ &       Golgi       &   D-Stellate    & Gaussian, centered at CF with spread \sGLGDS, uniform weight \wGLGDS, number \nGLGDS, delay \dGLGDS \\\hline
\end{tabularx}

\vspace{2ex}

% - D ------------------------------------------------------------------------------



%\noindent\begin{tabularx}{0.95\textwidth}{|p{0.150.95\textwidth}|X|}\hline
%\hdr{2}{D}{Neuron and Synapse Model}\\\hline
% \textbf{Name} & Poisson spike generator \\\hline
% \textbf{Type} & Leaky integrate-and-fire, $\delta$-current input\\\hline
% \raisebox{-4.5ex}{\parbox{0.95\textwidth}{\textbf{Subthreshold dynamics}}} &
% \rule{1em}{0em}\vspace*{-3.5ex}
%     \begin{equation*}
%       \begin{array}{r@{\;=\;}lll}
% \tau \dot{V}(t) & -V(t) + R I(t) & \text{if} & t > t^*+\tau_{\text{rp}} \\
%      V(t)       &  V_{\text{r}}  & \text{else} \\[2ex]
%      I(t)       & \multicolumn{3}{l}{\frac{\tau}{R} \sum_{\tilde{t}} w \delta(t-(\tilde{t}+\Delta))}
% \end{array}
%     \end{equation*}
% \vspace*{-2.5ex}\rule{1em}{0em}
%  \\\hline
 % \multirow{3}{*}{\textbf{Spiking}} &   If $V(t-)<\theta \wedge V(t+)\geq \theta$ \vspace*{-1ex}
 % \begin{enumerate}\setlength{\itemsep}{-0.5ex}
 % \item set $t^* = t$
 % \item emit spike with time-stamp $t^*$
 % \end{enumerate}
 % \vspace*{-4ex}\rule{1em}{0em} \\\hline
 % \end{tabularx}

\vspace{2ex}

% - E ------------------------------------------------------------------------------

\noindent
\begin{tabularx}{0.95\textwidth}{|l|X|}\hline %{0.95\textwidth}
\hdr{2}{E}{Optimisation} \\ \hline
      \textbf{Type}       & Principle-axis method \\\hline
   \textbf{Parameters}   & \\
 & $\wGLGDS \quad\to\quad [0.00001,0.05]\quad\mu{\rm S}$ \\
 & $\wANFDS \mathrm{ HSR }\quad [0.00001,0.05] \quad \mu{\rm S}$\\\hline
 & $\wANFDS \mathrm{ LSR }\quad [0.00001,0.05] \quad \mu{\rm S}$\\\hline
 & DS \GABAa synapse $ \tau_{GABA2} \quad [0.1,50.0]\quad {\rm ms}$\\\hline
 & DS leak conductance $ \bar{g}_{leak} \quad [0.00001,0.05] \quad \mathrm{Scm}^{-2}$\\\hline

\textbf{Fixed Parameters} & \\ \hline
$\nGLGDS = 25$ & \\ \hline
      $\wANFGLG$                    & \\ \hline
      $\nLSRGLG$                    & \\ \hline
      $\nHSRGLG$                    & \\ \hline
      $\sANFGLG $                   & \\ \hline
      $\dGLGDS = 0.5$ ms                    & Combination of conductance and synaptic delay. The effect of Golgi cells on DS is delayed by the extra 0.7~ms delay from ANF to Golgi, plus the slow peak of \GABAa inhibition.  \\\hline
  \textbf{Assumptions}    & The spread ANF to DS cells (\sANFDSh,\sANFDSl) is arbitrary at this point and will be explored in the next experiment.\\ \hline
   \textbf{Function}     & Weighted mean squared error see listing below  \\ \hline
\end{tabularx}
\vspace{2ex}

% - F -----------------------------------------------------------------------------

\noindent\begin{tabularx}{0.95\textwidth}{|X|}\hline
\hdr{1}{F}{Measurements}\\\hline
PSTHs were generated from 25 stimulus repetitions. Each response to a click is measured for 2 ms after the minimum first spike latency for the unit.  The unit used in the optimisation has a CF = 5.8~kHz (channel no. 50).\\ \hline
\begin{minipage}[c]{0.6\textwidth}
\vspace{1cm}
DS Ouput \hspace{2in} Golgi Output
\includegraphics[width=0.5\textwidth]{DS_ClickRecovery_DSpsth}\label{Ch3:fig:DSClickRecoveryPSTH}\includegraphics[width=0.5\textwidth]{DS_ClickRecovery_Gpsth}\label{Ch3:fig:DSClickRecoveryPSTH}\\
  \captionsize{PSTH response of a D-stellate cell from the click recovery stimulus used in the optimisation.}
  \end{minipage}\\ \hline
\end{tabularx}

 % ---------------------------------------------------------------------------------
% \newpage
% \begin{lstlisting}
% func fun() {local f
%       //Modify Variables
%       param.w.x[glg][ds] = $2
%       param.w.x[hsr][ds] = $3
%       param.w.x[lsr][ds] = $3
%       //Modify the network
%       {create_cells() connect_cells(fileroot) SetRates()}
%       // Simulate the network for N reps
%       for j=0, reps-1{
%          print j
%          GenSpikes()
%          run()
%          DSvec.append(dstellate[50][0].spiketimes)
%          //print startsw()-x, "secs"
%       }
%       DSvec = DSvec.histogram(0,tstop,0.1)
%
%       objref errorvec
%       errorvec = new Vector()
%       //Find the mean number of spikes in the first click
%       maxrate = (DSvec.sum(240,260) + DSvec.sum(740,760)+ DSvec.sum(1340,1360))/3
%       //Calc ratio of number of spikes in second click relative to mean first click
%       errorvec.append( DSvec.sum(260,280) / maxrate )
%       errorvec.append( DSvec.sum(780,800) / maxrate )
%       errorvec.append( DSvec.sum(1420,1440) / maxrate )
%       errorvec.plot(g
%     return errorvec.meansqerr(targetclick)
% }
% \end{lstlisting}




\begin{figure}
\includegraphics[angle=-90,width=0.8\textwidth]{DSClickRecoveryExpData}\label{Ch3:fig:DSClickRecoveryExpData}
\caption{Experimental Data of GABAergic influence on D-stellate cells from \citep{BackoffPalombiEtAl:1997}, Fig.~3.}
\end{figure}

%\parsep

From the command line type:
\begin{verbatim}
$ ./i686/special DS_ClickRecovery.hoc
\end{verbatim}
in the \texttt{cnstellate} directory to simulate the optimisation for D-stellate click recovery.  The first run may take some time if the AN filters have not been previously saved, since the Zilany \& Bruce model requires 500~kHz resolution in the stimulus that is downsampled to 50~kHz.

\clearpage
\subsection{Results}


\noindent\begin{tabularx}{0.95\textwidth}{|l|X|}\hline %{0.95\textwidth}
\hdr{2}{D}{Results} \\\hline
\textbf{Best Parameters} &
{\begin{minipage}[c]{0.6\textwidth}
$\wGLGDS = 0.532 \quad{\rm nS}$ \\
 $\wHSRDS = 0.16 \quad{\rm nS}$\\
 $\wLSRDS = 13.1 \quad{\rm nS}$\\
 $\tau_{GABA2} = 5.432 \quad{\rm ms}$\\
$\tau_{GABA1} = 0.262 \quad{\rm ms}$\\
 $\bar{g}_{leak} = 0.0163 \quad\mathrm{Scm}^{-2}$\\
\end{minipage}}\\\hline
\textbf{Error} & 0.006671    unweighted (MSE of recovery spike rate / mask rate)\\\hline
& 0.01447    final result (MSE of recovery spike rate / mask rate)\\\hline
\end{tabularx}

\begin{figure}[hp!]
  \centering
\includegraphics[keepaspectratio=true,angle=-90,width=0.9\textwidth]{./gfx/DS_ClickRecovery_result.eps}
%\includegraphics[keepaspectratio,angle=-90,width=0.8\textwidth]{./gfx/DSClickRecoveryExpData}\\
\caption{Experimental Data ({\color{green} Green}) of GABAergic influence on D-stellate cells from \citep{BackoffPalombiEtAl:1997}, Fig.~3.  Best result ({\color{blue} Blue}) shown in figure below. }
\label{fig:DS_ClickRecovery_result}  
\end{figure}


% \begin{figure}
%   \includegraphics[width=0.5\textwidth]{DS_ClickRecovery_OptVars.eps}\\
% %  \includegraphics[width=0.5\textwidth]{DS_ClickRecovery_Output.eps}\label{Ch3:fig:DSClickRecoveryOutput}
%   \caption{Final Output Data of the D-stellate Click Recovery optimisation }
% \end{figure}

% \begin{figure}
%   \includegraphics[keepaspectratio=true,width=0.8\textwidth]{DS_ClickRecovery_Example1.eps}\\
%   \includegraphics[keepaspectratio=true,width=0.8\textwidth]{DS_ClickRecovery_Example10.eps}\\
%   \includegraphics[keepaspectratio=true,width=0.8\textwidth]{DS_ClickRecovery_Example13.eps}\\
%   \includegraphics[keepaspectratio=true,width=0.8\textwidth]{DS_ClickRecovery_Example19.eps}\\
%   \caption{Click Recovery optimisation functions}
% \end{figure}


% \begin{figure}
%   \includegraphics[keepaspectratio=true,angle=-90,width=0.8\textwidth]{DS_ClickRecovery_result.eps}\\
% \end{figure}


% \begin{figure}
%   \includegraphics[keepaspectratio=true,angle=-90,width=0.8\textwidth]{DS_ClickRecovery_result1.eps}\\
% \end{figure}


% \begin{figure}
%   \includegraphics[keepaspectratio=true,angle=-90,width=0.8\textwidth]{DS_ClickRecovery_result2.eps}\\
%   \caption{Click Recovery optimisation }
% \end{figure}




% \begin{figure}
% \begin{center}
% \includegraphics[keepaspectratio=true]{DS_ClickRecovery_handtuned.eps}\\
% \includegraphics[keepaspectratio=true,angle=-90,width=0.8\textwidth]{DS_ClickRecovery_result_handtuned.eps}
% \caption{Handtuned}
% \label{hantuned}
% \end{center}
% \end{figure}

% \begin{figure}
% \begin{center}
% %\includegraphics[keepaspectratio=true]{DS_ClickRecovery_handtuned.eps}\\
% \includegraphics[keepaspectratio=true,angle=-90,width=0.8\textwidth]{gfx/DS_ClickRecovery_result_unweighted_8.eps}\\
% \includegraphics[keepaspectratio=true,angle=-90,width=0.8\textwidth]{gfx/DS_ClickRecovery_result_weighted_0.eps}
% \caption{Handtuned}
% \label{hantuned}
% \end{center}
% \end{figure}


%% Example optimisation points used by praxis 
% \begin{figure}
% \begin{center}
% %\includegraphics[keepaspectratio=true]{DS_ClickRecovery_handtuned.eps}\\
% \includegraphics[keepaspectratio=true,width=0.5\textwidth]{Praxis_123.eps}
% \includegraphics[keepaspectratio=true,width=0.5\textwidth]{Praxis_456.eps}
% \caption{Handtuned}
% \label{hantuned}
% \end{center}
% \end{figure}


% \clearpage
\newpage
\subsection{Verification}

% \subsection{Tone Response}
% \begin{figure}[h!]
% \centering\resizebox{0.95\textwidth}{!}{%
% \includegraphics{RateLevel/psthsingle90.2.eps}%
% \includegraphics{RateLevel/DS_ratelevel.eps}}
% \end{figure}
% \begin{figure}[h!]
% \centering\resizebox{0.95\textwidth}{!}{%
% \includegraphics{RateLevel/response_area.2.eps}%
% \includegraphics{RateLevel/response_area_log2.2.eps}}
% \end{figure}
% \begin{figure}[h!]
% \centering\resizebox{0.95\textwidth}{!}{%
% %\includegraphics{RateLevel/response_area.2.eps}
% \includegraphics{RateLevel/psthall90.2.eps}%
% \includegraphics{RateLevel/psthVlevel.2.eps}}
% \end{figure}


% \clearpage
% \subsection{Noise Response}
% \begin{figure}[h!]
% \centering\resizebox{0.95\textwidth}{!}{%
% \includegraphics{NoiseRateLevel/psthsingle120.2.eps}%
% \includegraphics{NoiseRateLevel/DS_ratelevel.eps}}
% \end{figure}
% \begin{figure}[h!]
% \centering\resizebox{0.95\textwidth}{!}{%
% \includegraphics{NoiseRateLevel/response_area.2.eps}%
% \includegraphics{NoiseRateLevel/response_area_log2.2.eps}}
% \end{figure}
% \begin{figure}[h!]
% \centering\resizebox{0.95\textwidth}{!}{%
% %\includegraphics{RateLevel/response_area.2.eps}
% \includegraphics{NoiseRateLevel/psthall90.2.eps}%
% \includegraphics{NoiseRateLevel/psthVlevel.2.eps}}
% \end{figure}


% \clearpage
% \subsection{Masked Noise and Tone}
% \begin{figure}[h!]
% \centering\resizebox{0.95\textwidth}{!}{\includegraphics{MaskedRateLevel/psthsingle90.2.eps}\includegraphics{MaskedRateLevel/DS_ratelevel.eps}}
% \end{figure}
% \begin{figure}[h!]
% \centering\resizebox{0.95\textwidth}{!}{%
% \includegraphics{MaskedRateLevel/response_area.2.eps}%
% \includegraphics{MaskedRateLevel/response_area_log2.2.eps}}
% \end{figure}

% \begin{figure}[h!]
% \centering\resizebox{0.95\textwidth}{!}{%
% %\includegraphics{RateLevel/response_area.2.eps}
% \includegraphics{MaskedRateLevel/psthall90.2.eps}%
% \includegraphics{MaskedRateLevel/psthVlevel.2.eps}}
% \end{figure}
% \clearpage
% \subsection{Masked Response Area}
% \begin{figure}[h!]
% \centering\resizebox{0.95\textwidth}{!}{%
% \includegraphics{MaskedResponseCurve/psthsingle5810.2.eps}%
% \includegraphics{MaskedResponseCurve/DS_masked.eps}}
% \end{figure}
% \begin{figure}[h!]
% \centering\resizebox{0.95\textwidth}{!}{%
% \includegraphics{MaskedResponseCurve/response_area.2.eps}%
% \includegraphics{MaskedResponseCurve/response_area_log2log2.2.eps}}
% \end{figure}

% \begin{figure}[h!] 
% \centering\resizebox{0.95\textwidth}{!}{%
% %\includegraphics{RateLevel/response_area.2.eps}
% \includegraphics{MaskedResponseCurve/psthall5810.2.eps}%
% \includegraphics{MaskedResponseCurve/psthVmod.2.eps}}
% \end{figure}
% \clearpage

% %
% %%%%%%%%%%%%%%%%%%%%%%%%%%%%%%%%%%%%%%%%%%%%%%%%%%%%%% 
%  \bibliographystyle{plainnat}%bmc_article} % Style BST file
%  \bibliography{../manuscript/bib/MyBib}
 
% \end{document}
 

%%% Local Variables: 
%%% mode: latex
%%% TeX-master: "SimpleResponses"
%%% TeX-PDF-mode: t
%%% End: 
