
%%
%% Custom Hyphenations
%%
%%
\hyphenation{cross-talk au-di-tory adap-ta-tion phe-nomeno-log-i-cal syn-a-pse
  co-inc-id-ence Tub-er-culo-vent-ral glyc-in-ergic psycho-phys-ical
  asym-met-ric ex-plor-at-ory potas-sium op-ti-mi-sa-tion phe-nomeno-log-i-cal
  au-di-t-ory system Neuro-in-for-mat-ics mar-g-in-al param-e-ters Rh-ode}

%%
%% Sample custom-configuration
%%
%%   You are encouraged to modify the following section with any of your
%%   own custom commands, packages, etc.
%%

%error 'You should modify this section and remove this error.'

% for URLs
\usepackage{url}

% AMS packages
%\usepackage{amsfonts}
\usepackage{amssymb}
\usepackage[fleqn]{amsmath}   % displayed equations flush left
\setlength{\mathindent}{0em}
\usepackage{amsthm}
\usepackage[mathscr]{eucal}

% Allow equations to break over pages...
\interdisplaylinepenalty=2500
% Command to stop equation breaks
% Note: enclose this in braces when used...
\newcommand{\donotsplitoverpages}{\interdisplaylinepenalty=10000}

%% Graphics
%\ifx\pdftexversion\undefined
%  \usepackage[dvips]{graphicx}
%\else
%  \usepackage[pdftex]{graphicx}
%\fi
\usepackage{ifpdf}
 \ifpdf
   \pdfoutput=1
   \usepackage[pdftex]{graphicx}  % uncomment if using graphicx
\usepackage[final,          % override "draft" which means "do nothing"
            colorlinks,     % rather than outlining them in boxes
            linkcolor=blue, % override truly awful colour choices
            citecolor=blue, %   (ditto)
            urlcolor=blue,  %   (ditto)
            ]{hyperref}

 \ifx\pdfoutput\undefined \usepackage[ps2pdf,bookmarks=true,bookmarksnumbered=true,breaklinks=true,
            final,          % override "draft" which means "do nothing"
            colorlinks,     % rather than outlining them in boxes
            linkcolor=blue, % override truly awful colour choices
            citecolor=blue, %   (ditto)
            urlcolor=blue,  %   (ditto)
            ]{hyperref}

  % \usepackage[pdftex]{hyperref}  % uncomment if using hyperref
%  \usepackage[ps2pdf]{thumbpdf}
 \DeclareGraphicsExtensions{.eps,.bmp}
  \else
 \DeclareGraphicsExtensions{.png,.pdf,.jpg,.JPEG}
  \usepackage{epstopdf}
 %\usepackage[pdftex,bookmarks=true,bookmarksnumbered=true,breaklinks=true]{hyperref}
  \pdfadjustspacing=1
  \usepackage[pdftex]{thumbpdf}
  \fi
 \else
   \usepackage[dvips]{graphicx}  % uncomment if using graphicx
    % comment if not using hyperref
 \usepackage[final,          % override "draft" which means "do nothing"
            colorlinks,     % rather than outlining them in boxes
            linkcolor=blue, % override truly awful colour choices
            citecolor=blue, %   (ditto)
            urlcolor=blue,  %   (ditto)
            ]{hyperref}
 \DeclareGraphicsExtensions{.eps,.bmp}

\fi


% Enable IEEE macros
%\usepackage{IEEEtrantools}
% Use a plain bibliography style
%\bibliographystyle{plain}
% Use the IEEE bibliography style (sorted)
%\bibliographystyle{IEEEtrans}
% Use the IEEE bibliography style (unsorted; order of reference)
%\bibliographystyle{IEEEtran}

% For isolated bibliographies
\usepackage{bibunits}

\usepackage{color}
%\usepackage[noadjust]{cite}
\usepackage{caption}

% For cool tables
\usepackage{array}
\usepackage{tabularx}  % automatically adjusts column width in tables
\usepackage{multirow}  % allows entries spanning several rows
\usepackage{colortbl}  % allows coloring tables

% For subfigures
\usepackage{subfig}
%\usepackage{subfigure}

% For algorithms
%\usepackage{algorithm}
%\usepackage{algorithmic}

% For cases
\usepackage{sublabel}

% For theroem numbers having the chapter included
%\usepackage{style/chngcntr}

% For cool theorem styles
%\usepackage[amsthm]{ntheorem}
%%\theorembodyfont{\normalfont}
%
%% Theorem definition
%\newtheorem{theorem}{Theorem}
%\counterwithin{theorem}{chapter}
%
%% Corollary definition
%\newtheorem{corollary}{Corollary}
%\counterwithin{corollary}{chapter}
%
%% Result definition
%\newtheorem{result}{Result}
%\counterwithin{result}{chapter}
%
%% Lemma definition
%\newtheorem{lemma}{Lemma}
%\counterwithin{lemma}{chapter}
%
%% Proposition definition
%\newtheorem{proposition}{Proposition}
%\counterwithin{proposition}{chapter}
%
%% Definition definition!
%\newtheorem{definition}{Definition}
%\counterwithin{definition}{chapter}
%
%% Remark definition (no counter?)
%\newenvironment{remark}{\emph{Remark:~}}{}
%
%% Fact definition (no counter?)
%\newenvironment{fact}{\emph{Fact:~}}{}

% (Re)Set the figure path
\newcommand{\setfigurepath}[1]{%
\ifx\figurepath\undefined
	\newcommand{\figurepath}{#1}
\else
	\renewcommand{\figurepath}{#1}
\fi%
}

% Used in the continued list environment below
\newcounter{continuedlist}

% Continued list environment
% \newenvironment{continuedlist}{ %
% 	\begin{enumerate}%
% 		% Space out each item
% 		\setlength{\itemsep}{1.25em}%
% 		% Start the enumeration from the previous value
% 		\setcounter{enumi}{\value{continuedlist}} %
% }{ %
%   % Save the counter to continue it later
%   \setcounter{continuedlist}{\value{enumi}}%
%   \end{enumerate}%
%   % \vspace{1.25em}% 
%   \vspace{1em}%
% }

% % Spaced out list environment
% \newenvironment{spacedoutlist}{%
% 	\begin{itemize}%
% 		% Space out each item
% 		\setlength{\itemsep}{1.25em}%
% }{\end{itemize}}


\usepackage[usenames,dvipsnames]{xcolor}
\usepackage{listings}
\lstset{language=C++,%[LaTeX]Tex,%
    keywordstyle=\color{RoyalBlue},%\bfseries,
    basicstyle=\small\sffamily,
%    identifierstyle=\color{NavyBlue},
    commentstyle=\color{Green}\rmfamily,
    stringstyle=\sffamily,
    numbers=left,%none,%
    numberstyle=\scriptsize,%\tiny
    stepnumber=5,
    numbersep=8pt,
    showstringspaces=false,
    breaklines=true,
    %frameround=ftff,
    %frame=single
    %frame=L
    lineskip=-5pt
}


% \lstset{language=Octave,                % choose the language of the code
% basicstyle=\footnotesize,       % the size of the fonts that are used for the code
% numbers=left,                   % where to put the line-numbers
% numberstyle=\footnotesize,      % the size of the fonts that are used for the line-numbers
% stepnumber=2,                   % the step between two line-numbers. If it's 1 each line will be numbered
% numbersep=5pt,                  % how far the line-numbers are from the code
% backgroundcolor=\color{white},  % choose the background color. You must add \usepackage{color}
% showspaces=false,               % show spaces adding particular underscores
% showstringspaces=false,         % underline spaces within strings
% showtabs=false,                 % show tabs within strings adding particular underscores
% frame=single,			% adds a frame around the code
% tabsize=2,			% sets default tabsize to 2 spaces
% captionpos=b,			% sets the caption-position to bottom
% breaklines=true,		% sets automatic line breaking
% breakatwhitespace=false,	% sets if automatic breaks should only happen at whitespace
% lineskip=-5pt  %
% }


\usepackage[sort,round,authoryear,nonamebreak]{natbib}
\setcitestyle{aysep={}} % J Neurophys formatting



\usepackage{xspace}
\usepackage{rotating}

\newcommand{\hdr}[3]{%
\multicolumn{#1}{|l|}{\color{white}\cellcolor[gray]{0.0}%
\textbf{\makebox[0pt]{#2}\hspace{0.5\linewidth}\makebox[0pt][c]{#3}}%
}}
 % Nordelie table environment
\newenvironment{ntab}[4]{\noindent\begin{tabularx}{\linewidth}{#2}\hline %
\multicolumn{#1}{|l|}{\color{white}\cellcolor[gray]{0.0}\textbf{#3\hfill{}{#4}\hfill{}}}%
}{\end{tabularx}
\vspace{1ex}
}



\usepackage[colorinlistoftodos,backgroundcolor=yellow!35,textsize=footnotesize]{todonotes}
\newcommand{\yellownote}[1]{\todo{#1}}



\usepackage{tikz}
\usepackage{calc}

%\usepackage{mparhack}

%\setlength{\parskip}{0ex}
%\setlength{\parindent}{0ex}

