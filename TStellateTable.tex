{%\vspace{2ex}
\small\linespread{0.5}
\begin{table}[htb]
    \caption{T~stellate cell model summary}
    \label{tab:TSModelSummary}
  \end{table}
\noindent%
\begin{tabularx}{\textwidth}{|l|X|}\hline %
\hdr{2}{A}{Model Summary}\\\hline
         \textbf{Populations}          & Six: \HSR \& \LSR ANFs, \GLG, \DS, \TV and \TS cells \\\hline
          \textbf{Topology}            & Tono-topicity of the rat AN and CN \\\hline
        \textbf{Connectivity}          & \ANF$\to$\{\GLG,\DS,\TV,\TS\}, Golgi$\to$\{\DS,\TS\}, \DS$\to$\{\TV,\TS\}, and \TVTS  \\\hline
         \textbf{Input model}          & ANF~model: instantaneous-rate Poisson model \citep{ZilanyBruce:2007} \\\hline
\multirow{4}{*}{\textbf{Neuron model}} & Golgi: instantaneous-rate Poisson model\\
                                       & D~stellate cell: HH-like single-compartment type I-II \RM model\\ 
                                       & Tuberculo-ventral cell:  HH-like single-compartment type I-c \RM model \\
                                       & T~stellate cell: HH-like single-compartment type I-t \RM model\\ \hline
       \textbf{Channel models}         & \INa, \Ileak, \IKHT, IKLT, \IKA, and \Ih \citep{RothmanManis:2003b}\\\hline
        \textbf{Synapse model}         & AMPA (\textit{ExpSyn}), \GABAa (\textit{Exp2Syn}), Glycine (\textit{Exp2Syn}) \\\hline
            \textbf{Input}             & Pure tones at CF of designated exemplar neuron.\\\hline
        \textbf{Measurements}          & Intracellular membrane voltage and spikes recorded over 25 repetitions. Membrane voltage clipped at threshold, $\Theta$, then averaged and processed for IV measurements discussed in text.  Spikes are processed for PSTH and CV statistics. \\\hline
\end{tabularx}
\vspace{1ex}

% - B -----------------------------------------------------------------------------
\noindent%
\begin{tabularx}{\textwidth}{|l|X|X|}\hline
\hdr{3}{B}{Populations}\\\hline
\textbf{Name} &      \textbf{Elements}      & \textbf{Size} \\\hline
     HSR      &      Poisson generator      & $N_{\text{HSR}} = 50$ per freq.\ channel \\\hline
     LSR      &      Poisson generator      & $N_{\text{LSR}}= 20$  per freq.\ channel \\\hline
     GLG      &      Poisson generator      & $N_{\text{GLG}}= 1$  per freq.\ channel  \\\hline
     DS       &    Type I-II \RM model     & $N_{\text{DS}}= 1$ per freq.\ channel \\\hline
     TV       &  Type I-classic \RM model  & $N_{\text{TV}}= 1$ per freq.\ channel\\\hline
     TS       & Type I-transient \RM model & $N_{\text{TV}}= 1$ per freq.\ channel\\\hline
\end{tabularx}
\vspace{1ex}

% - C ------------------------------------------------------------------------------
\noindent%
\begin{tabularx}{\textwidth}{|l|l|l|X|}\hline
\hdr{4}{C}{Connectivity}\\\hline
\textbf{Name} & \textbf{Source} & \textbf{Target}  & \textbf{Pattern} \\\hline
 %   \ANFDS   &       ANF       &    D~Stellate    & Skewed Gaussian, centered at CF, spread below CF \sANFDSl, spread above CF \sANFDSh \\\hline
 %   \ANFTV   &       ANF       & Tuberculoventral & Narrowband connection at CF, zero spread \\\hline
   \ANFTS     &       ANF       &        TS        & 
Narrowband connection at CF, zero spread, weight \wLSRTS and \wHSRTS, number \nLSRTS and \nHSRTS, delay \dANFTS \\\hline
 %   \GLGDS   &      Golgi      &    D~Stellate    & Gaussian, centered at CF with spread \sGLGDS \\\hline
 %    \DSTV   &   D~Stellate    & Tuberculoventral & Gaussian, centered at CF with spread \sDSTV \\\hline
   \GLGTS     &       GLG       &        TS        & 
Gaussian convergence, centered on CF, spread $\sigma^2 = \sGLGTS$, weight \wGLGTS, number \nGLGTS, delay $\dGLGTS=0.5$ ms \\\hline
    \DSTS     &       DS        &        TS        & 
Gaussian convergence, centered on CF, spread $\sigma^2 = \sDSTS$, weight \wDSTS, number \nDSTS, delay $\dDSTS=0.5$ ms \\\hline
    \TVTS     &       TV        &        TS        & 
Gaussian convergence, centered on CF, spread $\sigma^2 = \sTVTS$, weight \wTVTS, number \nTVTS, delay $\dTVTS=0.5$ ms \\\hline
\multicolumn{4}{|X|}{\ANFGLG, \ANFDS, \ANFTV, \GLGDS, \DSTV from previous CN model. }\\\hline
\end{tabularx}
% , uniform weight \wANFDS for all synapses, number \nLSRDS \& \nHSRDS, delay \dANFDS
\vspace{1ex}

% - D ------------------------------------------------------------------------------

\noindent%
\begin{tabularx}{\textwidth}{|l|X|}\hline
\hdr{2}{D}{Neuron and Synapse Model}\\\hline
 \textbf{Name} & DS, TV and TS cell models \\\hline
 \textbf{Type} & \RM \citep{RothmanManis:2003b}, conductance synapse input \\\hline
\textbf{Subthreshold dynamics} & \INa, \IKA, \IKHT, \Ih, and \Ileak currents \\\hline
 \textbf{Spiking} & Emit spike when $V(t)\geq \theta$  \\\hline
 \end{tabularx}
\vspace{1ex}
% \noindent\begin{tabularx}{\textwidth}{|p{0.150.95\textwidth}|X|}\hline
% \hdr{2}{D}{Neuron and Synapse Model}\\\hline
% \textbf{Name} &  \\\hline
% \textbf{Type} & \\\hline
% \raisebox{-4.5ex}{\parbox{0.95\textwidth}{\textbf{Subthreshold dynamics}}} &
% \rule{1em}{0em}\vspace*{-3.5ex}
%     \begin{equation*}
%       \begin{array}{r@{\;=\;}lll}
%       \tau \dot{V}(t) & -V(t) + R I(t) &\text{if} & t > t^*+\tau_{\text{rp}} \\
%       V(t) & V_{\text{r}} & \text{else} \\[2ex]
%       I(t) & \multicolumn{3}{l}{\frac{\tau}{R} \sum_{\tilde{t}} w
%         \delta(t-(\tilde{t}+\Delta))}
%       \end{array}
%     \end{equation*}
% \vspace*{-2.5ex}\rule{1em}{0em}
%  \\\hline
% \multirow{3}{*}{\textbf{Spiking}} &
%    If $V(t-)<\theta \wedge V(t+)\geq \theta$
% \vspace*{-1ex}
% \begin{enumerate}\setlength{\itemsep}{-0.5ex}
% \item set $t^* = t$
% \item emit spike with time-stamp $t^*$
% \end{enumerate}
% \vspace*{-4ex}\rule{1em}{0em}
% \\\hline
% \end{tabularx}

%\vspace{2ex}
\noindent
\begin{tabularx}{\textwidth}{|l|X|}\hline %
\hdr{2}{E}{Input\slash Ouput}\\\hline
\textbf{Input Stimulus} & Pure tone stimuli with 50 ms duration, 2 ms cosine squared on\slash off ramp, and 20 ms delay. Optimisation of each chopper unit used the CF and 30 dB re threshold sound level of the exemplar unit: CS (3.9~kHz 40 dB SPL), CT1 (8.2 kHz, 85 dB SPL), CT2 (12.4~kHz 35 dB SPL). \\\hline 
\textbf{Input} & Stimulus induced Poisson spike trains from \GLG units, \HSR and \LSR \ANFs, and natural synaptic input from \DS and \TV units\\\hline
%\multicolumn{2}{|c|}{\begin{minipage}[c]{0.8\textwidth}
%\includegraphics[width=0.8\textwidth,keepaspectratio]{./gfx/Notch-Wl-12.5kHz-0.5.eps}
%\end{minipage}}\\\hline
% \textbf{Output} & Output of 100 TV cells, across the network, with 25 repetitions\\\hline
%\multicolumn{2}{|c|}{\begin{minipage}[c]{0.8\textwidth}%
%\includegraphics[width=0.8\textwidth,keepaspectratio]{./gfx/AN_rateplace_12.5_0.5.eps}
%\end{minipage}}\\\hline
%\textbf{Measurements}    & PSTH sampled at each click for 2 ms to measure click recovery\\\hline
%\textbf{Optimisation}    & Parameters for \GLGDS are optimised based on experimental click recovery date from \citet{BackoffPalombiEtAl:1997}. The praxis method is used for optimisation.  \\\hline
\textbf{Measurements}    &  Intracellular voltage was recorded and special measures were calculated.  Spikes recorded and co-efficient of variation statistic was calculated in 10 ms windows.\\\hline
\end{tabularx}
\vspace{1ex}
}

%  \textbf{Assumptions}    & The spread ANF to DS cells (\sANFDSh,\sANFDSl) is arbitrary at this point and will be explored in the next experiment.\\ \hline
%   \textbf{Function}     & Weighted mean squared error see listing below  \\ \hline




%%% Local Variables: 
%%% mode: latex
%%% TeX-master: "SimpleResponses"
%%% TeX-PDF-mode: nil
%%% End: 
