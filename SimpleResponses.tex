

\documentclass[11pt,a4paper,twoside]{book} % Use the UniMelb Dissertation Template
\usepackage{../manuscript/style/uomthesis}
% User defined commands

%%
%% Custom Hyphenations
%%
%%

%\hyphenation{cross-talk au-di-tory adap-ta-tion phe-nomeno-log-i-cal syn-a-pse}

%%
%% Sample custom-configuration
%%
%%   You are encouraged to modify the following section with any of your
%%   own custom commands, packages, etc.
%%

%error 'You should modify this section and remove this error.'

% for URLs
\usepackage{url}

% AMS packages
%\usepackage{amsfonts}
\usepackage{amssymb}
\usepackage[fleqn]{amsmath}   % displayed equations flush left
\setlength{\mathindent}{0em}
\usepackage{amsthm}
\usepackage[mathscr]{eucal}

% Allow equations to break over pages...
\interdisplaylinepenalty=2500
% Command to stop equation breaks
% Note: enclose this in braces when used...
\newcommand{\donotsplitoverpages}{\interdisplaylinepenalty=10000}

%% Graphics
%\ifx\pdftexversion\undefined
%  \usepackage[dvips]{graphicx}
%\else
%  \usepackage[pdftex]{graphicx}
%\fi
\usepackage{ifpdf}
 \ifpdf
   \pdfoutput=1
   \usepackage[pdftex]{graphicx}  % uncomment if using graphicx
\usepackage[final,          % override "draft" which means "do nothing"
            colorlinks,     % rather than outlining them in boxes
            linkcolor=blue, % override truly awful colour choices
            citecolor=blue, %   (ditto)
            urlcolor=blue,  %   (ditto)
            ]{hyperref}

 \ifx\pdfoutput\undefined \usepackage[ps2pdf,bookmarks=true,bookmarksnumbered=true,breaklinks=true,
            final,          % override "draft" which means "do nothing"
            colorlinks,     % rather than outlining them in boxes
            linkcolor=blue, % override truly awful colour choices
            citecolor=blue, %   (ditto)
            urlcolor=blue,  %   (ditto)
            ]{hyperref}

  % \usepackage[pdftex]{hyperref}  % uncomment if using hyperref
%  \usepackage[ps2pdf]{thumbpdf}
 \DeclareGraphicsExtensions{.eps,.bmp}
  \else
 \DeclareGraphicsExtensions{.png,.pdf,.jpg,.JPEG}
  \usepackage{epstopdf}
 %\usepackage[pdftex,bookmarks=true,bookmarksnumbered=true,breaklinks=true]{hyperref}
  \pdfadjustspacing=1
  \usepackage[pdftex]{thumbpdf}
  \fi
 \else
   \usepackage[dvips]{graphicx}  % uncomment if using graphicx
    % comment if not using hyperref
 \usepackage[final,          % override "draft" which means "do nothing"
            colorlinks,     % rather than outlining them in boxes
            linkcolor=blue, % override truly awful colour choices
            citecolor=blue, %   (ditto)
            urlcolor=blue,  %   (ditto)
            ]{hyperref}
 \DeclareGraphicsExtensions{.eps,.bmp}

\fi


% Enable IEEE macros
%\usepackage{IEEEtrantools}
% Use a plain bibliography style
%\bibliographystyle{plain}
% Use the IEEE bibliography style (sorted)
%\bibliographystyle{IEEEtrans}
% Use the IEEE bibliography style (unsorted; order of reference)
%\bibliographystyle{IEEEtran}

% For isolated bibliographies
\usepackage{bibunits}

\usepackage{color}
%\usepackage[noadjust]{cite}
\usepackage{caption}

% For cool tables
\usepackage{array}
\usepackage{tabularx}  % automatically adjusts column width in tables
\usepackage{multirow}  % allows entries spanning several rows
\usepackage{colortbl}  % allows coloring tables

% For subfigures
\usepackage{subfig}
%\usepackage{subfigure}

% For algorithms
%\usepackage{algorithm}
%\usepackage{algorithmic}

% For cases
\usepackage{sublabel}

% For theroem numbers having the chapter included
%\usepackage{style/chngcntr}

% For cool theorem styles
%\usepackage[amsthm]{ntheorem}
%%\theorembodyfont{\normalfont}
%
%% Theorem definition
%\newtheorem{theorem}{Theorem}
%\counterwithin{theorem}{chapter}
%
%% Corollary definition
%\newtheorem{corollary}{Corollary}
%\counterwithin{corollary}{chapter}
%
%% Result definition
%\newtheorem{result}{Result}
%\counterwithin{result}{chapter}
%
%% Lemma definition
%\newtheorem{lemma}{Lemma}
%\counterwithin{lemma}{chapter}
%
%% Proposition definition
%\newtheorem{proposition}{Proposition}
%\counterwithin{proposition}{chapter}
%
%% Definition definition!
%\newtheorem{definition}{Definition}
%\counterwithin{definition}{chapter}
%
%% Remark definition (no counter?)
%\newenvironment{remark}{\emph{Remark:~}}{}
%
%% Fact definition (no counter?)
%\newenvironment{fact}{\emph{Fact:~}}{}

% (Re)Set the figure path
\newcommand{\setfigurepath}[1]{%
\ifx\figurepath\undefined
	\newcommand{\figurepath}{#1}
\else
	\renewcommand{\figurepath}{#1}
\fi%
}

% Used in the continued list environment below
\newcounter{continuedlist}

% Continued list environment
% \newenvironment{continuedlist}{ %
% 	\begin{enumerate}%
% 		% Space out each item
% 		\setlength{\itemsep}{1.25em}%
% 		% Start the enumeration from the previous value
% 		\setcounter{enumi}{\value{continuedlist}} %
% }{ %
%   % Save the counter to continue it later
%   \setcounter{continuedlist}{\value{enumi}}%
%   \end{enumerate}%
%   % \vspace{1.25em}% 
%   \vspace{1em}%
% }

% % Spaced out list environment
% \newenvironment{spacedoutlist}{%
% 	\begin{itemize}%
% 		% Space out each item
% 		\setlength{\itemsep}{1.25em}%
% }{\end{itemize}}


\usepackage[usenames,dvipsnames]{xcolor}
\usepackage{listings}
\lstset{language=C++,%[LaTeX]Tex,%
    keywordstyle=\color{RoyalBlue},%\bfseries,
    basicstyle=\small\sffamily,
%    identifierstyle=\color{NavyBlue},
    commentstyle=\color{Green}\rmfamily,
    stringstyle=\sffamily,
    numbers=left,%none,%
    numberstyle=\scriptsize,%\tiny
    stepnumber=5,
    numbersep=8pt,
    showstringspaces=false,
    breaklines=true,
    %frameround=ftff,
    %frame=single
    %frame=L
    lineskip=-5pt
}


% \lstset{language=Octave,                % choose the language of the code
% basicstyle=\footnotesize,       % the size of the fonts that are used for the code
% numbers=left,                   % where to put the line-numbers
% numberstyle=\footnotesize,      % the size of the fonts that are used for the line-numbers
% stepnumber=2,                   % the step between two line-numbers. If it's 1 each line will be numbered
% numbersep=5pt,                  % how far the line-numbers are from the code
% backgroundcolor=\color{white},  % choose the background color. You must add \usepackage{color}
% showspaces=false,               % show spaces adding particular underscores
% showstringspaces=false,         % underline spaces within strings
% showtabs=false,                 % show tabs within strings adding particular underscores
% frame=single,			% adds a frame around the code
% tabsize=2,			% sets default tabsize to 2 spaces
% captionpos=b,			% sets the caption-position to bottom
% breaklines=true,		% sets automatic line breaking
% breakatwhitespace=false,	% sets if automatic breaks should only happen at whitespace
% lineskip=-5pt  %
% }


\usepackage[sort,round,authoryear,nonamebreak]{natbib}
\setcitestyle{aysep={}} % J Neurophys formatting



\usepackage{xspace}
\usepackage{rotating}

\newcommand{\hdr}[3]{%
\multicolumn{#1}{|l|}{\color{white}\cellcolor[gray]{0.0}%
\textbf{\makebox[0pt]{#2}\hspace{0.5\linewidth}\makebox[0pt][c]{#3}}%
}}

\usepackage[colorinlistoftodos,backgroundcolor=yellow!35,textsize=footnotesize]{todonotes}
\newcommand{\yellownote}[1]{\todo{#1}}



% \setlength{\marginparwidth}{1.1in}
% \let\oldmarginpar\marginpar
% \renewcommand\marginpar[1]{\-\oldmarginpar[\raggedleft\footnotesize #1]%
% {\raggedright\footnotesize #1}}


\usepackage{tikz}
\usepackage{calc}
\usepackage{mparhack}

\setlength{\parskip}{0ex}
\setlength{\parindent}{0ex}

\newlength{\yellownotewidth}
\setlength{\yellownotewidth}{3.5cm}
\newlength{\yellownoteheight}
\setlength{\yellownoteheight}{3.5cm}
%   -   -   -   -   -   -   -   -   -   -   -   -
% Yellow note...
%   -   -   -   -   -   -   -   -   -   -   -   -
% \newcommand{\yellownote}[1]{
% \marginpar{
%     \vspace{-0.5\yellownoteheight}
%         \begin{center}
%         \begin{tikzpicture}
% %            \draw[white,fill=gray!25,opacity=0.75,shift={(-0.125,-0.125)}] 
% %                (0,0) rectangle (\yellownotewidth,\yellownoteheight);
%             \draw[fill=yellow!35] (0,0) rectangle (\yellownotewidth,\yellownoteheight);
% %            \draw[opacity=0.45,fill=gray!50] (0.7\yellownotewidth,0) -- 
% %                (0.9\yellownotewidth,0.45) -- (\yellownotewidth,0.4) -- cycle;
%             \node[blue,below] at (0.5\yellownotewidth,\yellownoteheight) {
%                 \begin{minipage}{\yellownotewidth-1em}
%                     \scriptsize\sf#1
%                 \end{minipage}
%             };
%         \end{tikzpicture}
%         \end{center}
%         \vspace{0.5\yellownoteheight}
%     }
% }

%   -   -   -   -   -   -   -   -   -   -   -   -
% Resizeable - Yellow note...
%   -   -   -   -   -   -   -   -   -   -   -   -
\newcommand{\resizeableyellownote}[3]{
\setlength{\yellownotewidth}{#1cm}
\setlength{\yellownoteheight}{#2cm}
\marginpar{
    \vspace{-0.5\yellownoteheight}
        \begin{center}
        \begin{tikzpicture}
%            \draw[white,fill=gray!25,opacity=0.75,shift={(-0.125,-0.125)}] 
%                (0,0) rectangle (\yellownotewidth,\yellownoteheight);
            \draw[fill=yellow!35] (0,0) rectangle (\yellownotewidth,\yellownoteheight);
%            \draw[opacity=0.45,fill=gray!50] (0.7\yellownotewidth,0) -- 
%               (0.9\yellownotewidth,0.45) -- (\yellownotewidth,0.4) -- cycle;
            \node[blue,below] at (0.5\yellownotewidth,\yellownoteheight) {
                \begin{minipage}{\yellownotewidth-1em}
                    \scriptsize\sf#3
                \end{minipage}
            };
        \end{tikzpicture}
        \end{center}
        \vspace{0.5\yellownoteheight}
    }
}

% User defined commands

%\usepackage{pstricks}

%\newenvironment{etabular}[1]{
%\noindent\begin{tabularx}{\linewidth}{#1}\hline %    \hdr{#2}{#3}{#4}\\ \hline
%}{\end{tabularx} \vspace{2ex}}

\usepackage{xspace}
%% Glossary

% ANF to Golgi
\newcommand{\ANFGLG}{\protect{\textrm{{ANF}}\ensuremath{\to}\textrm{{GLG}}\xspace}}
\newcommand{\HSRGLG}{\protect\ensuremath{\textrm{{HSR}}\to\textrm{{GLG}}\xspace}}
\newcommand{\LSRGLG}{\protect\ensuremath{\textrm{{LSR}}\to\textrm{{GLG}}\xspace}}
\newcommand{\wANFGLG}{\protect\ensuremath{w_{\ANFGLG}\xspace}}
\newcommand{\wLSRGLG}{\protect\ensuremath{w_{\LSRGLG}\xspace}}
\newcommand{\wHSRGLG}{\protect\ensuremath{w_{\HSRGLG}\xspace}}
\newcommand{\nLSRGLG}{\protect\ensuremath{n_{\LSRGLG}\xspace}}
\newcommand{\nHSRGLG}{\protect\ensuremath{n_{\HSRGLG}\xspace}}
\newcommand{\sANFGLG}{\protect\ensuremath{s_{\ANFGLG}\xspace}}
\newcommand{\sLSRGLG}{\protect\ensuremath{s_{\LSRGLG}\xspace}}
\newcommand{\sHSRGLG}{\protect\ensuremath{s_{\HSRGLG}\xspace}}
\newcommand{\dANFGLG}{\protect\ensuremath{d_{\ANFGLG}\xspace}}

%ANF to D-stellate
\newcommand{\ANFDS}{\protect\ensuremath{\textrm{{ANF}}\to\textrm{{DS}}\xspace}}
\newcommand{\HSRDS}{\protect\ensuremath{\textrm{{HSR}}\to\textrm{{DS}}\xspace}}
\newcommand{\LSRDS}{\protect\ensuremath{\textrm{{LSR}}\to\textrm{{DS}}\xspace}}
\newcommand{\wANFDS}{\ensuremath{w_{\ANFDS}\xspace}}
\newcommand{\wLSRDS}{\protect\ensuremath{w_{\LSRDS}\xspace}}
\newcommand{\wHSRDS}{\protect\ensuremath{w_{\HSRDS}\xspace}}
\newcommand{\nLSRDS}{\protect\ensuremath{n_{\LSRDS}\xspace}}
\newcommand{\nHSRDS}{\protect\ensuremath{n_{\HSRDS}\xspace}}
\newcommand{\dANFDS}{\protect\ensuremath{d_{\ANFDS}\xspace}}
\newcommand{\sANFDSh}{\protect\ensuremath{s^+_{\ANFDS}\xspace}}
\newcommand{\sANFDSl}{\protect\ensuremath{s^-_{\ANFDS}\xspace}}

%ANF to T-stellate
\newcommand{\ANFTS}{\protect\ensuremath{\textrm{{ANF}}\to\textrm{{TS}}\xspace}}
\newcommand{\HSRTS}{\protect\ensuremath{\textrm{{HSR}}\to\textrm{{TS}}\xspace}}
\newcommand{\LSRTS}{\protect\ensuremath{\textrm{{LSR}}\to\textrm{{TS}}\xspace}}
\newcommand{\wANFTS}{\protect\ensuremath{w_{\ANFTS}\xspace}}
\newcommand{\nLSRTS}{\protect\ensuremath{n_{\LSRTS}\xspace}}
\newcommand{\nHSRTS}{\protect\ensuremath{n_{\HSRTS}\xspace}}
\newcommand{\sANFTS}{\protect\ensuremath{s_{\ANFTS}\xspace}}
\newcommand{\dANFTS}{\protect\ensuremath{d_{\ANFTS}\xspace}}

%ANF to Tuberculoventral
\newcommand{\ANFTV}{\ensuremath{\textrm{{ANF}}\to\textrm{{TV}}\xspace}}
\newcommand{\HSRTV}{\ensuremath{\textrm{{HSR}}\to\textrm{{TV}}\xspace}}
\newcommand{\LSRTV}{\ensuremath{\textrm{{LSR}}\to\textrm{{TV}}\xspace}}
\newcommand{\wANFTV}{\ensuremath{w_{\ANFTV}\xspace}}
\newcommand{\nLSRTV}{\ensuremath{n_{\LSRTV}\xspace}}
\newcommand{\nHSRTV}{\ensuremath{n_{\HSRTV}\xspace}}
\newcommand{\sANFTV}{\ensuremath{s_{\ANFTV}\xspace}}
\newcommand{\dANFTV}{\ensuremath{d_{\ANFTV}\xspace}}

%GLG to T-stellate
\newcommand{\GLGTS}{\protect\ensuremath{\textrm{GLG}\to\textrm{TS}\xspace}}
\newcommand{\wGLGTS}{\protect\ensuremath{w_{\GLGTS}\xspace}}
\newcommand{\nGLGTS}{\protect\ensuremath{n_{\GLGTS}\xspace}}
\newcommand{\sGLGTS}{\protect\ensuremath{s_{\GLGTS}\xspace}}
\newcommand{\dGLGTS}{\protect\ensuremath{d_{\GLGTS}\xspace}}
%GLG to D-stellate
\newcommand{\GLGDS}{\protect\ensuremath{\textrm{{GLG}}\to\textrm{{DS}}\xspace}}
\newcommand{\wGLGDS}{\protect\ensuremath{w_{\GLGDS}\xspace}}
\newcommand{\nGLGDS}{\protect\ensuremath{n_{\GLGDS}\xspace}}
\newcommand{\sGLGDS}{\protect\ensuremath{s_{\GLGDS}\xspace}}
\newcommand{\dGLGDS}{\protect\ensuremath{d_{\GLGDS}\xspace}}

% % TS to Golgi
% \newcommand{\GLGTS}{\protect\ensuremath{\textrm{{GLG}}\to\textrm{{TS}}\xspace}}
% \newcommand{\wTSGLG}{\protect\ensuremath{w_{}}\xspace}}
% \newcommand{\nTSGLG}{\protect\ensuremath{n_{\textrm{TS}\to\textrm{GLG}}\xspace}}
% \newcommand{\sTSGLG}{\protect\ensuremath{s_{\textrm{TS}\to\textrm{GLG}}\xspace}}
% \newcommand{\dTSGLG}{\protect\ensuremath{d_{\textrm{TS}\to\textrm{GLG}}\xspace}}

%TS to D-stellate
\newcommand{\TSDS}{\protect\ensuremath{\textrm{{TS}}\to\textrm{{DS}}\xspace}}
\newcommand{\wTSDS}{\protect\ensuremath{w_{\TSDS}\xspace}}
\newcommand{\nTSDS}{\protect\ensuremath{n_{\TSDS}\xspace}}
\newcommand{\sTSDS}{\protect\ensuremath{s_{\TSDS}\xspace}}
\newcommand{\dTSDS}{\protect\ensuremath{d_{\TSDS}\xspace}}
%TS to T-stellate
\newcommand{\TSTS}{\protect\ensuremath{\textrm{TS}\to\textrm{TS}\xspace}}
\newcommand{\wTSTS}{\protect\ensuremath{w_{\TSTS}\xspace}}
\newcommand{\nTSTS}{\protect\ensuremath{n_{\TSTS}\xspace}}
\newcommand{\sTSTS}{\protect\ensuremath{s_{\TSTS}\xspace}}
\newcommand{\dTSTS}{\protect\ensuremath{d_{\TSTS}\xspace}}

%TS to Tuberculoventral
\newcommand{\TSTV}{\protect\ensuremath{\textrm{{TS}}\to\textrm{{TV}}\xspace}}
\newcommand{\wTSTV}{\protect\ensuremath{w_{\TSTV}\xspace}}
\newcommand{\nTSTV}{\protect\ensuremath{n_{\TSTV}\xspace}}
\newcommand{\sTSTV}{\protect\ensuremath{s_{\TSTV}\xspace}}
\newcommand{\dTSTV}{\protect\ensuremath{d_{\TSTV}\xspace}}

% DS to Golgi
\newcommand{\DSGLG}{\protect\ensuremath{\textrm{{DS}}\to\textrm{{GLG}}\xspace}}
\newcommand{\wDSGLG}{\protect\ensuremath{w_{\DSGLG}\xspace}}
\newcommand{\nDSGLG}{\protect\ensuremath{n_{\DSGLG}\xspace}}
\newcommand{\sDSGLG}{\protect\ensuremath{s_{\DSGLG}\xspace}}
\newcommand{\dDSGLG}{\protect\ensuremath{d_{\DSGLG}\xspace}}

% %DS to D-stellate
% \newcommand{\DSDS}{\protect\ensuremath{\textrm{{DS}}\to\textrm{{DS}}\xspace}}
% \newcommand{\wDSDS}{\protect\ensuremath{w_{\DSDS}\xspace}}
% \newcommand{\nDSDS}{\protect\ensuremath{n_{\DSDS}\xspace}}
% \newcommand{\sDSDS}{\protect\ensuremath{s_{\DSDS}\xspace}}
% \newcommand{\dDSDS}{\protect\ensuremath{d_{\DSDS}\xspace}}

%DS to T-stellate
\newcommand{\DSTS}{\protect\ensuremath{\textrm{DS}\to\textrm{TS}\xspace}}
\newcommand{\wDSTS}{\protect\ensuremath{w_{\DSTS}\xspace}}
\newcommand{\nDSTS}{\protect\ensuremath{n_{\DSTS}\xspace}}
\newcommand{\sDSTS}{\protect\ensuremath{s_{\DSTS}\xspace}}
\newcommand{\dDSTS}{\protect\ensuremath{d_{\DSTS}\xspace}}

%DS to Tuberculoventral
\newcommand{\DSTV}{\protect\ensuremath{\textrm{{DS}}\to\textrm{{TS}}\xspace}}
\newcommand{\wDSTV}{\protect\ensuremath{w_{\DSTV}\xspace}}
\newcommand{\nDSTV}{\protect\ensuremath{n_{\DSTV}\xspace}}
\newcommand{\sDSTV}{\protect\ensuremath{s_{\DSTV}\xspace}}
\newcommand{\dDSTV}{\protect\ensuremath{d_{\DSTV}\xspace}}
\newcommand{\oDSTV}{\protect\ensuremath{o_{\DSTV}\xspace}}

% TV to Golgi
\newcommand{\TVGLG}{\protect\ensuremath{\textrm{{TV}}\to\textrm{{GLG}}\xspace}}
\newcommand{\wTVGLG}{\protect\ensuremath{w_{\TVGLG}\xspace}}
\newcommand{\nTVGLG}{\protect\ensuremath{n_{\TVGLG}\xspace}}
\newcommand{\sTVGLG}{\protect\ensuremath{s_{\TVGLG}\xspace}}
\newcommand{\dTVGLG}{\protect\ensuremath{d_{\TVGLG}\xspace}}

%TV to D-stellate
\newcommand{\TVDS}{\protect\ensuremath{\textrm{{TV}}\to\textrm{{DS}}\xspace}}
\newcommand{\wTVDS}{\protect\ensuremath{w_{\TVDS}\xspace}}
\newcommand{\nTVDS}{\protect\ensuremath{n_{\TVDS}\xspace}}
\newcommand{\sTVDS}{\protect\ensuremath{s_{\TVDS}\xspace}}
\newcommand{\dTVDS}{\protect\ensuremath{d_{\TVDS}\xspace}}

%TV to T-stellate
\newcommand{\TVTS}{\protect\ensuremath{\textrm{{TV}}\to\textrm{{TS}}\xspace}}
\newcommand{\wTVTS}{\protect\ensuremath{w_{\TVTS}\xspace}}
\newcommand{\nTVTS}{\protect\ensuremath{n_{\TVTS}\xspace}}
\newcommand{\sTVTS}{\protect\ensuremath{s_{\TVTS}\xspace}}
\newcommand{\dTVTS}{\protect\ensuremath{d_{\TVTS}\xspace}}

%TV to Tuberculoventral
\newcommand{\TVTV}{\protect\ensuremath{\textrm{{TV}}\to\textrm{{TV}}\xspace}}
\newcommand{\wTVTV}{\protect\ensuremath{w_{\TVTV}\xspace}}
\newcommand{\nTVTV}{\protect\ensuremath{n_{\TVTV}\xspace}}
\newcommand{\sTVTV}{\protect\ensuremath{s_{\TVTV}\xspace}}
\newcommand{\dTVTV}{\protect\ensuremath{d_{\TVTV}\xspace}}


%Other common symbols
\newcommand{\GABAa}{\textrm{GABA}$_{\textrm{A}}\xspace$}


\graphicspath{{../../soma/netmod/cnstellate/golgi/}{../../soma/netmod/cnstellate/}}
\begin{document}

%%%%%%%%%%%%%%%%%%%%%%%%%%%%%%%%%%%%%%%%%%%%%%%%%%%%%%



%===================================
\subsection{Optimisation Results}

Figure~\ref{fig:GolgiTestResult} shows the output of the test optimisation trials for the Golgi cell model.
The testing trial used only five sound levels (0, 15, 55, 75 and 85 dB \SPL) and detected the mean rate from the instantaneous profile in its fitting routine.
The best response obtained a minimum root mean squared error of 11.63 spikes/sec against the five points in the target experimental data of unit S03-07 (CF=21~kHz) from \citep{GhoshalKim:1996}.
A rate-level curve (green circles, Figure~\ref{fig:GolgiTestResult}) was generated from the spiking output only to show a big discrepancy in the spike-based rate-level and the monotonic rate based rate-level.
The lack of low level response and a higher threshold indicated the need for some \HSR~input into the Golgi cell model.

%\smallskip{}


\begin{figure}[htb]
  \centering
\resizebox{0.6\textwidth}{!}{\includegraphics{GolgiRateLevel_result2.eps}}\\
  \caption[Initial results of Golgi cell model]{Initial trial results of the Golgi cell model optimisation.
Responses of the Golgi cell model (blue triangles) compared five five sound level (0,15, 55, 75 and 85 dB SPL) against 5 point in the target response (red squares).
The eventual best optimisation response obtained a minimum error of 11.63 spikes/s (root mean squared).
A spike response (green circles) was generated from the spiking output of the Golgi cell model using the final parameters. \label{fig:GolgiTestResult}}
\end{figure}

The final optimisation routine with 22 levels and a Golgi cell model with \HSR~and \LSR~\ANF~inputs was used to generate a closer fit to the \citeauthor{GhoshalKim:1996}  data.
Figure~\ref{fig:GolgiResult} shows the rate-level output of the best model response and its best combination of parameters are shown in Table~\ref{tab:GolgiCellModelSummary}E.
The root mean squared error of the best response was 4.48~spikes per second.

The parameters in Table~\ref{tab:GolgiCellResults} were within the range of expected values.
\LSR~inputs to the Golgi cell model out-weighted \HSR~inputs by more than a factor of 10.
The monotonic response of \LSR~fibres at high sound levels were necessary to create the large dynamic range in the Golgi cell model, the \HSR~fibres were just as necessary to provide some low level activity.
The spontaneous rate parameter matches the base response of unit S03-07 in Figure~\ref{fig:GolgiResult}.
The smoothing filter time constant of 5 ms is a typical value in membrane time constants for neural models and fits with the input resistance in intracellular recordings of Golgi cells \citep{FerragamoGoldingEtAl:1998}.

The input spread parameter is not well constrained by the optimisation fitness routine with a pure tone input and a single neuron, but the result is satisfactory given the uncertainty in \LSR~fibre's axonal organisation in the \GCD\@. 
The dendritic widths in Golgi cells are around 100 microns and the frequency separation laminae in the \VCN~core is approximately 70 microns, giving an expected result of 1.5 connectivity spread hence the result of 2.48 channels gives added frequency spread from \LSR~fibres.

%\smallskip{}


Table~\ref{tab:GolgiCellModelSummary}E result table.\\
{\small%% Result table
\noindent%
\begin{table}[htb]
  \centering
\begin{tabularx}{\textwidth}{|X|c|c|c|}\hline %{\textwidth}
\hdr{4}{}{GLG model parameters} \\ \hline
                \textbf{Parameters}                 & \textbf{Name} & \textbf{Range} & \textbf{Best Values} \\\hline
     Spatial spread LSR$\to$GLG (channel unit)      &  $\sANFGLG$   &     [0,10]     & 2.48  \\\hline
        Smoothing filter time constant (ms)         &    $\Gtau$    &     [0,20]     & 5.01  \\\hline
          Weighted sum of HSR~(unit-less)           &  $\wHSRGLG$   &     [0,5]      & 0.517 \\\hline
          Weighted sum of LSR~(unit-less)           &  $\wLSRGLG$   &     [0,5]      & 0.0487\\\hline
Spontaneous rate in Golgi cell model (spikes / sec) &   $\Gspon$    &     [0,50]     & 3.73  \\\hline
\end{tabularx}
  \caption{Golgi cell model optimisation parameters}
  \label{tab:GolgiCellResults}
\end{table}
}

\begin{figure}[htb]
  \centering
  % \resizebox{3.5in}{!}{\includegraphics{NoFigure}} \\
\includegraphics[keepaspectratio=true,width=0.6\textwidth]{GolgiRateLevel_result.eps}\\
  % \hspace{1cm}\figfont{A}\hfill\\
  %\resizebox{\textwidth}{!}{\includegraphics{GolgiRateLevel_result2.eps}} \\
  % \hspace{1cm}\figfont{B}\hfill \\
  \caption[Golgi cell model optimisation results]{Golgi cell model optimisation result trials against unit S03-07 (CF 21~kHz) from \citet{GhoshalKim:1996}.
A more detailed optimisation with 22 levels and included HSR inputs in the Golgi cell model generated a closer fit to the Ghoshal and Kim data.
The final root mean squared error was 4.48 spikes/s.
 \label{fig:GolgiResult}}
\end{figure}



%   % \includegraphics[width=0.6\textwidth,angle=-90]{GolgiRateLevelActualFit}\\
%   % \caption{Optimisation Results for Golgi Model using Rate Level data from
%   %     \label{Ch3:fig:GolgiFit}}
%   %   \includegraphics[width=0.8\textwidth]{GolgiRateLevel}\\
%   %   \caption{Optimisation Results for Golgi Model using Rate Level data from
%   %     \label{Ch3:fig:GolgiRL}}

%   %   \includegraphics[width=0.8\textwidth]{golgi_RateLevel_opt}\\
%   %   \caption{Optimisation Results for Golgi Model using Rate Level data from
%   %     \label{Ch3:fig:GolgiRL}}
%   % \includegraphics[width=0.8\textwidth,angle=-90]{GolgiRateLevel2}\\
%     %   \caption{Optimisation Results for Golgi Model using Rate Level data
%     %   from     \label{Ch3:fig:GolgiRL}}
%   \begin{figure}[htb]
%     \centering
% \includegraphics[width=0.6\textwidth,angle=-90]{GolgiRateLevelActualFit}\\
%     \caption{Optimisation Results for Golgi Model using Rate Level data from
%       \label{Ch3:fig:GolgiFit}}
%   \end{figure}
%   \begin{figure}[htb]
%     \centering
%     \includegraphics[width=0.8\textwidth]{GolgiRateLevel}\\
%     \caption{Optimisation Results for Golgi Model using Rate Level data from
%       \label{Ch3:fig:GolgiRL}}
%   \end{figure}
%   \begin{figure}[htb]
%     \centering
%     \includegraphics[width=0.8\textwidth]{golgi_RateLevel_opt}\\
%     \caption{Optimisation Results for Golgi Model using Rate Level data from
%       \label{Ch3:fig:GolgiRL}}
%   \end{figure}
%   \begin{figure}[htb]
%     \centering
% \includegraphics[width=0.8\textwidth,angle=-90]{GolgiRateLevel2}\\
%     \caption{Optimisation Results for Golgi Model using Rate Level data from
%       \label{Ch3:fig:GolgiRL}}
%   \end{figure}

%   \clearpage \newpage

%===================================
\subsection{Verification Results of Golgi Cell Model \label{sec:Golgi:verif-golgi-cell}}
%   \subsubsection{Tone Responses}


After settling with the above optimised parameters, the Golgi cell model was run with typical inputs to determine it's behaviour outside of the optimisation routine.
The Golgi cell model was tested across the entire network using tones, noise and tones plus noise stimuli. Figure~\ref{fig:Golgi_verification}A, B and D show the response of a Golgi cell model at the centre of the network (CF=5.8 kHz) and had monotonic responses to tones and noise similar to other Ghoshal and Kim units (Figure~\ref{fig:GolgiKimFig2}).  Figure~\ref{fig:Golgi_verification}C shows the response of all \GLG units in the network to a 5.8~kHz tone, increased from 0 to 90 dB~{SPL}.
 %\smallskip{}

\begin{figure}[htb]
%\centering
{\figfont{A}\hspace{0.5\textwidth}\figfont{B}\hfill}\\
%\resizebox{0.95\textwidth}{!}{
\includegraphics[keepaspectratio=true,width=0.48\textwidth]{ResponsesNoComp/G_ratelevel_combined.eps}%
\includegraphics[keepaspectratio=true,width=0.48\textwidth]{ResponsesNoComp/RateLevel/psthsingle90.3.eps}\\
%}\\
{\figfont{C}\hspace{0.5\textwidth}\figfont{D}\hfill}\\
%\resizebox{0.95\textwidth}{!}{
\includegraphics[keepaspectratio=true,width=0.48\textwidth]{ResponsesNoComp/RateLevel/response_area.3.eps}%
\includegraphics[keepaspectratio=true,width=0.48\textwidth]{ResponsesNoComp/MaskedResponseCurve3/15/G_masked.eps}\\
%}\\
% }}
%\resizebox{0.45\textwidth}{!}{\includegraphics{ResponsesNoComp/RateLevel/psthsingle90.3.eps}}\\
%\resizebox{0.45\textwidth}{!}{\includegraphics{ResponsesNoComp/RateLevel/psthsingle50.3.eps}}\\
\caption[Optimised Golgi cell model responses]{Response of optimised Golgi cell model at the centre of the network (CF=5.8~kHz). 
A. Rate level responses to tone, noise and tone plus noise. 
B. PSTH at 90 dB~SPL.  
C. Response area equivalent using all GLG units in the network. 
D. Masked noise-tone response of the central unit to 15 dB masking noise and frequencies one octave above and below its CF.} \label{fig:Golgi_verification}
\end{figure}




%   \begin{figure}[h]
%     \centering\resizebox{0.95\textwidth}{!}{%
%     \includegraphics{RateLevel/response_area.3.eps}%
%     \includegraphics{RateLevel/response_area_log2.3.eps}}
%   \end{figure}
%   \begin{figure}[h]
%     \centering\resizebox{0.95\textwidth}{!}{%
%     %     \includegraphics{RateLevel/response_area.3.eps}
%     \includegraphics{RateLevel/psthall90.3.eps}%
%     \includegraphics{RateLevel/psthVlevel.3.eps}}
%   \end{figure}



%   \clearpage
%   \subsubsection{Noise Responses}
%   \begin{figure}[h]
%     \centering\resizebox{0.95\textwidth}{!}{%
%     \includegraphics{NoiseRateLevel/psthsingle120.3.eps}%
%     \includegraphics{NoiseRateLevel/G_ratelevel.eps}}
%   \end{figure}
%   \begin{figure}[h]
%     \centering\resizebox{0.95\textwidth}{!}{%
%     \includegraphics{NoiseRateLevel/response_area.3.eps}%
%     \includegraphics{NoiseRateLevel/response_area_log2.3.eps}}
%   \end{figure}
%   \begin{figure}[h]
%     \centering\resizebox{0.95\textwidth}{!}{%
%     %     \includegraphics{RateLevel/response_area.3.eps}
%     \includegraphics{NoiseRateLevel/psthall90.3.eps}%
%     \includegraphics{NoiseRateLevel/psthVlevel.3.eps}}
%   \end{figure}


%   \clearpage
%   \subsubsection{Masking Responses}
%   \begin{figure}[h!]
% \centering\resizebox{0.95\textwidth}{!}{\includegraphics{MaskedRateLevel/psthsingle90.3.eps}\includegraphics{MaskedRateLevel/G_ratelevel.eps}}
%   \end{figure}
%   \begin{figure}[h!]
%     \centering\resizebox{0.95\textwidth}{!}{%
%     \includegraphics{MaskedRateLevel/response_area.3.eps}%
%     \includegraphics{MaskedRateLevel/response_area_log2.3.eps}}
%   \end{figure}

%   \begin{figure}[h!]
%     \centering\resizebox{0.95\textwidth}{!}{%
%     %     \includegraphics{RateLevel/response_area.3.eps}
%     \includegraphics{MaskedRateLevel/psthall90.3.eps}%
%     \includegraphics{MaskedRateLevel/psthVlevel.3.eps}}
%   \end{figure}
%   \clearpage

%   \begin{figure}[h!]
%     \centering\resizebox{0.95\textwidth}{!}{%
%     \includegraphics{MaskedResponseCurve/psthsingle5810.3.eps}%
%     \includegraphics{MaskedResponseCurve/G_masked.eps}}
%   \end{figure}
%   \begin{figure}[h!]
%     \centering\resizebox{0.95\textwidth}{!}{%
%     \includegraphics{MaskedResponseCurve/response_area.3.eps}%
% \includegraphics{MaskedResponseCurve/response_area_log2log2.3.eps}}
%   \end{figure}

%   \begin{figure}[h!]
%     \centering\resizebox{0.95\textwidth}{!}{%
%     %     \includegraphics{RateLevel/response_area.3.eps}
%     \includegraphics{MaskedResponseCurve/psthall5810.3.eps}%
%     \includegraphics{MaskedResponseCurve/psthVmod.3.eps}}
%   \end{figure}
%   \clearpage


%%% Local Variables:
%%% mode: latex
%%% mode: visual-line
%%% TeX-master: "SimpleResponses"
%%% TeX-PDF-mode: nil
%%% End:


\newpage

\section{Golgi Rate Level Curve}


% - A ------------------------------------------------------------------------------

\noindent
\begin{tabularx}{\linewidth}{|l|X|}\hline %
\hdr{2}{A}{Model Summary}\\\hline
\textbf{Populations}     & ANF (HSR,LSR) and Golgi \\\hline
\textbf{Topology}        & tonotopic, Auditory system of rat  \\\hline
\textbf{Connectivity}    & Gaussian spread dependent on morphology and afferent connections \\\hline
\textbf{Auditory model}    & \citep{ZilanyBruce:2008} ANF phenomenological model based on rat Audiograms\\\hline
\textbf{Neuron model}	& Golgi:  instantaneous-rate profile generated by weighted sum of HSR and LSR instantaneous-rate vectors followed by a smoothing? funciton (alpha function kernel). Poisson spikes are generated with refractory effects.\\\hline
\textbf{Channel models}  & - \\\hline
\textbf{Synapse model}   & [Alpha-function, tau = 5 ms \\\hline
\textbf{Input Stimulus}  & Rate Level function,  21~kHz tone at SPL -15 to 85 dB (20 ms delay, 2ms cosine squared ramp)\\\hline
\textbf{Measurements}    & Mean rate of instantaneous rate profile or PSTH sampled from poisson spike-generator  (25 repetitions). \\\hline
\textbf{Optimisation}    & Monotonic rate-level data from GCD in VCN \citep{GhoshalKim:1996} unit S03-07 (CF 21~kHz) was used to optimise parameters \textit{golgi\_spon}, \wLSRGLG, \wHSRGLG, \sLSRGLG using the praxis method. \\\hline
\end{tabularx}

\vspace{2ex}

% - B -----------------------------------------------------------------------------

\noindent\begin{tabularx}{\linewidth}{|l|X|X|}\hline %{\linewidth}
\hdr{3}{B}{Populations}\\\hline
  \textbf{Name}   & \textbf{Elements} & \textbf{Number} \\\hline
    HSR     &         & $N_{\text{HSR}} = 50$ per freq. channel \\\hline
    LSR     &         & $N_{\text{LSR}}= 20$  per freq. channel \\\hline
    GLG     & Poisson generator & $N_{\text{GLG}}= 1$  per freq. channel  \\\hline
\end{tabularx}
\vspace{2ex}

% - C ------------------------------------------------------------------------------

\noindent\begin{tabularx}{\linewidth}{|l|l|l|X|}\hline
\hdr{4}{C}{Connectivity}\\\hline
\textbf{Name} & \textbf{Source} & \textbf{Target} & \textbf{Pattern} \\\hline
  $\textrm{ANF} \to \textrm{GLG}$ & ANF (HSR and LSR) & Golgi & \begin{minipage}[c]{0.5\textwidth}
    Gaussian, centered at CF, spread of LSR \sLSRGLG was optimised, spread of HSR \sHSRGLG is fixed due to its replication of granule cells in the model, weight for LSR \wLSRGLG and HSR \wHSRGLG are determined  for all synapses, number \nLSRDS and \nHSRDS, delay \dANFGLG added to smoothing function to ensure conductance and dendritic filtering are included.
\end{minipage} \\\hline
 \end{tabularx}

\vspace{2ex}
% - D ------------------------------------------------------------------------------



\noindent\begin{tabularx}{\linewidth}{|p{0.15\linewidth}|X|}\hline
\hdr{2}{D}{Neuron and Synapse Model}\\\hline
\textbf{Name} & Golgi Phenomenological Model \\\hline
\textbf{Type} & Poisson instantaneous-rate model, ANF inst. rate input\\\hline
\textbf{Golgi Phenomenological Model} & \begin{minipage}[c]{0.6\textwidth}
$\mathbf{w}_{LSR} = N(\textrm{CF channel},\sLSRGLG)$,  $\mathbf{w}_{HSR} = N(\textrm{CF channel},\sHSRGLG)$  \\ 
\texttt{for \textit{i}=0, nchannels} \\
	$\mathbf{x}_i = \mathbf{w}_{LSR}(i)\mathbf{LSR}_i+\mathbf{w}_{HSR}(i)\cdot\mathbf{HSR}_i$ \\
\texttt{end}
	$\mathbf{x} = \mathbf{x}_i\circledast\mathbf{a}$  //Convolve profile with Alpha kernel\\
\end{minipage} \\\hline
% \multirow{3}{*}{\textbf{Spiking}} &
%    If $V(t-)<\theta \wedge V(t+)\geq \theta$
% \vspace*{-1ex}
% \begin{enumerate}\setlength{\itemsep}{-0.5ex}
% \item set $t^* = t$
% \item emit spike with time-stamp $t^*$
% \end{enumerate}
% \vspace*{-4ex}\rule{1em}{0em}
% \\\hline
\end{tabularx}

\vspace{2ex}

% - E ------------------------------------------------------------------------------

% %\begin{etabular}{|l|l|X|}%{3}{A}{Parameters}
% \noindent\begin{tabularx}{\linewidth}{|l|r|c|X|}\hline
% \hdr{4}{A}{Parameters}\\\hline
% \textbf{Name}   & \textbf{Initial}&\textbf{Range}&\textbf{Comments}   \\ \hline
% \wANFGLG& 1 	&	& Spontaneous rate  \\
% \nLSRGLG& 0.5 	&	&weight of LSR ANFs to Golgi cells          \\
% \nHSRGLG& 0.1 	&	&weight of HSR ANFs to Golgi cells        \\
% \sANFGLG& 3 	&	&spatial spread of LSR ANFs to DS cells             \\
% \dANFGLG& - 	&	&delay of ANFs to Golgi     \\ \hline \hline
% \end{tabularx}
% \vspace{2ex}
% %\end{etabular}


\noindent\begin{tabularx}{\linewidth}{|l|l||X|}\hline %{\linewidth}
\hdr{2}{E}{Optimisation} \\ \hline
\textbf{Parameters} & $\wLSRGLG \quad\to\quad [0,0.005]\quad \mu{S}$ \\
                    & $\wANFDS \quad [0,0.005]\quad \mu{S}$\\\hline
\textbf{Function} &  see listing below  \\\hline
\textbf{Fixed Parameters} & \\\hline
                    & \wANFGLG\\\hline
                    & \nLSRGLG\\\hline
                    & \nHSRGLG\\\hline
                    & \sANFGLG\\\hline
                    & \dANFGLG)\\\hline
\textbf{Assumed Parameters} &(\nLSRDS,\nHSRDS,\sANFDSh,\sANFDSl,\dANFDS)\\\hline
\end{tabularx}
\vspace{2ex}

% - F -----------------------------------------------------------------------------

\noindent\begin{tabularx}{\linewidth}{|X|}\hline
\hdr{1}{F}{Measurements}\\\hline


% \begin{minipage}[c]{0.6\textwidth}
% \includegraphics[width=0.5\textwidth]{DSpsth}\label{Ch3:fig:DSClickRecoveryPSTH}\\
%   \captionsize{PSTH response of a D-stellate cell from the click recovery stimulus used in the optimisation.}
%   \end{minipage}\\\hline
%\end{tabularx}

% ---------------------------------------------------------------------------------

%\end{tabularx}

  % \includegraphics[width=0.6\textwidth,angle=-90]{GolgiRateLevelActualFit}\\
  % \caption{Optimisation Results for Golgi Model using  Rate Level data from }\label{Ch3:fig:GolgiFit}
  % \includegraphics[width=0.8\textwidth]{GolgiRateLevel}\\
  % \caption{Optimisation Results for Golgi Model using  Rate Level data from }\label{Ch3:fig:GolgiRL}

  % \includegraphics[width=0.8\textwidth]{golgi_RateLevel_opt}\\
  % \caption{Optimisation Results for Golgi Model using  Rate Level data from }\label{Ch3:fig:GolgiRL}
  % \includegraphics[width=0.8\textwidth,angle=-90]{GolgiRateLevel2}\\
    % \caption{Optimisation Results for Golgi Model using  Rate Level data from }\label{Ch3:fig:GolgiRL}





\begin{figure}[htb]
\centering
  \includegraphics[width=0.6\textwidth,angle=-90]{GolgiRateLevelActualFit}\\
  \caption{Optimisation Results for Golgi Model using  Rate Level data from }\label{Ch3:fig:GolgiFit}
\end{figure}

\begin{figure}[htb]
\centering
  \includegraphics[width=0.8\textwidth]{GolgiRateLevel}\\
  \caption{Optimisation Results for Golgi Model using  Rate Level data from }\label{Ch3:fig:GolgiRL}
\end{figure}

\begin{figure}[htb]
\centering
  \includegraphics[width=0.8\textwidth]{golgi_RateLevel_opt}\\
  \caption{Optimisation Results for Golgi Model using  Rate Level data from }\label{Ch3:fig:GolgiRL}
\end{figure}

\begin{figure}[htb]
\centering
  \includegraphics[width=0.8\textwidth,angle=-90]{GolgiRateLevel2}\\
  \caption{Optimisation Results for Golgi Model using  Rate Level data from }\label{Ch3:fig:GolgiRL}
\end{figure}




%%%%%%%%%%%%%%%%%%%%%%%%%%%%%%%%%%%%%%%%%%%%%%%%%%%%%%
 \bibliographystyle{plainnat}%bmc_article} % Style BST file
 \bibliography{../manuscript/bib/MyBib}

\end{document}
