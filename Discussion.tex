%===================================
\section{Discussion    \label{sec:3:discussion}}

\yellownote{Discuss the relevance of creating the CN stellate model,
  especially the methods.}  
The T~stellate cell microcircuit in the
cochlear nucleus is one of the most interesting neural features of the
auditory system.  It enables a high fidelity sensory input to be
passed to higher order centres, by reproducing the spectrum in a
robust fashion \citep{BlackburnSachs:1990,May:2003} and
synchronisation to significant periodic frequencies
\citep{KeilsonRichardsEtAl:1997}.

%===================================
\subsection{Verification of CN cell models    \label{sec:3:verification-cn-cell}}

\subsubsection{Golgi cell model}

\yellownote{Discuss the Golgi cell optimisation.  Is the filter-based
  approach effective at meeting the goal of simulating data from
GhoshalKim:1997. The Golgi cell model optimised in section~\ref{sec:GolgiCellModel} reproduces the experimental data of a single unit in the marginal shell of the VCN \citep{GhoshalKim:1997}.}

%===================================
\subsubsection{D~stellate cell model}

\yellownote{Discuss the DS cell model optimisation.  What were the reasons for
using cell-based parameters, and why is this necessary?  Why aren't
the spread of ANF inputs to DS cells optimised based on your methods?
How does the optimisation routine replicate DS responses to other
stimuli, when there are already plenty?}

%===================================
\subsubsection{Tuberculoventral cell model}

\yellownote{Discuss the TV cell optimisation. Is the goal of replicating wide-band
inhibition offset necessary, or is rate ratio between TS/DS/TV the
most important?  Mammalian studies show that the TV cells receive
input from TS cells, but not in rats}

%===================================
\subsection{Limitations of the CN stellate microcircuit model    \label{sec:3:limit-cn-stell}}

\yellownote{AM coding in model is not verified by \citet{ZilanyBruce:2006} model,
and it was superseded by the addition of a power-law functions in the
synapse \citep{ZilanyBruceEtAl:2009}}

%===================================
\subsection{Implications of sequential microcircuit optimisation}


\yellownote{Probability distribution of delay (within cell-to-cell connection) not included.}



\yellownote{Probability of AP-to-synaptic release ratio not included (Liley assumption of 1\% vesicle release in cortical buttons.) }




%%% Local Variables: 
%%% mode: latex
%%% mode: tex-fold
%%% TeX-master: "SimpleResponses"
%%% TeX-PDF-mode: nil
%%% End: 
