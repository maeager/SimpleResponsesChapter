{\small%\linespread{0.5}
  \begin{table}[htb]
    \caption{Tuberculoventral cell model summary}
    \label{tab:TVcellModelSummary}
  \end{table}
\noindent%
\begin{tabularx}{\textwidth}{|l|X|}\hline %
\hdr{2}{A}{Model Summary}\\\hline
         \textbf{Populations}          & Five: HSR \& LSR ANFs, Golgi, DS, and TV cells \\\hline
          \textbf{Topology}            & Tono-topicity of the rat AN and CN \\\hline
        \textbf{Connectivity}          & ANF$\to${GLG,DS,TV}, GLG$\to$DS, DS$\to$TV  \\\hline
         \textbf{Input model}          & ANF~model: instantaneous-rate Poisson model \citep{ZilanyBruce:2007} \\ \hline
\multirow{3}{*}{\textbf{Neuron model}} & Golgi: instantaneous-rate Poisson spike trains\\
                                       & D~stellate: HH-like single-compartment model (Type 1-2 R\&M model)\\ 
                                       & Tuberculoventral cell:  HH-like single-compartment model (Type-I classic R\&M model) \\\hline
       \textbf{Channel models}         & $I_{\textrm{Na}}$, $I_{\textrm{KHT}}$, $I_{\textrm{KLT}}$, $I_{\textrm{KA}}$ and $I_{\textrm{h}}$ \citep{RothmanManis:2003b}\\\hline
\multirow{2}{*}{\textbf{Synapse model}} & Excitatory: AMPA glutamatergic receptor (single-exponential)\\
&  Inhibitory: GABA$_{\rm A}$ GABAergic receptor (double-exponential), Glycinergic receptor (double-exponential) \\\hline
            \textbf{Input}             & Notch-noise stimulus (Stop-band filtered white noise 60 dB SPL, 50 ms, 5ms on\slash off ramp, 20 ms delay)\\\hline
%\textbf{Optimisation}    & Parameters for \GLGDS are optimised based on experimental click recovery date from \citet{BackoffPalombiEtAl:1997}. The praxis method is used for optimisation.  \\\hline
\textbf{Measurements}    &  Spikes of TV units recorded and PSTH genereated. First spike latency, mean rate and variance of TV units calculated. Fitting data was compared against experimental data of a Type-II~\DCN~unit~\citep{ReissYoung:2005}, Fig.~9.
\end{tabularx}
\vspace{2ex}

% - B -----------------------------------------------------------------------------
\noindent%
\begin{tabularx}{\textwidth}{|l|X|X|}\hline
\hdr{3}{B}{Populations}\\\hline
\textbf{Name} &    \textbf{Elements}    & \textbf{Size} \\\hline
     HSR      &    Poisson generator    & $N_{\text{HSR}} = 50$ per freq.\ channel \\\hline
     LSR      &    Poisson generator    & $N_{\text{LSR}}= 30$  per freq.\ channel \\\hline
     GLG      &    Poisson generator    & $N_{\text{GLG}}= 1$  per freq.\ channel  \\\hline
     DS       &   Type I-II R\&M model    & $N_{\text{DS}}= 1$ per freq.\ channel \\\hline
     TV       & Type I-classic R\&M model & $N_{\text{TV}}= 1$ per freq.\ channel\\\hline
\end{tabularx}
\vspace{2ex}

% - C ------------------------------------------------------------------------------
\noindent%
\begin{tabularx}{\textwidth}{|l|l|l|X|}\hline
\hdr{4}{C}{Connectivity}\\\hline
\textbf{Name} & \textbf{Source} & \textbf{Target}  & \textbf{Pattern} \\\hline
%   ANF$\to$DS     &       ANF       &    D~Stellate    & Skewed Gaussian, centered at CF, spread below CF \sANFDSl, spread above CF \sANFDSh \\\hline
   ANF$\to$TV     &       ANF       & Tuberculoventral & Narrowband connection at CF, zero spread \\\hline
%   GLG$\to$DS     &      Golgi      &    D~Stellate    & Gaussian, centered at CF with spread \sGLGDS \\\hline
   DS$\to$TV     &   D~Stellate    & Tuberculoventral & Gaussian, centered at CF with spread \sGLGDS \\\hline
\multicolumn{4}{|X|}{\ANFGLG, \ANFDS, and \GLGDS from previous CN model. }\\\hline
\end{tabularx}
% , uniform weight \wANFDS for all synapses, number \nLSRDS \& \nHSRDS, delay \dANFDS
\vspace{2ex}

% - D ------------------------------------------------------------------------------
\noindent
\begin{tabularx}{\textwidth}{|l|X|}\hline
\hdr{2}{D}{Neuron and Synapse Model}\\\hline
 \textbf{Name} & TV cell model \\\hline
 \textbf{Type} & Type 1-classic R\&M model \citep{RothmanManis:2003b}, conductance synapse input \\\hline
 \textbf{Spiking} & Emit spike when $V(t)\geq \theta$  \\\hline
 \end{tabularx}
\vspace{2ex}
% \noindent\begin{tabularx}{\textwidth}{|p{0.150.95\textwidth}|X|}\hline
% \hdr{2}{D}{Neuron and Synapse Model}\\\hline
% \textbf{Name} &  \\\hline
% \textbf{Type} & \\\hline
% \raisebox{-4.5ex}{\parbox{0.95\textwidth}{\textbf{Subthreshold dynamics}}} &
% \rule{1em}{0em}\vspace*{-3.5ex}
%     \begin{equation*}
%       \begin{array}{r@{\;=\;}lll}
%       \tau \dot{V}(t) & -V(t) + R I(t) &\text{if} & t > t^*+\tau_{\text{rp}} \\
%       V(t) & V_{\text{r}} & \text{else} \\[2ex]
%       I(t) & \multicolumn{3}{l}{\frac{\tau}{R} \sum_{\tilde{t}} w
%         \delta(t-(\tilde{t}+\Delta))}
%       \end{array}
%     \end{equation*}
% \vspace*{-2.5ex}\rule{1em}{0em}
%  \\\hline
% \multirow{3}{*}{\textbf{Spiking}} &
%    If $V(t-)<\theta \wedge V(t+)\geq \theta$
% \vspace*{-1ex}
% \begin{enumerate}\setlength{\itemsep}{-0.5ex}
% \item set $t^* = t$
% \item emit spike with time-stamp $t^*$
% \end{enumerate}
% \vspace*{-4ex}\rule{1em}{0em}
% \\\hline
% \end{tabularx}
\vspace{2ex}

% \noindent
% \begin{tabularx}{\textwidth}{|l|X|}\hline %
% \hdr{2}{E}{Input\slash Ouput}\\\hline
% \textbf{Input Stimulus}  & Notch-noise (Stop-band filtered white noise)  \\\hline
% %\multicolumn{2}{|c|}{\begin{minipage}[c]{0.8\textwidth}
% %\includegraphics[width=0.8\textwidth,keepaspectratio]{./gfx/Notch-Wl-12.5kHz-0.5.eps}
% %\end{minipage}}\\\hline
% % \textbf{Output} & Output of 100 TV cells, across the network, with 25 repetitions\\\hline
% %\multicolumn{2}{|c|}{\begin{minipage}[c]{0.8\textwidth}%
% %\includegraphics[width=0.8\textwidth,keepaspectratio]{./gfx/AN_rateplace_12.5_0.5.eps}
% %\end{minipage}}\\\hline
% %\textbf{Measurements}    & PSTH sampled at each click for 2 ms to measure click recovery\\\hline
% %\textbf{Optimisation}    & Parameters for \GLGDS are optimised based on experimental click recovery date from \citet{BackoffPalombiEtAl:1997}. The praxis method is used for optimisation.  \\\hline
% \textbf{Measurements}    &  First spikes and PSTH of TV cells, calculated for first spike latency, mean rate and variance. Fitting data was compared against experimental data of a Type-II~\DCN~unit~\citep{ReissYoung:2005}, Fig.~9. \\\hline
% \end{tabularx}
% \vspace{2ex}

%  \textbf{Assumptions}    & The spread ANF to DS cells (\sANFDSh,\sANFDSl) is arbitrary at this point and will be explored in the next experiment.\\ \hline
%   \textbf{Function}     & Weighted mean squared error see listing below  \\ \hline


% % D~----------------------------------
% \begin{tabularx}{\linewidth}{|X|c|c|c|}
% \hdr{4}{F}{Optimisation} \\ \hline
%               \textbf{Parameters}                & \textbf{Name} & \textbf{Range} & \textbf{Best Values} \\\hline 
%         Weight of DS syn on TV  ($\mu$S)         &    \wDSTV     &  [1e-5,0.005]  & 0.0029 \\
%        Weight of ANF syn on TV  ($\mu$S)         &    \wANFTV    &  [1e-5,0.005]  & 0.00017 \\
%          Number of synapses, LSR to TV           &    \nLSRTV    &     [0,64]     & 8           \\
%          Number of synapses, HSR to TV           &    \nHSRTV    &     [0,64]     & 14          \\
% Spread of DS connections onto TV (channel units) &    \sDSTV     &     [0,10]     & 2.1         \\
% Offset of DS connections onto TV (channel units) &    \oDSTV     &     [0,10]     & 0.24        \\ \hline
% \end{tabularx}
}


%%% Local Variables: 
%%% mode: latex
%%% TeX-master: "SimpleResponses"
%%% TeX-PDF-mode: nil
%%% End: 
