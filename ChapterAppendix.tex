\graphicspath{{/media/data/Work/cnstellate/golgi/}{/media/data/Work/cnstellate/Responses/}{../figures/}{./gfx/}}

%\renewcommand{\setthesubsection}{\Alph{subsection}}
\begin{appendix}
\section{Appendix}
\label{sec:chp3appendix}
\subsection{Conversion of Auditory model}


The steps for converting \citet{ZilanyBruce:2007} auditory model to
accommodate a rat are the position on the basilar membrane to
characteristic frequency mapping function (\mbox{\textsf{cochlea\_x2f()}}), the
reciprocal frequency to position function (\mbox{\textsf{cochlea\_f2x()}}), a
delay function, and an appropriate audiogram mapping using a MATLAB
routine \mbox{\textsf{fitaudiogram.m}}.  Specific Added support for rat species
are included in the NMODL file \mbox{\textsf{an\_bruce.mod}} and \mbox{\textsf{an\_zilany\_v4.mod}}, so that specific
parameters can be passed from NEURON to the auditory model.  Rat
basilar membrane frequency mapping in \mbox{\textsf{cochlea\_f2x()}} and
\mbox{\textsf{cochlea\_x2f()}} included using data obtained from the Boston University Earlab
website, \url{earlab.bu.edu}.  Compression variables for the IHC and OHC
components of the Zilany Bruce model are fit using audiogram data of a
rat.  The audiogram in Fig.~\ref{fig:AudThresholdRat} was used to
generate the compression data for the rat model.

\medskip{}

\begin{lstlisting}[label=lst:makeaudiogram,caption=Using fitaudiogram.m to create COHC and CIHC vectors for the cat.]
 x=dlmread("heffner_1985a_felis_p347_3.txt",'\t',2,0)
 [Cohc,Cihc,OHC_Loss]=fitaudiogram(x(:,1),x(,2))
 dlmwrite ("cat_audiogram.txt",[x co' ci'],'\t',0,0,"precision","%g")
\end{lstlisting}


\begin{lstlisting}[label=lst:cataudiogram,caption=Portion of cat\_audiogram.txt]
...
125	37	0	0.025
250	22	0	0.3
500	8	0.65	0.7
1000	3	0.9	0.9
2000	1	0.95	1
4000	1	0.95	1
...
\end{lstlisting}


\begin{lstlisting}[label=lst:getaudiogramdata,caption= Procedure to get audiogram data and interpolate to freuencies in \textsf{cf} vector (Utilities.hoc)]
...
objref audiogram,cohc,cihc
proc GetAudiogramData(){
  ...
   file.ropen(audiogram_file)
   audiogram.scanf(file,nrows,4)
  ...
/* Interpolate compression data to cf positions */
   cohc.interpolate(cf, audiogram.getcol(0).c, audiogram.getcol(2).c)
   cihc.interpolate(cf, audiogram.getcol(0).c, audiogram.getcol(3).c)
}
GetAudiogramData() 
...
\end{lstlisting}

\medskip{}

\begin{figure}[htb]
\begin{center}
\resizebox{5in}{!}{\includegraphics[keepaspectratio=true]{NoFigure}}%Audiograms.jpg}}
\caption{Hearing threshold of the domestic cat by Heffner and Heffner
  1985 \citep{HeffnerHeffner:1985} (Data file
  \mbox{heffner\_1985a\_felis\_p347\_3.txt} obtained from [earlab.bu.edu])}
\label{fig:AudThresholdRat}
\end{center}
\end{figure}


The formula for the latency of acoustic stimulation to reach a
particular point on the basilar membrane comprises a fixed conduction
delay plus an additional delay that is an exponential function of the
distance from the stapes. The frequency mapping function is defined
as:
\[
 f = A\times10^{\left(a*d/L\right)} - K
 \]
where \emph{d} is distance along the basilar membrane from the stapes.

\medskip{}

This equation is suitable since it uses the mapping function
\mbox{\textsf{cochlea\_x2f()}} and its inverse, \mbox{\textsf{cochlea\_f2x()}}, which is
different for cat, rat and humans.  The data listed in
Table~\ref{tab:f2x} shows the currently accepted parameters for each
species.


\begin{table}[h]
  \centering
  \begin{tabular}{lccccc}
    \hline
    % after \\: \hline or \cline{col1-col2} \cline{col3-col4} ...
& A & a & k & L \\
Human &165.4	&2.1	&1.0(0.88)	&35     \\
Cat&456&	2.1&0.8&25 \\
Rat&7613.3	&0928	&1.0&	8.03     \\
    \hline
  \end{tabular}
  \caption{Frequency to basilar membrane distance function parameters \citep{FitzGeraldBurkittEtAl:2001}. Data obtained from \url{earlab.bu.edu.}}\label{tab:f2x}
\end{table}


In the cat, \citet{CarneyYin:1988} fitted the latency vs CF curve from
click responses in the cat to obtain the equation: \( delay (msec) =
A_0 * {\rm exp}( -x / A_1 ) * 1e-3 - 1.0/CF \) where $x = \mathsf{cochlea\_f2x(species, cf)}$ is thedistance alongthe basilar membrane, $A_0 = 8.13$ (ms), $A_1 = 6.49$
(nm). \citet{HeinzZhangEtAl:2001} then corrected the peak click to
match the onset delay of ANFs and this has been retained in the model
used here \citep{ZilanyBruceEtAl:2009}: \(delay = A_0 * {\rm exp}(-x / A_1) * 1e-3
\) where $A_0 = 3.0$, $A_1 = 12.5$. In
humans, the delay function is: \( delay = 4.915 +
0.3631 * {\rm exp}(0.11324*x) , 5<x<35 (mm) \).

\medskip{}

Steps for converting any Carney Auditory Model written in C for
MATLAB to a NEURON NMODL compatible C file.  
\begin{enumerate} 
\item Remove mex headers 
\item remove \textsf{mexfunction}, this is
  replaced with equivalent NMODL function that retrieves variables and
  returns the equivalent vectors
\item Find and replace all vector or memory allocation routines with functions
  in scopmath.h
\begin{itemize} 
\item \textsf{mxCalloc}$\Rightarrow$\textsf{makevector}
\item \textsf{mxFree} $\Rightarrow$ \textsf{freevector}
\end{itemize} 
\item Replace random generator functions \textsf{drand48()} to
  \mbox{\textsf{scop\_random()}} and let random seed be set in NEURON or replace \textsf{srand(seed)} with
  \mbox{\textsf{set\_seed(seed)}}. 
\item Calls to MATLAB functions within a mex file The most recent model  within the mex file.  This was
  converted to a simple double array of random values with scoplib's
  \mbox{\textsf{normrand(0,1)}} function.
\begin{itemize}
\item builtin \textsf{randn} $\Rightarrow$ generate an array gaussian random numbers with scoplib's
  \mbox{\textsf{normrand(0,1)}} function
\item builtin \textsf{resample} $\Rightarrow$ implementation of a reliable resampling function in C, using
  a real-time resample library (libresample [reference needed]).
\item M-file \textsf{ffGn} $\Rightarrow$ generate a C function that implements a the fast fractional Gaussian noise generator
\end{itemize}

\end{enumerate}





\subsection{Golgi Cell Model}

 \yellownote{More detail in golgi model}
Each golgi cell template consists of a spike generator, \emph{s}, and vector
objects representing the instantaneous rate, the spike times and
accumulated  spike times of the golgi cell. Parameters identifying
each cell include the \emph{channel} number, CF and bandwidth of ANF
input (actually the variance of the weight each auditory filter
channel contributes to the firing of the cell).

The instantaneous rate vector of a golgi cell model at CF channel
\emph{i} is created by :
\begin{itemize}
 \item a selection of ANF input channels centred at \emph{i} with a
 spread, \sLSRGLG;
 \item the instantaneous rate vectors of the LSR ANF's in each channel
 are weighted (with a gaussian function (mean=\emph{i},
 s.d.=$\sqrt{\sLSRGLG}$) and the weighted vectors are averaged; and
 \item the weighted average vector is convolved with an alpha function
 of time constant 5, to simulate the synaptic and membrane dynamics of
 golgi cells
\end{itemize}

 \medskip{}

\begin{lstlisting}[label=GolgiTemplate,caption=Create golgi cell rate vector within Golgi template (in CNcell.tem)]
weight_sum_LSR = 1
weight_sum_HSR = 0.01
golgi_spon = 1                             /* spikes per second */
golgi_spatial_filter_variance = 4          /* channel */
golgi_syn_filter_tau = 0.0005              /* seconds */
golgi_syn_filter_scale_factor=1
objref golgi_synfilter

func alpha(){ /*Alpha function synaptic/membrane filter*/
  return $1*sg_tdres*exp(-($1*sg_tdres)/golgi_syn_filter_tau)
}
proc CreateGolgiSynFilter(){  /* create alpha function */
  golgi_synfilter = new Vector(sg_rate*10*golgi_syn_filter_tau)
  golgi_synfilter.indgen().apply("alpha")  
  golgi_synfilter.mul(golgi_syn_filter_scale_factor/golgi_synfilter.sum()) 
}
CreateGolgiSynFilter()
...
begintemplate Golgicell
...
proc SetRate2() {local i,j,spon_factor
...
/* Calculate weight vectors based on this cell's position */
  gaussian_mean = channel
  gaussian_variance = golgi_spatial_filter_variance_LSR
  wLSR.apply("gaussian").mul(weight_sum_LSR)
  gaussian_variance = golgi_spatial_filter_variance_HSR
  wHSR.apply("gaussian").mul(weight_sum_HSR)

/* Add LSR and HSR vectors to tempsout, check weights */
  for i=0,nchannels-1  {
    if (wLSR.x[i]>0.001) tempsout.add(LSRout[i].c.add(-Lowspont).mul(wLSR.x[i]))
    if (wHSR.x[i]>0.001) tempsout.add(HSRout[i].c.add(-Highspont).mul(wHSR.x[i]))
  }
/* Correct for spontaneous rate */
  tempsout.add(golgi_spon)
/* Smooth AN input to simulate dendritic filtering */
    convolve(tempsout,golgi_synfilter,sout)
/* Add conduction delay to rate profile */
    for i = 0, param.delay.x[lsr][glg]*sg_rate/1000{
        sout.insrt(0,golgi_spon)
        sout.remove(sout.size-1)
    }
/* Push vector to spike generator object, s */
    s.SetFibreRate(sout,spiketimes,sg_tdres)
}
...
endtemplate Golgicell
\end{lstlisting}


\subsection{Cochlear nucleus cell templates}

The cochlear nucleus cell models were introduced by \citet{RothmanManis:2003b}, after exploratory studies on the potassium currents of murine cochlear nucleus neurons \citep{RothmanManis:2003,RothmanManis:2003a}. Listing~\ref{lst:CellTemplate}  shows 

\begin{lstlisting}[label=lst:CellTemplate,caption=Rothman and Manis cochlear nucleus cell template (in CNcell.tem)]
begintemplate CNcell
  public soma, dend, AMPA, GlyR, GABAA, spiketimes, spikecount
  public spikes,connect2target, synlist,cf,model,channel
...
proc init() { local model
  create soma
   synlist = new List()
    spiketimes = new Vector()
    lastspikecount = 0
    spikes = new Vector()
    
    if($1 == 0) set_tstellate() //Tstellate cell
    if($1 == 1) set_tuberc() //Tubervuloventral cell
    if($1 == 2) set_dstellate() //Dstellate cell
    if($1 == 3) set_golgi() //golgi cell
    cf = $2
    channel=$3

    soma {
	insert rm  /* Na, KHT, Ih, leak currents */
        eh_rm = -43 

      AMPA = new ExpSyn(0.5)
      AMPA.tau  = ampa_decay
      AMPA.e 	= ampa_erev

      GlyR = new Exp2Syn(0.5)
      GlyR.tau2	= gly_decay
      GlyR.tau1	= gly_rise
      GlyR.e 	= gly_erev

      GABAA = new Exp2Syn(0.5)
      GABAA.tau2 	= gaba_decay
      GABAA.tau1 	= gaba_rise
      GABAA.e 	= gaba_erev
    }


      ena = 50		// mV
      ek  = -70		//  mV


  }	// end init()


func nstomho() {//convert S to nS/cm2
 return (1E-9*$1/somaarea)
}

func qt(){ //temperature quotient in current channels
 if(numarg()==1){//quality, temperature coefficient for Ih conductance	
 	return 1.3^((celsius-22)/10)
 } else {
	return 2^((celsius-22)/10)
 }
}
proc set_tstellate() {
 soma {
	insert ka
	...
	ghbar_rm = nstomho(0.5)*qt(1)
        ...  
 }
}
...
endtemplate CNcell 

\end{lstlisting}


\end{appendix}


%%% Local Variables: 
%%% mode: latex
%%% TeX-master: "SimpleResponses"
%%% TeX-PDF-mode: nil
%%% End: 
