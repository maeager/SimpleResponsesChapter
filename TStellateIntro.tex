 

%===================================
\section[TS Cell Model]{T~Stellate Cell Model: Optimisation of Three Chopper Subtypes}

\subsection{Background}
 
Accurate modelling of the cochlear nucleus, in particular chopper units and T~stellate cells.  %%

\yellownote{Expand background}

This work expands on the extracellular classification of supposed TS cells into choppers \citep{Bourk:1976,Pfeiffer:1966}, then  into slowly or transiently adapting and sustained choppers \citep{BlackburnSachs:1989,YoungRobertEtAl:1988}.  %%
\citet{SmithRhode:1989} was the first to do simultaneous labelling and physiological responses in cats, which was followed by others in gerbils \citep{OstapoffFengEtAl:1994,FengKuwadaEtAl:1994}, guinea pigs \citep{PalmerWallaceEtAl:2003,ArnottWallaceEtAl:2004}, and rats \citep{PaoliniClarkEtAl:1997,PaoliniClark:1999,PaoliniClareyEtAl:2004}.  %%


Averaging intracellular responses to acoustic input to determine stochastic excitatory\slash inhibitory inputs was first used in the chopper units of gerbils \citep{OstapoffFengEtAl:1994,FengKuwadaEtAl:1994}.  %%
\citet{PaoliniClareyEtAl:2005} used similar averaging of TS and DS cell intracellular responses in rats as the basis for a  thorough statistical analysis to separate chopper units into three distinct subtypes.  %%
Figure~\ref{fig:PaoliniAIV} shows the intracellular acoustic classification of chopper units in the rat into three distinct types \citep{PaoliniClareyEtAl:2005}: CS chopper sustained, and two transient choppers, CT1 and CT2.  %%
The work by Tony Paolini, Janine Clarey, Karina Needham and others at the Royal Victorian Eye and Ear Hospital were pivotal in collecting the T stellate cell experimental data used in this section.  %%


The intracellular traces in Figure~\ref{fig:PaoliniAIV} along with the CV statistics Figure~\ref{fig:PaoliniCVdata}  will form the basis for the optimisation routine of the T~stellate cell model of each chopper type.  %%


\begin{figure}[htb]
\centering%
\subfloat[Chopper Sustained]{\includegraphics[keepaspectratio,width=0.3\textwidth]{TStellate/CS-01-864-004-a}}\hfill%\quad%
\subfloat[Chopper Transient 1]{\includegraphics[keepaspectratio,width=0.3\textwidth]{TStellate/CT1-01-857-007}}\hfill%\\
\subfloat[Chopper Transient 2]{\includegraphics[keepaspectratio,width=0.3\textwidth]{TStellate/CT2-01-305-014}}%\quad%
%\subfloat[Onset Chopper]{\resizebox{0.35\textwidth}{!}{\includegraphics{TStellate/OC-99-812-013}}}
\caption[Average intracellular response data in stellate cells in rats.]{Average intracellular response to CF tone 30dB above depolarisation threshold in stellate  cells~\citep[Reproduced from Fig.~2, ][]{PaoliniClareyEtAl:2005}.
A. Sustained chopper unit 01-864-004, CF 3.8~kHz,
B. Transient chopper type 1 unit 01-857-007, CF 8.9~kHz.
C. Transient chopper type 2 unit 01-305-014 CF 12.3~kHz.
%D. Onset chopper unit  99-812-013.
Hyper polarisation after tone indicated by asterisk.  \label{fig:PaoliniAIV}}
\end{figure}
\yellownote{Permission needed for Paolini plots}

The categorisation via coefficient of variation is shown in Figure~\ref{fig:PaoliniCVdata}.  %%
Sustained choppers maintain a stable CV below 0.2 throughout the entire stimulus. The transient chopper optimisation had two types defined by \citep{PaoliniClareyEtAl:2005}.  %%
The first CT type is categorised with CV starting below 0.2 then rising, hence the name transient, but stays below 0.3.  %%
The second CT type is regular in the first 10 ms period, but rises to 0.3 or above throughout the stimulus.  %%


\begin{figure}[htb]
\centering%
%\resizebox{0.6\textwidth}{!}{
\includegraphics{TStellate/PaoliniCV.eps}
%}
\caption{Regularity in chopper units \citep[Data reproduced from Fig.~2,~][]{PaoliniClareyEtAl:2005}}
\label{fig:PaoliniCVdata}
\end{figure}

%===================================
\subsection{Implementation}

\yellownote{Para 1: Nordlie table~\ref{tab:TSModelSummary}A. Using \RM  type I-t, Other previous models}
\yellownote{Synaptic inputs known and unknown, included and not included in model.  Previous work include and excludes}

Figure~\ref{fig:TSinputs} shows the expected response of a T~stellate cell to individual connections from different cells in the CN stellate network.  %%
The membrane parameters for the single compartment T~stellate cell model are default except for sodium conductance set to zero.  %%
In this example, excitation from the afferent \ANF~inputs (\LSR~Figure~\ref{fig:TSinputs}A and \HSR~Figure~\ref{fig:TSinputs}B) show a large depolarisation.  %%
\HSR~inputs show a rapid onset and a slowly adapting sustained depolarisation.  %%



\yellownote{Actual parameters: diameter =19.5$\mu$m, $\gNa=0$, $\gKHT=0.0189416$ S cm$^{-2}$, $\gleak=0.000473539$ S cm$^{-2}$, $\gh=$6.20392e-05 S cm$^{-2}$, $\gKA=0.01539$ S cm$^{-2}$, $\Eleak=-65$ mV, $\ENa=50$ mV, $\EK=-70$ mV.}


\begin{figure}[htb]
\centering%
\includegraphics[keepaspectratio,width=0.9\textwidth]{TStellate/baseline_exc}\\
\includegraphics[keepaspectratio,width=0.9\textwidth]{TStellate/baseline_inh}
\caption[Response of T~stellate cells to isolated synaptic inputs]%
{Intracellular membrane voltage response of a T~stellate cell model to isolated synaptic inputs.
A pure tone stimulus of 8.2~kHz at 85 dB~SPL was presented to the CN network. The CF of the recorded TS unit was 8.267~kHz.
Single stimulus responses are shown as a thin line and average response over 25 repetitions is shown as the dark line.
A. 30 LSR ANF synapses.
B. 20 HSR ANF synapses.
C. 20 D stellate cell glycinergic synapses.
D. 15 Golgi cell \GABAa synapses.
All weights were set to $0.0005\,\mu{\rm S}$ and the sodium conductance (\gNa)set to zero.
The parameters for synapse's were: excitatory (tau = 0.36 ms), glycinergic (tau1=0.4 and tau2=2.5 ms), and GABAergic (tau1=0.26 and tau2=5.43 ms).\label{fig:TSinputs}}
\end{figure}

\begin{figure}[htb]
\centering%
% \resizebox{0.9\textwidth}{!}{\includegraphics{TStellate/baseline_jitter}}
\includegraphics[keepaspectratio=true,width=0.9\textwidth]{TStellate/baseline_jitter}
  % \caption[Response of T~stellate cells to isolated synaptic
  % inputs]{Intracellular membrane voltage response of a T~stellate cell
  %   model to isolated synaptic inputs. A pure tone stimulus of 8.2~kHz at
  %   85 dB~SPL was presented to the CN network. The CF of the recorded
  %   T~stellate cell was 8.267~kHz.  Single stimulus responses are shown as
  %   a thin line and average response over 25 repetitions is shown as the
  %   dark line. A. 30 LSR ANF synapses. B. 20 HSR ANF synapsexs. C. 20 D
  %   stellate cell glycinergic synapses. D. 15 Golgi cell GABA$_{\rm A}$
  %   synapses. All weights were set to $0.0005\,\mu{\rm S}$ and the sodium
  %   conductance set to zero.  The parameters for synapases were: excitatory
  %   (tau = 0.36 ms), glycinergic (tau1=0.4 and tau2=2.5 ms), and
  %   GABAergic (tau1=0.26 and tau2=5.43 ms).\label{fig:TSExcinputs}}
\caption[]{Jitter of AN input T~stellate Optimisation results}\label{fig:CSjitter}
\end{figure}





%%% Local Variables:
%%% mode: latex
%%% mode: tex-fold
%%% mode: visual-line
%%% TeX-master: "SimpleResponses"
%%% TeX-PDF-mode: nil
%%% End:
